\documentclass{article}

\usepackage[utf8]{inputenc}
\usepackage[english]{babel}
\usepackage[margin=3cm]{geometry}
\usepackage[normalem]{ulem}
\usepackage{hyperref}
\usepackage[shortlabels]{enumitem}
\usepackage{mathtools, amsmath, amssymb, amsthm, mdframed, bbm, graphicx, float, physics, xcolor, cleveref}

\hypersetup{
    colorlinks   = true, %Colours links instead of ugly boxes
    urlcolor     = blue, %Colour for external hyperlinks
    linkcolor    = blue, %Colour of internal links
    citecolor   = red %Colour of citations
}

% Definition of numbered environments.
% Usage: \begin{theorem} ... \end{theorem}
% Remark has no numbering.
\theoremstyle{plain}
\newtheorem{question}{Question}
\newtheorem{theorem}{Theorem}
\newtheorem{lemma}{Lemma}
\theoremstyle{remark}
\newtheorem*{remark}{Remark}

% Question with box around it
\newenvironment{qbox}{\begin{mdframed}\begin{question}}{\end{question}\end{mdframed}}

% A proof with "solution" instead of "proof" and no QED symbol
\newenvironment{solution}{\begin{proof}[Solution]\renewcommand\qedsymbol{}}{\end{proof}}

% Some renewed commands
\renewcommand{\vec}{\boldsymbol}
\renewcommand{\emptyset}{\varnothing}
\renewcommand{\epsilon}{\varepsilon}
\renewcommand{\theta}{\vartheta}
\renewcommand{\phi}{\varphi}

% Frequently used math alphabets
\newcommand{\Bb}{\mathbb}
\newcommand{\Cal}{\mathcal}
\newcommand{\Bf}{\mathbf}
\newcommand{\Rm}{\mathrm}

% Frequently used letters in the blackboard alphabet
\newcommand{\CC}{\Bb C}
\newcommand{\NN}{\Bb N}
\newcommand{\PP}{\Bb P}
\newcommand{\QQ}{\Bb Q}
\newcommand{\RR}{\Bb R}
\newcommand{\EE}{\Bb E}
\newcommand\HH{\Cal H}
\newcommand\FF{\Cal F}
\newcommand\Nn{\Cal N}
\renewcommand\SS{\Cal S}

% Usage: \ang{...} is equivalent to \langle ... \rangle, while \ang*{...} is equivalent to \left\langle ... \right\rangle
% For other delimiters: use \qty from the physics package (i.e., \qty(...))
\DeclarePairedDelimiter{\ang}{\langle}{\rangle}
\DeclarePairedDelimiter{\floor}{\lfloor}{\rfloor}
\DeclarePairedDelimiter{\ceil}{\lceil}{\rceil}

% Frequently used commands
\newcommand{\T}{^\top} % Matrix transpose A\T
\newcommand{\C}{^\complement} % Set complement A\C
\newcommand\ceq\coloneqq % Definitions :=
\newcommand\pow{\Cal P} % Power sets
\newcommand\eps\epsilon
\newcommand\ind{\mathbbm 1} % Blackboard 1 for indicator functions
\newcommand\restr{\mathord\restriction}
\newcommand\TODO{{\color{red} TODO: }}
\newcommand\toprob{\overset{\Rm{p}}{\to}}
\newcommand\todist{\overset{\Rm{d}}{\to}}
\newcommand\toas{\overset{\Rm{a.s.}}{\to}}
\newcommand\deq{\overset{\Rm{d}}=}
\newcommand\iid{\overset{\Rm{iid}}{\sim}}

% Functions that appear frequently
\DeclareMathOperator{\sign}{sign}
\DeclareMathOperator{\Int}{Int}
\DeclareMathOperator{\Span}{Span}
\DeclareMathOperator{\Var}{Var}
\DeclareMathOperator*{\argmin}{arg\,min}
\DeclareMathOperator*{\argmax}{arg\,max}
\DeclareMathOperator\Exp{Exp}
\DeclareMathOperator\MSE{MSE}
\DeclareMathOperator\MISE{MISE}
\DeclareMathOperator\Bias{Bias}
\DeclareMathOperator\diag{diag}
\DeclareMathOperator\CV{CV}

\newcommand\XX{\Cal X}
\renewcommand\AA{\Cal A}
\newcommand\BB{\Cal B}
\newcommand\TT{\Cal T}

\title{Topics  --- Example Sheet  2} % subject
\author{Lucas Riedstra}
\date{22 November 2020} % date

\begin{document}
\maketitle

\setcounter{question}{1}

\begin{question}
	For $\beta \in (0, 1]$ and $L > 0$, let $\FF_2(\beta, L)$ denote the class of densities on $\RR^2$ that satisfy
	\[
	\abs{f(x, y) - f(x_0, y_0)} \leq L(\abs{x - x_0}^\beta + \abs{y - y_0}^\beta)
	\]
	for all $(x, y), (x_0, y_0) \in \RR^2$. Let $K$ be a non-negative kernel on $\RR$ with $\mu_\beta(K)$ and $R(K)$ finite. Given i.i.d.\ pairs $(X_1, Y_1), \dotsc, (X_n, Y_n)$, consider the kernel density estimator $\hat f_n$ obtained using a product kernel, i.e., 
	\[
	\hat f_n(x_0, y_0) \ceq \frac1{nh^2} \sum_{i=1}^n K\qty(\frac{x_0 - X_i}{h}) K\qty(\frac{y_0 - Y_i}{h}). 
	\]
	Find a bound on $\MSE\qty{\hat f_n(x_0, y_0)}$ that holds uniformly for all $f \in \FF_2(\beta, L)$ and $(x_0, y_0) \in \RR^2$. 
\end{question}

\begin{proof}
	First we compute a variance bound following the proof of proposition 19: noting that $\hat f_n(x, y) = \frac1n \sum_{i=1}^n K_h(x - X_i) K_h(y - Y_i)$, we have
	\begin{align*}
		\Var \hat f_n(x, y) &= \frac1n \Var(K_h(x - X_i) K_h(y - Y_i)) \leq \frac1n \EE[K_h^2(x - X_i) K_h^2(y - Y_i)] \\
		&= \frac1{nh^4}  \iint_{\RR^2} K^2(\frac{x-w}{h}) K^2(\frac{y - z}{h}) f(w, z) \dd{w} \dd{z} \\
		&\leq  \frac{\norm{f}_\infty}{nh^2} \iint_{\RR^2} K^2(s) K^2(t) \dd{s} \dd{t} = \frac{\norm{f}_\infty R(K)^2}{nh^2}.
	\end{align*}

	Next, we compute a bias bound following the proof of proposition 22. Note that we have
	\begin{align*}
		\Bias \hat f_n(x, y) &= \frac1{h^2} \iint_{\RR^2} K(\frac{x-w}{h}) K(\frac{y-z}{h}) f(w, z) \dd{w}\dd{z} - f(x, y) \\
		&= \iint_{\RR^2} K(s) K(t)\qty{f(x - sh, y - th) - f(x, y)} \dd{s} \dd{t}, 
	\end{align*}
	and taking absolute values gives
	\begin{align*}
		\abs{\Bias\hat f_n(x, y)} &\leq \iint_{\RR^2} {K(s)} {K(t)} \abs{f(x - sh, y - th) - f(x, y)} \dd{s} \dd{t} \\
		&\leq L \iint_{\RR^2} {K(s)} {K(t)} \qty(\abs{sh}^\beta + \abs{th}^\beta) \dd{s} \dd{t} \\
		&= 2Lh^\beta \int_\RR {K(s)} \int_\RR \abs{t} {K(t)} \dd{t} \dd{s} \\
		&= 2Lh^\beta \mu_\beta(K) \int_\RR {K(s)} \dd{s} = 2Lh^\beta \mu_\beta(K). 
	\end{align*}
	
	We therefore have
	\[
	\MSE \hat f_n(x, y) \leq \frac1{nh^2} \norm{f}_\infty R(K)^2 + 4L^2\mu_\beta(K)^2 h^{2\beta}. 
	\]
	Completely analogous to the proof of theorem 23, we can show that $\norm{f}_\infty$ is bounded uniformly over $\FF_2(\beta, L)$, and the minimiser of $\MSE$ is of order $n^{-1/(2\beta + 2)}$. Plugging this into the expression gives 
	\[
	\sup_{(x, y)} \sup_{f \in \FF} \MSE \hat f_n(x, y) \leq C n^{-2\beta/(2\beta + 2)},
	\]
	for some $C$ depending only on $\beta, L, K$. 
\end{proof}


\setcounter{question}{9}
\begin{question}
Let $n \geq 3$, let $a \leq x_1 < \dotsb < x_n \leq b$, and let $\vec g \in \RR^n$. Prove that the natural cubic spline interpolant to $\vec g$ at $x_1, \dotsc, x_n$ is the unique minimiser of $R(\tilde g) = \int_a^b \tilde g''(x)^2 \dd{x}$ over all $\tilde g \in \SS_2[a, b]$ that interpolate $\vec g$ at $x_1, \dotsc, x_n$. 

\emph{Hint:} Let $g$ be the natural cubic spline interpolant, $h \ceq \tilde g - g$, and consider $\int_a^b g''(x) h''(x) \dd{x}$. 
\end{question}

\begin{solution}
As in the hint we let $g$ be the natural cubic spline interpolant, $\tilde g\in \SS_2[a, b]$, and $h \ceq \tilde g - g$. We know that $g$ is a cubic polynomial $p_i$ on each interval $[x_i, x_{i+1}]$ (denote its leading coefficient by $c_i$), that $g''$ is continuous, and that $g'' = 0$ on $[a, x_1]$ and on $[x_n, b]$. Furthermore, we know that $h(x_i) = 0$ for all $i$. 
Using this, we can write
\begin{align*}
	\int_a^b g''(x) h''(x) \dd{x} &= \sum_{i=1}^{n-1} \int_{x_i}^{x_{i+1}} p_i''(x) h''(x) \dd{x} \\
	&= \sum_{i=1}^{n-1} \qty(\qty[p_i''(x) h'(x)]_{x = x_i}^{ x_{i+1}} - \int_{x_i}^{x_{i+1}} p_i'''(x) h'(x) \dd{x}) \\
	&= - \sum_{i=1}^{n-1} c_i \int_{x_i}^{x_{i+1}} h'(x) \dd{x} \\
	&= - \sum_{i=1}^{n-1} c_i \qty(h(x_{i+1}) - h(x_i)) = 0. 
\end{align*} 
Now we find
\begin{align*}
	R(g) &= \int_a^b g''(x)^2 \dd{x} = \int_a^b g''(x) \qty(\tilde g''(x) - h''(x)) \dd{x} = \int_a^b g''(x) \tilde g''(x) \overset{\Rm{CS}}{\leq} \sqrt{R(g)} \sqrt{R(\tilde g)},
\end{align*}
with equality if and only if $g'' = \tilde g''$ (by Cauchy-Schwarz).
Rearranging gives $\sqrt{R(g)} \leq \sqrt{R(\tilde g)} \implies R(g) \leq R(\tilde g)$. Since $\tilde g \in \SS_2[a, b]$ was arbitrary, we deduce that $g$ is a minimiser of $R$ over all function in $\SS_2$. 

For uniqueness: we already know that any other minimiser $h \in \SS_2$ must satisfy $g'' = h''$ a.e., so $g - h$ is a polynomial of degree 1 a.e., and by continuity we know that $g - h$ is a polynomial of degree 1. However, since $g$ and $h$ must agree on $n \geq 3$ points it follows that $g = h$. 
\end{solution}
\end{document}