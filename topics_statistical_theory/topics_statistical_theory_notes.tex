\documentclass{article}

\usepackage[utf8]{inputenc}
\usepackage[english]{babel}
\usepackage[margin=3cm]{geometry}
\usepackage[normalem]{ulem}
\usepackage{hyperref}
\usepackage[shortlabels]{enumitem}
\usepackage{mathtools, amsmath, amssymb, amsthm, mdframed, bbm, graphicx, float, physics, xcolor, cleveref}

\hypersetup{
    colorlinks   = true, %Colours links instead of ugly boxes
    urlcolor     = blue, %Colour for external hyperlinks
    linkcolor    = blue, %Colour of internal links
    citecolor   = red %Colour of citations
}


% Definition of numbered environments.
% Usage: \begin{theorem} ... \end{theorem}
% Remark and Convention have no numbering.
\theoremstyle{plain}
\newtheorem{theorem}{Theorem}[section]
\newtheorem{lemma}[theorem]{Lemma}
\newtheorem{corollary}[theorem]{Corollary}
\newtheorem{proposition}[theorem]{Proposition}
\theoremstyle{definition}
\newtheorem{definition}[theorem]{Definition}
\newtheorem{example}[theorem]{Example}
\newtheorem{algorithm}[theorem]{Algorithm}
\newtheorem{discussion}[theorem]{Discussion}
\newtheorem{consequence}[theorem]{Consequence}
\newtheorem{rec}[theorem]{Recap}
\theoremstyle{remark}
\newtheorem*{remark}{Remark}
\newtheorem*{convention}{Convention}


\newenvironment{recap}{\begin{mdframed}\begin{rec}}{\end{rec}\end{mdframed}}

% Some renewed commands
\renewcommand{\vec}{\boldsymbol}
\renewcommand{\emptyset}{\varnothing}
\renewcommand{\epsilon}{\varepsilon}
\renewcommand{\theta}{\vartheta}
\renewcommand{\phi}{\varphi}

% Frequently used math alphabets
\newcommand{\Bb}{\mathbb}
\newcommand{\Cal}{\mathcal}
\newcommand{\Bf}{\mathbf}
\newcommand{\Rm}{\mathrm}

% Frequently used letters in the blackboard alphabet
\newcommand{\CC}{\Bb C}
\newcommand{\NN}{\Bb N}
\newcommand{\PP}{\Bb P}
\newcommand{\QQ}{\Bb Q}
\newcommand{\RR}{\Bb R}
\newcommand{\EE}{\Bb E}
\newcommand\HH{\Cal H}
\newcommand\FF{\Cal F}
\newcommand\Nn{\Cal N}
\renewcommand\SS{\Cal S}

% Usage: \ang{...} is equivalent to \langle ... \rangle, while \ang*{...} is equivalent to \left\langle ... \right\rangle
% For other delimiters: use \qty from the physics package (i.e., \qty(...))
\DeclarePairedDelimiter{\ang}{\langle}{\rangle}
\DeclarePairedDelimiter{\floor}{\lfloor}{\rfloor}
\DeclarePairedDelimiter{\ceil}{\lceil}{\rceil}

% Frequently used commands
\newcommand{\T}{^\top} % Matrix transpose A\T
\newcommand{\C}{^\complement} % Set complement A\C
\newcommand\ceq\coloneqq % Definitions :=
\newcommand\pow{\Cal P} % Power sets
\newcommand\eps\epsilon
\newcommand\ind{\mathbbm 1} % Blackboard 1 for indicator functions
\newcommand\restr{\mathord\restriction}
\newcommand\TODO{{\color{red} TODO: }}
\newcommand\toprob{\overset{\Rm{p}}{\to}}
\newcommand\todist{\overset{\Rm{d}}{\to}}
\newcommand\toas{\overset{\Rm{a.s.}}{\to}}
\newcommand\deq{\overset{\Rm{d}}=}
\newcommand\iid{\overset{\Rm{iid}}{\sim}}

% Functions that appear frequently
\DeclareMathOperator{\sign}{sign}
\DeclareMathOperator{\Int}{Int}
\DeclareMathOperator{\Span}{Span}
\DeclareMathOperator{\Var}{Var}
\DeclareMathOperator*{\argmin}{arg\,min}
\DeclareMathOperator*{\argmax}{arg\,max}
\DeclareMathOperator\Exp{Exp}
\DeclareMathOperator\MSE{MSE}
\DeclareMathOperator\MISE{MISE}
\DeclareMathOperator\Bias{Bias}
\DeclareMathOperator\diag{diag}
\DeclareMathOperator\CV{CV}
\DeclareMathOperator\Div{Div}
\DeclareMathOperator\KL{KL}
\DeclareMathOperator\Hh{H}
\DeclareMathOperator\TV{TV}

\newcommand\XX{\Cal X}
\renewcommand\AA{\Cal A}
\newcommand\BB{\Cal B}
\newcommand\TT{\Cal T}
\newcommand\YY{\Cal Y}
\newcommand\ac{_\Rm{ac}}
\newcommand\sing{_\Rm{sing}}

\title{Topics in Statistical Theory --- Summary}
\author{Lucas Riedstra}

\begin{document}
\maketitle
\tableofcontents
\section{Distributions}
%\begin{recap}
%    A \emph{topological field} $\KK$ is a field endowed with a topology such that addition, multiplication, and inversion are continuous.
%    
%    A \emph{topological vector space} $V$ over a topological field $\KK$ is a vector space endowed with a topology such that vector addition and scalar multiplication are continuous. 
%    
%    The \emph{topological dual space} $V^*$ of $V$ is the set of all continuous linear maps from $V$ to $\KK$. 
%\end{recap}
\subsection{Test functions and distributions}
\begin{definition}
    Let $X \subseteq \RR^n$ be open, then we define the set of \emph{test functions} on $X$ as
    \[
    \Cal D(X) \ceq C^\infty_0(X) = \qty{f \colon X \to \CC \mid f \text{ is smooth with compact support}}. 
    \]
\end{definition}

\begin{definition} \label{def:test_convergence}
    Let $(\phi_m) \subseteq \Cal D(X)$. We say that $(\phi_m) \to 0$ in $\Cal D(X)$ if
    \begin{enumerate}
        \item  there exists a compact $K \subseteq X$ such that $\supp\phi_m \subseteq K$ for all $m$;
        \item $\pt^\alpha \phi_m \to 0$ uniformly for each multi-index $\alpha$.
    \end{enumerate}
\end{definition}

Note that, for any $\phi, \psi \in \Cal D(X)$ and any multi-index $\alpha$ we have
\[
\int_X \phi\cdot  \pt^\alpha \psi \dd{x} = (-1)^{\abs{\alpha}} \int_X \psi \cdot \pt^\alpha \phi \dd{x}, 
\]
which follows from partial integration and the fact that all boundary terms vanish since $\phi$ and $\psi$ have compact support. 

Also, by Taylor's theorem, for any $\phi \in \Cal D(X)$, $x, h \in X$ and $N \in \NN$ we have
\[
\phi(x + h) = \sum_{\abs{\alpha} \leq N} \frac{h^\alpha}{\alpha!} \pt^\alpha \phi(x) + R_N(x, h) \quad\text{where } R_N(x, h) = o(\abs{h}^N) \text{ uniformly in $x$}. 
\]

\begin{definition}
    A \emph{distribution} on $X$ is a linear map $u \colon \Cal D(X) \to \CC$ if for every compact set $K \subseteq X$ there exist constants $C, N$ such that for all $\phi \in \Cal D(X)$ with $\supp\phi\subseteq K$ we have
    \begin{equation} \label{eq:seminorm_cond}
        \abs{u(\phi)} \leq C \sum_{\abs{\alpha} \leq N} \sup \abs{\pt^\alpha \phi}.
    \end{equation}

Condition \ref{eq:seminorm_cond} is called the \emph{seminorm condition}. If, in the seminorm condition, the same $N$ can be used for every compact set $K \subseteq X$, then the least such $N$ is called the \emph{order} of $u$, written $\ord(u)$. 

The set of all distributions in $X$ is denoted $\Cal D'(X)$. 
\end{definition}

If $u \in \DD'(X)$ and $\phi \in \DD(X)$, then instead of $u(\phi)$ we usually write $\ang{u, \phi}$.

\begin{recap}
    A function $f \colon X \to \CC$ is called \emph{locally integrable} if $\int_K \abs{f} \dd{x} < \infty$ for all compact $K \subseteq X$. 
    
    The set of locally integrable functions on $X$ is denoted $\locint(X)$. 
\end{recap}

\begin{example}
    Let $M \in \NN$ and let $f_\alpha \in \locint(X)$ for all $\abs{\alpha} \leq M$. Define the linear map $T \colon \DD(X) \to \CC$ by
    \[
    \ang{T, \phi} \ceq \sum_{\abs{\alpha} \leq M} \int_X f_\alpha \cdot \pt^\alpha \phi \dd{x}. 
    \]
    It is trivial that $T$ is linear, and we verify that $T$ is a distribution as follows: take $\phi \in \Cal D(X)$ with $\supp\phi \subseteq K$. Then we have
    \begin{align*}
        \abs{\ang{T, \phi}} &\leq \sum_{\abs{\alpha} \leq M} \int_K \abs{f_\alpha} \cdot \abs{\pt^\alpha \phi} \dd{x} \\
        &\leq \sum_{\abs{\alpha} \leq M} \sup \abs{\pt^\alpha \phi} \cdot \int_K \abs{f_\alpha} \dd{x} \\
        &\leq \qty(\max_\alpha \int_K \abs{f_\alpha} \dd{x})   \sum_{\abs{\alpha} \leq M} \sup \abs{\pt^\alpha \phi}. 
    \end{align*}
Therefore, the seminorm condition is satisfied with $N = M$. From this, it also follows that $\ord(T) \leq M$. 
\end{example}

A special case of the previous example is the case $M = 0$: in this case the distribution simply becomes
\[
\ang{\tau_f, \phi} = \int_X f \phi \dd{x}. 
\]
Henceforth we will abuse notation: if $f \in \locint(X)$, then we will write $f$ instead of $\tau_f$, i.e., $\ang{f, \phi} = \int_X f\phi \dd{x}$. 

\begin{lemma}[Sequential continuity] \label{lem:seq_continuity}
    Let $u \colon \DD(X) \to \CC$ be a linear map. Then $u$ is a distribution if and only if, for every sequence $(\phi_m) \subseteq \DD(X)$ with $\phi_m \to 0$ as in \cref{def:test_convergence}, we have $\ang{u, \phi_m} \to 0$. 
\end{lemma}

\begin{proof}
    `$\implies$' If $u$ is a distribution and $(\phi_m) \to 0$, then $\supp\phi_m \subseteq K$ for some compact $K$, and by \cref{eq:seminorm_cond} there exist $C, N$ such that
    \[
    \abs{\ang{u, \phi_m}} \leq C \sum_{\abs{\alpha} \leq N} \sup \abs{\pt^\alpha \phi_m} \to 0. 
    \]
    
    `$\impliedby$' Suppose there is a compact set $K$ such that \cref{eq:seminorm_cond} is not valid for any $C, N$. Let $m \in \NN$ and $C = N = m$, then there is some $\phi_m$ with $\supp(\phi_m) \subseteq K$, and 
    \[
    \abs{\ang{u, \phi_m}} > m \sum_{\abs{\alpha} \leq m} \sup \abs{\pt^\alpha \phi_m}. 
    \]
    By dividing $\phi_m$ by $\ang{u, \phi_m} \neq 0$, we may assume w.l.o.g.\ that $\ang{u, \phi_m} = 1$. 
    We now have a sequence $(\phi_m)$ such that
    \[
    \frac1m > \sum_{\abs{\alpha} \leq m} \sup \abs{\pt^\alpha \phi_m} \implies  \abs{\pt^\alpha \phi_m} < \frac1m \quad\text{ for $\abs{\alpha} \leq m$} \implies \pt^\alpha \phi_m \to 0 \text{ uniformly for all $\alpha$}. 
    \]
    Since each $\phi_m$ also satisfies $\supp\phi_m\subseteq K$, by \cref{def:test_convergence} we have that $\phi_m \to 0$, but also $\ang{u, \phi_m} \to 1$, a contradiction. 
\end{proof}

\subsection{Limits in the distribution space}
\begin{definition}
    We say that a sequence $(u_m) \subseteq \DD'(X)$ converges to $u \in \DD'(X)$ and write $u_m \to u$ if 
    \[
    \ang{u_m, \phi} \to \ang{u, \phi} \quad\text{for all $\phi \in \DD(X)$}. 
    \]
\end{definition}

\begin{mdframed}
The following theorem is non-examinable but interesting: 
\begin{theorem}
    Let $(u_m)$ be a sequence in $\DD'(X)$ such that $\lim\limits_{m\to\infty} \ang{u_m, \phi}$ exists for all $\phi \in \DD(X)$. Then the map $\ang{u, \phi} \ceq \lim\limits_{m\to\infty} \ang{u_m, \phi}$ is a distribution in $X$.
\end{theorem}

\begin{proof}
    This is a direct application of the uniform boundedness principle. 
\end{proof}
\end{mdframed}

\begin{example}
    Let $X = \RR$ and consider the sequence of functions $u_m \in \locint(\RR)$ defined by $u_m(x) = \sin(mx)$. Then, for all $\phi \in \DD(\RR)$, we have
    \[
    \ang{u_m, \phi} = \int_\RR \sin(mx) \phi(x) \dd{x} = \frac1m \int_\RR \cos(mx) \phi'(x) \dd{x} \leq \frac1m \int \abs{\phi'(x)} \dd{x} \to 0.   \]
    Therefore, it holds that $u_m \to 0$ in $\DD'(\RR)$. With our abuse of notation we write this as $\lim\limits_{m\to\infty} \sin(mx) = 0$ in $\DD'(\RR)$. 
\end{example}

\subsection{Basic operations}

\subsubsection{Differentiation and multiplication by smooth functions}
For $u \in C^\infty(X)$ and $\phi \in \DD(X)$, we have noted that
\[
\ang{\pt^\alpha u, \phi} = \int_X \pt^\alpha u \cdot \phi \dd{x} = (-1)^\abs{\alpha}  \int_X u \cdot \pt^\alpha \phi \dd{x} = \ang{u, (-1)^\abs{\alpha} \pt^\alpha \phi}. 
\]

Since the RHS makes sense for any distribution $u$, we define 
\begin{definition}
    For $f \in C^\infty(X)$, $u \in \DD'(X)$, we define $\pt^\alpha(fu)$ by
    \[
    \ang{\pt^\alpha(fu), \phi}  \ceq \ang{u, (-1)^\abs{\alpha} f \cdot  \pt^\alpha \phi}
    \]
\end{definition}

\begin{remark}
    This definition outlines a more general pattern when working with distributions: first we take some well-defined operator on the collection of smooth maps, then we rewrite it to a form that is sensible for any distribution, and then we \emph{define} that new form as the operator on distributions. This process is called \emph{extending the definition by duality}.
\end{remark}

\begin{example}
    Let $u = \delta_x$, then we have
    \[
    \ang{\pt^\alpha \delta_x, \phi} = \ang{\delta_x, (-1)^\abs{\alpha} \pt^\alpha\phi} = (-1)^\abs{\alpha} \pt^\alpha \phi(x)
    \]
    
    Furthermore, consider the Heaviside function $H(x) = \ind_{x \geq 0}$. We have
    \[
    \ang{H', \phi} = - \ang{H, \phi'} = -\int_\RR H(x) \phi'(x) \dd{x} = - \int_0^\infty \phi'(x) \dd{x} = \phi(0) = \ang{\delta_0, \phi}, 
    \]
    so we write $H'= \delta_0$ \emph{in the distributional sense}. 
\end{example}

\begin{lemma}
    Suppose $u' \in \DD'(\RR)$ satisfies $u' = 0$. Then $u$ is constant (i.e., $\ang{u, \phi} = \ang{c, \phi} = c \int_\RR \phi \dd{x}$ for some $c$). 
\end{lemma}

\begin{proof}
    Fix  any $\theta \in \DD(\RR)$ with $\ang{1, \theta} = 1$. Let $\phi \in \DD(\RR)$ and define
    \[
    \phi_A \ceq \phi - \ang{1, \phi}\theta, \quad \phi_B \ceq \ang{1, \phi}\theta \quad\text{such that }\phi = \phi_A + \phi_B.
    \]
    
    Note that $\ang{1, \phi_A} = \ang{1, \phi} - \ang{1, \phi}\ang{1, \theta} = 0$.
    
    We claim that the function $\Phi_A(x) \ceq \int_{-\infty}^x \phi_A(y) \dd{y}$ has compact support: since $\supp\phi_A \subseteq [a, b]$ for some $a, b \in \RR$, clearly $\Phi_A(x) = 0$ for $x < a$, while for $x > b$ we have $\Phi_A(x) = \ang{1, \phi_A} = 0$. 
    Obviously, it holds that $\Phi_A' = \phi_a$. Now we compute
    \[
    \ang{u, \phi} = \ang{u, \phi_A} + \ang{u, \phi_B} = \ang{u, \Phi_A'} + \ang{1, \phi}\ang{u, \theta} = - \ang{u', \Phi_A} + c \ang{1, \phi} = \ang{c, \phi}. 
    \]
    Since $\phi$ was chosen arbitrarily this shows that $u$ is constant. 
\end{proof}

\subsubsection{Reflection and translation} 
For $\phi \in \DD(\RR^n), h \in \RR^n$, define the \emph{translation of $\phi$ by $h$} by 
\[
(\tau_h\phi)(x) \ceq \phi(x - h),
\]
and the \emph{reflection of $\phi$} by $\refl\phi(x) \ceq \phi(-x)$. 

Extending the definitions of translation and reflection by duality yields the following:
\begin{definition}
    For $u \in \DD'(\RR^n)$, $h \in \RR^n$, $\phi \in \DD(\RR^n)$, define
    \[
    \ang{\tau_hu, \phi} \ceq \ang{u, \tau_{-h}\phi} \quad\text{and}\quad \ang{\refl u, \phi} \ceq \ang{u, \refl{\phi}}. 
    \]
\end{definition}

\begin{lemma}
    For $u \in \DD'(\RR^n)$, define $V_h \in \DD'(\RR^n)$ for $0 \neq h \in \RR^n$ by
    \[
    V_h \ceq \frac{\tau_{-h}u - u}{\norm{h}}
    \]
    If $(h_j) \subseteq \RR^n$ is a sequence for which $\lim_{j \to \infty} \frac{h_j}{\norm{h_j}} = m \in S^{n-1}$, then $V_{h_j} \to \sum_i m_i \pdv{u}{x_i}$ in $\DD'(\RR^n)$. 
\end{lemma}

\begin{proof}
    By definition, we can write $\ang{V_h, \phi} = \frac{1}{\norm{h}} \ang{u, \tau_h\phi - \phi}$. Now Taylor's theorem tells us that
    \[
    (\tau_h\phi - \phi)(x) = \phi(x - h) - \phi(x) = - \sum_i h_i \pdv{\phi}{x_i}(x) + R(x, h),
    \]
    where $R(x, h) = o(\norm{h})$ in $D(\RR^n)$ (exercise sheet 1, question 2). 
    
    By sequential continuity, we have
    \[
    \lim_{j \to \infty} \ang{V_{h_j}, \phi} = \ang{u, - \sum_i m_i \pdv{\phi}{x_i}} = \ang{\sum_i m_i \pdv{u}{x_i}, \phi}, 
    \]
    which shows that $V_{h_j} \to \sum_i m_i \pdv{u}{x_i}$ in $\DD'(\RR^n)$. 
\end{proof}

\subsubsection{Convolution}
For $\phi \in \DD(\RR^n)$, note that $(\tau_x \refl\phi)(y) = \refl\phi(y-x) = \phi(x-y)$. 
\begin{definition}
    For $u \in C^\infty(\RR^n), \phi \in \DD(\RR^n)$, we define the \emph{convolution} $u * \phi \colon \RR^n \to \CC$ as
    \[
    (u * \phi)(x) \ceq \int_{\RR^n} u(x-y) \phi(y) \dd{y} = \int_{\RR^n} u(y) \phi(x - y) \dd{y} = \ang{u, \tau_x \refl\phi}. 
    \]
    Since the RHS makes sense for any $u \in \DD'(\RR^n)$, we extend the definition this way: for $u \in \DD'(\RR^n), \phi \in \DD(\RR^n)$, we define the convolution $u * \phi$ as
    \[
    (u * \phi)(x) \ceq \ang{u, \tau_x\refl\phi}. 
    \]
\end{definition}


\begin{lemma}
    Let $\phi \in C^\infty(\RR^n \times \RR^m)$ and define $\Phi_x(y) \ceq \phi(x, y)$. Suppose for any $x \in \RR^n$ there exists a neighbourhood $N(x)$ and a compact $K \subseteq \RR^m$ such that $\phi(x, y)$ for all $x \in N(x), y \notin K$. 
    
    Then $x \mapsto \ang{u, \Phi_x}$ is differentiable with
    \[
    \pt_x^\alpha \ang{u, \Phi_x} = \ang{u, \pt_x^\alpha \Phi_x}
    \]
    for any $u \in \DD'(\RR^m)$. 
\end{lemma}

\begin{proof}
    Fix $x \in \RR^n$, then by Taylor's formula we have
    \[
    \Phi_{x + h}(y) = \Phi_x(y) + \sum_{i=1}^n h_i \pdv{\phi}{x_i} (x, y)+ R(x, y, h), 
    \]
    where $\pt_y^\alpha R(x, y, h) = o(\norm{h})$, uniformly in $y$, for any multi-index $\alpha$. Furthermore, by assumption there exists a compact $K$ such that for $h$ small enough, $\supp R(x, \cdot, h) \subseteq K$. Therefore, $R(x, \cdot, h)$ is a test function for $h$ small enough.
    
    Combining the previous two facts shows that $R(x, \cdot, h) = o(\norm{h})$ in $\DD(\RR^m)$ as $h \to 0$. 
    
    Let $u \in \DD'(\RR^m)$, then we find by sequential continuity that $\ang{u, R(x, \cdot, h)}$ is also $o(\norm{h})$, and therefore
    \[
    \ang{u, \Phi_{x+h}} = \ang{u, \Phi_x} + \sum_{i=1}^n h_i \ang{u, \pdv{\Phi_x}{x_i}} + o(\norm{h}). 
    \]
    This shows that $x \mapsto \ang{u, \Phi_x}$ is differentiable with 
    \[
    \pdv{x_i} \ang{u, \Phi_x}  = \ang{u, \pdv{\Phi_x}{x_i}}. 
    \]
    From this the result follows. 
\end{proof}

\begin{corollary}
    If $u \in \DD'(\RR^n)$, $\phi \in \DD(\RR^n)$, then $u * \phi$ is differentiable with $\pt^\alpha (u * \phi) = u * \pt^\alpha\phi$. 
\end{corollary}

\begin{proof}
    Apply the previous lemma with $\Phi_x(y) \ceq  \phi(x-y)$. 
\end{proof}

Due to the previous corollary, we often call $u * \phi$ a \emph{regularisation} of $u$. 

\begin{convention}
	If $u \in \DD'(X)$ and $\phi \in \DD(X)$, then instead of $\ang{u, \phi}$ we also write $\ang{u(t), \phi(t)}$ (or with any other dummy variable) when the variable used for $\phi$ is not directly clear. 
\end{convention}

\subsection{Density of test functions in distribution space}
\begin{lemma} \label{lem:convolution_associative}
	If $u \in \DD'(\RR^n)$ and $\phi, \psi \in \DD(\RR^n)$, then
	\[
	(u * \phi) * \psi = u * (\phi * \psi). 
	\]
\end{lemma}

\begin{proof}
	Fix $x \in \RR^n$. Now we write
	\begin{align*}
		((u * \phi) * \psi)(x) &= \int_{\RR^n} \ang{u(z), (\tau_y \refl\phi)(z)} \cdot \psi(x-y) \dd{y} \\
		&= \int_{\RR^n} \ang{u(z), (\tau_{x-y}\refl\phi)(z)} \cdot \psi(y) \dd{y} \\
		&= \int_{\RR^n} \ang{u(z), \psi(y) (\tau_{x-y} \refl\phi)(z)} \dd{y}.
	\end{align*}
We would like to interchange integral and application of $u$, and we will have to justify this using Riemann sums: 
\begin{align*}
\int_{\RR^n} \ang{u(z), \psi(y) (\tau_{x-y} \refl\phi)(z)} \dd{y} &= \lim_{\eps \downarrow 0} \sum_{m \in \ZZ^n} \ang{u(z), \psi(\eps m) \phi(x - z - \eps m)} \eps^n \\
&\overset*= \lim_{\eps\downarrow0} \ang{u(z), \sum_{m \in \ZZ^n} \psi(\eps m) \phi(x - z - \eps m)\eps^n } \\
&\overset{**}= \ang*{u(z), \int_{\RR^n} \psi(y) \phi(x - z - y) \dd{y}} \\
&= \ang{u(z), (\phi * \psi)(x - z)} = \ang{u(z), (\tau_x\widecheck{\phi * \psi})(z)} = (u * (\phi * \psi))(x). 
\end{align*}
Here, $*$ is by the fact that the sum is finite since $\psi$ has compact support, while $**$ is by sequential continuity of $u$ and the fact that the Riemann sum converges to the convolution integral \emph{in the space of test functions} (non-examinable fact). 
\end{proof}

We will use the following trick many times:
\begin{proposition}
	For any $u \in \DD'(\RR^n), \phi \in \DD(\RR^n)$ we have $\ang{u, \phi} = (u * \refl\phi)(0)$. 
\end{proposition}

\begin{proof}
	We have $(u * \refl\phi)(0) = \ang{u, \tau_0 \phi} = \ang{u, \phi}$. 
\end{proof}
For example, from this trick it follows that if $u * \phi = 0$ for all $\phi$, then $u = 0$. 

\begin{theorem}
	If $u \in \DD'(\RR^n)$, there exists a sequence $(\phi_k) \subseteq \DD(\RR^n)$ such that $\phi_k \to u$ in $\DD'(\RR^n)$. 
\end{theorem}

\begin{proof}
	Fix $\psi \in \DD(\RR^n)$ with $\int_{\RR^n} \psi \dd{x} = 1$, and set $\psi_k(x) \ceq k^n \psi(kx)$. Note that $\int_{\RR^n} \psi_k \dd{x} = 1$.
	
	Now, fix any $\chi \in \DD(\RR^n)$ with $\chi \equiv 1$ on $\qty{\norm{x} < 1}$ and $\chi \equiv 0$ on $\qty{\norm{x} < 2}$. Define $\chi_k(x) \ceq \chi(x/k)$, so that $\lim_{k\to\infty} \chi_k(x) = 1$ for all $x$. We will set
	\[
	\phi_k(x) \ceq \chi_k(x) (u * \psi_k)(x). 
	\]
	Clearly we have $\phi_k \in \DD(\RR^n)$ since each $\chi_k$ has compact support. 
	
	Now, take any $\theta \in \DD(\RR^n)$, then we have
	
	\begin{align*}
		\ang{\phi_k, \theta} &= \ang{\chi_k (u * \psi_k), \theta} = \ang{u * \psi_k, \chi_k \theta} = \qty[ (u * \psi_k) * \refl{\chi_k\theta}](0) \\
		&= \qty[u * \qty(\psi_k * \refl{\chi_k \theta})](0).
	\end{align*}

	Now we compute $\psi_k * \refl{\chi_k\theta}$: note that 
	\begin{align*}
		(\psi_k * \refl{\chi_k\theta})(x) &= \int_{\RR^n} k^n\psi(k(x-y)) \chi\qty(-\frac yk) \theta(-y) \dd{y} \\
		&= \int_{\RR^n} \psi(y) \chi\qty(\frac{y}{k^2} - \frac xk) \theta(\frac yk - x) \dd{y} \\
		&= \theta(-x) + R_k(-x) = (\refl{\theta + R_k})(x)
	\end{align*}
where $R_k(x) = \int_{\RR^n} \psi(y) \qty[ \chi\qty(\frac y{k^2} + \frac xk) \theta\qty(\frac yk + x )- \theta(x)] \dd{y}$.

So \[
\ang{\phi_k, \theta} = (u * (\refl{\theta + R_k}))(0) = (u *\refl\theta)(0) + (u * \refl{R_k})(0) = \ang{u, \theta} + \ang{u, R_k}. 
\]

We must now only prove that $R_k \to 0$ in $\DD(\RR^n)$, and then by sequential continuity it follows that $\phi_k \to u$  in $\DD'(\RR^n)$. 
\end{proof}


\section{The Lasso}
\subsection{Model selection}
We go back to the linear model $Y = X\beta^0 + \eps$ with $\EE[\eps] = 0$ and $\Var(\eps) = \sigma^2 I$. Using the trace trick, one can easily compute that the MSPE of the OLS estimator is given by
\[
\frac1n \EE\norm{X\beta^0 - X\hat\beta^\Rm{OLS}}_2^2 = \frac{\sigma^2 p}{n}. 
\]
Defining $S = \qty{k \mid (\beta^0)_k \neq 0}$, there is often reason to assume that $S$ is small, i.e., $s \ceq \abs{S} \ll p$. If we could fit a model using only the variables in $S$, the MSPE would be much $\frac{\sigma^2 s}{n} \ll \frac{\sigma^2 p}{n}$. 

\paragraph{Best subset selection} A natural way to find $S$ is to consider all possible subsets of $\qty{1, \dotsc, p}$, and pick the best regression procedure using, for example, cross-validation. However, this can become computationally infeasible for moderately large $p$ (say $p \approx 10$). 

\paragraph{Forward selection} This is a greedy way of performing best subset regression. Given a target model size $m$, we first compute the intercept-only model $M_0$, and then one-by-one add the predictor variable that reduces the residual sum of squares the most, until we have a model with $m$ variables. 

\subsection{Lasso estimator}
The \emph{Least absolute shrinkage and selection operator} or \emph{Lasso} is given by
\[
(\hat\mu_\lambda^\Rm{L}, \hat\beta_\lambda^\Rm{L}) \ceq \argmin_{(\mu, \beta) \in \RR \times \RR^p} \frac1{2n}\norm{Y- \mu \vec 1 - X\beta}_2^2 + \lambda \norm{\beta}_1. 
\]
As with ridge regression, we usually centre and scale the matrix $X$, as well as centre the responses $Y$, in which case we find
\[
\hat\beta_\lambda^\Rm{L} = \argmin_{\beta \in \RR^p} \frac1{2n}\norm{Y - X\beta}_2^2 + \lambda \norm{\beta}_1. 
\]

The main difference between the lasso estimator and the ridge regression estimator is that it is likely that the lasso estimator has some zero components. This means that the lasso estimator also estimates which variables are relevant. 

We have the following:
\begin{theorem}[Slow rate]
	Assume $X$ has centred and scaled columns, and assume that $Y$ has been centred, so $Y = X\beta^0 + \eps - \bar\eps \vec 1$. Let $A > 0$ and suppose
	\[
	\lambda = A\sigma\sqrt{\frac{\log(p)}{n}}.
	\]
	Let $\hat\beta = \hat\beta_\lambda^\Rm{L}$, then with probability at least $1 - 2p^{-(A^2/2 - 1)}$ we have that the MSPE satisfies
	\[
	\frac1n \norm{X(\beta^0 - \hat\beta)}_2^2 \leq 4\lambda \norm{\beta^0}_1 = 4A\sigma\sqrt{\frac{\log(p)}{n}} \norm{\beta^0}_1. 
	\]
\end{theorem}

\begin{proof}
	By definition we have
	\[
	\frac1{2n}\norm{Y - X\hat\beta }_2^2 + \lambda \norm{\hat \beta}_1 \leq \frac1{2n}\norm{Y - X\beta^0}_2^2 + \lambda \norm{\beta^0}_1,
	\]
	and rearranging the terms gives
	\[
	\frac1{2n} \norm{X(\beta^0 - \hat\beta)}_2^2 \leq\frac1n \eps\T X (\hat\beta - \beta^0) + \lambda\norm{\beta^0}_1 - \lambda \norm{\hat\beta}_1. 
	\]
	By H\"older's inequality we have $\abs{\eps\T X(\beta^0 - \hat\beta)} \leq \norm{X\T\eps}_\infty \norm{\hat\beta - \beta^0}_1$. 
	Define the event \\ $\Omega = \qty{\norm{X\T\eps}_\infty/n \leq \lambda}$, then conditional on $\Omega$ we find
	\[
	\frac1{n} \norm{X(\beta^0 - \hat\beta)}_2^2 \leq 2\lambda \qty(\norm{\beta^0 - \hat\beta}_1 + \norm{\beta^0}_1 - \norm{\hat\beta}_1) \leq 4\lambda \norm{\beta^0}_1.
	\]
	In \cref{lem:probability_omega_bound}, we will show that $\PP(\Omega) \geq 1 - 2p^{-(A^2/2 - 1)}$, which completes the proof. 
\end{proof}

\subsection{Concentration inequalities}
Let $W$ be any random variable and $\phi \colon \RR\to[0,\infty)$ strictly increasing. Then by Markov's inequality we have
\[
\PP(W \geq t) = \PP(\phi(W) \geq \phi(t)) \leq \frac{\EE[\phi(W)]}{\phi(t)}. 
\]
Plugging in $\phi(x) = e^{\alpha x}$ (for some $\alpha > 0$), we get
\[
\PP(W \geq t) \leq e^{-\alpha t} \EE[e^{\alpha W}] = e^{-\alpha t} M_W(\alpha).
\]
Now we can take the infimum over all $\alpha$ on the right-hand side, and we get what is called the \emph{Chernoff bound}:
\[
\PP(W \geq t) \leq \inf_{\alpha > 0} e^{-\alpha t} M_W(\alpha). 
\]

\begin{definition}
	A random variable  $W$ with mean $\mu$ is called \emph{sub-Gaussian with parameter $\sigma > 0$} or $\sigma$-sub-Gaussian if 
	\[
	M_{W - \mu} \leq M_{N(0, \sigma^2)} \quad\text{or equivalently}\quad \EE[e^{\alpha(W - \mu)}] \leq e^{\alpha^2\sigma^2/2} \text{ for all $\alpha \in \RR$}. 
	\]
\end{definition}

We need the following lemma, which characterises an important class of sub-Gaussian random variables:
\begin{lemma}[Hoeffding]
	If $W$ is a mean-zero random variable which takes values in $[a, b]$, then $W$ is sub-Gaussian with parameter $(b-a)/2$. 
\end{lemma}

By Chernoff bounding, we obtain for a $\sigma$-sub-Gaussian random variable $W$ that \[\PP(W - \mu \geq t) \leq e^{-t^2/(2\sigma^2)}.\]

\begin{proposition}
	Let $W_1, \dotsc, W_n$ be independent, mean-zero random variables where $W_i$ is $\sigma_i$-sub-Gaussian. For any $\gamma \in \RR^n$, the random variable $\gamma\T W$ is sub-Gaussian with parameter $\qty(\sum_i \sigma_i^2\gamma_i^2)^{1/2}$.
\end{proposition}
\begin{proof}
	Since the $W_i$ are independent we have for all $\alpha \in \RR$ that
	\begin{align*}
		\EE[e^{\alpha \sum_i \gamma_i W_i}] = \prod_{i=1}^n \EE[e^{\alpha \gamma_i W_i}]  \leq \prod_{i=1}^n e^{\alpha^2\gamma_i^2\sigma_i^2/2} = e^{\alpha^2 \sum_i \gamma_i^2\sigma_i^2/2}.
	\end{align*}
\end{proof}

\begin{lemma} \label{lem:probability_omega_bound}
	Suppose $\eps_1, \dotsc, \eps_n$ are independent mean-zero $\sigma$-sub-Gaussian random variables and let $\lambda \ceq A\sigma \sqrt{\log(p)/n}$. Then
	\[
	\PP\qty(\frac{\norm{X\T\eps}_\infty}{n} \leq \lambda) \geq 1 - 2p^{-(A^2/2 - 1)}. 
	\]
\end{lemma}

\begin{proof}
	We have
	\[
	\PP\qty(\frac{\norm{X\T\eps}_\infty}{n} > \lambda) = \PP\qty(\bigcup_i \frac{\abs{X_i\T\eps}}{n} > \lambda) \leq \sum_i \PP\qty(\frac{\abs{X_i\T\eps}}{n} > \lambda). 
	\]
	By the previous proposition, both $X_i\T\eps$ and $-X_i\T\eps$ are sub-Gaussian with parameter $\sigma/\sqrt n$, so we have
	\[
	\sum_i \PP\qty(\frac{\abs{X_i\T\eps}}{n} > \lambda) \leq 2p \exp(\frac{-\lambda^2}{2(\sigma/\sqrt n)^2}) = 2p\exp(-A^2\log(p)/2) = 2 p^{-(A^2/2 - 1)}.
	\]
\end{proof}

We now move to finding tail bounds for products of sub-Gaussian random variables. To this end, we consider Bernstein's in equality. 
\begin{definition}
	A random variable $W$ with mean $\mu$ is said to satisfy \emph{Bernstein's condition} with parameters $\sigma > 0, b > 0$ if 
	\[
	\EE\qty(\abs{W - \mu}^k) \leq \frac12 k! \sigma^2 b^{k-2} \qquad k = 2, 3, \dotsc
	\]
\end{definition}

\begin{proposition}[Bernstein's inequality]
	Let $W_1, W_2, \dotsc$ be independent random variables with mean $\mu$, where each $W_i$ satisfies Bernstein's condition with parameter $(\sigma, b)$. Then 
	\begin{enumerate}
		\item $\EE[\exp(\alpha(W_i - \mu))] \leq \exp\qty(\frac{\alpha^2\sigma^2}{2(1 - b \abs{\alpha})})$ for all $\abs{\alpha} < b^{-1}$;
		\item $\PP\qty(\overline{W} - \mu \geq t) \leq \exp\qty(\frac{-nt^2}{2(\sigma^2 + bt)})$ for all $t \geq 0$.  
	\end{enumerate}
\end{proposition}

\begin{proof}
	See notes from topics in statistical theory.
\end{proof}

\begin{lemma}
	Let $W, Z$ be mean-zero, sub-Gaussian random variables with parameters $\sigma_W, \sigma_Z$. Then $WZ$ satisfies the Bernstein condition with parameters $(8\sigma_W\sigma_Z, 4\sigma_W\sigma_Z)$. 
\end{lemma}

\begin{proof}
	Using the tail probability formula for expectations one can prove that any mean-zero sub-Gaussian random variable $X$ satisfies $\EE[X^{2k}] \leq 2^{k+1} \sigma_X^{2k} k!$. 
	
	Furthermore for any random variable $Y$ we have
	\begin{align*}
		\EE\abs{Y - \EE Y}^k &\leq \EE\qty(\abs{Y} + \abs{\EE Y})^k \\
		&= \EE \sum_{t=0}^k \binom kt \abs{Y}^t \abs{\EE Y}^{k-t} \\
		&\overset*\leq \sum_{t=0}^k \binom kt \EE(\abs{Y}^t) \EE(\abs{Y}^{k-t}) \\
		&\overset*\leq \sum_{t=0}^k \binom kt \EE(\abs{Y}^k)^{t/k} \EE(\abs{Y}^k)^{k - t/k} \\
		&= \EE(\abs{Y}^k) \sum_{t=0}^k \binom kt = 2^k \EE\abs{Y}^k,
	\end{align*}
where both equations $*$ are applications of Jensen's inequality. 

From the above bound, we obtian 
\begin{align*}
\EE\abs{WZ - \EE WZ}^k &\leq 2^k \EE(\abs{W}^k \abs{Z}^k) \overset{\Rm{CS}}\leq 2^k (\EE W^{2k})^{1/2} (\EE Z^{2k})^{1/2} \leq 2^{2k+1} \sigma_W^k \sigma_Z^k k! \\
&= \frac{k!}{2} (8\sigma_W\sigma_Z)^2 (4\sigma_W\sigma_Z)^{k-2},
\end{align*}
which proves the claim. 
\end{proof}

\subsubsection{Optimisation theory and convex analysis}
\begin{proposition}
	Let $f \colon \RR^d \to \RR$ be twice continuously differentiable. Then $f$ is convex (resp.\ strictly convex) if and only if its Hessian is positive semi-definite (resp.\ positive definite) for all $x \in \RR^d$.
\end{proposition}
\section{Nonparametric regression}
\subsection{Fixed and random design}
In \emph{fixed design}, we assume we have data $x_1 \leq \dotsb \leq x_n$ and the response variables satisfy 
\[
Y_i \ceq m(x_i) + \sqrt{v(x_i)} \eps_i,
\]
where the $\eps_i$ are independent, mean-zero random variables with $\Var(\eps_i) = 1$. The function $m$ is called the \emph{regression function}, and the function $v$ is the \emph{variance function}. If $v$ is constant, the model is called \emph{homoscedastic}, else it is called \emph{heteroscedastic}. 

In \emph{random design}, we assume we have i.i.d.\ data pairs $(X_i, Y_i)$ with 
\[
Y_i = m(X_i) + \sqrt{v(X_i)} \eps_i, 
\]
where the $\eps_i$ are again independent with $\EE[\eps_1 | X_1] = 0$ and $\Var(\eps_1 | X_1) = 1$. The regression function is given by $m(x) = \EE(Y_1 | X_1 = x)$ and the variance function by $v(x) = \Var(Y_1 | X_1 = x)$. 

\subsection{Local polynomial estimators}
We will assume the fixed design setting. 
\begin{definition}
	Let $K$ be a kernel, $h > 0$ a bandwidth and $p \in \NN$. Then the \emph{local polynomial estimator of degree $p$ with bandwidth $h$ and kernel $K$}, denoted $\hat m_n (\cdot; p) \equiv \hat m_n(\cdot; p, h, K)$, is constructed at $x \in \RR$ by fitting a polynomial $p$ to the data using weighted least squares, where the pair $(x_i, Y_i)$ is assigned weight $K_h(x_i - x)$. 
\end{definition}

To write this in formulas, define $Q(u) \ceq (1, u, \frac{u^2}{2}, \dotsc, \frac{u^p}{p!}) \in \RR^{p + 1}$ and $Q_h(\cdot) = Q(\cdot/h)$, we then have
\[
\hat m_n(x; p) = \hat\beta_0, \quad\text{where } \hat\beta \ceq \argmin_{\beta \in \RR^{p+1}} \sum_{i=1}^n \qty(Y_i - \beta\T Q_h(x_i - x))^2 K_h(x_i - x). 
\]


In matrix-vector notation, writing 
\begin{align*}
X &\equiv X(x;p, h) \ceq \mqty(Q_h(x_1 - x)\T \\ \vdots \\ Q_h(x_n - x)\T) \in \RR^{n \times (p + 1)}, \quad Y \ceq \mqty(Y_1 \\ \vdots \\ Y_n) \in \RR^n, \\
W&\equiv W(x; h, K) \ceq \diag(K_h(x_1 - x), \dotsc, K_h(x_n - x)) \in \RR^{n \times n}, 
\end{align*}
we have 
\[
\hat\beta = \argmin_{\beta \in \RR^{p+1}} (Y - X\beta)\T W (Y - X\beta).
\]

By standard weighted least squares theory, we know that $\hat\beta$ must satisfy $X\T W X\hat\beta = X\T W Y$. 
\begin{proposition}
	Suppose $X\T W X$ is positive definite. Then we have
	\[
	\hat\beta = (X\T W X)^{-1} X\T W Y. 
	\]

\end{proposition}

We will assume from here on that $X\T W X$ is indeed positive definite. 
In this case, since the entries of $W$ and $X$ are functions of $x_i - x$, we can write the local polynomial estimator in the form
\[
\hat m_n(x) = n^{-1} \sum_{i=1}^n w(x, x_i) Y_i.
\]
(??)
The set of weights $\qty{w(x, x_i)}$ is called the \emph{effective kernel} at $x$. 

For $p = 0$ and $p = 1$, there exist explicit formulas for the local polynomial estimator of degree $p$. 

\begin{proposition}[Reproducing property] \label{prop_reproducing}
	Let $\qty{w_p(x, x_i)}$ denote the effective kernel of a local polynomial estimator of degree $p$ based on data $(x_1, Y_1), \dotsc, (x_n, Y_n)$, and let $R$ denote a polynomial of degree at most $p$. If $X\T W X$ is positive definite, then 
	\[
	\frac1n \sum_{i=1}^n w_p(x, x_i) R(x_i) = R(x). 
	\]
\end{proposition}

\begin{proof}
See example sheet 2 question 3.
\end{proof}

Before we can study the bias and variance of local polynomial estimators, we require the following lemma:
\begin{lemma}
	Let $K$ be a kernel that vanishes outside $[-1, 1]$, and suppose that $n^{-1} X\T W X$ is positive definite with minimal eigenvalue $\lambda_0 \equiv \lambda_{0, n, x} > 0$. Then 
	\begin{enumerate}[(i)]
		\item $\sup_{x\in[0, 1]} \max_{i = 1, \dotsc, n} \frac1n \abs{w(x, x_i)} \leq \frac{2\norm{K}_\infty}{\lambda_0 n h}$;
		\item $n^{-1} \sum_{i=1}^n \abs{w(x, x_i)} \leq \frac{2\norm{K}_\infty}{\lambda_0 n h}\cdot  \sum_{i=1}^n \ind_{\abs{x_i - x} \leq h}$;
		\item $w(x, x_i) = 0$ when $\abs{x_i - x} > h$. 
	\end{enumerate}
\end{lemma}

\begin{proof}
	\begin{enumerate}[(i)]
		\item
	Note that $n^{-1} w(x, x_i)$ is the $(0, i)$ entry of the matrix $(X\T W X)^{-1} X\T W$, and it is therefore less than the norm of the $i$-th column of $(X\T W X)^{-1} X\T W$.  For $x \in [0, 1]$ and $i = 1, \dotsc, n$, we find
	\begin{align*}
		\frac1n \abs{w(x, x_i)} &\leq \norm{(X\T W X)^{-1} Q_h(x_i - x) K_h(x_i - x)} \overset\star\leq  \norm{K_h}_\infty \norm{(X\T W X)}^{-1} \norm{Q_h(x_i - x)} \ind_{\abs{x_i - x} \leq h} \\
		&= \frac{\norm{K}_\infty}{h} \frac{1}{\lambda_0 n}\norm{Q_h(x_i - x)} \ind_{\abs{x_i - x} \leq h}  \leq \frac{\norm{K}_\infty}{\lambda_0 n h} \norm{Q(1)} = \frac{\norm{K}_\infty}{\lambda_0 n h} \qty(\sum_{j=0}^p \frac1{(j!)^2})^{1/2} \\
		&\leq \frac{\norm{K}_\infty}{\lambda_0 n h} e^{1/2} \leq \frac{2 \norm{K}_\infty}{\lambda_0 nh}. 
	\end{align*}
	Here, the indicator function in $\star$ appears because $K$ vanishes outside $[-1, 1]$. 
	Since the upper bound is independent of both $x$ and $i$, the claim is proven. 
	
	\item Similarly as before, we find 
	\[
	\frac1n \sum_{i=1}^n \abs{w(x, x_i)} \leq \frac{\norm{K}_\infty}{\lambda_0 n h} \sum_{i=1}^n \norm{Q_h(x_i - x)} \ind_{\abs{x_i - x} \leq h} \leq \frac{2 \norm{K}_\infty}{\lambda_0 n h} \sum_{i=1}^n \ind_{\abs{x_i - x} \leq h}. 
	\]
	
	\item This follows immediately from inequality $\star$ in the proof of (i). 

	\end{enumerate}
\end{proof}

Now, we can compute bounds on the variance and bias of our local polynomial estimator. For simplicity, we will assume $x_i = i/n$. 
\begin{proposition}
Assume the model $Y_i = m(x_i) + v^{1/2}\eps_i$ with $m \in \HH(\beta, L)$ on $[0, 1]$ and $\max_i v(x_i) \leq \sigma_\Rm{max}^2$. Let $K$ be a kernel that vanishes outside $[-1, 1]$, and suppose that $\lambda_0$, the minimal eigenvalue of $n^{-1} X\T W X$, is strictly positive. Then, for $p \geq \ceil{\beta} - 1 \eqqcolon \beta_0$ and for each $x_0 \in [0, 1]$, $n \in \NN$ and $h \geq 1/(2n)$, we have 
\[
\Var \hat m_n(x_0;p) \leq \frac{16 \norm{K}_\infty^2\sigma_\Rm{max}^2}{\lambda_0^2 n h}, \quad \abs{\Bias \hat m_n(x_0;p)} \leq \frac{8L \norm{K}_\infty}{\lambda_0 \beta_0!} h^\beta. 
\]
\end{proposition}

\begin{proof}
	Using the previous lemma, we obtain
	\begin{align*}
		\Var \hat m_n(x_0;p) &= \Var\qty(\frac1n \sum_{i=1}^n w(x_0, x_i) Y_i) = \frac1{n^2} \sum_{i=1}^n w(x_0, x_i)^2 \Var(Y_i) \\
		&\leq \frac{\sigma_\Rm{max}^2}{n^2} \sum_{i=1}^n w(x_0, x_i)^2  \leq \sigma_\Rm{max}^2 \qty(\sup_{x \in [0, 1]} \max_i \frac1n \abs{w(x, x_i)}) \frac1n \sum_{i=1}^n \abs{w(x_0, x_i)} \\
		&\leq \frac{4 \norm{K}_\infty^2 \sigma_\Rm{max}^2}{\lambda_0^2 n^2 h^2} \sum_{i=1}^n \ind_{\abs{i/n - x_0} \leq h}.
	\end{align*}
Now, we have $\abs{i/n - x_0} \leq h \iff i \in [n x_0 - nh, n x_0 + nh]$, and there are at most $2nh + 1$ integers in this interval. Recalling that $1 \leq 2nh$ we obtain 
\[
\Var \hat m_n(x_0; p) \leq \frac{4(2nh + 1) \norm{K}_\infty \sigma_\Rm{max}^2}{\lambda_0^2 n^2 h^2} \leq \frac{16 \norm{K}_\infty \sigma_\Rm{max}^2}{\lambda_0^2 nh} . 
\]

For the bias, we will first use a Taylor expansion combined with the reproducing property \cref{prop_reproducing}. 
Firstly, since $\frac1n\sum_{i=1}^n w(x_0, x_i) = 1$ by this property, we have
\[
\Bias \hat m_n(x_0;p) = \qty(\frac1n \sum_{i=1}^n w(x_0, x_i) m(x_i)) - m(x_0) = \frac1n \sum_{i=1}^n w(x_0, x_i) \qty{m(x_i) - m(x_0)}.
\] 
Now, we apply a Taylor expansion to write \[
m(x_i) - m(x_0) = P(x_i - x_0) + \frac{1}{\beta_0!} m^{(\beta_0)}(x_0 + \tau_i(x_i - x_0)) (x_i - x_0)^{\beta_0}, 
\]
where $P$ has degree at most $\beta_0 - 1 < p$ and a constant coefficient equal to 0,  and $\tau_i \in [0, 1]$. By the reproducing property we have $n^{-1} \sum_{i=1}^n w(x_0, x_i) P(x_i - x_0) = P(0) = 0$, so we obtain
\begin{align*}
\Bias \hat m_n(x;p) &= \frac1n w(x_0, x_i) \sum_{i=1}^n w(x_0, x_i) \frac{m^{(\beta_0)}(x_0 + \tau_i(x_i - x_0))}{\beta_0!} (x_i - x_0)^{\beta_0} \\
&= \frac1n w(x_0, x_i) \sum_{i=1}^n w(x_0, x_i) \frac{m^{(\beta_0)}(x_0 + \tau_i(x_i - x_0)) - m^{(\beta_0)}(x_0)}{\beta_0!} (x_i - x_0)^{\beta_0},
\end{align*}
where the last line follows again from the reproducing property. Now we apply the fact that
$m \in \HH(\beta, L)$ with the previous lemma and find
\begin{align*}
	\abs{\Bias \hat m_n(x;p)} &\leq \frac Ln \sum_{i=1}^n \abs{w(x_0, x_i)} \frac{\abs{x_i - x_0}^\beta}{\beta_0!} = \frac Ln \sum_{i=1}^n \abs{w(x_0, x_i)} \frac{\abs{x_i - x_0}^\beta}{\beta_0!} \ind_{\abs{x_i - x} \leq h} \\
	&\leq \frac{Lh^\beta}{n\beta_0!} \sum_{i=1}^n \abs{w(x_0, x_i)} \leq \frac{8 L \norm{K}_\infty}{\lambda_0 \beta_0!}h^\beta. 
\end{align*}
\end{proof}

So again, we have a variance term of order $1/nh$ and a bias term of order $h^\beta$. However, both our bounds depend on $\lambda_0$, which depends on both $n$ and $x$. 
\begin{proposition}
Suppose $x_i = i/n$ and that $K(u) \geq K_0 \ind_{\abs{u} \leq \Delta}$ for some $K_0, \Delta > 0$. Then, for $n \geq 2$ and $h \leq \frac1{4\Delta}$, 
\[
\inf_{x \in [0, 1]} \lambda_{0, n, x} \geq K_0 \inf_{v \in S^p} \min \qty{\int_0^\Delta (v\T Q(u))^2 \dd{u}, \int_{-\Delta}^0 (v\T Q(u))^2 \dd{u}} - \frac{(4\Delta + 2) K_0 e^{\Delta^2}}{nh}. 
\]
\end{proposition}

\begin{proof}
Since $\lambda_{0, n, x} = \inf_{v \in S^p} v\T \qty(\frac1n X\T W X)v$, we will try to bound this quantity from below. 
	
Let $u_i \ceq \frac{x_i - x}{h}$ for $i = 1, \dotsc, n$, so $u_1 \leq \frac{x_1}{h} = \frac{1}{nh}$, and let $u_0 \ceq 0$. 

First, we assume $x < 1 - h\Delta$, so that $u_n > \frac{1 - (1 - h\Delta)}{h} = \Delta$. Then, for any $v \in S^p$, we have
\begin{align*}
v\T \qty(\frac1n X\T W X) &= \frac1n (Xv)\T W (Xv) = \frac1n \sum_{i=1}^n (Xv)_i^2  W_{ii}\\
& = \frac1{nh} \sum_{i=1}^n (v\T Q(u_i))^2 K(u_i) \geq \frac{K_0}{nh} \sum_{i=1}^n (v\T Q(u_i))^2 \ind_{u_i \in [0, \Delta]} .
\end{align*}
This is a Riemann sum (since $u_{i+1} - u_i = 1/(nh)$) and we will try to approximate it by the corresponding integral $K_0 \int_0^\Delta (v\T Q(u))^2 \dd{u}$. 

To do this, note that we have for $u \in [0, \Delta]$ that
\[
\norm{Q(u)}  \leq \qty{\sum_{\ell = 0}^p \frac{u^{2\ell}}{(\ell!)}}^{1/2} \leq \qty(\sum_{\ell = 0}^\infty \frac{u^{2\ell}}{\ell!})^{1/2} \leq e^{\Delta^2/2}. 
\] 
Furthermore, for $a, b \geq 0$, it can be shown that
\[
\abs{b^\ell - a^\ell} \leq \max(1, \ell / 2) (b^{\ell - 1} + a^{\ell - 1})\abs{b - a},
\]
and using this we obtain for $u, v \in [0, \Delta]$ that 
\begin{align*}
	\norm{Q(u)- Q(v)} &\leq \qty{\sum_{\ell =1 }^p \frac{\max(1, \ell^2/4) (u^{\ell - 1} + v^{\ell - 1})^2 (u - v)^2}{(\ell !)^2}}^{1/2} \\
	&\leq \abs{u - v} \qty{\sum_{\ell = 1}^p \frac{\max(1, \ell^2/4)(2\Delta^{\ell - 1})^2}{(\ell!)^2}}^{1/2} \leq \abs{u - v} \qty{\sum_{\ell = 1}^p \frac{\max(4, \ell^2) \Delta^{2\ell - 2}}{(\ell!)^2}}^{1/2} 
	 \\
	 &\overset\star\leq  2\abs{u - v} \qty{\sum_{\ell = 0}^{p - 1} \frac{\Delta^{2\ell}}{(\ell!)^2}}^{1/2} \leq 	2 \abs{u - v} \qty{\sum_{\ell = 0}^\infty \frac{\Delta^{2\ell}}{\ell!}}^{1/2} = 2 e^{\Delta^2/2}\abs{u-v}, 
\end{align*}
where in $\star$ we used that $\max(4, \ell^2) \leq 4\ell^2$.

Now we estimate the difference between the Riemann sum and the corresponding integral. Write $i_- \ceq \min\qty{i : u_i > 0}$ and $i_+ \ceq \min\qty{i \mid u_i > \Delta}$.  Then 
\begin{align*}
\frac1{nh} \sum_{i=1}^n (v\T Q(u_i))^2 \ind_{u_i \in [0, \Delta]} &= \sum_{i=i_-}^{i_+ -1 } \int_{u_{i-1}}^{u_i} (v\T Q(u_i))^2 \dd{u} \\
&\leq \sum_{i = i_-}^{i_+} \int_{\max(u_{i-1}, 0)}^{\min(u_i, \Delta)} (v\T Q(u_i))^2 \dd{u} + \frac1{nh} \sup_{u \in [0, \Delta]} (v\T Q(u))^2.
\end{align*}
(Note that the last integral, from $u_{i-1}$ to $\Delta$, is 0 since )
Therefore we obtain
\begin{align}
	&\abs{\frac{1}{nh} \sum_{i=1}^n (v\T Q(u_i))^2 \ind_{u_i \in [0, \Delta]} -  \int_0^\Delta (v\T Q(u))^2 \dd{u}} \nonumber \\
	&\leq \abs{\sum_{i = i_-}^{i_+} \int_{\max(u_{i-1}, 0)}^{\min(u_i, \Delta)}(v\T Q(u_i))^2 - (v\T Q(u))^2 \dd{u}} + \frac1{nh} \sup_{u \in [0, \Delta]} (v\T Q(u))^2. \label{eq:integral_expr}
\end{align}
Now note that by Cauchy-Schwarz since $\norm{v} = 1$ we have 
\begin{align*}
\abs{(v\T Q(u_i))^2 - (v\T Q(u))^2} &= \abs{(v\T Q(u_i) + v\T Q(u))}\abs{(v\T Q(u_i) - v\T Q(u))}\\
&\leq (\norm{Q(u_i)} + \norm{Q(u)}) \norm{Q(u_i) - Q(u)} \\
&\leq 2 e^{\Delta^2/2} \cdot 2e^{\Delta^2/2} \abs{u_i - u} = 4 e^{\Delta^2} \abs{u_i - u}, 
\end{align*}
and  therefore
\[
\eqref{eq:integral_expr} \leq 4e^{\Delta^2} \sum_{i=i_-}^{i_+} \int_{\max(u_{i-1}, 0)}^{\min(u_i, \Delta)}(u_i - u) \dd{u} + \frac{e^{\Delta^2}}{nh} \leq \frac{(4\Delta + 1) e^{\Delta^2}}{nh},
\]
which concludes the case $x < 1- h\Delta$. 

Suppose $x \geq 1 - h\Delta$, then we have $u_1 \leq -\Delta$ and $u_n \geq 0$, and we can apply very similar arguments to reach the desired conclusion. 
\end{proof}

Now that we have bounds on the variance and bias, we can prove our uniform bound: 
\begin{theorem}
	Under the conditions of the previous two propositions, if we choose $h = \alpha n^{-1/(2\beta + 1)}$ for some $\alpha > 0$, there exists $n_0 \in \NN, C > 0$, depending only on $\beta, L, \alpha, K, \sigma_\Rm{max}^2$, such that
	\[
	\sup_{m \in \HH(\beta, L)} \sup_{x_0 \in [0, 1]} \EE\qty[ \qty{\hat m_n(x_0;p) - m(x_0)}^2] \leq Cn^{-2\beta/(2\beta + 1)}.
	\]
\end{theorem}


%Suppose that $X\T W X$ is positive definite (this is a very weak condition), then we have
%\begin{align*}
%	\hat\beta = \argmin_{\beta \in \RR^{p+1}} (\beta - (X\T W X)^{-1} X\T W Y)\T (X\T W X) (\beta - (X\T W X)^{-1} X\T W Y),
%\end{align*}
%which is minimised at $\hat\beta = (X\T W X)^{-1} X\T WY$. 

\subsection{Splines}
\subsubsection{Cubic splines}
Let $n \geq 3$ and $a \leq x_1 < \dotsb < x_n \leq b$.  
\begin{definition}
	A function $g \colon [a, b] \to \RR$ is called a \emph{cubic spline} with \emph{knots} $x_1, \dotsc, x_n$ if 
	\begin{enumerate}
		\item $g$ is a cubic polynomial on each interval $(a, x_1), (x_1, x_2), \dotsc, (x_n, b)$;
		\item $g \in C^2[a, b]$. 
	\end{enumerate}
	Furthermore, $g$ is called \emph{natural} if it is linear on $[a, x_1]$ and $[x_n, b]$, (i.e., $g''(a) = g'''(a) = g''(b) = g'''(b) = 0$). 
\end{definition}

We often represent a natural cubic spline by the vectors $\vec g \in \RR^n$ with $g_i = g(x_i)$, and $\vec \gamma \in \RR^{n-2}$ with $\gamma_i = g''(x_i)$ (excluding $\gamma_1$ and $\gamma_n$). Writing $h_i \ceq x_{i+1} - x_i$ we have for $x \in [x_i, x_{i+1}]$ that
\[
g(x) = \frac{(x - x_i) g_{i+1} - (x_{i+1} - x) g_i}{h_i} - \frac16(x - x_i)(x_{i+1} - x) \qty{\qty(1 + \frac{x - x_i}{h_i})\gamma_{i+1} + \qty(1 + \frac{x_{i+1} - x}{h_i}\gamma_i)}.
\]
\begin{proposition}
	Given $\vec g \in \RR^n$, there exists a unique natural cubic spline $g$ with knots at $x_1, \dotsc, x_n$ satisfying $g(x_i) = g_i$ for all $i$, and there exists $K \succeq 0$ (depending on $x_1, \dotsc, x_n$) such that
	\[
	\int_a^b g''(x)^2 \dd{x} = \vec g\T K \vec g. 
	\]
\end{proposition}
We call $g$ the \emph{natural cubic spline interpolant to $\vec g$ at $x_1, \dotsc, x_n$}. 

\begin{definition}
	We define $\SS_2[a, b]$ as the set of real-valued functions on $[a, b]$ with an absolutely continuous first derivative. For $f \in \SS_2[a, b]$, we define the \emph{roughness} of $f$ by $R(f) \ceq \int_a^b f''(x)^2 \dd{x}$. 
\end{definition}

\begin{proposition}
	For any $\vec g \in \RR^n$, the natural cubic spline interpolant to $\vec g$ at $x_1, \dotsc, x_n$ is the unique minimiser of $R$ over all $g \in \SS_2[a, b]$ that satisfy $g(x_i) = g_i$ for all $i$. 
\end{proposition}

\subsubsection{Natural cubic smoothing splines}
Consider the nonparametric regression model $Y_i = g(x_i) + \sigma \eps_i$, where $\EE[\eps_i] = 0$ and $\Var[\eps_i] = 1$. A way to estimate a nonparametric regression function is to balance data fidelity against roughness of the curve, which can be done by minimising 
\[
S_\lambda(g) \ceq \sum_{i=1}^n (Y_i - g(x_i))^2 + \lambda R(g),
\]
where $\lambda > 0$ is a regularisation parameter. For small $\lambda$, this is almost an exact fit to the data. For large $\lambda$, we are basically minimising $\abs{g''}$, which means we will approximate the linear regression fit. 

\begin{theorem}
For each $\lambda \in (0, \infty)$, there exists a unique minimiser $\hat g_\lambda$ of $S_\lambda$ over $\SS_2[a, b]$. It is the natural cubic spline with knots at $x_1, \dotsc, x_n$ and $\vec g = (I + \lambda K)^{-1} \vec Y$. 
\end{theorem}

\begin{proof}
	If $g$ is not a natural cubic spline, we know that the natural cubic spline $g^*$ which interpolates $g(x_1), \dotsc, g(x_n)$ at $x_1, \dotsc, x_n$ has a strictly lower value of $S_\lambda$, so we know the minimiser must be a natural cubic spline. 
	
	If $g$ is a natural cubic spline, then there exists $K \succeq 0$ such that
	\begin{align*}
		S_\lambda(g) &= (\vec Y - \vec g)\T (\vec Y - \vec g) + \lambda \vec g\T K \vec g \\
		&= \vec g\T (I + \lambda K)\vec g - 2 \vec Y\T \vec g + \vec Y\T\vec Y,
	\end{align*}
and by ``completing the square'' we write, for some $Z$ independent of $\vec g$, 
\[
S_\lambda(g) = \qty(\vec g - (I + \lambda K)^{-1} \vec Y)\T (I + \lambda K) \qty(\vec g - (I + \lambda K)^{-1} \vec Y) + Z
\]
Since $I + \lambda K$ is positive definite it follows that $\vec g = (I + \lambda K)^{-1} \vec Y$ gives the minimiser. 
\end{proof}

The function $\hat g_\lambda$ is called the \emph{natural cubic smoothing spline} for data $(x_1, Y_1), \dotsc, (x_n, Y_n)$ and smoothing parameter $\lambda$. 

\subsubsection{Choice of smoothing parameter}
We are left with the question of how to choose the smoothing parameter $\lambda$. A standard method is to minimise the cross-validation score
\[
\CV(\lambda) \ceq \frac1n \sum_{i=1}^n \qty(Y_i - \hat g_{-i, \lambda}(x_i))^2, 
\]
where $\hat g_{-i, \lambda}$ is the natural cubic smoothing spline for all data points except $(x_i, Y_i)$. It seems like computing $\CV(\lambda)$ requires the computation of $n$ natural cubic smoothing splines, but it turns out that this is not the case:
\begin{proposition}
	Write $A(\lambda) = (I + \lambda K)^{-1}$, then we have
	\[
	\CV(\lambda) = \frac1n \sum_{i=1}^n \qty(\frac{Y_i - \hat g_\lambda(x_i)}{1 - A_{ii}(\lambda)})^2. 
	\]
\end{proposition}

\begin{proof}
	Example sheet 3. 
\end{proof}

In the above formula, we can consider the quantity $A_{ii}(\lambda)$ as the ``leverage'' of the $i$-th observation. In the \emph{generalised cross-valudation} score, we give every observation equal leverage: it is defined as
\[
\Rm{GCV}(\lambda) \ceq \frac1n \sum_{i=1}^n \qty(\frac{Y_i - \hat g_\lambda(x_i)}{1 - n^{-1} \tr A(\lambda)})^2. 
\]

\paragraph{Regression splines}
There are many different types of splines and different directions to go in. For example, one disadvantage of natural cubic smoothing splines is that we have a ``parameter'' of dimension $n$ to estimate (namely, the vector $\vec g$). We can also reduce the number of knots to $\xi_1, \dotsc, \xi_K$ and locate $\xi_k$ at the $\qty(\frac{k+1}{K+2})$-th sample quantile of $x_1, \dotsc, x_n$. Splines of order $p$ can then be expanded in the \emph{truncated power series basis}
\[
1, x, x^2, \dotsc, x^p, (x - \xi_1)_+^p, \dotsc, (x - \xi_K)_+^p. 
\]
Therefore we can minimise the residual error over all polynomials in the span of this basis, which gives a parameter in $\RR^{p+1+K}$ to estimate using least squares. The solution is called a \emph{regression spline}. Here, $K$ playes the rolw of the smoothing parameter. 
\section{Minimax lower bounds}
We seek lower bounds on the worst-case risk of any procedure, which provide a `benchmark'  against which we can measure the performance of a proposed method. 

\subsection{Reduction to testing}
We will assume our parameter space $(\Theta, d)$ is a metric space or a semi-metric space (where we don't require that $d(\theta, \theta') = 0 \implies \theta = \theta'$). We denote our collections of distributions depending on our parameters by $\qty{P_\theta \mid \theta \in \Theta}$, which are probability measures on some measurable space $(\XX, \AA)$. 

Now, we let $(\Omega, \FF)$ be any measurable space with a collection of probability measures $\qty{\PP_\theta \mid \theta \in \Theta}$ and a measurable function $X \colon \Omega \to \XX$ so that $X \sim P_\theta$ on $(\Omega, \FF, \PP_\theta)$. Let $\hat\Theta$ denote the set of possible estimators for $\theta$, i.e., all measurable functions $\XX \to\Theta$. 

Now, suppose we wish to estimate $\theta$ with a loss function of the form
\[ 
L(\theta', \theta) = g(d(\theta', \theta)) \qquad\text{$g$ increasing, $\theta', \theta \in \Theta$}. 
\]
We then define the \emph{minimax risk} as 
\[
\Cal M \ceq \inf_{\hat\theta \in \hat\Theta} \sup_{\theta \in \Theta} \EE_\theta L(\hat\theta(X), \theta),
\]
i.e., the lowest worst-case estimated loss of any possible estimator $\hat\theta$. 


\begin{example}
Suppose we are trying to estimate the mean of a normally distributed random variable with variance 1, and we have a sample $(X_1, \dotsc, X_n)$. Then $\Theta = \RR$ with the Euclidian distance, $(\XX, \AA) = (\RR^n, \BB\RR^n)$, and $P_\theta(F) = \int_F f_\theta(x_1)\dotsb f_\theta(x_n) \dd{\lambda^n(x)}$, where $f_\theta$ is the density function of a $N(\theta, 1)$ distribution. 

Now, let $X_\theta \colon (\Omega, \FF) \to (\RR^n, \BB\RR^n)$ be a $N_n(\theta \vec 1, I_n)$ distributed random variable. In this case, $\PP_\theta$ is a probability measure on $\Omega$ such that $P_\theta(F) = \PP_\theta(X \in F)$. 

Let $\hat\Theta$ denote the set of all estimators of $\theta$, which are functions $\RR^n \to \RR$. Our loss function could simply be $L(\theta', \theta) = \abs{\theta' - \theta}$ or $L(\theta', \theta) = (\theta' - \theta)^2$. 
\end{example}

For $M \in \NN$, let $[M] \ceq \qty{1, \dotsc, M}$, and let $\hat \TT$ denote the set of measurable functions $\XX \to [M]$. Given any $\theta_1, \dotsc, \theta_M \in \Theta$ and $\hat\theta \in \hat\Theta$, we can define $T_{\hat\theta} \in \hat\TT$ by
\[
T_{\hat\theta}(x) \ceq \argmin_{j \in [M]} d(\hat\theta(x), \theta_j), 
\]
where we pick the smallest $j$ in case of a tie. Intuitively, we are simply approximating $\hat\theta$ by the closest $\theta_j$. Now, we will lower-bound the minimax risk by an expression that only depends on estimators in $\hat\TT$. 

Writing $\eta = \frac12 \min_{jk} d(\theta_j, \theta_k)$, we can lower-bound the worst-case loss of any fixed estimator $\hat\theta$ by 
\begin{align*}
	\sup_{\theta \in \Theta} \EE_\theta L(\hat\theta(X), \theta) &\geq \max_{j \in [M]} \EE_{\theta_j} g(d(\hat\theta, \theta_j)) \\
	&= \max_{j \in [M]} \EE_{\theta_j} \qty{g(d(\hat\theta, \theta_j)) \ind_{T_{\hat\theta} \neq j}} \\
	&\overset\star\geq g(\eta) \max_{j \in [M]} \EE_{\theta_j} \ind_{T_{\hat\theta} \neq j} \\
	&= g(\eta) \max_{j \in [M]} P_{\theta_j}(T_{\hat\theta} \neq j), 
\end{align*}
where $\star$ holds because if $T_{\hat\theta} \neq j$, then $d(\hat\theta(x), \theta_j) \geq \eta$. 

We therefore have
\begin{align*}
	\Cal M &\geq g(\eta) \inf_{\hat\theta \in \hat\Theta} \max_{j \in [M]} P_{\theta_j}(T_{\hat\theta} \neq j) \geq g(\eta) \inf_{T \in \hat\TT} \max_{j \in [M]} P_{\theta_j}(T \neq j) \\
	&= g(\eta) \qty{1 - \sup_{T \in \hat\TT} \min_{j \in [M]} P_{\theta_j}(T = j)} \geq g(\eta) \qty{1 - \sup_{T \in \hat\TT} \frac1M \sum_{j=1}^M P_{\theta_j}(T = j)}. 
\end{align*}
Therefore, we have now reduced the problem of lower-bounding $\Cal M$ to the problem of upper-bounding $\sup_{T \in \hat T} \frac1M \sum_{j=1}^M P_{\theta_j}(T = j)$, which is a testing problem. We repeat the main result:
\begin{equation} \label{eq:red_to_testing}
	\Cal M \geq g(\eta) \qty{1 - \sup_{T \in \hat\TT} \frac1M\sum_{j=1}^M P_{\theta_j} (T = j)}, 
\end{equation}
where $\eta = \frac12 \min_{jk} d(\theta_j, \theta_k)$. 
\subsection{Divergences}
\begin{definition}
	Let $\mu, \nu$ be measures on $(\XX, \AA)$. We say that $\mu$ is \emph{absolutely continuous} w.r.t.\ $\nu$, notation $\mu \ll\nu$, if
	\[
	\nu(A) = 0 \implies \mu(A) = 0. 
	\]
	
	We say that $\mu, \nu$ are \emph{mutually singular}, notation $\mu \perp \nu$, if there exists $A \in \AA$ such that $\mu(A) = 0$ and $\nu(A\C) = 0$.
\end{definition}
Note that mutual singularity means that $\mu$ ``lives on'' $A\C$, while $\nu$ ``lives on'' $A$. 

\begin{theorem}[Lebesgue]
	If $\mu, \nu$ are $\sigma$-finite measures on $(\XX, \AA)$, then there exists measures $\mu_\Rm{ac}$ and $\mu_\Rm{sing}$ on $(\XX, \AA)$ such that $\mu$ can be decomposed as $\mu = \mu_\Rm{ac} + \mu_{\Rm{sing}}$, where $\mu_\Rm{ac} \ll \nu$ and $\mu_\Rm{sing} \perp \nu$. Furthermore, this decomposition is unique. 
\end{theorem}

Let $f \colon (0, \infty) \to \RR$ be convex. Then for any $y > 0$, the function $x \mapsto \frac{f(x) - f(y)}{x - y}$ is increasing on $(y, \infty)$ (this is easy to check). Furthermore, we have
\[
\lim_{x \to \infty} \frac{f(x) - f(y)}{x - y} = \lim_{x \to \infty} \frac{f(x)}{x - y} - \lim_{x \to \infty} \frac{f(y)}{x - y} = \lim_{x \to \infty} \frac{f(x)}{x}, 
\]
so in particular the limit is independent of $y$ and we can define the \emph{maximal slope} of $f$ by
\[
M_f \ceq \lim_{x\to\infty} \frac{f(x)}{x} \in (-\infty, \infty].
\]
We define $f(0) \ceq \lim_{x \downarrow 0} f(x) \in (-\infty, \infty]$ (a convex function is continuous on an open interval, so this limit exists). In this case, we have
\[
f(x + y) = f(x) + y \frac{f(x+y) - f(x)}{y} \leq f(x) + y M_f \qquad\forall x, y \geq 0. 
\]

\begin{definition}
	Given a convex function $f \colon (0, \infty) \to \RR$ with $f(1) = 0$, and probability measures $P, Q$ on a measurable space $(\XX, \AA)$, we define the $f$-divergence 
	\[
	\Div_f(P, Q) \ceq \int_\XX f\qty(\dv{P_\Rm{ac}}{Q}) \dd{Q} + P_\Rm{sing}(\XX) \cdot M_f. 
	\]
\end{definition}

By Jensen's inequality we have
\begin{align*}
	\Div_f(P, Q) &=  \int_\XX f\qty(\dv{P_\Rm{ac}}{Q}) \dd{Q} + P_\Rm{sing}(\XX) \cdot M_f \geq f\qty(\int_\XX \dv{P_\Rm{ac}}{Q} \dd{Q}) + P_\Rm{sing}(\XX) \cdot M_f \\
	&= f(P_\Rm{ac}(\XX)) + P_\Rm{sing}(\XX) \cdot M_f = f(P_\Rm{ac}(\XX) + P_\Rm{sing}(\XX)) = f(P(\XX)) = f(1) = 0, 
\end{align*}
so $f$-divergences are nonnegative. 

\begin{example}
	\begin{enumerate}
		\item If $f(x) = x\log x$, then $M_f = \infty$. If $P \ll Q$ (i.e., $P_\Rm{sing} = 0$), then we have
		\[
		\Div_f(P, Q) = \int_\XX  \dv{P}{Q} \log(\dv{P}{Q}) \dd{Q} = \int_\XX \log(\dv{P}{Q}) \dd{P},
		\]
		and otherwise $\Div_f(P, Q) = 0$.
		
		This divergence is known as the \emph{Kullbach-Leibler} divergence from $Q$ to $P$, denoted $\KL(P, Q)$. 
		
		If $P \ll Q$ and $P$ and $Q$ have densities $p$ and $q$ w.r.t.\ a  measure $\mu$, we have $\KL(P, Q) = \int_\XX p \log(\frac pq) \dd{\mu}$. 
		
		\item If $f(x) = x^2 - 1$, then $M_f = \infty$. If $P \ll Q$ we have
		\[
		\Div_f(P, Q) = \int_\XX \qty(\dv{P}{Q})^2 \dd{Q} - \int_\XX \dd{Q} = \int_\XX \qty(\dv{P}{Q})^2 \dd{Q} - 1, 
		\]
		and otherwise $\Div_f(P, Q) = \infty$. 
		
		This divergence is known as the $\chi^2$ divergence from $Q$ to $P$, denoted $\chi^2(P, Q)$. 
		
		If $P \ll Q$ and  $P$ and $Q$ have densities $p$ and $q$ w.r.t.\ a measure $\mu$, we have $\chi^2(P, Q) = \int_\XX \frac{p^2}{q} \dd{\mu} - 1$.
		
		\item If $f(x) = (\sqrt x - 1)^2 = x + 1 - 2\sqrt x$ (note that this is convex since $\sqrt x$ is concave), then $M_f = 1$ and therefore
		\[
		\Div_f(P, Q) = \int_\XX \qty(\sqrt{\dv{P_\Rm{ac}}{Q}} - 1)^2 \dd{Q} + P_\Rm{sing}(\XX) \eqqcolon \Hh^2(P, Q),
		\]
		the \emph{squared Hellinger distance} between $P$ and $Q$. 
		
		If $P$ and $Q$ have densities $p$ and $q$ w.r.t.\ a $\sigma$-finite measure $\mu$, then $H^2(P, Q) = \int_\XX (\sqrt p - \sqrt q)^2 \dd{\mu}$ (example sheet). 
		
		\item If $f(x) = \frac{\abs{x-1}}{2}$, then $M_f = \frac12$ and we have
		\begin{align*}
			\Div_f(P, Q) \overset{\text{\TODO}}{=} \sup_{A \in \AA} \abs{P(A) - Q(A)} \eqqcolon \TV(P, Q),
		\end{align*}
	the \emph{total variation} divergence between $P$ and $Q$. 
	\end{enumerate}
\end{example}

All $f$-divergences are \emph{jointly convex}: for all $\lambda \in [0, 1]$ we have (see Example Sheet)
\[
\Div_f\qty((1 - \lambda) P_1 + \lambda P_2, (1 - \lambda) Q_1 + \lambda Q_2) \leq (1-\lambda) \Div_f(P_1, Q_1) + \lambda \Div_f(Q_1, Q_2)
\]

\subsection{Le Cam's two point lemma}
Plugging $M = 1$ into \cref{eq:red_to_testing} yields the trivial result $\Cal M \geq 0$. Surprisingly, when we plug in $M = 2$, we obtain Le Cam's two point lemma, which can often provide optimal rates for estimating real-valued parameters (though not optimal constants). 
\begin{lemma}
	In the set-up of section 4.1, we have for any $\theta_1, \theta_2 \in \Theta$ that 
	\[
	\Cal M \geq \frac{g(\eta)}{2} \qty{1 - \TV(P_{\theta_1}, P_{\theta_2})}. 
	\]
\end{lemma}

\begin{proof}
	For $T \in \hat\TT_2$, let $A \ceq T^{-1}(\qty{1})$, then we have by \cref{eq:red_to_testing}
	\begin{align*}
		\Cal M &\geq g(\eta) \qty{1 - \sup_{T \in \hat T_2} \frac{P_{\theta_1}(T = 1) + P_{\theta_2}(T = 2)}{2}} = \frac{g(\eta)}{2} \qty{1 - \sup_{T \in \hat\TT_2} \qty(P_{\theta_1}(T = 1) - P_{\theta_2}(T = 1))} \\
		&\geq \frac{g(\eta)}{2} \qty{ 1 - \TV(P_{\theta_1}, P_{\theta_2})}. 
	\end{align*}
\end{proof}

\begin{example}
	Let $X_1, \dotsc, X_n \iid N(\theta, 1)$ for some $\theta \in \RR$. Let $\theta_1 = 0$ and $\theta_2 = c n^{-1/2}$ for some $c > 0$, and let $P_{\theta_j} \ceq N(\theta_j, 1)$ for $j = 1, 2$. Then we have (where $\star$ denotes equalities that will be proved on the example sheet):
	\[
	\TV(P_{\theta_1}^{\times n}, P_{\theta_2}^{\times n}) \overset\star\leq \sqrt{\frac{\KL(P_0^{\times n}, P_{cn^{-1/2}}^{\times n})}{2}} \overset\star= \sqrt{\frac n2 \KL(P_0, P_{cn^{-1/2}})} \overset\star= \frac c2. 
	\]
	Plugging this into Le Cam's two point lemma yields (using the squared error loss $L(x, y) = (x-y)^2$ that
	\[
	\Cal M = \inf_{\hat\theta \in \hat\Theta} \sup_{\theta \in \RR} \EE_\theta \qty[(\hat\theta(X_1, \dotsc, X_n) - \theta)^2] \geq \sup_{c > 0} \frac{c^2}{8n} \qty(1 - \frac c2) = \frac2{27n}.
		\]
		
	In this problem, it can be shown that $\Cal M = 1/n$, so Le Cam's two point lemma does give the optimal rate, but it does not give the optimal constant. 
\end{example}

\begin{example}
	\TODO Understand remark 45 in lecture notes
\end{example}

\subsection{Assouad's lemma}
\begin{lemma}
	Let $m \in \NN$, $\Phi \ceq \qty{0, 1}^m$ and $\qty{P_\phi \mid \phi \in \Phi}$ a family of probability measures on $(\XX, \AA)$. Write $\phi \sim \phi'$ when $\phi$ and $\phi'$ differ in precisely one coordinate and $\phi \sim_j \phi'$ when this coordinate is the $j$-th. 
	
	Suppose the loss function is of the form 
	\[
	L(\phi', \phi) = \sum_{j=1}^m L_j(\phi', \phi) = \sum_{j=1}^m g(d_j(\phi', \phi)), 
	\]
	where $d_1, \dotsc, d_m$ are semimetrics satisfying $d_j(\phi', \phi) \geq\alpha_j$ whenever $\phi \sim_j \phi'$, and where $g$ is increasing and  satisfies $g(x+y) \leq A\qty{g(x) + g(y)}$ for all $x, y \geq 0$ and some $A > 0$. Then
	\[
	\inf_{\hat\phi \in \hat\Phi} \max_{\phi\in\Phi} \EE_\phi L(\hat\phi,\phi) \geq \frac1{2A} \qty{1 - \max_{\phi \sim \phi'} \TV(P_\phi, P_{\phi'})} \sum_{j=1}^m g(\alpha_j). 
	\]
	
\end{lemma}

\begin{proof}
	For any $\phi \in \Phi, j \in [m]$, let $\phi^{[j]}$ be the unique element of $\Phi$ with $\phi \sim_j \phi^{[j]}$. Letting $\hat\Phi$ denote the set of measurable functions from $\XX$ to $\Phi$, we have
\begin{equation} \label{eq:minimax}
		 \max_{\phi \in \Phi} \EE_\phi L(\hat\phi, \phi) \geq \frac1{2^m} \sum_{\phi \in \Phi} \sum_{j=1}^m \EE_\phi L_j(\hat\phi, \phi) = \frac1{2^{m+1}} \sum_{j=1}^m \sum_{\phi \in \Phi} \qty{\EE_\phi L_j(\hat\phi,\phi) + \EE_{\phi^{[j]}} L_j(\hat\phi, \phi^{[j]})},
	 	\end{equation}
 	where the last equality follows from the fact that in the sum we count every element of $\phi$ twice (so we must divide by 2). By the definition of $L_j$ and the triangle inequality,
 	\[
 	L_j(\hat\phi, \phi) + L_j(\hat \phi, \phi^{[j]}) \geq \frac1A g(d_j(\hat\phi, \phi) + d_j(\hat\phi, \phi^{[j]})) \geq \frac1A g(d_j(\phi, \phi^{[j]})) \geq \frac{g(\alpha_j)}{A}.
 	\]
 	If we multiply and divide \cref{eq:minimax} by $\sum_{j=1}^m L_j(\hat\phi, \phi) + L_j(\hat\phi, \phi^{[j]})$, we obtain, writing $\FF$ for the set of measurable functions from $\XX\to [0, 1]$, 
 	\begin{align*}
 	\max_{\phi \in \Phi} \EE_\phi L(\hat\phi, \phi) &\geq \qty(\sum_{j=1}^m L_j(\hat\phi,\phi) + L_j(\hat\phi, \phi^{[j]}))\frac1{2^{m+1}}  \frac{ \sum_{j=1}^m \sum_{\phi \in \Phi} \qty{\EE_\phi L_j(\hat\phi,\phi) + \EE_{\phi^{[j]}} L_j(\hat\phi, \phi^{[j]})}}{\sum_{j=1}^m L_j(\hat\phi, \phi) + L_j(\hat\phi, \phi^{[j]})} \\
 	&\overset{??}\geq \frac{\sum_{j=1}^m g(\alpha_j)}{2^{m+1}A} \sum_{\phi \in \Phi} \inf_{f_1, f_2 \in \FF: f_1 + f_2 =1 } \qty{\EE_\phi(f_1) + \EE_{\phi^{[j]}} (f_2)} \\ 
 	&= \frac{\sum_{j=1}^m g(\alpha_j)}{2^{m+1}A}  \sum_{\phi \in \Phi}\qty(1 - \sup_{f \in \FF}\qty{ \EE_\phi f - \EE_{\phi^{[j]}} f}) \\ 
 	&\overset\star\geq \frac1{2A} \qty{1 - \max_{\phi \sim \phi'} \TV(P_\phi, P_{\phi'})} \sum_{j=1}^m g(\alpha_j), 
  	\end{align*}
  \TODO explain $\star$ (related to the alternative expression for $\TV$ as the divergence of $\abs{x-1}/2$).
  % where $\star$ follows from the fact that $\TV(P_\phi, P_{\phi'})
\end{proof}

\begin{example}
	Let $X_1, \dotsc, X_n \iid N_d(\theta, \Sigma) \eqqcolon P_\theta$ for some $\theta \in \RR^d$, where $\Sigma = \Rm{diag}(\sigma_1^2, \dotsc, \sigma_d^2)$. Fix $c > 0$, and for $\phi \in \qty{0, 1}^d$ define $\theta^\phi \in \RR^d$ by $\theta_j^\phi = c\sigma_j n^{-1/2} \ind_{\phi_j = 1}$ for $j \in [d]$. 
	
	If we used the squared error loss $L(\theta, \theta') = \norm{\theta - \theta'}_2^2$, then we have $d_j(\theta, \theta') = \abs{\theta_j - \theta'_j}$, so $\alpha_j = c \sigma_j n^{-1/2}$ and $g(x) = x^2$, and  $(x+y)^2 \leq 2(x^2 + y^2)$ implies $A = 2$. 
	
	Write $\hat\Theta$ for the set of measurable functions from $(\RR^d)^{\times n}$ to $\RR^d$, then Assouad's lemma tells us that, for any $\hat\theta(X_1, \dotsc, X_n) = \hat\theta \in \Theta$, 
	\begin{align*}
		\sup_{\theta \in \RR^d} \EE_\theta \qty(\norm{\hat\theta - \theta}^2) &\geq \max_{\phi \in \Phi} \EE_{\theta^\phi} \qty(\norm{\hat\theta - \theta^\phi}^2) \\
		&\geq \frac14 \qty{1 - \max_{\phi \sim \phi'} \TV(P_{\theta^\phi}, P_{\theta^{\phi'}})} \sum_{j=1}^d \frac{c^2\sigma_j^2}{n} \\
		&\overset\star\geq \frac{c^2}{4n} \qty{1 - \max_{\phi \sim \phi'} \sqrt{\frac{\KL(P_{\theta^\phi}^{\times n}, P_{\theta^{\phi'}}^{\times n})}{2}}} \sum_{j=1}^d \sigma^2 \\
		&= \frac{c^2}{4n} \qty(1 - \frac c2) \sum_{j=1}^d \sigma_j^2, 
	\end{align*}
and taking the supremum over all $c > 0$ on the right-hand side gives $\frac4{27n} \sum_{j=1}^d \sigma_j^2$.  Here, $\star$ follows from Pinsker's inequality (example sheet). 

It can be shown that this is the optimal rate in terms of $n$ and the $\sigma_j$, but again, $4/27$ is not the optimal constant. 

\subsection{The data processing inequality}
If $(\XX, \AA, \mu)$ is a measure space and $(\YY, \BB)$ a measurable space, and $g \colon \XX \to \YY$ is measurable, we denote the pushforward measure by $\mu^g \ceq \mu \circ g^{-1}$. 
\begin{lemma}[Data processing inequality]
	Let $(\XX, \AA)$ and $(\YY, \BB)$ be measurable spaces, $P, Q$ probability measures on $\XX$, and $g \colon \XX \to \YY$ measurable. Then for any $f$-divergence $\Div_f$ we have
	\[
	\Div_f (P^g, Q^g) \leq \Div_f(P, Q). 
	\]
\end{lemma}

\begin{proof}
	We decompose $P = P\ac + P\sing$ w.r.t.\ $Q$. We will show that $(P\ac)^g \ll Q^g$ with Radon-Nikodym derivative $\gamma$, where
	\[
	\gamma(y) \ceq \EE_Q \qty( \dv{P\ac}{Q}(X) \mid g(X) = y). 
	\]
	To see this, let $B \in \BB$, then
	\begin{align*}
		(P\ac)^g(B) &= P\ac(g^{-1}(B)) = \int_\XX \ind_{g^{-1}(B)} \dv{P\ac}{Q} \dd{Q} = \EE_Q\qty[ \EE_Q \qty[ \ind_{g^{-1}(B)}(X) \dv{P\ac}{Q}\qty(X) \mid g(X)]] \\
		&\overset\star= \EE_Q \qty[ \ind_{g^{-1}(B)}(X) \EE_Q\qty[\dv{P\ac}{Q}\qty(X) \mid g(X)]] \\
		&= \int_\XX \ind_{g^{-1}(B)} \cdot (\gamma \circ g) \dd{Q} 
		\overset\star= \int_\YY \ind_B \cdot \gamma \dd{Q^g} = \int_B \gamma \dd{Q^g}, 
	\end{align*}
where $\star$ follows from the transformation theorem. This establishes the claim. 

Now, writing $(P\sing)^g = ((P\sing)^g)\ac + ((P\sing)^g)\sing$ (the Lebesgue decomposition w.r.t.\ $Q^g$), we have
\[
P^g = (P\ac)^g + (P\sing)^g = (P\ac)^g + ((P\sing)^g)\ac + ((P\sing)^g)\sing, 
\]
and since $(P\ac)^g \ll Q^g$, this gives the Lebesgue decomposition of $P^g$ w.r.t.\ $Q^g$, namely $(P^g)\ac = (P\ac)^g$ and $(P^g)\sing = ((P\sing)^g)\sing$. We now obtain, using $f(x+y) \leq f(x) + M_f$, that
\begin{align*}
	\Div_f(P^g, Q^g) &= \int_\YY f\qty(\dv{(P\ac)^g}{Q^g} + \dv{((P\sing)^g)\ac}{Q^g}) \dd{Q^g} + ((P\sing)^g)\sing(\YY) \cdot M_f \\
	&\leq \int_\YY f\qty(\dv{(P\ac)^g}{Q^g}) \dd{Q^g} + (P\sing)^g(\YY) \cdot M_f \\
	&= \int_\YY f \circ \gamma \dd{Q^g} + (P\sing)^g(\YY) \cdot M_f \\
	&= \int_\XX f\circ\gamma\circ g \dd{Q} + P\sing(\XX) \cdot M_f \\
	&= \EE_Q\qty[ f\qty(\EE_Q\qty[ \dv{P\ac}{Q}\qty(X) \mid g(X)])] + P\sing(\XX) \cdot M_f \\
	&\leq \EE_Q \qty[ \EE_Q \qty{ f\qty(\dv{P\ac}{Q} \qty(X) \mid g(X))}] + P\sing(\XX) \cdot M_f \\
	&= \int_\XX f\qty(\dv{P\ac}{Q}) \dd{Q} + P\sing(\XX) \cdot M_f = \Div_f(P, Q), 
\end{align*}
where the last inequality follows from the conditional version of Jensen. 
\end{proof}
\end{example}
\end{document}
