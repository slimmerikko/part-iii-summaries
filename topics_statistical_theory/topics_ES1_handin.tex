\documentclass{article}

\usepackage[utf8]{inputenc}
\usepackage[english]{babel}
\usepackage[margin=3cm]{geometry}
\usepackage[normalem]{ulem}
\usepackage{hyperref}
\usepackage[shortlabels]{enumitem}
\usepackage{mathtools, amsmath, amssymb, amsthm, mdframed, bbm, graphicx, float, physics, xcolor, cleveref}

\hypersetup{
    colorlinks   = true, %Colours links instead of ugly boxes
    urlcolor     = blue, %Colour for external hyperlinks
    linkcolor    = blue, %Colour of internal links
    citecolor   = red %Colour of citations
}

% Definition of numbered environments.
% Usage: \begin{theorem} ... \end{theorem}
% Remark has no numbering.
\theoremstyle{plain}
\newtheorem{question}{Question}
\newtheorem{theorem}{Theorem}
\newtheorem{lemma}{Lemma}
\theoremstyle{remark}
\newtheorem*{remark}{Remark}

% Question with box around it
\newenvironment{qbox}{\begin{mdframed}\begin{question}}{\end{question}\end{mdframed}}

% A proof with "solution" instead of "proof" and no QED symbol
\newenvironment{solution}{\begin{proof}[Solution]\renewcommand\qedsymbol{}}{\end{proof}}

% Some renewed commands
\renewcommand{\vec}{\mathbf}
\renewcommand{\emptyset}{\varnothing}
\renewcommand{\epsilon}{\varepsilon}
\renewcommand{\theta}{\vartheta}
\renewcommand{\phi}{\varphi}

% Frequently used math alphabets
\newcommand{\Bb}{\mathbb}
\newcommand{\Cal}{\mathcal}
\newcommand{\Bf}{\mathbf}
\newcommand{\Rm}{\mathrm}

% Frequently used letters in the blackboard alphabet
\newcommand{\CC}{\Bb C}
\newcommand{\NN}{\Bb N}
\newcommand{\PP}{\Bb P}
\newcommand{\QQ}{\Bb Q}
\newcommand{\RR}{\Bb R}
\newcommand{\EE}{\Bb E}
\newcommand\HH{\Cal H}
\newcommand\FF{\Cal F}
\newcommand\Nn{\Cal N}
\renewcommand\SS{\Cal S}

% Usage: \ang{...} is equivalent to \langle ... \rangle, while \ang*{...} is equivalent to \left\langle ... \right\rangle
% For other delimiters: use \qty from the physics package (i.e., \qty(...))
\DeclarePairedDelimiter{\ang}{\langle}{\rangle}
\DeclarePairedDelimiter{\floor}{\lfloor}{\rfloor}
\DeclarePairedDelimiter{\ceil}{\lceil}{\rceil}

% Frequently used commands
\newcommand{\T}{^\top} % Matrix transpose A\T
\newcommand{\C}{^\complement} % Set complement A\C
\newcommand\ceq\coloneqq % Definitions :=
\newcommand\pow{\Cal P} % Power sets
\newcommand\eps\epsilon
\newcommand\ind{\mathbbm 1} % Blackboard 1 for indicator functions
\newcommand\restr{\mathord\restriction}
\newcommand\TODO{{\color{red} TODO: }}
\newcommand\toprob{\overset{\Rm{p}}{\to}}
\newcommand\todist{\overset{\Rm{d}}{\to}}
\newcommand\toas{\overset{\Rm{a.s.}}{\to}}
\newcommand\deq{\overset{\Rm{d}}=}
\newcommand\iid{\overset{\Rm{iid}}{\sim}}

% Functions that appear frequently
\DeclareMathOperator{\sign}{sign}
\DeclareMathOperator{\Int}{Int}
\DeclareMathOperator{\Span}{Span}
\DeclareMathOperator{\Var}{Var}
\DeclareMathOperator*{\argmin}{arg\,min}
\DeclareMathOperator*{\argmax}{arg\,max}
\DeclareMathOperator\Exp{Exp}
\DeclareMathOperator\MSE{MSE}
\DeclareMathOperator\MISE{MISE}
\DeclareMathOperator\Bias{Bias}
\DeclareMathOperator\Beta{Beta}
\DeclareMathOperator\Gam{Gamma}
\DeclareMathOperator\Cov{Cov}
\DeclareMathOperator\diag{diag}

\title{Topics in Statistical Theory --- Example Sheet 1} % subject
\author{Lucas Riedstra}
\date{8 November 2020} % date

\begin{document}
\maketitle

\setcounter{question}{3}

\begin{question}
	Let $X \sim \Rm{Bin}(n, p)$. Compare the Hoeffding, Bennett, and Bernstein upper bounds on $\PP(X/n \geq \frac12)$ as $p \to 0$. 
\end{question}

\begin{solution}
	Note that $X/n$ is the average of $n$ i.i.d.\ random variables $Y_i \sim \Rm{Bern}(p)$, where $Y_i \in [0, 1]$ for all $i$.  
	\begin{enumerate}
		\item 

	We start with Hoeffding's inequality. In this case, we have
	\[
	\PP\qty(X/n - p \geq \frac12) \leq \exp(-\frac{2n^2(1/2)^2}{n}) = \exp(- \frac n2),
	\]
	and this bound also holds as $p \to 0$. 
	
\item 	We continue with Bennett's inequality. We consider the mean-zero random variables $Y_i - p$, which are bounded from above by $b = 1 - p$. Now Bennett's inequality tells us that
	\begin{align*}
		\PP\qty(X/n \geq \frac12) &\leq \exp(- \frac{np(1-p)}{(1-p)^2} h\qty(\frac{1-p}{2p(1-p)})) = \exp\qty( -\frac{np}{1-p} \cdot  h\qty(\frac{1}{2p})).
	\end{align*}
We compute
\[
h\qty(\frac1{2p}) = \qty(1 + \frac1{2p}) \log(1 + \frac1{2p}) - \frac1{2p} = \log(1 + \frac1{2p}) + \frac1{2p} \qty(\log(1 + \frac1{2p}) - 1),
\]
and therefore
\[
\frac{np}{1-p} h\qty(\frac1{2p}) \geq  \frac{n}{2(1-p)} \qty(\log(1 + \frac1{2p}) - 1) \overset{p \to 0}\to \infty. 
\]
It follows that the Bennett upper bound converges to 0 as $p \to 0$. 
	
%   Bennett's inequality tells us that
%	\[
%	\PP(X/n \geq \frac12) \leq \exp\qty(- n\nu h(1/(2\nu))), 
%	\]
%	where $h(u) = (1 + u) \log(1+ u)$ and $\nu = p(1-p)$. Plugging this in yields
%	\[
%	\exp(-n\nu h(x/\nu)) = \exp(-n p(1-p) \qty(1 + \frac{1}{2p(1-p)})\log(1 + \frac{1}{2p(1-p)}))
%	\]

\item We finish with Bernstein's inequality.  We have for $q \geq 3$ that
\[
\frac{2(1-p)}{q!} \leq \frac2{q!} \leq 3^{2-q}, 
\]
and therefore we have
\begin{align*}
	\EE[(Y_i - p)_+^q] &= p(1-p)^q = \sigma_p^2 (1-p)^{q-1} = (q! \sigma_p^2 (1-p)^{q-2}/2)\cdot (2(1-p)/q!) \\
	&\leq q!\sigma_p^2 ((1-p)/3)^{q-2}/2,
\end{align*}
so $Y_i - p$ satisfies Bernstein's condition with $c = \frac{1-p}{3}$.
Now Bernstein's inequality tells us that
\[
\PP(X/n \geq \frac12 \leq \exp\qty(- \frac{n(1/2)^2}{2(\sigma_p^2 + (1-p)/6)}) = \exp(- \frac{n}{8\sigma_p^2 + 4(1-p)/3}) \overset{p \to 0}{\to} \exp(-\frac{3n}{4}). \]
	\end{enumerate}
Of course, the true limit is 0 for any $n$, which is only given by Bennett's inequality. We also see that Hoeffding's inequality gives the most loose bound. 

\end{solution}

\setcounter{question}{9}



\begin{question}
	\begin{enumerate}[(a)]
		\item 	Verify the algebraic identity
		\[
		\phi_\sigma(x-\mu)\phi_{\sigma'}(x-\mu') = \phi_{\sigma\sigma'/(\sigma^2 + \sigma'^2)^{1/2}}(x - \mu^*) \phi_{(\sigma^2 + \sigma'^2)^{1/2}}(\mu - \mu')
		\]
		where $\mu^* \ceq (\sigma'^2 \mu + \sigma^2\mu') / (\sigma^2 + \sigma'^2)$, and $\phi_\sigma$ is the $N(0, \sigma^2)$ density. 
		
		\item Let $X_1, \dotsc, X_n \iid N(0, \sigma^2)$. Taking $K$ to be the $N(0, 1)$ density, show that the MISE of the kernel density estimate $\hat f_n$ with kernel $K$ and bandwidth $h$ can be expressed exactly as
		\[
		\MISE(\hat f_n) = \frac1{2\pi^{1/2}} \qty{\frac1{nh} + (1 - \frac1n) \frac{1}{(h^2 + \sigma^2)^{1/2}} - \frac{2^{3/2}}{(h^2 + 2\sigma^2)^{1/2}} + \frac1\sigma}. 
		\]
	\end{enumerate}

\end{question}

\begin{proof}
	\begin{enumerate}[(a)]
		\item We have
		\begin{align*}
			&\phantom{=\ }\frac{(x-\mu)^2}{\sigma^2} + \frac{(x - \mu')^2}{\sigma'^2} \\
			&= \frac{\sigma'^2(x - \mu)^2 + \sigma^2(x - \mu')^2}{\sigma^2\sigma'^2} \\
			&= \frac{(\sigma^2 + \sigma'^2)x^2 - 2 (\sigma'^2 \mu + \sigma^2 \mu')x + \sigma'^2 \mu^2 + \sigma^2\mu'^2}{\sigma^2 \sigma'^2} \\
			&=  \frac{(\sigma^2 + \sigma'^2)(x^2 - 2\mu^*x) + \sigma'^2 \mu^2 + \sigma^2\mu'^2}{\sigma^2 \sigma'^2} \\
			&= \frac{(\sigma^2 + \sigma'^2)(x^2 - 2\mu^* + \mu^{*2}) - (\sigma^2 + \sigma'^2)\mu^{*2} + \sigma'^2 \mu^2 + \sigma^2\mu'^2}{\sigma^2 \sigma'^2} \\
			&= \frac{(x - \mu^*)^2}{(\sigma\sigma'/(\sigma^2 + \sigma'^2)^{1/2})^2} + \frac{\sigma'^2 \mu + \sigma^2 \mu'^2 - (\sigma'^2\mu + \sigma^2\mu')^2/(\sigma^2 + \sigma'^2)}{\sigma^2\sigma'^2} \\
			&= \frac{(x - \mu^*)^2}{(\sigma\sigma'/(\sigma^2 + \sigma'^2)^{1/2})^2} + \frac{(\sigma^2 + \sigma'^2)(\sigma'^2\mu + \sigma^2\mu'^2) - (\sigma'^2\mu + \sigma^2\mu')^2}{\sigma^2 \sigma'^2(\sigma^2 + \sigma'^2)} \\
			&= \frac{(x - \mu^*)^2}{(\sigma\sigma'/(\sigma^2 + \sigma'^2)^{1/2})^2} + \frac{(\mu - \mu')^2}{\sigma^2 + \sigma'^2} \\
			&= \frac{(x - \mu^*)^2}{(\sigma\sigma'/(\sigma^2 + \sigma'^2)^{1/2})^2} + \frac{(\mu - \mu')^2}{((\sigma^2 + \sigma'^2)^{1/2})^2}, 
		\end{align*}
	which proves the claim. 
	\item Let $K = \phi_1$ and define $K_h(x) \ceq h^{-1} K(x/h)$ so $K_h = \phi_h$. Then recall from the lectures that
	\[
	\MISE(\hat f_n) = \frac1n \int_\RR \qty[(\phi_h^2 * \phi_\sigma)(x) - (\phi_h * \phi_\sigma)^2(x)] \dd{x} + \int_{-\infty}^\infty \qty[ (\phi_h * \phi_\sigma)(x) - \phi_\sigma(x)]^2 \dd{x}
	\]
	
	We will use the previous exercise to compute all these terms. 
	We have in general 
	\begin{align}
		(\phi_h * \phi_\sigma)(x) &= \int_\RR \phi_h(x-z) \phi_\sigma(z) \dd{z}\nonumber \\
		&= \int_\RR \phi_\sigma(z) \phi_h(z-x) \dd{z} \nonumber\\
		&= \phi_{(\sigma^2 + h^2)^{1/2}}(x) \int_\RR \phi_\xi(z - \mu^*) \dd{z} \nonumber\\
		&= \phi_{(\sigma^2 + h^2)^{1/2}}(x) \label{eq:normal_convolve}. 
	\end{align}

	We also have
	\begin{equation} \label{eq:normal_squared}
	\phi_\sigma^2(x-\mu) = \phi_{\sigma/\sqrt 2}(x - \mu) \phi_{\sqrt 2 \sigma}(0) = \frac{1}{2 \sigma \sqrt\pi} \phi_{\sigma/\sqrt 2}(x - \mu). 
	\end{equation}

	Combining \cref{eq:normal_convolve,eq:normal_squared} we get 
	\begin{align*}
		(\phi_h^2 * \phi_\sigma)(x) &= \int_\RR \phi_h^2(x-z) \phi_\sigma(z) \dd{z} \\
		%&= \phi_{\sqrt 2 h}(0) \int_\RR \phi_{h/\sqrt 2}(x - z)  \phi_\sigma(z) \dd{z} \\
		&= \frac{1}{2h \sqrt \pi } \int_\RR \phi_{h/\sqrt 2}(x - z) \phi_{\sigma}(z)  \dd{z} \\
		&= \frac{1}{2h\sqrt \pi} (\phi_{h/\sqrt 2} * \phi_\sigma)(x) \\
		&= \frac{1}{2h \sqrt\pi} \phi_{(\sigma^2 + h^2/2)^{1/2}}(x)
%		&= \frac{1}{2h \sqrt \pi} \phi_{\xi'}(x) \int_\RR \phi_\xi(z - \mu^*) \dd{z} \\
%		&= \frac{1}{2h\sqrt\pi} \phi_{\xi'}(x),
	%\\ &= \frac{1}{2\sqrt{\pi h}} \int_\RR \phi_{\xi}(z - \mu^*) \phi_{(\sigma^2 + h^2/2)^{1/2}}(x-z)
	\end{align*}
	\end{enumerate}

	We also get
	\[
	(\phi_h * \phi_\sigma)^2(x) = \phi_{(\sigma^2 + h^2)^{1/2}}^2(x) = \frac{1}{2 (\sigma^2 + h^2)^{1/2} \sqrt\pi} \phi_{(\sigma^2 + h^2)^{1/2}/\sqrt 2}(x).
	\]
	
	Finally, we have
	\begin{align*}
		((\phi_h * \phi_\sigma)(x) - \phi_\sigma(x))^2 &= (\phi_h * \phi_\sigma)^2(x) - 2 (\phi_h * \phi_\sigma)(x) \phi_\sigma(x) + \phi_\sigma^2(x).
	\end{align*}
	The first term we already computed, the third term is $\frac{1}{2\sigma\sqrt\pi} \phi_{\sigma/\sqrt 2}(x)$, so we only need to compute
	\begin{align*}
		(\phi_h * \phi_\sigma)(x) \phi_\sigma(x) &= \phi_{(\sigma^2 + h^2)^{1/2}}(x) \phi_\sigma(x) = \phi_{\xi}(x) \phi_{(2\sigma^2 + h^2)^{1/2}}(0) = \frac{1}{\sqrt{2\pi} (2\sigma^2 + h^2)^{1/2}} \phi_\xi(x),
	\end{align*}
where $\xi$ is an irrelevant constant. 


Combining all these terms and using that $\phi_\sigma(x-\mu)$ integrates to 1 for any $\mu, \sigma$, we get
\begin{align*}
	\MISE \hat f_n &= \frac1n \qty(\frac1{2h\sqrt\pi} - \frac{1}{2(\sigma^2 + h^2)^{1/2}\sqrt\pi}) + \frac{1}{2(\sigma^2 + h^2)^{1/2}\sqrt\pi} - \frac{\sqrt 2}{\sqrt\pi (2\sigma^2 + h^2)^{1/2}} + \frac{1}{2\sigma\sqrt\pi} \\
	&= \frac{1}{2\sqrt\pi} \qty(\frac{1}{nh} + (1 - \frac1n) \frac{1}{(\sigma^2 + h^2)^{1/2}} - \frac{2^{3/2}}{(2\sigma^2 + h^2)^{1/2}} + \frac1\sigma). 
\end{align*}
\end{proof}

\begin{question}
	Use the expression from 10(b) to derive an upper bound of the form $\MISE \hat f_n \leq C_1/(nh) + C_2^2 h^4$ (where you should specify $C_1, C_2$). Show that $\phi_\sigma \in \FF_\Nn(2, L)$ with $L^2 = 3/(8 \pi^{1/2} \sigma^5)$, and hence compare the bound from the first part of this question with that obtained from the general theory from lectures. 
\end{question}

\begin{solution}
	We have 
	\[
	\qty(1 - \frac1n) \frac{1}{(\sigma^2 + h^2)^{1/2}} - \frac{2^{3/2}}{(2\sigma^2 + h^2)^{1/2}} \leq \frac{1}{(\sigma^2 + h^2)^{1/2}}  - \frac{2}{(\sigma^2 + h^2/2)^{1/2}} < 0,
	\]
%	and therefore
%	\[
%	\MISE \hat f_n < \frac{1}{2\sqrt\pi} \qty(\frac1{nh} + \frac1\sigma). 
%	\]
	I do not know how to obtain an upper bound of the form $C_1/(nh) + C_2^2 h_4$ from this expression. 
	
	To show that $\phi_\sigma \in \FF_\Nn(2, L)$,  we must show that $\phi_\sigma \in \FF_\Nn(2, L)$. By question 9, it suffices to show that $\norm{\phi''_\sigma}_{L^2}^2 \leq L^2$. 
	A simple computation gives
	\[
	\phi_\sigma''(x) = \frac{1}{\sqrt{2\pi} \sigma^5} \qty(x^2 - \sigma^2) \exp(-\frac{x^2}{2\sigma^2}) \leq \frac1{\sqrt{2\pi} \sigma^5} x^2 \exp(-\frac{x^2}{2\sigma^2})
	\]
	so
	\begin{align*}
	\norm{\phi_\sigma''}_{L^2}^2 &\leq \frac{1}{2\pi\sigma^{10}} \int_\RR x^4 \exp(-\frac{x^2}{\sigma^2}) \dd{x} \overset\star= \frac{1}{2\pi \sigma^{10}} \cdot \frac34 \sqrt{\pi} \sigma^5 = \frac{3}{8 \sqrt\pi \sigma^5} = L^2, 
	\end{align*}
where $\star$ can be computed using the fact that the integral is, up to scaling, the fourth moment of $N(0, \sqrt2\sigma)$ distribution. 

Note that for $K = \phi_1$, we have 
\[
R(K) = \int_{-\infty}^\infty \phi_1^2(x) \dd{x} = \frac1{2\sqrt\pi}, 
\]
while
\[
\mu_2^2(K)=  \int_{-\infty}^\infty x^2 \phi_1(x) \dd{x} = 1.
\]

Plugging all the above into theorem 27 shows that 
\[
\MISE(\hat f_n) \leq \frac{1}{2\sqrt\pi} \frac1{nh} + \frac{3}{8\sqrt\pi \sigma^5} h^4.
\]
\end{solution}

\end{document}