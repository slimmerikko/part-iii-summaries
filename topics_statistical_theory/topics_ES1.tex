\documentclass{article}

\usepackage[utf8]{inputenc}
\usepackage[english]{babel}
\usepackage[margin=3cm]{geometry}
\usepackage[normalem]{ulem}
\usepackage{hyperref}
\usepackage{mathtools, amsmath, amssymb, amsthm, enumerate, mdframed, bbm, graphicx, float, physics, xcolor, cleveref}

\hypersetup{
    colorlinks   = true, %Colours links instead of ugly boxes
    urlcolor     = blue, %Colour for external hyperlinks
    linkcolor    = blue, %Colour of internal links
    citecolor   = red %Colour of citations
}

% Definition of numbered environments.
% Usage: \begin{theorem} ... \end{theorem}
% Remark has no numbering.
\theoremstyle{plain}
\newtheorem{question}{Question}
\newtheorem{theorem}{Theorem}
\newtheorem{lemma}{Lemma}
\theoremstyle{remark}
\newtheorem*{remark}{Remark}

% Question with box around it
\newenvironment{qbox}{\begin{mdframed}\begin{question}}{\end{question}\end{mdframed}}

% A proof with "solution" instead of "proof" and no QED symbol
\newenvironment{solution}{\begin{proof}[Solution]\renewcommand\qedsymbol{}}{\end{proof}}

% Some renewed commands
\renewcommand{\vec}{\mathbf}
\renewcommand{\emptyset}{\varnothing}
\renewcommand{\epsilon}{\varepsilon}
\renewcommand{\theta}{\vartheta}
\renewcommand{\phi}{\varphi}

% Frequently used math alphabets
\newcommand{\Bb}{\mathbb}
\newcommand{\Cal}{\mathcal}
\newcommand{\Bf}{\mathbf}
\newcommand{\Rm}{\mathrm}

% Frequently used letters in the blackboard alphabet
\newcommand{\CC}{\Bb C}
\newcommand{\NN}{\Bb N}
\newcommand{\PP}{\Bb P}
\newcommand{\QQ}{\Bb Q}
\newcommand{\RR}{\Bb R}
\newcommand{\EE}{\Bb E}

% Usage: \ang{...} is equivalent to \langle ... \rangle, while \ang*{...} is equivalent to \left\langle ... \right\rangle
% For other delimiters: use \qty from the physics package (i.e., \qty(...))
\DeclarePairedDelimiter{\ang}{\langle}{\rangle}
\DeclarePairedDelimiter{\floor}{\lfloor}{\rfloor}
\DeclarePairedDelimiter{\ceil}{\lceil}{\rceil}

% Frequently used commands
\newcommand{\T}{^\top} % Matrix transpose A\T
\newcommand{\C}{^\complement} % Set complement A\C
\newcommand\ceq\coloneqq % Definitions :=
\newcommand\pow{\Cal P} % Power sets
\newcommand\eps\epsilon
\newcommand\ind{\mathbbm 1} % Blackboard 1 for indicator functions
\newcommand\restr{\mathord\restriction}
\newcommand\TODO{{\color{red} TODO: }}
\newcommand\toprob{\overset{\Rm{p}}{\to}}
\newcommand\todist{\overset{\Rm{d}}{\to}}
\newcommand\toas{\overset{\Rm{a.s.}}{\to}}
\newcommand\deq{\overset{\Rm{d}}=}
\newcommand\iid{\overset{\Rm{iid}}{\sim}}

% Functions that appear frequently
\DeclareMathOperator{\sign}{sign}
\DeclareMathOperator{\Int}{Int}
\DeclareMathOperator{\Span}{Span}
\DeclareMathOperator{\Var}{Var}
\DeclareMathOperator*{\argmin}{arg\,min}
\DeclareMathOperator*{\argmax}{arg\,max}
\DeclareMathOperator\Exp{Exp}
\DeclareMathOperator\Beta{Beta}
\DeclareMathOperator\Gam{Gamma}

\title{Topics in Statistical Theory --- Example Sheet 1} % subject
\author{Lucas Riedstra}
\date{...} % date

\begin{document}
\maketitle

\begin{question}
    Let $U_1, \dotsc, U_n \iid U(0, 1)$ and let $Y_1, \dotsc, Y_{n+1} \iid \Exp(1)$. Writing $S_j \ceq \sum_{i=1}^j Y_i$ for $j = 1, \dotsc, n+1$, show that
    \[
    U_{(j)} \deq \frac{S_j}{S_{n+1}} \sim \Beta(j, n - j + 1)
    \]
    for $j = 1, \dotsc, n$. 
\end{question}

\begin{solution}
%    First we compute the distribution function of $U_{(j)}$. If $U_{(j)} \leq x$, then at least $j$ of the $U_{i}$ must be $\leq x$. 
%
%For $k = j, j+1, \dotsc, n$, there are $\binom{n}{k}$ ways to choose $k$ of the $U_{i}$'s that must be $\leq j$. Fix $x \in [0, 1]$, then we have
%\begin{align*}
%    \PP(U_{(j)} \leq x) &= \sum_{k=j}^n \binom{n}{k} \PP(X_1 \leq x, \dotsc, X_k \leq x, X_{k+1} > x, \dotsc, X_n > x) \\
%    &= \sum_{k=j}^n \binom{n}{k} (F(x))^k \cdot (1 - F(x))^{n-k} \\
%    &=  \sum_{k=j}^n \binom{n}{k} x^k (1-x)^{n-k}.
%\end{align*}
%
%Now, the density function of $X_{(k)}$ is given by the derivative of $F$. This is a hefty calculation, see for example \href{https://math.stackexchange.com/a/1179378/696417}{this StackExchange link}, but it turns out to be
%\[
%\dv{x} F_{(j)}(x) = f_{(j)}(x) = \frac{n!}{(j-1)!(n-j)!} x^{j-1}(1-x)^{n-j},
%\]
%which is also the density function of the $\Rm{Beta}(j, n-j+1)$ distribution.

We compute the density function of $U_{(j)}$ as follows: let $x \in (0, 1)$, then we know that 
\[
f_{(j)}(x) = \dv{x} F_{(j)}(x) = \lim_{h\to 0} \frac{F_{(j)}(x+h) - F_{(j)}(x)}{h} = \lim_{h \to 0} \frac{\PP(x < U_{(j)} \leq x + h)}{h}. 
\]
The probability $\PP(x < U_{(j)} \leq x + h)$ is the probability that exactly $j-1$ of the $U_{i}$ are less than $x$, and that at least one of the $U_{i}$ is in $(x, x+ h]$. 

The probability that two or more of the $U_{i}$ lie in $(x, x+h]$ is $O(h^2)$ and therefore negligible, so we must compute the probability that exactly $j-1$ of the $U_{i}$ are smaller than $x$, one of the $U_{i}$ is in $(x, x+h]$, and the other $U_{i}$ are greater than $x+h$. This is easily seen to be
\begin{align*}
&\binom{n}{j-1} \PP(U \leq x)^{j-1} \cdot (n - j + 1) \PP(x < U \leq x + h) \cdot \PP(U > x + h)^{n-j} \\
&= \frac{n!}{(j-1)!(n-j+1)!} (n-j+1) x^{j-1} h (1 - x - h)^{n-j} \\
&= \frac{n!}{(j-1)!(n-j)!} x^{j-1} (1-x-h)^{n-j} h. 
\end{align*}
Therefore, we easily compute
\[
f_{(j)}(x) = \lim_{h \to 0} \frac{\frac{n!}{(j-1)!(n-j)!} x^{j-1} (1-x-h)^{n-j} h}{h} = \frac{n!}{(j-1)!(n-j)!} x^{j-1} (1-x)^{n-j}. 
\]
This is also the density function of a $\Beta(j, n-j+1)$ distribution. 

Finally, define $T = S_{n+1} - S_j$, so that $S_j$ and $T$ are independent. It is known that $S_j \sim \Gam(j, 1)$, $T \sim \gamma(n - j + 1)$, and furthermore that
\[
\frac{S_j}{S_{n+1}} = \frac{S_j}{S_j + T} \deq \frac{\Gamma(j, 1)}{\Gamma(j, 1) + \Gamma(n - j + 1, 1)} \sim \Beta(j, n - j + 1). 
\]
\end{solution}

\begin{question}
    Let $X$ be a random variable with mean zero that satisfies $a \leq X \leq b$. Use convexity to show that for every $t \in \RR$, we have
    \[
    \log \EE(e^{tX}) \leq - \alpha u + \log(\beta + \alpha e^u),
    \]
    where $u \ceq t(b-a)$ and $\alpha \ceq 1 - \beta = -a/(b-a)$. Using a second-order Taylor expansion around the origin, deduce that $\log \EE(e^{tX}) \leq t^2 (b-a)^2/8$. 
\end{question}

\begin{proof}
    Let $x \in [a, b]$, then we know there exists a unique $\lambda \in [0, 1]$ such that $x = (1-\lambda) a + \lambda b$. A simple computation gives $\lambda = (x-a)/(b-a)$, $1 - \lambda = (b-x)/(b-a)$. By convexity of $t \mapsto e^{tx}$ we find
    \[
    e^{tx} \leq \frac{b-x}{b-a} e^{t a} + \frac{x-a}{b-a} e^{t b}. 
    \]
    
    From this we deduce that 
    \[
    \EE[e^{tX}] \leq \EE\qty[\frac{b-X}{b-a} e^{t a} + \frac{X-a}{b-a} e^{t b}] = \frac{b}{b - a} e ^{ta} + \frac{-a}{b - a} e^{tb} = \beta e^{ta} + \alpha e^{tb} = e^{-\alpha u + \log(\beta + \alpha e^u)}. 
    \]
    Since $\log$ is increasing, we can take the logarithm on both sides to conclude
    \[
    \log \EE[e^{tX}] \leq -\alpha u + \log(\beta + \alpha e^u). 
    \]
    
    Now, we compute the taylor polynomial of $f(u) \ceq -\alpha u + \log(\beta + \alpha e^u)$ in $u = 0$: we have
    \begin{align*}
        f(0) &= \log(\beta + \alpha) = \log(1) = 0; \\
        f'(u) &= -\alpha + \frac{\alpha e^u}{\beta + \alpha e^u}; \\
        f'(0) &= -\alpha + \frac{\alpha}{\beta + \alpha} = 0; \\
        f''(u) &= \frac{(\beta + \alpha e^u)\alpha e^u - (\alpha e^u)^2}{(\beta + \alpha e^u)^2} = \frac{\alpha e^u}{(\beta + \alpha e^u)} \qty(1 - \frac{\alpha e^u}{(\beta + \alpha e^u)})
    \end{align*}

Note that $\frac{\alpha e^u}{\beta + \alpha e^u} \in [0, 1]$ since $\alpha, \beta \geq 0$ (this holds because $a$ must be negative and $b$ must be positive due to the condition $\EE X = 0$). For $y \in [0, 1]$, the polynomial $y(1-y)$ takes values in $[0, \frac14]$. Therefore, by Taylor's theorem with remainder, we conclude
\[
\log \EE[e^{tX}] \leq \qty(\sup_{u \in \RR} \frac{f''(u)}{2} )u^2 = \frac18 u^2 = \frac{t^2(b-a)^2}{8}. 
\]
\end{proof}

\begin{question}
    Let $X_1, \dotsc, X_n$ be independent with distribution $P$ on a measurable space $(\Cal X, \Cal A)$, and let $\hat P_n$ be the empirical measure of $X_1, \dotsc, X_n$; thus $\hat P_n(A) = n^{-1} \sum_{i=1}^n \ind_{U_{i} \in A}$. Show that, for all $\eps > 0$ and $A \in \Cal A$, we have
    \[
    \PP\qty(\abs{\hat P_n(A) - P(A)} > \eps) \leq 2 e^{-2n\eps^2}. 
    \]
\end{question}

\begin{proof}
    Define a new distribution $Y = \ind_{X \notin A}$. Its distribution function is given by
    \[
    F_Y(y) = \begin{cases}
        0 &\text{$y < 0$}; \\
        P(A) &\text{$y \in [0, 1)$}; \\
        1 &y \geq 1. 
    \end{cases}, \quad F_Y(0) = P(A). 
    \]
    The empirical distribution function of $Y_1, \dotsc, Y_n \iid Y$ is given by
    \[
    \hat F_n(y) = \frac1n \sum_{i=1}^n \ind_{Y_i \leq y},
    \]
    and thus for $y = 0$ we have
    \[
    \hat F_n(0) = \frac1n \sum_{i=1}^n \ind_{Y_i \leq 0} = \frac1n \sum_{i=1}^n \ind_{U_{i} \in A} = \hat P_n(A).
    \]
    
    The result now follows from the DKW inequality. \TODO make more precise. 
\end{proof}
\end{document}