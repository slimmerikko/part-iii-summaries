\section{Classical regularisation theory}
Let $A \in \BB(\XX, \YY)$ such that $\Ran(A)$ is not closed (this happens for example when $A$ is compact and does not have finite rank), and consider the inverse problem $Au = f$. Suppose we measure not $f$, but noisy data $f_\delta$ such that $\norm{f_\delta - f} \leq \delta$. Then since $A\D$ is discontinuous, we cannot expect that $A\D f_\delta \to A\D f$ as $\delta \to 0$. Therefore, we must replace $A\D$ by operators that approximate it.

\begin{definition}
	Let $A\in \BB(\XX, \YY)$. A family $(R_\alpha)_{\alpha > 0}$ of continuous operators is called a \emph{regularisation} of $A\D$ if
	\[
	\lim_{\alpha \to 0} R_\alpha f = A\D f \quad\text{for all $f \in \Dom(A\D)$}. 
	\]
	If all $R_\alpha$ are linear (\TODO and bounded?), then we speak of a \emph{linear regularisation} of $A\D$. 
\end{definition}

\begin{theorem}[Banach-Steinhaus]
	Let $\XX, \YY$ be Hilbert spaces and $\qty{A_\alpha} \subseteq \BB(\XX, \YY)$ a family of pointwise bounded operators. Then $\qty{A_\alpha}$ is bounded in norm. 
\end{theorem}

\begin{corollary} \label{cor:bounded_converge}
	Let $\XX, \YY$ be Hilbert spaces and $(A_j) \subseteq \BB(\XX, \YY)$. Then $(A_j)$ converges pointwise to some $A \in \BB(\XX, \YY)$ if and only if $\qty{A_j}$ is norm-bounded and converges pointwise on some dense subset $\XX' \subseteq \XX$. 
\end{corollary}

\begin{theorem}
	Let $\XX, \YY$ be Hilbert spaces, $A \in \BB(\XX, \YY)$ and $(R_\alpha)_{\alpha > 0}$ a linear regularisation. If $A\D$ is not continuous, $(R_\alpha)$ is not norm-bounded. In particular, there exists $f \in \YY$ with $\norm{R_\alpha f} \to \infty$. 
\end{theorem}

\begin{proof}
	Suppose $(R_\alpha)$ is norm-bounded. Let $\alpha_j \to 0$, then we know that $R_{\alpha_j} \to A\D$ pointwise on $\Dom(A\D)$. Since $\Dom(A\D)$ is dense in $\YY$, \cref{cor:bounded_converge} then  tells us that $A\D$ is bounded and therefore continuous, a contradiction. 
	
	By the Banach-Steinhaus theorem, if $(R_\alpha)$ is not norm-bounded, it is not pointwise bounded, so there must exist $f \in \YY$ such that $\qty{\norm{R_\alpha f}}$ is not bounded. 
\end{proof}

\begin{recap}
	Recall that any bounded sequence in a Hilbert space has a weakly convergent subsequence. 
\end{recap}
\begin{theorem}
	Let $A \in \BB(\XX, \YY)$ and $(R_\alpha)$ a linear regularisation of $A\D$. If $\qty{\norm{A R_\alpha}}_{\alpha > 0}$ is bounded, then $\norm{R_\alpha f} \to \infty$ as $\alpha \to 0$ for every $f \notin \Dom(A\D)$. 
\end{theorem}

\begin{proof}
	Define $u_\alpha \ceq R_\alpha f$ for $f \notin \Dom(A\D)$, and assume there exists a sequence $\alpha_k \to 0$ such that $\qty{\norm{u_{\alpha_k}}}$ is bounded. After taking a subsequence if necessary, we may assume that $u_{\alpha_k} \rightharpoonup u$ for some $u \in \XX$, and therefore we also have $Au_{\alpha_k} \rightharpoonup Au$. 
	
	We also have $\lim_{\alpha \to 0} A R_\alpha f = AA\D f = P_{\clos{\Ran(A)}}f$ for $f \in \Dom(A\D)$, and since we assumed $\qty{A R_\alpha}$ was norm-bounded, by \cref{cor:bounded_converge} we have $\lim_{\alpha \to 0} AR_\alpha f = P_{\clos{\Ran(A)}} f$ for all $f \in \YY$. 
	
	Since $Au_{\alpha_k}$ is convergent and has weak limit $Au$, it must also have limit $Au$, so we find $Au = P_{\clos{\Ran(A)}} f$ so $f \in \Dom(A\D)$, a contradiction. 
\end{proof}

We need some process to choose a parameter. To this end, note that we have
\begin{equation} \label{eq:error_terms}
\norm{R_\alpha f_\delta - A\D f} \leq \norm{R_\alpha (f_\delta - f)} + \norm{(R_\alpha - A\D) f} \\
\leq  \delta \norm{R_\alpha} +\norm{(R_\alpha - A\D) f}  . 
\end{equation}
The first term is called the \emph{data error} and is unbounded for $\alpha \to 0$, and the second term is called the \emph{approximation error} which does vanish for $\alpha \to 0$. Therefore, we want to choose $\alpha$ small enough to have a low approximation error, while keeping the data error at bay. 


\subsection{Parameter choice rules}
\begin{definition}
	A function $\alpha \colon  \RR_{> 0} \times \YY \to \RR_{> 0} \colon (\delta, f_\delta) \mapsto \alpha(\delta, f_\delta)$ is called a \emph{parameter choice rule} (PCR). We distinguish three types:
	\begin{enumerate}
		\item An \emph{a priori} PCR depends only on $\delta$;
		\item An \emph{a posteriori} PCR depends on both $\delta$ and $f_\delta$;
		\item A \emph{heuristic} PCR depends only on $f_\delta$. 
	\end{enumerate}
\end{definition}

\begin{definition}
	Let $(R_\alpha)_{\alpha > 0}$ be a regularisation of $A\D$ and $\alpha$ a parameter choice rule. We call $(R_\alpha, \alpha)$ a \emph{convergent regularisation} if
	\[
	\lim_{\delta \to 0} \sup_{f_\delta : \norm{f - f_\delta} \leq \delta} \norm{R_\alpha f_\delta - A\D f} = 0
	\]
	and
	\begin{equation} \label{eq:alpha_to_0}
	\lim_{\delta \to 0} \sup_{f_\delta : \norm{f - f_\delta} \leq \delta} \alpha(\delta, f_\delta) = 0. 
	\end{equation}
\end{definition}

\subsubsection{A priori parameter choice rules}

We will not prove the following theorem, which guarantees the existence of a priori PCRs: 
\begin{theorem}
	Let $(R_\alpha)_{\alpha > 0}$ be a regularisation of $A\D$. Then there exists an a priori PCR $\alpha = \alpha(\delta)$ such that $(R_\alpha, \alpha)$ is convergent. 
\end{theorem}

We can characterise PCRs in the following way:
\begin{theorem}
	Let $(R_\alpha)_{\alpha > 0}$ be a linear regularisation of $A\D$, and $\alpha = \alpha(\delta)$ an a priori PCR. Then $(R_\alpha, \alpha)$ is convergent if and only if
	\[
	\lim_{\delta \to 0} \delta \norm{R_{\alpha(\delta)}} = 0 \quad\text{and}\quad \lim_{\delta\to 0} \alpha(\delta) = 0. 
	\]
\end{theorem}

\begin{proof}
	``$\implies$'' Suppose $(R_\alpha, \alpha)$ is convergent. It is clear that $\lim_{\delta \to 0} \alpha(\delta) = 0$ by \cref{eq:alpha_to_0}. Suppose $\lim_{\delta\to 0} \delta \norm{R_{\alpha(\delta)}} \neq 0$. Then there exists a sequence $(\delta_k) \to 0$ and a constant $C > 0$ such that $\delta_k \norm{R_{\alpha(\delta_k)}} \geq C$ for all $k$. This implies we can find a sequence $(g_k) \subseteq \YY$ with $\norm{g_k} = 1$ and $\delta_k \norm{R_{\alpha(\delta_k)}g_k} \geq C$ for all $k$. 
	
	Now let $f \in \Dom(A\D)$ and define $f_k \ceq f + \delta_k g_k$, then clearly we have $f_k \to f$, but also
	\[
	C \leq \norm{R_{\alpha(\delta_k)}(\delta_k g_k)} = \norm{R_{\alpha(\delta_k)}(f_{\delta_k} - f)} \leq \norm{R_{\alpha(\delta_k)}f_{\delta_k} - A\D f} + \norm{(R_{\alpha(\delta_k)} - A\D) f}.
	\]
	In particular we find that $\norm{(R_{\alpha(\delta_k)} - A\D) f} \geq C$, so clearly $R_\alpha$ is not convergent. 
	
	``$\impliedby$'' This follows immediately from \cref{eq:error_terms}. 
\end{proof}

A problem with a priori PCRs is that they are scale-invariant: if $\alpha = \alpha(\delta)$ gives a convergent regularisation, then $\hat\alpha = \alpha(k\delta)$ also gives a convergent regularisation for any $k$. In practice, it is not always clear which scale should be chosen.

\subsubsection{A posteriori parameter choice rules}
Let $f \in \Dom(A\D)$ and $f_\delta$ s.t.\ $\norm{f- f_\delta} \leq \delta$. Letting $u\D$ denote the minimum-norm solution of the problem $Au = f$, and defining $\mu \ceq \norm{Au\D - f} = \inf_{u \in \XX} \norm{Au - f}$, we see that
\[
\norm{Au\D - f_\delta} \leq \norm{Au\D - f} + \norm{f - f_\delta} \leq \mu + \delta. 
\]
Therefore, it is not useful to choose $\alpha(\delta, f_\delta)$ with $\norm{Au_\alpha - f_\delta} < \mu + \delta$: if this is the case, we are most likely overfitting. 

This motivates \emph{Morozov's discrepancy principle}: 

\begin{definition}
	Let $(R_\alpha)$ be a (\TODO linear?) regularisation of $A\D$ and assume $\Ran(A)$ is dense in $\YY$. Fix $\eta > 1$, and define
	\[
	\alpha(\delta, f_\delta) = \sup\qty{\alpha > 0 : \norm{A R_{\alpha} f_\delta - f_\delta} \leq \eta\delta}.
	\]
	Then $\alpha(\delta, f_\delta)$ is said to satisfy \emph{Morozov's discrepancy principle}. 
\end{definition}

It can be shown that the above $\alpha$ indeed gives a convergent regularisation.

\subsubsection{Heuristic parameter choice rules}
Heuristic parameter choice rules unfortunately only work if the original problem was well-posed:
\begin{theorem}[Bakushinskii]
	Let $(R_\alpha)$ be a regularisation of $A\D$ and suppose there exists a heuristic parameter choice rule $\alpha$ such that $(R_\alpha, \alpha)$ is convergent. Then $A\D$ is continuous from $\YY$ to $\XX$. 
\end{theorem}

\subsection{Spectral regularisation}
We will now start with specific examples of regularisations. Spectral regularisations are derived from the spectral decomposition
\[
A\D f = \sum_{j=1}^\infty \sigma_j^{-1} \ang{f, y_j} x_j. 
\]
We construct a regularisation by replacing $\sigma_j^{-1}$ by some function $g_\alpha(\sigma_j)$, i.e.,
\begin{equation} \label{eq:spectral_reg}
R_\alpha f = \sum_{j=1}^\infty g_\alpha(\sigma_j) \ang{f, y_j} x_j. 
\end{equation}
Let us explore which conditions $g_\alpha$ must satisfy: 

\begin{theorem} \label{thm:spectral_reg}
	Let, for $\alpha > 0$, the function $g_\alpha \colon \RR_{>0} \to \RR_{>0}$ satisfy
	\begin{enumerate} \itemsep=0em
		\item $\lim_{\alpha \to 0} g_\alpha(\sigma) = \frac1\sigma$ for all $\sigma > 0$;
		\item $g_\alpha(\sigma) \leq C_\alpha$ for some $C_\alpha > 0$; 
		\item $\sup_{\alpha, \sigma}  \sigma g_\alpha(\sigma) \leq \gamma$ for some $\gamma > 0$. 
	\end{enumerate}
	Then collection $(R_\alpha)$ defined by \cref{eq:spectral_reg} is a linear regularisation of $A\D$, and in particular, we have $\norm{R_\alpha} \leq C_\alpha$. 
\end{theorem}

\begin{proof}
	From condition 2 it follows that all $R_\alpha$ are bounded. Since 
	\[\ang{f, y_j} = \ang{P_{\clos{\Ran(A)}} f, y_j} = \ang{AA\D f, y_j} = \ang{A\D f, A^* y_j} = \sigma_j \ang{u\D, x_j},\]
	we compute
	\[
	(R_\alpha - A\D) f = \sum_j (g_\alpha(\sigma_j) - \sigma_j)^{-1} \ang{f, y_j} x_j = \sum_j (\sigma_j g_\alpha(\sigma_j) - 1) \ang{u\D, x_j} x_j, 
	\]
	and since $\sigma g_\alpha(\sigma_j) \leq \gamma$, we have $(\sigma_j g_\alpha(\sigma_j) - 1)^2 \leq 1 + \gamma^2$, so that 
	\[
	\norm{(R_\alpha - A\D) f}^2 \leq (1 + \gamma^2) \norm{u\D}^2 < \infty. 
	\]
	
	Since $\norm{(R_\alpha - A\D) f}$ is finite, we may apply the reverse Fatou lemma to the sum and obtain
	\[
	\limsup_{\alpha \to 0} \norm{(R_\alpha - A\D)f}^2 \leq \sum_j \qty(\sigma_j \limsup_{\alpha \to 0} g_\alpha(\sigma_j) - 1)^2 \ang{u\D, x_j}^2 = 0,
	\]
	and therefore $R_\alpha f \to A\D f$ as $\alpha \to 0$. 
\end{proof}

\begin{example}
	The first, very simple example is the \emph{truncated SVD}: we simply define
	\[
	g_\alpha(\sigma) = \begin{cases} 1/\sigma &\sigma \geq \alpha, \\ 0 &\sigma < \alpha. \end{cases}
	\]
	It is easy to check that $g_\alpha$ satisfies the conditions of \cref{thm:spectral_reg}, and that all $R_\alpha$ are finite-rank operators with $\norm{R_\alpha} \leq \frac1\alpha$. Therefore, if we choose $\alpha = \alpha(\delta)$ such that $\delta/\alpha(\delta) \to 0$, then we obtain a convergent regularisation. 
	
	This also highlights the problem with this method: as $\delta$ gets smaller, we need more and more singular vectors which are generally expensive to compute. 
\end{example}

\begin{example}
	The second example is \emph{Tikhonov regularisation}. Here, we define $g_\alpha(\sigma) = \frac{\sigma}{\sigma^2 + \alpha}$, and again it is easily checked that the conditions of \cref{thm:spectral_reg} are satisfied, noting that
	\[
	\frac{\sigma}{\sigma^2 + \alpha} \leq \frac{\sigma}{2 \sigma \sqrt{\alpha}} = \frac{1}{2\sqrt\alpha} \eqqcolon C_\alpha. 
	\]
	Therefore, if $\delta/\sqrt{\alpha(\delta)} \to 0$, the regularisation is convergent. 
	
	This method does not require computing the SVD of $A$: it is easily shown that $u_\alpha \ceq R_\alpha f$ is the unique solution to the \emph{regularised normal equation}
	\[
	(A^*A + \alpha I) u_\alpha = A^* f. 	
	\]
	
	While $A^* A + \alpha I$ is always invertible, computing the inverse is expensive, so we usually use some approximation of the inverse.
	
	
	Finally, it can also be shown that 
	\[
	u_\alpha = \min_{u \in \XX} \norm{A u - f}^2 + \alpha \norm{u}^2,
	\]
	so we can also view $u_\alpha$ as the solution of an optimisation problem. 
\end{example}