\documentclass{article}

\usepackage[utf8]{inputenc}
\usepackage[english]{babel}
\usepackage[margin=3cm]{geometry}
\usepackage[normalem]{ulem}
\usepackage{hyperref}
\usepackage[shortlabels]{enumitem}
\usepackage{mathtools, amsmath, amssymb, amsthm, mdframed, bbm, graphicx, float, physics, xcolor, cleveref}

\hypersetup{
    colorlinks   = true, %Colours links instead of ugly boxes
    urlcolor     = blue, %Colour for external hyperlinks
    linkcolor    = blue, %Colour of internal links
    citecolor   = red %Colour of citations
}

% Definition of numbered environments.
% Usage: \begin{theorem} ... \end{theorem}
% Remark has no numbering.
\theoremstyle{plain}
\newtheorem{question}{Question}
\newtheorem{theorem}{Theorem}
\newtheorem{lemma}{Lemma}
\theoremstyle{remark}
\newtheorem*{remark}{Remark}

% Question with box around it
\newenvironment{qbox}{\begin{mdframed}\begin{question}}{\end{question}\end{mdframed}}

% A proof with "solution" instead of "proof" and no QED symbol
\newenvironment{solution}{\begin{proof}[Solution]\renewcommand\qedsymbol{}}{\end{proof}}

% Some renewed commands
\renewcommand{\vec}{\mathbf}
\renewcommand{\emptyset}{\varnothing}
\renewcommand{\epsilon}{\varepsilon}
\renewcommand{\theta}{\vartheta}
\renewcommand{\phi}{\varphi}

% Frequently used math alphabets
\newcommand{\Bb}{\mathbb}
\newcommand{\Cal}{\mathcal}
\newcommand{\Bf}{\mathbf}
\newcommand{\Rm}{\mathrm}

% Frequently used letters in the blackboard alphabet
\newcommand{\CC}{\Bb C}
\newcommand{\NN}{\Bb N}
\newcommand{\PP}{\Bb P}
\newcommand{\QQ}{\Bb Q}
\newcommand{\RR}{\Bb R}
\newcommand{\EE}{\Bb E}
\newcommand\XX{\Cal X}
\newcommand\YY{\Cal Y}
\newcommand\UU{\Cal U}
\newcommand\VV{\Cal V}
\newcommand\LL{\Cal L}
\newcommand\KK{\Cal K}
\newcommand\BB{\Cal B}
\newcommand\JJ{\Cal J}
\newcommand\LLL{\Bb L}
\newcommand\DD{\Cal D}

% Usage: \ang{...} is equivalent to \langle ... \rangle, while \ang*{...} is equivalent to \left\langle ... \right\rangle
% For other delimiters: use \qty from the physics package (i.e., \qty(...))
\DeclarePairedDelimiter{\ang}{\langle}{\rangle}
\DeclarePairedDelimiter{\floor}{\lfloor}{\rfloor}
\DeclarePairedDelimiter{\ceil}{\lceil}{\rceil}

% Frequently used commands
\newcommand{\T}{^\top} % Matrix transpose A\T
\newcommand{\C}{^\complement} % Set complement A\C
\renewcommand{\P}{^\perp}
\newcommand{\D}{^\dagger}
\newcommand\ceq\coloneqq % Definitions :=
\newcommand\pow{\Cal P} % Power sets
\newcommand\eps\epsilon
\newcommand\ind{\mathbbm 1} % Blackboard 1 for indicator functions
\newcommand\restr{\mathord\restriction}
\newcommand\TODO{{\color{red} TODO: }}
\newcommand\clos\overline

% Functions that appear frequently
\DeclareMathOperator{\sign}{sign}
\DeclareMathOperator{\Int}{Int}
\DeclareMathOperator{\Span}{Span}
\DeclareMathOperator{\Var}{Var}
\DeclareMathOperator*{\argmin}{arg\,min}
\DeclareMathOperator*{\argmax}{arg\,max}
\DeclareMathOperator\Dom{\Cal D}
\DeclareMathOperator\Nul{\Cal N}
\DeclareMathOperator\Ran{\Cal R}
\DeclareMathOperator\TV{TV}
\DeclareMathOperator\BV{BV}

\newcommand\cw\rightharpoonup
\newcommand\cws{\overset{*}{\rightharpoonup}}
\newcommand\rext{\overline\RR}
\newcommand\pt\partial
\DeclareMathOperator\dom{dom}

\title{Inverse Problems --- Example Sheet 2} % subject
\author{Lucas Riedstra}
\date{16 November 2020} % date

\begin{document}
\maketitle

\begin{question}
	Let $\UU$ be a Banach space and $J \colon \UU\to \rext$ a functional. We define the \emph{subdifferential} of $J$ at any $v \in \UU$ as
	\begin{equation} \nonumber %\label{def:subdifferentiable}
	\pt\JJ(v) \ceq \qty{p \in \UU^* \mid J(u) \geq J(v) + \ang{p, u - v} \text{ for all $u \in \UU$}}. 
	\end{equation}
	Characterise the subdifferential for the
	\begin{enumerate}[(a)]
		\item absolute value function: $\UU= \RR$, $J(v) = \abs{v}$,
		\item $\ell^1$-norm: $\UU = \ell^2$,
		\[
		J(u) = \norm{u}_{\ell^1} \ceq \begin{cases}
			\sum_{j=1}^\infty \abs{u_j} &\text{if $u \in \ell^1$}; \\ \infty &\text{else.}
		\end{cases}
		\]
		\item characteristic function of the unit ball in $\RR$: $\UU = \RR$, $J(u) = \chi_C(u)$, $C \ceq \qty{ u\in \RR : \abs{u} \leq 1}$. 
		\item Total Variation $\TV \colon L^1(\Omega) \to \rext$, where $\Omega \subset \RR^n$ is bounded Lipschitz
		\[
		\TV(u) = \sup_{\phi \in \DD} \ang{u, \div\phi}, \quad \DD = \qty{\phi \in \Cal C_0^\infty(\Omega; \RR^n) : \norm{\phi(x)}_2 \leq 1 \ \forall x \in \Omega}. 
		\]
	\end{enumerate}
\end{question}

\begin{solution}
	Note: the spaces $\UU$ in parts (a) to (c) are Hilbert spaces, which means we can identify $\UU^*$ with $\UU$ (since any functional in $\UU^*$ is of the form $\ang{u, \cdot}$ for some $u \in \UU$). 
\begin{enumerate}[(a)]
\item Let $v \in \RR$. We know that $\abs{\cdot}$ is differentiable at $v \neq 0$, so 
\[
v > 0 \implies \pt J(v) = \qty{1} \quad\text{and}\quad v < 0 \implies \pt J(v) = \qty{-1}. 
\]
For $v = 0$ we have
\begin{align*}
	p \in \pt J(v) &\iff \abs{u} \geq p \cdot u  \text{ for all $u \in \RR$} \\
	&\iff p \in [-1, 1],
\end{align*}
so $\pt J(0) = [-1, 1]$. 

\item Let $v \in \ell^2$. Firstly, if $v \notin \ell^1 = \dom(J)$, then we have $\pt J(v) = \emptyset$. Assume now that $v \in \ell_1 \cap \ell_2$. Then we have, for $p \in \ell^2$, that
\begin{align}
	p \in \pt J(v) &\iff \norm{u}_{\ell^1} \geq \norm{v}_{\ell^1} + \ang{p, u - v} &\text{for all $u \in \ell^2$}  \nonumber \\
	&\iff \norm{u}_{\ell^1} - \norm{v}_{\ell^1} - \ang{p, u- v} \geq 0 &\text{for all $u \in \ell^2$} \nonumber \\
	&\iff \sum_{j=1}^\infty \abs{u_i} - \abs{v_i} - p_i(u_i - v_i) \geq 0 &\text{for all $u \in \ell^2$} \label{subdiff1} \\
	&\overset\star\iff \abs{x} - \abs{v_i} - p_i(x - v_i) \geq 0 &\text{for all $i \in \NN$ and $x \in \RR$}. \label{subdiff2}  
\end{align}
We first prove the bi-implication $\star$. If \eqref{subdiff2} holds, it is clear that \eqref{subdiff1} holds. If \eqref{subdiff2} does not hold, then we can find $x, i$ such that $\abs{x} - \abs{v_i} - p_i(x - v_i) < 0$. By now letting $u = x e_i$ in \eqref{subdiff1} we find  that \eqref{subdiff1} does not hold. 

However, if we define $H(x) \ceq \abs{x}$,  we see that \cref{subdiff2} is equivalent to $p_i \in \pt H(v_i)$ for all $i$. Therefore, by (a) we have
\[
\pt J(v) = \qty{p \in \ell^2 \mid p_i = \sign(v_i) \text{ if $v_i \neq 0$ and } p_i \in [-1, 1] \text{ for all $i$}}. 
\]

\item Clearly, if $\abs{v} < 1$, then $\chi_C$ is differentiable with derivative 0 so $\pt J(v) = \qty{0}$. If $\abs{v} > 1$, then $v \notin \dom(J)$, and therefore $\pt J(v) = \emptyset$. 

Consider the point $v = 1$, then we have
\[
p \in \pt J(\chi_C) \iff \chi_C(u) \geq p \cdot (u - 1) \ \forall u.
\]
For $u > 1$, this equation is satisfied regardless of $p$. Therefore, the above equation is equivalent to
\[
p \cdot(u-1) \leq 0 \ \forall u \leq 1, 
\]
which is satisfied for all $p \geq 0$, so we conclude $\pt J(1) = [0, \infty)$. 
Analogously, we find $\pt J(-1) = (-\infty, 0]$. We conclude that
\[
\pt J(v) = \begin{cases}
	\emptyset &\text{if $\abs{v} > 1$}; \\
	(-\infty, 0] &\text{if $v = -1$}; \\
	\qty{0} &\text{if $v \in (-1, 1)$}; \\
	[0, \infty) &\text{if $v = 1$}. 
\end{cases}
\]

\item Let $f \in L^1(\Omega) \setminus \BV(\Omega)$, then clearly $\pt \TV(f) = \emptyset$. Now suppose $f \in \BV(\Omega)$.  It is known that the dual of $L^1(\Omega)$ is $L^\infty(\Omega)$. Therefore, we have for $p \in L^\infty(\Omega)$ that 
\begin{align*}
	p \in \pt \TV(f) &\iff \TV(g) \geq \TV(f) + \int_{\Omega} p(x) (g - f)(x) \dd{x} \ \forall g \in L^1(\Omega)
\end{align*}
I do not know how to continue from here. 
\end{enumerate}
\end{solution}

\begin{question}
	Recall that a function $J \colon \UU\to \rext$ is called absolutely one-homogeneous if $J(\lambda u) = \abs{\lambda} J(u)$ for all $\lambda \in \RR, u \in \UU$. Let $J$ be convex, proper, l.s.c.\ and absolutely one-homogeneous. 
	\begin{enumerate}[(a)]
		\item Show that $p \in \pt J(v)$ if and only if $p \in \pt J(0)$ and $J(v) = \ang{p, v}$.
		Therefore, 
		\[
		D_J^p(u, v) = J(u) - \ang{p, u}. 
		\]
		Show that
		\[
		\pt J(0) = \bigcup_{u \in \UU} \pt J(u). 
		\]
		
		\item Show that the Bregman distances associated with absolutely one-homogeneous functionals fulfill the triangle inequality in the first argument, i.e., for all $u, v, w \in \UU$ and $p \in \pt J(w)$ there is
		\[
		D_J^p(u + v, w) \leq D_J^p(u, w) + D_J^p(v, w). 
		\]
		
		\item Show that the convex conjugate $J^*(\cdot)$ is the characteristic function of the convex set $\pt J(0)$. Compare this to the results of Exercise 2(a)(i). 
	\end{enumerate}
\end{question}

\begin{proof}
	It is clear that $J(0) = 0$. 
	\begin{enumerate}[(a)]
		\item Suppose $p \in \pt J(v)$. Then we have $J(u) \geq J(v) + \ang{p, u - v}$ for all $u$, which we can rewrite as $J(u) - \ang{p, u} \geq J(v) - \ang{p, v}$. Plugging in $u = 0$ we obtain $J(v) - \ang{p, v} \leq 0$, but plugging in $u = 2v$ we obtain
		\[
		2 \qty(J(v) - \ang{p, v}) = J(2v) - \ang{p, 2v} \geq J(v) - \ang{p, v} \implies J(v) - \ang{p, v} \geq 0, 
		\]
		so we conclude $J(v) - \ang{p, v} = 0$ or $J(v) = \ang{p, v}$. 	
		This also implies that
		\[
		J(u) \geq \ang{p, u} \text{ for all $u$} \implies p \in \pt J(0). 		
		\]
		
		Conversely, if $p \in \pt J(0)$ and $J(v) = \ang{p, v}$, then for all $u$ we have
		\[
		J(u) \geq \ang{p, u} + (J(v) - \ang{p, v}) \implies p \in \pt J(v).
		\]
		This concludes the first claim.
		
		From this claim, it follows that $\pt J(u) \subseteq \pt J(0)$ for all $u \in \UU$, and therefore trivially $\pt J(0) = \cup_u \pt J(u)$. 
		
		\item Note that we have
		\[
		J(u+v) = 2 J\qty(\frac12u + \frac12 v) \leq 2\qty(\frac12 J(u) + \frac12J(v)) = J(u) + J(v), 
		\]
		and therefore
		\[
		D_J^p(u+v, w) = J(u+v) - \ang{p, u +v} \leq J(u) + J(v) - \ang{p, u} - \ang{p, v} = D_J^p(u, w) + D_J^p(v, w). 
		\]
		
		\item We can reason analogously to 2(a)(i): we have
		\[
		J^*(v) = \sup_{u \in U} (\ang{v, u} - J(u)).
		\]
		Suppose that $v \notin \pt J(0)$, i.e., $\ang{v, u^*} - J(u) = \xi > 0$ for some $u^*$. Then we have for all $\lambda > 0$ that 
		\[
		\ang{v, \lambda u^*} - J(\lambda u^*) = \lambda \xi, 
		\]
		and letting $\lambda \to\infty$ shows $J^*(v) = \infty$. 
		
		On the other hand, suppose that $v \in \pt J(0)$, i.e., $\ang{v, u} - J(u) \leq 0$ for all $u$. Then the supremum is attained in $u = 0$ and therefore we have $J^*(v) = 0$. 
		
		It follows that $J^*(v) = \pt J(0)$, which is indeed also what we saw in 2(a)(i), since the subdifferential of the norm at 0 is exactly $\qty{v \in \UU^* : \norm{v}_{\UU^*} \leq 1}$. 
	\end{enumerate}
\end{proof}
\end{document}