\documentclass{article}

\usepackage[utf8]{inputenc}
\usepackage[english]{babel}
\usepackage[margin=3cm]{geometry}
\usepackage[normalem]{ulem}
\usepackage{hyperref}
\usepackage[shortlabels]{enumitem}
\usepackage{mathtools, amsmath, amssymb, amsthm, mdframed, bbm, graphicx, float, physics, xcolor, cleveref}

\hypersetup{
    colorlinks   = true, %Colours links instead of ugly boxes
    urlcolor     = blue, %Colour for external hyperlinks
    linkcolor    = blue, %Colour of internal links
    citecolor   = red %Colour of citations
}

% Definition of numbered environments.
% Usage: \begin{theorem} ... \end{theorem}
% Remark has no numbering.
\theoremstyle{plain}
\newtheorem{question}{Question}
\newtheorem{theorem}{Theorem}
\newtheorem{lemma}{Lemma}
\theoremstyle{remark}
\newtheorem*{remark}{Remark}

% Question with box around it
\newenvironment{qbox}{\begin{mdframed}\begin{question}}{\end{question}\end{mdframed}}

% A proof with "solution" instead of "proof" and no QED symbol
\newenvironment{solution}{\begin{proof}[Solution]\renewcommand\qedsymbol{}}{\end{proof}}

% Some renewed commands
\renewcommand{\vec}{\mathbf}
\renewcommand{\emptyset}{\varnothing}
\renewcommand{\epsilon}{\varepsilon}
\renewcommand{\theta}{\vartheta}
\renewcommand{\phi}{\varphi}

% Frequently used math alphabets
\newcommand{\Bb}{\mathbb}
\newcommand{\Cal}{\mathcal}
\newcommand{\Bf}{\mathbf}
\newcommand{\Rm}{\mathrm}

% Frequently used letters in the blackboard alphabet
\newcommand{\CC}{\Bb C}
\newcommand{\NN}{\Bb N}
\newcommand{\PP}{\Bb P}
\newcommand{\QQ}{\Bb Q}
\newcommand{\RR}{\Bb R}
\newcommand{\EE}{\Bb E}
\newcommand\XX{\Cal X}
\newcommand\YY{\Cal Y}
\newcommand\UU{\Cal U}
\newcommand\VV{\Cal V}
\newcommand\LL{\Cal L}
\newcommand\KK{\Cal K}
\newcommand\BB{\Cal B}
\newcommand\JJ{\Cal J}
\newcommand\LLL{\Bb L}
\newcommand\DD{\Cal D}

% Usage: \ang{...} is equivalent to \langle ... \rangle, while \ang*{...} is equivalent to \left\langle ... \right\rangle
% For other delimiters: use \qty from the physics package (i.e., \qty(...))
\DeclarePairedDelimiter{\ang}{\langle}{\rangle}
\DeclarePairedDelimiter{\floor}{\lfloor}{\rfloor}
\DeclarePairedDelimiter{\ceil}{\lceil}{\rceil}

% Frequently used commands
\newcommand{\T}{^\top} % Matrix transpose A\T
\newcommand{\C}{^\complement} % Set complement A\C
\renewcommand{\P}{^\perp}
\newcommand{\D}{^\dagger}
\newcommand\ceq\coloneqq % Definitions :=
\newcommand\pow{\Cal P} % Power sets
\newcommand\eps\epsilon
\newcommand\ind{\mathbbm 1} % Blackboard 1 for indicator functions
\newcommand\restr{\mathord\restriction}
\newcommand\TODO{{\color{red} TODO: }}
\newcommand\clos\overline

% Functions that appear frequently
\DeclareMathOperator{\sign}{sign}
\DeclareMathOperator{\Int}{Int}
\DeclareMathOperator{\Span}{Span}
\DeclareMathOperator{\Var}{Var}
\DeclareMathOperator*{\argmin}{arg\,min}
\DeclareMathOperator*{\argmax}{arg\,max}
\DeclareMathOperator\Dom{\Cal D}
\DeclareMathOperator\Nul{\Cal N}
\DeclareMathOperator\Ran{\Cal R}
\DeclareMathOperator\TV{TV}
\DeclareMathOperator\BV{BV}

\newcommand\cw\rightharpoonup
\newcommand\cws{\overset{*}{\rightharpoonup}}
\newcommand\rext{\overline\RR}
\newcommand\pt\partial
\DeclareMathOperator\dom{dom}

\title{Inverse Problems --- Example Sheet 2} % subject
\author{Lucas Riedstra}
\date{16 November 2020} % date

\begin{document}
\maketitle

\begin{question}
	Let $\UU$ be a Banach space and $J \colon \UU\to \rext$ a functional. We define the \emph{subdifferential} of $J$ at any $v \in \UU$ as
	\begin{equation} \nonumber %\label{def:subdifferentiable}
	\pt\JJ(v) \ceq \qty{p \in \UU^* \mid J(u) \geq J(v) + \ang{p, u - v} \text{ for all $u \in \UU$}}. 
	\end{equation}
	Characterise the subdifferential for the
	\begin{enumerate}[(a)]
		\item absolute value function: $\UU= \RR$, $J(v) = \abs{v}$,
		\item $\ell^1$-norm: $\UU = \ell^2$,
		\[
		J(u) = \norm{u}_{\ell^1} \ceq \begin{cases}
			\sum_{j=1}^\infty \abs{u_j} &\text{if $u \in \ell^1$}; \\ \infty &\text{else.}
		\end{cases}
		\]
		\item characteristic function of the unit ball in $\RR$: $\UU = \RR$, $J(u) = \chi_C(u)$, $C \ceq \qty{ u\in \RR : \abs{u} \leq 1}$. 
		\item Total Variation $\TV \colon L^1(\Omega) \to \rext$, where $\Omega \subset \RR^n$ is bounded Lipschitz
		\[
		\TV(u) = \sup_{\phi \in \DD} \ang{u, \div\phi}, \quad \DD = \qty{\phi \in \Cal C_0^\infty(\Omega; \RR^n) : \norm{\phi(x)}_2 \leq 1 \ \forall x \in \Omega}. 
		\]
	\end{enumerate}
\end{question}

\begin{solution}
	Note: the spaces $\UU$ in parts (a) to (c) are Hilbert spaces, which means we can identify $\UU^*$ with $\UU$ (since any functional in $\UU^*$ is of the form $\ang{u, \cdot}$ for some $u \in \UU$). 
\begin{enumerate}[(a)]
\item Let $v \in \RR$. We know that $\abs{\cdot}$ is differentiable at $v \neq 0$, so 
\[
v > 0 \implies \pt J(v) = \qty{1} \quad\text{and}\quad v < 0 \implies \pt J(v) = \qty{-1}. 
\]
For $v = 0$ we have
\begin{align*}
	p \in \pt J(v) &\iff \abs{u} \geq p \cdot u  \text{ for all $u \in \RR$} \\
	&\iff p \in [-1, 1],
\end{align*}
so $\pt J(0) = [-1, 1]$. 

\item Let $v \in \ell^2$. Firstly, if $v \notin \ell^1 = \dom(J)$, then we have $\pt J(v) = \emptyset$. Assume now that $v \in \ell_1 \cap \ell_2$. Then we have, for $p \in \ell^2$, that
\begin{align}
	p \in \pt J(v) &\iff \norm{u}_{\ell^1} \geq \norm{v}_{\ell^1} + \ang{p, u - v} &\text{for all $u \in \ell^2$}  \nonumber \\
	&\iff \norm{u}_{\ell^1} - \norm{v}_{\ell^1} - \ang{p, u- v} \geq 0 &\text{for all $u \in \ell^2$} \nonumber \\
	&\iff \sum_{j=1}^\infty \abs{u_i} - \abs{v_i} - p_i(u_i - v_i) \geq 0 &\text{for all $u \in \ell^2$} \label{subdiff1} \\
	&\overset\star\iff \abs{x} - \abs{v_i} - p_i(x - v_i) \geq 0 &\text{for all $i \in \NN$ and $x \in \RR$}. \label{subdiff2}  
\end{align}
We first prove the bi-implication $\star$. If \eqref{subdiff2} holds, it is clear that \eqref{subdiff1} holds. If \eqref{subdiff2} does not hold, then we can find $x, i$ such that $\abs{x} - \abs{v_i} - p_i(x - v_i) < 0$. By now letting $u = x e_i$ in \eqref{subdiff1} we find  that \eqref{subdiff1} does not hold. 

However, if we define $H(x) \ceq \abs{x}$,  we see that \cref{subdiff2} is equivalent to $p_i \in \pt H(v_i)$ for all $i$. Therefore, by (a) we have
\[
\pt J(v) = \qty{p \in \ell^2 \mid p_i = \sign(v_i) \text{ if $v_i \neq 0$ and } p_i \in [-1, 1] \text{ for all $i$}}. 
\]
From this it follows that
\[
\pt J(v) \neq \emptyset \iff v \text{ has finitely many non-zero entries.}
\]

\item Clearly, if $\abs{v} < 1$, then $\chi_C$ is differentiable with derivative 0 so $\pt J(v) = \qty{0}$. If $\abs{v} > 1$, then $v \notin \dom(J)$, and therefore $\pt J(v) = \emptyset$. 

Consider the point $v = 1$, then we have
\[
p \in \pt J(\chi_C) \iff \chi_C(u) \geq p \cdot (u - 1) \ \forall u.
\]
For $u > 1$, this equation is satisfied regardless of $p$. Therefore, the above equation is equivalent to
\[
p \cdot(u-1) \leq 0 \ \forall u \leq 1, 
\]
which is satisfied for all $p \geq 0$, so we conclude $\pt J(1) = [0, \infty)$. 
Analogously, we find $\pt J(-1) = (-\infty, 0]$. We conclude that
\[
\pt J(v) = \begin{cases}
	\emptyset &\text{if $\abs{v} > 1$}; \\
	(-\infty, 0] &\text{if $v = -1$}; \\
	\qty{0} &\text{if $v \in (-1, 1)$}; \\
	[0, \infty) &\text{if $v = 1$}. 
\end{cases}
\]

\item Let $f \in L^1(\Omega) \setminus \BV(\Omega)$, then clearly $\pt \TV(f) = \emptyset$. Now suppose $f \in \BV(\Omega)$.  It is known that the dual of $L^1(\Omega)$ is $L^\infty(\Omega)$. Therefore, we have for $p \in L^\infty(\Omega)$ that 
\begin{align*}
	p \in \pt \TV(f) &\iff \TV(g) \geq \TV(f) + \int_{\Omega} p(x) (g - f)(x) \dd{x} \ \forall g \in L^1(\Omega)
\end{align*}
I do not know how to continue from here. 
\end{enumerate}
\end{solution}

\begin{question}
	Let $\UU$ be a Banach space and let $E \colon \UU\to \rext$ be proper, lower semi-continuous and convex. Then the \emph{Fenchel conjugate} or \emph{convex conjugate} of $E$ is defined to be the mapping $E^* \colon \UU^* \to \RR$ with
	\[
	E^*(v) \ceq \sup_{u \in \UU} \qty{\ang{v, u} - E(u)}. 
	\]
	\begin{enumerate}[(a)]
		\item Compute the convex conjugates of the following functionals.
		\begin{enumerate}[(i)]
			\item $E(u) = \norm{u}_\UU$ for a Banach space $\UU$,
			\item $E(u \mid f) = \sum_{i=1}^n u_i \log(\frac{u_i}{f_i})$, where $f \in \RR^n_{>0}$ is a positive vector and $u \in \RR^n$. What is the effective domain of $E$? (here we define $\log(x) = -\infty$ for $x < 0$). 
		\end{enumerate}
	\item Let $E \colon \UU\to\rext$ be a proper, lower semi-continuous and convex functional. Show that
	\[
	p \in \pt E(u) \iff u \in \pt E^*(p)
	\]
	for all $u, p \in \UU$. 
	
	\emph{Hint:} You may exploit the fact that under the stated assumptions $E = E^{**}$ holds true. 
	\end{enumerate}
\end{question}

\begin{solution}
	\begin{enumerate}[(a)]
		\item \begin{enumerate}[(i)]
			\item We have
			\[
			E^*(v) = \sup_{u \in \UU} \qty(\ang{v, u} - \norm{u}). 
			\]
			Suppose $\ang{v, u^*} - \norm{u^*} = \xi > 0$ for some $u^*$. Then we have for $\alpha > 0$ that
			\[
			\ang{v, \alpha u^*} - \norm{\alpha u^*} = \alpha \qty(\ang{v, u^*} - \norm{u^*}) = \alpha\xi, 
			\]
			and therefore clearly $E^*(v) = \infty$. 
			
			On the other hand, if $\ang{v, u} - \norm{u}\leq0$ for all $u$, then the supremum is attained in $u = 0$ with value $0$, and therefore $E^*(v) = 0$. 
			
			We see that
			\begin{align*}
				&\ang{v, u^*} - \norm{u^*} > 0 &\text{for some $u^* \in \UU$}; \\
				&\iff \ang{v, u^*} > \norm{u^*} &\text{for some $u^* \in \UU$}; \\
				&\iff \norm{v}_{\UU^*} > 1. 
			\end{align*}
		We conclude that
		\[
		E^*(v) = \chi_\qty{\norm{v} \leq 1} = \begin{cases}
			0 &\text{if $\norm{v} \leq 1$}, \\ \infty &\text{else.}
		\end{cases}
		\]
		
		\item Suppose first that $p \in \pt E(u)$. Then we have
		\begin{align*}
&\phantom{\implies}	p \in \pt E(u) \\
&\implies E(v) \geq E(u) + \ang{p, v - u} &\text{for all $v$} \\
&\implies \ang{p, u} - E(u) \geq \ang{p, v} - E(v) &\text{for all $v$} \\
&\implies \ang{p, u} - E(u) \geq \sup_v \qty(\ang{p, v} - E(v)) = E^*(p) \\
&\implies \ang{p, u} - E(u) + \ang{q - p, u} \geq E^*(p) + \ang{q - p, u} &\text{for all $q$} \\
&\implies \ang{q, u} - E(u) \geq E^*(p) + \ang{q - p, u} &\text{for all $q$} \\
&\implies \sup_v \qty(\ang{q, v} - E(v)) \geq E^*(p) + \ang{q - p, u} &\text{for all $q$} \\
&\implies E^*(q) \geq E^*(p) + \ang{q -p, u} &\text{for all $q$} \\
&\implies u \in \pt E^*(p). 
		\end{align*}
	Now, for the reverse implication, note that by what we just proved we have
	\[
	u \in \pt E^*(p) \implies p \in \pt E^{**}(u) \iff p \in \pt E(u), 
	\]
	which proves the claim.
		\end{enumerate}
	\end{enumerate}
\end{solution}

\begin{question}
	Let $u, v \in \UU$ and $p \in \pt J(v)$. Recall that the Bregman distance of $J$ at $u, v$ is defined as
	\[
	D_J^p(u, v) \ceq J(u) - J(v) - \ang{p, u - v}.
	\]
	\begin{enumerate}[(a)]
		\item Show that Bregman distances are non-negative. 
		\item Show that Bregman distances may not be symmetric, i.e., there exists a $J$ and $u, v \in \UU$ with $p \in \pt J(v), q \in \pt J(u)$ such that $D_J^p(u, v) \neq D_J^q(v, u)$.
		\item Show that a vanishing Bregman distance may not imply that the two arguments are the same. What if $J$ is strictly convex?
	\end{enumerate}

\end{question}

\begin{proof}
	\begin{enumerate}[(a)]
		\item Since $p \in \pt J(v)$, we have $J(u) \geq J(v) + \ang{p, u - v}$, or equivalently $J(u) - J(v) - \ang{p, u- v} \geq 0$. 
		
		\item Let $\UU = \RR$, $J(x) = \abs{x}$, and choose $u = 0, v = 1$ and $p = 1, q = 0$. Then 
		\[
		D^p_J(u, v) = -1 - (-1) =  0 \quad\text{and}\quad D^q_J(v, u) = 1. 
		\]
		
		\item In the previous part we had an example $J(x)= \abs{x}, u = 0, v = 1, p = 1$, where $D_J^p(u, v) = 0$ while $u \neq v$. 
		
		Suppose that $J$ is strictly convex and that $u \neq v$ but $D_J^p(u, v) = 0$, so $J(u) = J(v) + \ang{p, u-v}$. Then we have for all $t \in (0, 1)$ that 
		\begin{align*}
			J(v) + \ang{p, (1-t)(u-v)} &= J(v) + \ang{p,  \qty(tv + (1-t)u) -v} \\
			&\overset\star\leq J(tv + (1-t) u) \\
			&< t J(v) + (1-t) J(u) \\
			&= t J(v) + (1-t) \qty(J(v) + (1-t)\ang{p, u-v}) \\
			&= J(v) + \ang{p, (1-t)(u-v)}, 
		\end{align*}
	a contradiction (here $\star$ follows from $p \in \pt J(v)$). We conclude that, if $J$ is strictly convex, the Bregman distance does satisfy $u \neq v \implies D_J^p(u, v) > 0$. 
		
	\end{enumerate}
\end{proof}

\begin{question}
	Recall that a function $J \colon \UU\to \rext$ is called absolutely one-homogeneous if $J(\lambda u) = \abs{\lambda} J(u)$ for all $\lambda \in \RR, u \in \UU$. Let $J$ be convex, proper, l.s.c.\ and absolutely one-homogeneous. 
	\begin{enumerate}[(a)]
		\item Show that $p \in \pt J(v)$ if and only if $p \in \pt J(0)$ and $J(v) = \ang{p, v}$.
		Therefore, 
		\[
		D_J^p(u, v) = J(u) - \ang{p, u}. 
		\]
		Show that
		\[
		\pt J(0) = \bigcup_{u \in \UU} \pt J(u). 
		\]
		
		\item Show that the Bregman distances associated with absolutely one-homogeneous functionals fulfill the triangle inequality in the first argument, i.e., for all $u, v, w \in \UU$ and $p \in \pt J(w)$ there is
		\[
		D_J^p(u + v, w) \leq D_J^p(u, w) + D_J^p(v, w). 
		\]
		
		\item Show that the convex conjugate $J^*(\cdot)$ is the characteristic function of the convex set $\pt J(0)$. Compare this to the results of Exercise 2(a)(i). 
	\end{enumerate}
\end{question}

\begin{proof}
	It is clear that $J(0) = 0$. 
	\begin{enumerate}[(a)]
		\item Suppose $p \in \pt J(v)$. Then we have $J(u) \geq J(v) + \ang{p, u - v}$ for all $u$, which we can rewrite as $J(u) - \ang{p, u} \geq J(v) - \ang{p, v}$. Plugging in $u = 0$ we obtain $J(v) - \ang{p, v} \leq 0$, but plugging in $u = 2v$ we obtain
		\[
		2 \qty(J(v) - \ang{p, v}) = J(2v) - \ang{p, 2v} \geq J(v) - \ang{p, v} \implies J(v) - \ang{p, v} \geq 0, 
		\]
		so we conclude $J(v) - \ang{p, v} = 0$ or $J(v) = \ang{p, v}$. 	
		This also implies that
		\[
		J(u) \geq \ang{p, u} \text{ for all $u$} \implies p \in \pt J(0). 		
		\]
		
		Conversely, if $p \in \pt J(0)$ and $J(v) = \ang{p, v}$, then for all $u$ we have
		\[
		J(u) \geq \ang{p, u} + (J(v) - \ang{p, v}) \implies p \in \pt J(v).
		\]
		This concludes the first claim.
		
		From this claim, it follows that $\pt J(u) \subseteq \pt J(0)$ for all $u \in \UU$, and therefore trivially $\pt J(0) = \cup_u \pt J(u)$. 
		
		\item Note that we have
		\[
		J(u+v) = 2 J\qty(\frac12u + \frac12 v) \leq 2\qty(\frac12 J(u) + \frac12J(v)) = J(u) + J(v), 
		\]
		and therefore
		\[
		D_J^p(u+v, w) = J(u+v) - \ang{p, u +v} \leq J(u) + J(v) - \ang{p, u} - \ang{p, v} = D_J^p(u, w) + D_J^p(v, w). 
		\]
		
		\item We can reason analogously to 2(a)(i): we have
		\[
		J^*(v) = \sup_{u \in U} (\ang{v, u} - J(u)).
		\]
		Suppose that $v \notin \pt J(0)$, i.e., $\ang{v, u^*} - J(u) = \xi > 0$ for some $u^*$. Then we have for all $\lambda > 0$ that 
		\[
		\ang{v, \lambda u^*} - J(\lambda u^*) = \lambda \xi, 
		\]
		and letting $\lambda \to\infty$ shows $J^*(v) = \infty$. 
		
		On the other hand, suppose that $v \in \pt J(0)$, i.e., $\ang{v, u} - J(u) \leq 0$ for all $u$. Then the supremum is attained in $u = 0$ and therefore we have $J^*(v) = 0$. 
		
		It follows that $J^*(v) = \pt J(0)$, which is indeed also what we saw in 2(a)(i), since the subdifferential of the norm at 0 is exactly $\qty{v \in \UU^* : \norm{v}_{\UU^*} \leq 1}$. 
	\end{enumerate}
\end{proof}
\end{document}