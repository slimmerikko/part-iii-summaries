\section{Variational regularisation}
\subsection{Background}
\subsubsection{Banach spaces and weak convergence}
A \emph{Banach space} $\XX$ is a complete normed vector space. We define the \emph{dual space} $\XX^* \ceq \LL(X, \RR)$, and for $p \in \XX^*, u \in \XX$ we usually write $\ang{p, u}$ instead of $p(u)$.  For any $A \in \LL(\XX, \YY)$ we define the \emph{adjoint} $A^* \colon \YY^* \to \XX^*$ by $\ang{A^*p, u} \ceq \ang{p, Au}$ for all $p \in \XX^*, u \in \XX$. The dual space $\XX'$ is equipped with the norm
\[
\norm{p}_{\XX^*} \ceq \sup_{\norm{u}\leq 1} \ang{p, u}, 
\]
and with this norm $\XX^*$ is a Banach space. 

The \emph{bi-dual space} is defined as $\XX^{**} \ceq (\XX^*)^*$. The mapping $E \colon \XX \to (\XX)^{**}$ defined by $\ang{E(u), p} \ceq \ang{p, u}$ is a continuous linear isometry, and we will regard $\XX$ as a subspace of $\XX^{**}$ using this isometry. If $\XX = \XX^{**}$ (i.e., $E$ is surjective), the space $\XX$ is called \emph{reflexive}.
A space $\XX$ is called \emph{separable} if $\XX$ has a countable dense subset. 

A sequence $(u_k) \subseteq \XX$ is said to \emph{converge weakly} to $u \in \XX$, denoted $u_k \cw u$, if $\ang{p, u_k} \to \ang{p, u}$ for all $p \in \XX^*$. 

A sequence $(p_k) \subseteq \XX^*$ is said to \emph{converge weakly-*} to $p \in \XX'$, denoted $p_k \cws p$, if $\ang{p_k, u} \to \ang{p, u}$ for all $u \in \XX$. 

\begin{theorem}
	Let $\XX$ be Banach, then the unit ball is compact in $\XX^*$ w.r.t.\ the weak-$*$ topology. If $\XX$ is separable, then the weak-$*$ topology is metrisable and every bounded sequence in $\XX^*$ has a weakly-$*$ convergent subsequence. 
\end{theorem}

\begin{theorem}
	Let $\XX$ be reflexive, then every bounded sequence in $\XX$ has a weakly convergent subsequence. 
\end{theorem}

We define $\rext \ceq \RR\cup \qty{\pm\infty}$. 

\begin{definition}
	Let $\XX$ be a Banach space with topology $\tau_X$. A functional $E \colon \XX \to\rext$ is said to be \emph{sequentially lower-semicontinuous with respect to $\tau_\XX$} or simply $\tau_\XX$-LSC if
	\[
	E(u) \leq \liminf_{n\to\infty} E(u_n)\quad\text{if $u_n \overset\tau\to u$}. 
	\]
	
	Specifically, if $\tau_\XX$ is the weak topology, then $E$ is called \emph{weakly} LSC. If $\tau_\XX$ is the topology induced by the norm on $\XX$, then $E$ is called \emph{strongly} LSC or simply LSC.
\end{definition}


\begin{lemma} \label{lem:lsc_closed_sets}
	Let $E \colon \XX \to \rext$ be $\tau_\XX$-LSC and $a \in \RR$. Then $\qty{u \in \XX \mid E(u) \leq a}$ is $\tau_\XX$-closed. 
\end{lemma}

\begin{proof}
	Let $(u_n) \overset\tau\to u$ and suppose $E(u_n) \leq a$ for all $n$. Then clearly $E(u) \leq \liminf E(u_n) \leq a$. 
\end{proof}

\subsubsection{Convex analysis}
\begin{definition}
	Let $C \subseteq \XX$. Then the \emph{characteristic function} of $C$ is defined as 
	\[
	\chi_C(u) \ceq \begin{cases} 0, &u \in C, \\ \infty,& u \notin C. \end{cases}
	\]
\end{definition}
Using characteristic functions, we have $\min_{u \in C} E(u) = \min_{u \in \XX} E(u) + \chi_C(u)$. 

\begin{definition}
	Let $E \colon \XX \to \rext$, then the \emph{effective domain} is $\dom(E) \ceq \qty{u \mid E(u) < \infty}$. 
	
	The functional $E$ is called \emph{proper} if $\dom(E) \neq\emptyset$. 
\end{definition}

\begin{definition}
	A functional $E \colon \XX \to \rext$ is called:
	\begin{enumerate}
		\item \emph{convex} if for all $u \neq v \in \XX$ and $\lambda \in (0, 1)$ we have $E(\lambda u + (1-\lambda) v) \leq \lambda E(u) + (1-\lambda) E(v)$;
		\item \emph{strictly convex} if the above inequality is strict;
		\item \emph{strongly convex} with constant $\theta > 0$ if $u \mapsto E(u) - \theta \norm{u}^2$ is convex. 
	\end{enumerate}

Note that $C \subseteq \XX$ is a convex set if and only if $\chi_C$ is a convex function. 
\end{definition}

\begin{lemma}
	Nonnegative linear combinations of convex functionals are convex. If one of the components is strictly convex, then the nonnegative linear combination is also strictly convex. 
\end{lemma}

\begin{definition}
	Let $E \colon \XX \to \rext$ be a functional. We define the \emph{Fenchel conjugate}
	\[
	E^* \colon \XX^* \to \rext \colon p \mapsto \sup_{u \in \XX} \qty[ \ang{p, u} - E(u)]. 
	\]
\end{definition}

\begin{theorem}
	For any $E \colon \XX \to \RR$ we have $E^{**} \restr_{\XX} \leq E$. If $E$ is proper and LSC, then $E^{**} \restr_{\XX} = E$. 
\end{theorem}

\begin{definition}
	A functional $E \colon \XX \to \rext$ is called \emph{subdifferentiable} at $u \in \XX$ if there exists a $p \in \XX^*$ such that
	\[
	E(v) \geq E(u) + \ang{p, v - u} \quad\text{for all $v \in \XX$}. 
	\]
	In this case, we call $p$ a \emph{subgradient} of $E$ at position $u$. The collection of all subgradients of $E$ at $u$ is denoted by $\pt E(u)$ and is called the \emph{subdifferential} of $E$ at $u$. 
\end{definition}

\begin{lemma}
	Let $E \colon \XX \to \rext$ be convex, then $E$ is subdifferentiable at all points $u \in \dom(E)$ where $E$ is continuous. If $E$ is also proper, then $E$ is not subdifferentiable at any $u \notin \dom(E)$. 
\end{lemma}

\begin{theorem}
	Let $E \colon \XX \to \rext$ be proper and convex and $u \in \dom(E)$. Then $\pt E(u)$ is convex and weakly-$*$ compact in $\XX^*$. 
\end{theorem}

\begin{theorem}
	Let $E, F$ be proper LSC convex functionals and $u \in \dom(E) \cap \dom(F)$ such that at least one of $E$ and $F$ is continuous at $u$. Then $\pt(E + F)(u) = \pt E(u) + \pt F(u)$. 
\end{theorem}

\begin{theorem}
	Let $E$ be convex. Then $u$ is a global minimiser of $E$ if and only if $0 \in \pt E(u)$. 
\end{theorem}


\begin{definition}
	Let $E$ be convex, $u, v \in \XX$, $E(v) < \infty$ and $q \in \pt E(v)$. Then the \emph{Bregman distance} of $E$ between $u$ and $v$ is defined as
	\[
	D_E^q(u, v) \ceq E(u) - E(v) - \ang{q, u -v} \geq 0. 
	\]
	
	If we also have $E(u) < \infty, p \in \pt E(u)$, then we define the \emph{symmetric Bregman distance}
	\[
	D_E^{p, q}(u, v) \ceq D_E^p(v, u) + D_E^q(u, v) = \ang{p - q, u - v}. 
	\]
\end{definition}

\begin{definition}
	For $p > 0$, a functional $E$ is called \emph{absolutely $p$-homogeneous} if $E(\lambda u) = \abs{\lambda}^p E(u)$ for all $\lambda \in \RR, u \in \XX$. 
\end{definition}

\begin{proposition}
	Let $E$ be a convex, proper and absolutely one-homogeneous, and $p \in \pt E(u)$. Then:
	\begin{enumerate}
		\item $E(u) = \ang{p, u}$;
		\item $D^p(v, u) = E(v) - \ang{p, v}$ for all $v \in \XX$;
		\item $E^*(p) = \chi_{\pt E(0)}(p)$.
	\end{enumerate}  
\end{proposition}

Furthermore, we have the following:
\begin{proposition}
	Let $E$ be proper, convex, and absolutely one-homogeneous, and let $u \in \XX$. Then $p \in \pt E(u)$ if and only if $p \in \pt E(0)$ and $\ang{p, u} = E(u)$. 
\end{proposition}

\subsubsection{Minimisers}
\begin{definition}
	We say that $u^* \in \XX$ is a \emph{minimiser} of a  functional $E$ if $u$ minimises $E$ and $E(u) < \infty$. 
\end{definition}

\begin{definition}
	A functional $E$ is called \emph{coercive} if $\norm{u_j} \to \infty \implies \abs{E(u_j)} \to \infty$. 
\end{definition}

\begin{lemma}
	Let $E$ be proper, coercive and bounded from below. Then $\inf_{u \in \XX} E(u) > -\infty$ and there exists a (bounded) minimising sequence $(u_j)$ with $E(u_j) \to \inf_u E(u)$. \end{lemma}

\begin{theorem}[Direct method] \label{thm:direct_method}
	Let $\XX$ be Banach and $\tau_\XX$ a topology on $\XX$ such that any bounded sequence in $\XX$ has a $\tau_\XX$ convergent subsequence. Then any proper, bounded from below, coercive, $\tau_X$-LSC functional has a minimiser.  
\end{theorem}

\begin{proof}
	Since $E$ is bounded from below, we have $\inf_u E(u) > -\infty$, so there exists a bounded minimising sequence $(u_j)$, which we can assume is $\tau_\XX$ convergent with limit $u^*$ after taking a subsequence if necessary. By lower-semicontinuity of $E$ we have
	\[
	E(u^*) \leq \liminf_{k\to\infty} E(u_j) = \lim_{j\to\infty} E(u_j) = \inf_u E(u),
	\]
	so $u^*$ is a minimiser. 
\end{proof}

\begin{theorem}
	If a strictly convex functional has a minimiser, it is unique. 
\end{theorem}

\begin{proof}
	Suppose $u \neq v$ are two minimisers, then by strict convexity, we have $E(\frac12 u + \frac12 v) < E(u)$, a contradiction. 
\end{proof}

\subsubsection{Duality in convex optimisation}
Consider the \emph{primal} optimisation problem 
\[
(P) \ceq \inf_{u \in \XX} E(Au) + F(u),
\]
where $E, F$ are proper, convex and LSC, and $A \in \BB(\XX, \YY)$. Since $E$ is convex and LSC, we have $E = E^{**}$ so we can rewrite the primal problem as the \emph{saddle point problem}
\[
\inf_{u \in \XX} \sup_{\eta \in \YY^*} \ang{\eta, Au} - E^*(\eta) + F(u).
\]
Since $\inf\sup \geq \sup\inf$ always holds we have
\[
(P) \geq \sup_{\eta \in \YY^*} \inf_{u \in \XX} \ang{\eta, y} - E^*(\eta) + F(u) = \sup_{\eta \in \YY^*} -E^*(\eta) - F^*(-A^*\eta) \eqqcolon (D). 
\] 

The problem (D) is called the \emph{dual problem}, and the fact that $(D) \leq (P)$ is called \emph{weak duality}. The value $(P) - (D)$ is called the \emph{duality gap}, and if $(P) = (D)$, we speak of \emph{strong duality}. 

We have the following:
\begin{theorem} \label{thm:primal_dual_gap}
	Suppose the function $E(Au) + F(u)$ is proper, convex, LSC and coercive. Suppose also that there exists $u_0 \in \XX$ s.t.\ $F(u)< \infty$, $E(Au_0) < \infty$, and $E(y)$ is continuous at $y = A u_0$. Then:
	\begin{enumerate}
		\item The dual problem (D) has at least one solution $\hat\eta$;
		\item There is no duality gap;
		\item If $(P)$ has an optimal solution $\hat u$, then we have
		\[
		A^*\hat\eta \in \pt F(\hat u), \quad -\hat\eta \in \pt E(A\hat u). 
		\]
	\end{enumerate}
\end{theorem}

\subsection{Regularisation properties}
We consider the problem $Au = f$ where $A \in \BB(\XX, \YY)$ and where $\YY$ is a Banach space and $\XX$ is the dual of a separable Banach space.  Recall that in Tychonoff regularisation, we have
\[
R_\alpha(f_\delta) = \argmin_{u \in \XX} \norm{Au - f_\delta}^2 + \alpha \norm{u}^2. 
\]
Here, the term $\norm{u}^2$ is known as the \emph{regularisation term}, and penalising large values of $\norm{u}$ ensures regularity of the solution. Inspired by this, suppose we have any function $\JJ(u)$ as regulariser, then the \emph{variational regularisation problem} is given by
\[
R_\alpha f_\delta \in \argmin_{u \in \XX} \frac12 \norm{Au - f_\delta}^2 + \alpha \JJ(u), 
\]
\begin{definition}
	Let $u_\JJ\D$ be a least-squares solution that minimises $\JJ$ over all least-squares solutions. Then we call $u_\JJ\D$ a \emph{$\JJ$-minimising solution} of the problem $Au = f$. 
\end{definition}

\begin{convention}
	We will assume there exists t least one least-squares solution with a finite value of $\JJ$. 
\end{convention}

\begin{lemma} \label{lem:null_space_subspace}
	Let $\JJ(u) \ceq \sum_{i=1}^n \JJ_i(u)$, where each $\JJ_i(u)$ is convex and absolutely $p_i$-homogeneous ($p_i > 0$). Then $\Nul(J)$ is a linear subspace of $\XX$. 
\end{lemma}

\begin{proof}
	We note that $\JJ_i$ is nonnegative, since
	\[
	0 = \JJ_i(0) = \JJ_i\qty(\frac12 u - \frac12 u) \leq \frac12\JJ_i(u) + \frac12\JJ_i(-u) = \JJ_i(u).
	\]
	Therefore we have $\Nul(\JJ) = \cap_i \Nul(\JJ_i)$, and we will show that every $\Nul(\JJ_i)$ is a subspace. Let $\lambda \in \RR, u, v \in \Nul(\JJ_i)$, then 
	\[
	\JJ_i(\lambda u + v) = 2^{p_i} \JJ_i\qty(\frac{\lambda u}{2} + \frac v2) \leq 2^{p_i - 1} \qty(\abs{\lambda}^{p_i} \JJ_i(u) + \JJ_i(v)) = 0. 
	\]
	so $\lambda u + v \in \Nul(\JJ_i)$. This completes the proof. 
\end{proof}

\begin{lemma}
	Let the assumptions of \cref{lem:null_space_subspace} be satisfied, and suppose $u \in \XX, v \in \Nul(\JJ)$. Then $\JJ(u + v) = \JJ(u)$. 
\end{lemma}

\begin{proof}
	Clearly it suffices to prove this lemma for each $\JJ_i$. It is easily seen that for any $t \in (0, 1)$ we have
	\[
	\JJ_i(u + v) = \JJ_i \qty(t \frac ut + (1-t) \frac{v}{1-t}) \leq t \JJ_i \qty(\frac ut) = t^{1 - p_i} \JJ(u). 
	\]
	Letting $t \to 1$, we find $\JJ_i(u + v) \leq \JJ_i(u)$. 
	
	Similarly, we have
	\[
	\JJ_i(u) = \JJ_i(u + v - v) = \JJ_i \qty(t \frac{ u+ v}{t} + (1-t) \frac{-v}{1-t}) \leq t \JJ_i\qty(\frac{u+ v}{t}) = t^{1 - p_i} \JJ_i(u + v), 
	\]
	and letting $t \to 1$ we obtain $\JJ_i(u) \leq \JJ_i(u+v)$, so $\JJ_i(u) = \JJ_i(u+v)$, so $\JJ(u) = \JJ(u + v)$. 
\end{proof}

\begin{recap}
	Let $\XX$ be a Banach space and let $U \subseteq \XX$ be a closed subspace of $\XX$. Then $U$ is called \emph{complemented} in $\XX$ if there exists a closed subspace $V \subseteq \XX$ such that $\XX = U \oplus V$. 
	
	In general, it is difficult to determine which closed subspaces of $\XX$ are complemented. However, it is known that all \textbf{finite-dimensional} subspaces are complemented. 
\end{recap}
\begin{lemma} \label{lem:regulariser_coercive}
	Suppose $\JJ$ is proper, convex, and satisfies the conditions of \cref{lem:null_space_subspace}. Suppose also that:
	\begin{enumerate}[(i)]
		\item $\dim \Nul(\JJ) < \infty$ (so that $\XX = \Nul(\JJ) + \XX_0$ for some closed subspace $\XX_0$) and $\JJ$ is coercive on $\XX_0$;
		\item $\Nul(A) \cap \Nul(\JJ) = \qty{0}$. 
	\end{enumerate}

Then the function $\Phi_\alpha(u) \ceq \frac12\norm{Au - f}^2 + \alpha \JJ(u)$ is coercive on $\XX$ for any $\alpha > 0$. 
\end{lemma}

\begin{proof}
	Let $(u_j) \subseteq \XX$ be a sequence such that $(\Phi_\alpha(u_j))$ is bounded. We will prove that $(u_j)$ is bounded. 
	
	Decompose every $u_j$ as $u_j = u_j^0 + u_j^{\Cal N}$ with $u_j^0 \in \XX_0$ and $u_j^{\Cal N} \in \Nul(\JJ)$. Now, since $(\Phi_\alpha(u_j))$ is bounded, the sequence $(\JJ(u_j)) = (\JJ(u_j^0))$ is bounded as well, and by coercivity of $\JJ$ on $\XX^0$ we conclude that $(u_j^0)$ is bounded. 
	
	Now define $\tilde A \ceq A\restr_{\Nul(\JJ)}$. Since $\Nul(A) \cap \Nul(\JJ) = \qty{0}$, $\tilde A$ has an inverse on $A \Nul(\JJ)$, and since $\dim A\Nul(\JJ) < \infty$, that inverse is bounded. Therefore we find
	\begin{align*}
		\norm{u_j^{\Cal N}} &= \norm{\tilde A^{-1} ( A u_j^{\Cal N})} \leq C \norm{\tilde A u_j^{\Cal N}} = C \norm{\tilde A u_j - f - (A u_j^0 - f)} \leq C \qty(\norm{Au_j - f} + \norm{A} \norm{u_j^0} + \norm{f}). 
	\end{align*}
	Since $\Phi_\alpha(u_j)$ is bounded, $\norm{Au_j - f}$ is also bounded, and the other terms are also all bounded, so we conclude that $(u_j^{\Cal N})$ is bounded, and therefore that $(u_j)$ is bounded. 
\end{proof}

\begin{theorem} \label{thm:minimiser_existence}
	Let $\XX, \YY$ be Banach spaces with topologies $\tau_\XX, \tau_\YY$. Assume that:
	\begin{enumerate}[(a)]
		\item bounded sequences in $\XX$ have $\tau_\XX$-convergent subsequences;
		\item $\JJ\colon \XX \to [0, \infty]$ is proper, convex, $\tau_\XX$-LSC and satisfies the assumptions of \cref{lem:regulariser_coercive};
		\item $A$ is continuous w.r.t.\ $\tau_\XX$ and $\tau_\YY$;
		\item $\norm{\cdot}_\YY$ is $\tau_\YY$-LSC. 
	\end{enumerate}

Then:
\begin{enumerate}
	\item There exists a $\JJ$-minimising solution $u_\JJ\D$ of the equation $Au = f$;
	\item For any $\alpha > 0$, $f \in \YY$ there exists a minimiser
	\[
 	u_\alpha = R_\alpha f \in \argmin \frac12 \norm{Au - f}^2 + \alpha \JJ(u). 
	\]
\end{enumerate}
\end{theorem}

\begin{proof}
	\begin{enumerate}
		\item Let $\LLL$ be the set of least-squares solutions and $\mu \ceq \inf\qty{\norm{Au - f} \mid u \in \XX}$, then we can write
		\[
		\LLL = \qty{u \in \XX : \norm{Au - f} \leq \mu}. 
		\]
		Since $\norm{\cdot}_\YY$ is $\tau_\YY$-LSC and $A \colon \tau_\XX \to \tau_\YY$ is continuous, we have by \cref{lem:lsc_closed_sets} that $\LLL$ is $\tau_\XX$-closed.
		
		Now consider the problem 
		\[
		\inf_{u \in \LLL} \JJ(u) = \inf_{u \in \XX} \JJ(u) + \chi_\LLL(u). 
		\] 
		By assumption there exists $u \in \LL$ with $\JJ(u) < \infty$, and the objective function $\JJ + \chi_\LL$ is bounded from below (by 0). Using similar arguments as in the previous lemma (???), it can be shown that it is coercive. Finally, characteristic functions of closed sets are LSC, so $\chi_\LLL$ and $\JJ$ are both $\tau_\XX$-LSC, and therefore their sum is $\tau_\XX$-LSC as well. By the direct method (\cref{thm:direct_method}), we conclude that a $\JJ$-minimising solution exists. 
		
		\item The objective function $\Phi_\alpha$ is coercive by the previous lemma, and also bounded from below. It is easily seen that $\Phi_\alpha$ is $\tau_\XX$-LSC, and using the direct method we conclude that $\Phi_\alpha$ has a minimiser. 
	\end{enumerate}
\end{proof}

\begin{theorem} \label{thm:minimiser_convergence}
	Let the assumptions of \cref{thm:minimiser_existence} be satisfied and suppose that $\inf\limits_{v \in \XX} \norm{Av - f} = 0$. 
	Let $\alpha = \alpha(\delta)$ satisfy \[
	\lim_{\delta\to0} \alpha(\delta) = 0  = \limsup_{\delta \to 0} \frac{\delta^2}{\alpha(\delta)}. 
	\]
	Then any sequence $(\delta_n) \to 0$ has a subsequence $(\delta_n) \to 0$ such that $u_{\delta_n} \ceq u_{\alpha(\delta_n)} \to u_\JJ \D$ and $\JJ(u_\delta) \to \JJ(u_\JJ\D)$, where $u_\JJ\D$ is a $\JJ$-minimising solution. 
\end{theorem}

\begin{proof} 
	By \cref{lem:regulariser_coercive}, we know that $\Phi_\alpha$ is coercive, and by \cref{thm:minimiser_existence}, we know there exists a $\JJ$-minimising solution $u_\JJ\D$. For any $(u_{\delta_n})$ with $\delta_n \to 0$ we find
	\begin{equation} \label{eq:phi_alpha}
	\Phi_\alpha(u_{\delta_n}) \leq \Phi_\alpha(u_\JJ\D) = \frac12\norm{A u_\JJ\D - f_\delta}^2 + \alpha(\delta) \JJ(u_\JJ\D) \leq \frac{\delta^2}{2} + \alpha(\delta)\JJ(u_\JJ\D),
	\end{equation}
	which is bounded, so by coercivity of $\Phi_\alpha$ we conclude that $(u_{\delta_n})$ is bounded. 
	
	After taking a subsequence if necessary, we conclude that $u_{\delta_n} \overset{\tau_X}\to u_0$. We show that $u_0$ is a $\JJ$-minimising solution. Since $A \colon \tau_X \to \tau_Y$ is continuous and $\norm{\cdot}_\YY$ is $\tau_\YY$-LSC, we get
	\[
	\norm{Au_0 - f} \leq \liminf_{n\to\infty} \norm{Au_{\delta_n} - f} \leq \liminf_{n\to\infty} \norm{Au_{\delta_n} - f_{\delta_n}} + \norm{f - f_{\delta_n}} = 0,
	\]
	since $\norm{Au_{\delta_n} - f\delta_n} \leq 2 \Phi_\alpha(u_{\delta_n}) \to 0$ by \cref{eq:phi_alpha}. This shows that $u_0$ is a least-squares solution.
	
	Furthermore, we have 
	\[
	\JJ(u_0) \leq \liminf_{n\to\infty} \JJ(u_{\delta_n}) \leq \limsup_{n\to\infty} \JJ(u_{\delta_n}) \leq \limsup_{n\to\infty} \frac{\Phi_\alpha(u_{\delta_n})}{\alpha(\delta)} \leq \limsup_{n\to\infty} \frac12 \frac{\delta_n^2}{\alpha(\delta_n)} + \JJ(u_\JJ\D) = \JJ(u_\JJ\D)
	\]
	which proves that $u_0$ minimises $\JJ$. 
	
	To prove that $u_{\delta_n}$ is actually a minimising sequence, note that all above inequalities are in fact equalities, so $\liminf_{n\to\infty} \JJ(u_{\delta_n}) = \limsup_{n\to\infty} \JJ(u_{\delta_n}) = \lim_{n\to\infty} \JJ(u_{\delta_n}) = \JJ(u_\JJ\D)$.

%	By the previous theorem, we know there exists a $\JJ$-minimising solution $u_0$, and for every $\alpha > 0$ we know there exists
%	\[
%	u_\delta \ceq u_{\alpha(\delta)} \in \argmin_u \frac12 \norm{Au - f}^2 + \alpha \JJ(u). 
%	\]
%	
%	Therefore, we have 
%	\begin{equation} \label{eq:estimate}
%	\frac12 \norm{Au_\delta - f_\delta}^2 + \alpha(\delta) \JJ(u_\delta) \leq \frac12 \norm{A u_0 - f_\delta}^2 + \alpha(\delta) \JJ(u_0) \leq \frac{\delta^2}{2} + \alpha(\delta) \JJ(u_0). 
%	\end{equation}
%	\Cref{eq:estimate} gives us two estimates: the first one is
%	\[
%	\alpha(\delta) \JJ(u_\delta) \leq \frac{\delta^2}{2} + \alpha(\delta) \JJ(u_0) \implies \JJ(u_\delta) \leq \frac12 \frac{\delta^2}{\alpha(\delta)} + \JJ(u_0), 
%	\]
%	so $\JJ(u_\delta)$ is bounded uniformly in $\delta$. From \cref{eq:estimate} we also infer
%	\[
%	\frac12 \norm{Au_\delta - f_\delta}^2 \leq \frac{\delta^2}{2} + \alpha(\delta) \JJ(u_0),
%	\]
%	from which it follows that $\norm{Au_\delta - f_\delta}$ is also uniformly bounded in $\delta$. 
%	
%	Writing $u_\delta = u_\delta^0 + u_\delta^{\Cal N}$
\end{proof}

\subsection{Total variation regularisation}
\begin{definition}
	Let $\Omega \subseteq \RR^n$ be a bounded comain and $u \in L^1(\Omega)$. Let
	\[
	\DD(\Omega, \RR^n) \ceq \qty{\phi \in C_0^\infty(\Omega, \RR^n) \mid \sup_{x \in \Omega} \norm{\phi(x)}_2 \leq 1}. 
	\]
	We define the \emph{total variation} of $u \in L^1(\Omega)$ as
	\[
	\TV(u) \ceq \sup_{\phi \in \DD(\Omega, \RR^n)} \int_\Omega u(x) (\div \phi)(x) \dd{x}. 
	\]
\end{definition}

\begin{example}
	If $u \in L^1(\Omega)$ has a weak derivative $\grad u \in L^1$ (i.e., $u \in W^{1, 1}$), then since any $\phi \in \DD(\Omega, \RR^n)$ is 0 on the boundary of $\Omega$, we have
	\[
	\TV(u) = \sup_{\phi \in \DD(\Omega, \RR^n)} \int_\Omega u(x) (\div\phi)(x) \dd{x} =  \sup_{\phi \in \DD(\Omega, \RR^n)} \int_\Omega -\grad u \vdot \phi \dd{x}, 
	\]
	By Cauchy-Schwarz we have $\abs{\nabla u(x) \vdot \phi(x)} \leq \norm{\nabla u(x)}_2 \norm{\phi(x)}_2$ which gives an upper bound for $\TV(u)$, and it can be shown that this upper bound is indeed an equality, so
	\[
	\TV(u) = \int_\Omega \norm{\nabla u}_2 \dd{x} = \norm{\nabla u}_{L^1}.
	\]
\end{example}

\begin{proposition}
	The functional $\TV$ is proper, convex, and absolutely one-homogeneous, and for any constant $c$ and any $u \in L^1(\Omega)$ we have
	\[
	\TV(c) = 0, \quad \TV(u + c) = \TV(u). 
	\]
\end{proposition}

\begin{proof}
	For any test function $\phi$ we have 
	\[
	\int_\Omega \div\phi \dd{x} = \int_{\pt\Omega} \phi \vdot \vec n \dd{x} = 0
	\]
	since $\phi = 0$ on the boundary. This shows immediately that $\TV(c) = 0$ and that $\TV(u + c) = \TV(u)$. 
\end{proof}

\begin{remark}
	In fact, the converse also holds: $\TV(u) = 0$ if and only if $u$ is constant. 
\end{remark}

\begin{example}
	Let $C \subseteq \Omega$ be a bounded domain with smooth boundary. Then we have
	\[
	\TV(\ind_C) = \sup_{\phi \in \DD(\Omega, \RR^n)} \int_C\div\phi \dd{x} = \sup_{\phi \in \DD(\Omega, \RR^n)} \int_{\pt C} \phi \vdot \vec n \dd{s} \leq \sup_{\phi \in \DD(\Omega, \RR^n)} \int_{\pt C} \norm{\phi}_2 \norm{\vec n}_2  \dd{s} = \int_{\pt C} 1 \dd{s},
	\]
	and we can show that the inequality here is in fact an equality, i.e., $\TV(\ind_C) = \int_{\pt C} 1 \dd{s}$, which is the perimeter of $C$. 
\end{example}

\begin{definition}
	We define the space of functions of \emph{bounded variation} as
	\[
	\BV(\Omega) \ceq \qty{u \in L^1(\Omega) : \norm{u}_{\BV} \ceq \norm{u}_{L^1} + \TV(u) < \infty}.
	\]
\end{definition}
\begin{remark}
	It can be shown that BV is the dual of a separable Banach space, and that weak-$*$ convergence in BV is equivalent to strong convergence in $L^1$ and convergence of the values $\TV(u_n) \to \TV(u)$. 
\end{remark}

\begin{definition}
	We define a \emph{domain} in $\RR^n$ as an open connected subset, and it is called a \emph{Lipschitz domain} if its boundary is locally the graph of a Lipschitz continuous function (see Wikipedia for the details). 
\end{definition}

\begin{theorem}
	Let $\Omega \subseteq \RR^n$ be a bounded Lipschitz domain and $p, m \in \NN$. Define 
	\[
	p^* \ceq \begin{cases} \frac{np}{n - mp}, &n > mp, \\ \infty,& n \leq mp. \end{cases}
	\]
	Then the embedding $W^{m, p} \to L^q(\Omega)$ is continuous for all $q \in [1, p^*]$ and compact for all $q \in [1, p^*)$. 
\end{theorem}

A consequence ($p = m = 1$) is the following:
\begin{corollary}
	For any bounded Lipschitz domain $\Omega \subseteq \RR^n$, the $\BV(\Omega)$ is compactly embedded in $L^1(\Omega)$ for $n \geq 2$. For $n = 2$, the embedding $\BV(\Omega) \to L^2(\Omega)$ is continuous. 
\end{corollary}

\begin{theorem}
	Let $\Omega \subseteq \RR^n$ be open and bounded. Then $\TV$ is LSC in $L^1(\Omega)$. 
\end{theorem}

\begin{proof}
	Let $(u_j) \subseteq \BV(\Omega)$ converge to $u \in L^1(\Omega)$. Then for any test function we have
	\begin{align*}
		\TV(u) &= \sup_{\phi \in \DD(\Omega, \RR^n)}\int_\Omega u_j(x) (\div \phi)(x) \dd{x} \\
		&= \sup_{\phi \in \DD(\Omega, \RR^n)} \lim_{j\to\infty} \int_\Omega u_j(x) (\div \phi)(x) \dd{x} \\
		&\leq \liminf_{j\to\infty} \sup_{\phi \in \DD(\Omega, \RR^n)} \int_\Omega u_j(x) (\div\phi)(x) \dd{x} \\
		&= \liminf_{j\to\infty} \TV(u_j). 
	\end{align*}
\end{proof}

\TODO what is the point/use of proposition 4.3.12 and corollary 4.3.13? 
%\begin{proposition}
%	Let $\Omega \subseteq \RR^n$ be a bounded Lipschitz domain, then there exists $C > 0$ such that for all $u \in \BV(\Omega)$, we have
%	\[
%	\norm{u - u_\Omega}_{L^1} \leq C \TV(u),
%	\]
%	where $u_\Omega$ is the mean value of $u$ over $\Omega$. 
%\end{proposition}

\begin{theorem}
	Let $\XX = L^1(\Omega)$, where $\Omega \subseteq \RR^n$ is bounded Lipschitz, and $\YY$ a Banach space. Let $A \in \BB(L^1, \YY)$ such that $A(1) \neq 0$. Then $u_\delta$ converges strongly in $L^1$ (along a subsequence) to a TV-minimising solution as $\delta \to 0$ if $\alpha(\delta)$ is chosen as required by \cref{thm:minimiser_convergence}. 
\end{theorem}

\begin{proof}
	We need to check that the conditions of  \cref{thm:minimiser_convergence} are satisfied. We have assumed that $\Ran(A)$ is dense, so $\inf_{v\in \XX} \norm{Av - f} = 0$. Since $\BV(\Omega)$ is compactly embedded in $L^1(\Omega)$, we know that sequences with a finite TV-value have (strongly) convergent subsequences. We have checked that $\TV$ is convex and LSC, that $\dim(\Nul(\TV)) < \infty$, and by assumption we have $\Nul(A) \cap \Nul(\TV) = \qty{0}$. Finally, $\TV$ is absolutely one-homogeneous, so all assumptions are valid. 
\end{proof}