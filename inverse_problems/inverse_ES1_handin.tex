\documentclass{article}

\usepackage[utf8]{inputenc}
\usepackage[english]{babel}
\usepackage[margin=3cm]{geometry}
\usepackage[normalem]{ulem}
\usepackage{hyperref}
\usepackage{mathtools, amsmath, amssymb, amsthm, enumerate, mdframed, bbm, graphicx, float, physics, xcolor, cleveref}

\hypersetup{
    colorlinks   = true, %Colours links instead of ugly boxes
    urlcolor     = blue, %Colour for external hyperlinks
    linkcolor    = blue, %Colour of internal links
    citecolor   = red %Colour of citations
}

% Definition of numbered environments.
% Usage: \begin{theorem} ... \end{theorem}
% Remark has no numbering.
\theoremstyle{plain}
\newtheorem{question}{Question}
\newtheorem{theorem}{Theorem}
\newtheorem{lemma}{Lemma}
\theoremstyle{remark}
\newtheorem*{remark}{Remark}

% Question with box around it
\newenvironment{qbox}{\begin{mdframed}\begin{question}}{\end{question}\end{mdframed}}

% A proof with "solution" instead of "proof" and no QED symbol
\newenvironment{solution}{\begin{proof}[Solution]\renewcommand\qedsymbol{}}{\end{proof}}

% Some renewed commands
\renewcommand{\vec}{\mathbf}
\renewcommand{\emptyset}{\varnothing}
\renewcommand{\epsilon}{\varepsilon}
\renewcommand{\theta}{\vartheta}
\renewcommand{\phi}{\varphi}

% Frequently used math alphabets
\newcommand{\Bb}{\mathbb}
\newcommand{\Cal}{\mathcal}
\newcommand{\Bf}{\mathbf}
\newcommand{\Rm}{\mathrm}

% Frequently used letters in the blackboard alphabet
\newcommand{\CC}{\Bb C}
\newcommand{\NN}{\Bb N}
\newcommand{\PP}{\Bb P}
\newcommand{\QQ}{\Bb Q}
\newcommand{\RR}{\Bb R}
\newcommand{\EE}{\Bb E}
\newcommand\ZZ{\Bb Z}
\newcommand\FF{\Cal F}

% Usage: \ang{...} is equivalent to \langle ... \rangle, while \ang*{...} is equivalent to \left\langle ... \right\rangle
% For other delimiters: use \qty from the physics package (i.e., \qty(...))
\DeclarePairedDelimiter{\ang}{\langle}{\rangle}
\DeclarePairedDelimiter{\floor}{\lfloor}{\rfloor}
\DeclarePairedDelimiter{\ceil}{\lceil}{\rceil}

% Frequently used commands
\newcommand{\T}{^\top} % Matrix transpose A\T
\newcommand{\C}{^\complement} % Set complement A\C
\renewcommand{\P}{^\perp}
\newcommand{\D}{^\dagger}
\newcommand\ceq\coloneqq % Definitions :=
\newcommand\pow{\Cal P} % Power sets
\newcommand\eps\epsilon
\newcommand\ind{\mathbbm 1} % Blackboard 1 for indicator functions
\newcommand\restr{\mathord\restriction}
\newcommand\TODO{{\color{red} TODO: }}
\newcommand\clos\overline

% Functions that appear frequently
\DeclareMathOperator{\sign}{sign}
\DeclareMathOperator{\Int}{Int}
\DeclareMathOperator{\Span}{Span}
\DeclareMathOperator{\Var}{Var}
\DeclareMathOperator*{\argmin}{arg\,min}
\DeclareMathOperator*{\argmax}{arg\,max}

\newcommand\XX{\Cal X}
\newcommand\YY{\Cal Y}
\newcommand\UU{\Cal U}
\newcommand\VV{\Cal V}
\newcommand\LL{\Cal L}
\newcommand\KK{\Cal K}
\newcommand\BB{\Cal B}
% \newcommand\LLL{\Bb L}
\DeclareMathOperator\Dom{\Cal D}
\DeclareMathOperator\Nul{\Cal N}
\DeclareMathOperator\Ran{\Cal R}
\DeclareMathOperator\diag{diag}

\title{Inverse Problems --- Example Sheet 1} % subject
\author{Lucas Riedstra}
\date{2 November 2020} % date

\begin{document}
\maketitle

\begin{mdframed}
		Note: when writing a norm of a vector $v \in V$, I will simply write $\norm{v}$ and not $\norm{v}_V$, unless it is unclear in which space $v$ lives. The same holds for inner products. 	
\end{mdframed}
\begin{question}
	For $\Omega = [0, 1]^2$ and $\XX \in L^2(\Omega)$, we consider the integral operator $A \colon \XX \to \XX$ with
	\[
	(Au)(y) \ceq \int_\Omega k(x, y) u(x) \dd{x},
	\]
	for $k \in L^2(\Omega \times \Omega)$.  Show that
	\begin{enumerate}[(a)]
		\item $A$ is linear with respect to $u$,
		\item $A$ is a bounded linear operator, i.e.\ $\norm{Au}_\XX \leq \norm{A}_{\LL(\XX, \XX)} \norm{u}_\XX$. Give also an estimate for $\norm{A}_{\LL(\XX, \XX)}$,
		\item the adjoint $A^*$ is given via
		\[
		(A^* v)(y) = \int_\Omega k(y, x) v(x) \dd{x}.
		\]
		\item $A$ is a compact operator, i.e.\ $A \in \KK(\XX, \XX)$. 
		
		\emph{Hint:} you may use the fact that if an operator $A$ can be written as a limit (in the operator norm) of finite-rank operators then $A$ is compact. An operator $B$ is called finite-rank if $\dim(B) < \infty$. 
	\end{enumerate}
\end{question}

\begin{solution}

	
	\begin{enumerate}[(a)]
		\item Let $\alpha, \beta \in \RR$, $u, v \in L^2(\Omega)$ and $y \in \Omega$. Then we have
		\begin{align*}
			(A(\alpha u + \beta v))(y) &= \int_\Omega k(x, y) (\alpha u + \beta v)(x) \dd{x} \\
			&= \int_\Omega k(x, y) \qty(\alpha u(x) + \beta v(x)) \dd{x}\\
			&= \alpha \int_\Omega k(x, y) u(x) \dd{x} + \beta \int_\Omega k(x, y) v(x) \dd{x} \\
			&= (\alpha A u)(y) + (\beta Av)(y) = (\alpha Au + \beta Av)(y). 
		\end{align*}
	Since equality holds for all $y \in \Omega$ we find $A(\alpha u + \beta v) = \alpha A u + \beta A v$, which proves that $A$ is linear. 
	\item Let $u \in L^2(\Omega)$, then we have
	\begin{align*}
	\norm{Au}^2 &= \int_\Omega ((Au)(y))^2 \dd{y} = \int_\Omega \qty( \int_\Omega k(x, y) u(x) \dd{x})^2 \dd{y} = \int_\Omega \ang{k(\cdot, y), u(\cdot)}^2 \dd{y}. 
	\end{align*}
Now we apply Cauchy-Schwarz and find
\begin{align*}
	\int_\Omega \ang{k(\cdot, y), u(\cdot)}^2 \dd{y} &\leq \int_\Omega \norm{k(\cdot, y)}^2 \norm{u}^2 \dd{y} \overset\star= \norm{u}^2 \iint_{\Omega^2} k^2(x, y) \dd{x}\dd{y} = \norm{u}^2 \norm{k}^2, 
\end{align*}
where $\star$ follows from Fubini's theorem since the integrand is nonnegative. 
Taking square roots on both sides we find that $\norm{Au}\leq \norm{k}\norm{u}$, so $A$ is bounded with $\norm{A} \leq \norm{k}$. 

\item We know that the adjoint is the unique operator that satisfies $\ang{Au, v} = \ang{u, A^*v}$ for all $u, v \in \XX$. Let $u, v \in \XX$, then we compute
\begin{align*}
	\ang{Au, v} &= \int_\Omega (Au)(y)\cdot  v(y) \dd{y} = \int_\Omega \qty(\int_\Omega k(x, y) u(x) \dd{x}) v(y) \dd{y} \\
	&= \int_\Omega\int_\Omega k(x, y) u(x) v(y) \dd{x} \dd{y} \overset\star= \int_\Omega\int_\Omega k(x, y) u(x) v(y) \dd{y} \dd{x} \\
	&= \int_\Omega u(x) \qty(\int_\Omega k(x, y) v(y) \dd{y}) \dd{x} = \ang{u, A^*v}
\end{align*}
where $(A^* v)(x) = \int_\Omega k(x, y) v(y) \dd{y}$ as required. Here $\star$ follows from Fubini's theorem.

\item It is known that for any compact set $X \subseteq \RR^n$, polynomials lie dense in $L^2(X)$. Therefore, there exists a sequence of polynomials $p_n$ such that $p_n \to k$ in $L^2([0, 1]^4)$. It is easily seen that for any polynomial $p$, the operator
\[
(A_p u)(y) \ceq \int_\Omega p(x, y) u(x) \dd{x} 
\] 
has finite rank: let $p(z) = \sum_{\abs{\alpha} \leq n} c_\alpha z^\alpha$ (where $z \in [0, 1]^4$ and $\alpha$ is a multi-index), then we find
\[
(A_p u)(y) = \sum_{\abs{\alpha} \leq n} c_\alpha \int_\Omega x_1^{\alpha_1} x_2^{\alpha_2} y_1^{\alpha_3} y_2^{\alpha_4} u(x) \dd{x} = \sum_{\abs{\alpha} \leq n} c_\alpha \qty(\int_\Omega x_1^{\alpha_1} x_2^{\alpha_2} u(x) \dd{x}) y_1^{\alpha_3} y_2^{\alpha_4},
\]
so $A_p u$ lies in the $\Span\qty{y_1^{\alpha_1} y_2^{\alpha_2} \mid \alpha_1 + \alpha_2 \leq n}$, and therefore has finite rank. 
By (b), we find that $\norm{A - A_n} \leq \norm{k - p_n} \to 0$, which shows that $A_n \to A$ in operator norm. We conclude that $A$ is compact. 
%\[
%(A_n u)(y) \ceq \int_\Omega p_n()
%\]
	\end{enumerate}
\end{solution}

\begin{question}
	We consider the problem of differentiation, formulated as the inverse problem of finding $u$ from $Au = f$ with the integral operator $A \colon L^2([0, 1]) \to L^2([0, 1])$ defined as
	\[
	(Au)(y) \ceq \int_0^y u(x) \dd{x}.
	\]
	
	\begin{enumerate}[(a)]
		\item Let $f$ be given by
		\[
		f(x) \ceq \begin{cases} 0 & x < \frac12, \\ 1 &x > \frac12. \end{cases}
		\]
		Show that $f \in \clos{\Ran(A)}$. 
		\item Let $f$ be given as in a). Show that $f \in \clos{\Ran(A)} \setminus \Ran(A)$. \emph{Hint: Consider the Picard criterion}. 
		
		\item Prove or falsify: ``The Moore-Penrose inverse of $A$ is continuous.''
	\end{enumerate}
\end{question}

\begin{solution}
	\begin{enumerate}[(a)]
		\item We want to show that we can approximate $f$ by a sequence $(Au_n)$ for some $(u_n) \subseteq L^2[0,1]$. To this end, define for $n \geq 2$
		\[
		u_n(x) =
		\begin{cases}
			0 &\abs{x - \frac12} > \frac1n, \\
			\frac n2 &\abs{x - \frac12} \leq \frac1n. 
		\end{cases}
		\]
		Clearly $u \in L^2[0, 1]$, and we have
		\[
		f_n(y) \ceq (Au_n)(y) = \int_0^y u_n(x) \dd{x} = \begin{cases}
			0 &\text{if $y < \frac12 - \frac1n$,}\\
			\frac n2(y - \frac12 + \frac1n) &\text{if $\frac12 - \frac1n \leq y \leq \frac12 + \frac1n$} \\
			1 &\text{if $y > \frac12 + \frac1n$}. 
		\end{cases}
		\]
		
		Therefore we find
		\begin{align*}
			\norm{f_n - f}^2 &= \int_0^1 (f_n - f)^2(x) \dd{x} \\
			&= 2 \int_{\frac12 - \frac1n}^{\frac12} \frac{n^2}{4} (x - \frac12 - \frac1n)^2 \dd{x} \\
			&= \frac{n^2}{2} \int_0^{1/n}  x^2 \dd{x} = \frac{1}{6n} \to 0, 
		\end{align*}
	so $f_n \to f$ in $L^2[0, 1]$. Since $f_n \in \Ran(A)$ this shows $f \in \clos{\Ran(A)}$. 
	

	\item 	In example 2.2.12, it is shown that for this operator, the Picard criterion is
	\begin{equation} \label{eq:picard}
	2\sum_{j=1}^\infty  \sigma_j^{-2} \qty(\int_0^1 f(s) \sin(\sigma_j^{-1} s) \dd{s})^2 < \infty,
	\end{equation}
	where $\sigma_j = \frac2{(2j - 1)\pi}$.
	
	We compute
	\[
	\int_0^1 f(s) \sin(\sigma_j^{-1} s) \dd{s} = \int_{1/2}^1 \sin(\sigma_j^{-1} s) \dd{s} = \sigma_j \qty[ \cos(\frac12 \sigma_j^{-1}) - \cos(\sigma_j^{-1})]. 
	\]
	We have
	\[
	\cos(\sigma_j^{-1}) = \cos(\frac{(2j - 1)\pi}{2}) = 0 \quad\text{and}\quad \cos(\frac12\sigma_j^{-1}) = \cos(\frac{(2j-1)\pi}{4}) = \pm\frac1{\sqrt2 }. 
	\]
	
	Plugging this into \cref{eq:picard} gives that
	\[
	2\sum_{j=1}^\infty  \sigma_j^{-2} \qty(\int_0^1 f(s) \sin(\sigma_j^{-1} s) \dd{s})^2 = 2 \sum_{j=1}^\infty \sigma_j^{-2} (\sigma_j^2/2) = \sum_{j=1}^\infty 1 = \infty,
	\]
	so $f$ does not satisfy the Picard criterion and therefore $f \in \clos{\Ran(A)} \setminus \Ran(A)$. 
%	Note that $A$ is compact, since we can write $(Au)(y) = \int_{[0, 1]} k(x, y) u(x) \dd{x}$ with $k(x, y) = \ind_{x \leq y}$. 
%	
%	To apply the Picard criterion we must find the singular values and right singular vectors of $A$, which are equal to the square roots of the eigenvalues of $AA^*$ and the eigenvectors of $AA^*$. 
%	
%
%	\begin{align*}
%		\ang{Au, v} &= \int_0^1 (Au)(y)\cdot v(y) \dd{y} \\
%		&= \int_0^1 \int_0^y u(x) \dd{x} v(y) \dd{y} \\
%		&= \int_0^1 \int_0^y u(x) v(y) \dd{x} \dd{y} \\
%		&= \int_0^1 \int_x^1 u(x) v(y) \dd{y} \dd{x} \\ 
%		&= \int_0^1 u(x) \int_x^1 v(y) \dd{y} \dd{x} \\
%		&= \ang{u, A^* v}
%	\end{align*}
%where $v(x) = \int_x^1 v(y) \dd{y}$.  Therefore, we find
%\begin{align*}
%	(AA^* u)(x) = \int_0^x \int_y^1 u(z) \dd{z} \dd{y}.
%\end{align*}
%Now we solve the eigenequation. Firstly, note that $\Nul(A) = \qty{0}$, so let $\lambda^2 > 0$. Then we have
%\begin{align}
%	\int_0^x \int_y^1 u(z) \dd{z} \dd{y} &= \lambda^2 u(x) \\
%	\int_x^1 u(z) \dd{z} &= \lambda^2 u'(x) \\
%	- u(x) &= \lambda^2 u''(x). 
%\end{align}
%Furthermore, from (1) we infer $u(0) = 0$ while from (2) we infer $u'(1) = 0$. 
%The general solution to (3) is given by $u(x) = a \cos(\lambda^{-1} x) + b \sin(\lambda^{-1} x)$. 
%Plugging $u(0) = 0$ gives $a = 0$, and $u'(1) = 0$ gives
%\[
%b\lambda^{-1} \cos(\lambda^{-1}) = 0 \implies b = 0 \text{ or } \lambda^{-1} = \pi\qty(n - \frac12) \text{ for some $n \in \ZZ_{>0}$}.
%\]
%
%Our singular values are therefore $\frac{1}{(n - \frac12)\pi}$ ($n \in \ZZ_{>0}$) with right singular vectors $y_n(x) = \sqrt 2\sin((n - \frac12) \pi x)$. 
%We now compute
%\[
%\ang{f, y_n} = \sqrt 2\int_{1/2}^1 \sin(\qty(n - \frac12) \pi x) \dd{x} = ...
%\]

\item The Moore-Penrose inverse of $A$ is discontinuous. This can be seen by theorem 2.1.11: we have in (b) an element $f \in \clos{\Ran(A)} \setminus \Ran(A)$, so   $\Ran(A)$ is not closed, so $A\D$ is discontinuous. 

\end{enumerate}
\end{solution}

\end{document}