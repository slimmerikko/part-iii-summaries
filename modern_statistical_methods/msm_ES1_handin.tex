\documentclass{article}

\usepackage[utf8]{inputenc}
\usepackage[english]{babel}
\usepackage[margin=3cm]{geometry}
\usepackage[normalem]{ulem}
\usepackage{hyperref}
\usepackage{mathtools, amsmath, amssymb, amsthm, enumerate, mdframed, bbm, graphicx, float, physics, xcolor, cleveref}

\hypersetup{
    colorlinks   = true, %Colours links instead of ugly boxes
    urlcolor     = blue, %Colour for external hyperlinks
    linkcolor    = blue, %Colour of internal links
    citecolor   = red %Colour of citations
}

% Definition of numbered environments.
% Usage: \begin{theorem} ... \end{theorem}
% Remark has no numbering.
\theoremstyle{plain}
\newtheorem{question}{Question}
\newtheorem{theorem}{Theorem}
\newtheorem{lemma}{Lemma}
\theoremstyle{remark}
\newtheorem*{remark}{Remark}

% Question with box around it
\newenvironment{qbox}{\begin{mdframed}\begin{question}}{\end{question}\end{mdframed}}

% A proof with "solution" instead of "proof" and no QED symbol
\newenvironment{solution}{\begin{proof}[Solution]\renewcommand\qedsymbol{}}{\end{proof}}

% Some renewed commands
\renewcommand{\vec}{\mathbf}
\renewcommand{\emptyset}{\varnothing}
\renewcommand{\epsilon}{\varepsilon}
\renewcommand{\theta}{\vartheta}
\renewcommand{\phi}{\varphi}

% Frequently used math alphabets
\newcommand{\Bb}{\mathbb}
\newcommand{\Cal}{\mathcal}
\newcommand{\Bf}{\mathbf}
\newcommand{\Rm}{\mathrm}

% Frequently used letters in the blackboard alphabet
\newcommand{\CC}{\Bb C}
\newcommand{\NN}{\Bb N}
\newcommand{\PP}{\Bb P}
\newcommand{\QQ}{\Bb Q}
\newcommand{\RR}{\Bb R}
\newcommand{\EE}{\Bb E}

\newcommand\XX{\Cal X}
\newcommand\HH{\Cal H}

% Usage: \ang{...} is equivalent to \langle ... \rangle, while \ang*{...} is equivalent to \left\langle ... \right\rangle
% For other delimiters: use \qty from the physics package (i.e., \qty(...))
\DeclarePairedDelimiter{\ang}{\langle}{\rangle}
\DeclarePairedDelimiter{\floor}{\lfloor}{\rfloor}
\DeclarePairedDelimiter{\ceil}{\lceil}{\rceil}

% Frequently used commands
\newcommand{\T}{^\top} % Matrix transpose A\T
\newcommand{\C}{^\complement} % Set complement A\C
\newcommand\ceq\coloneqq % Definitions :=
\newcommand\pow{\Cal P} % Power sets
\newcommand\eps\epsilon
\newcommand\ind{\mathbbm 1} % Blackboard 1 for indicator functions
\newcommand\restr{\mathord\restriction}
\newcommand\TODO{{\color{red} TODO: }}
\renewcommand\P{^\perp}

% Functions that appear frequently
\DeclareMathOperator{\sign}{sign}
\DeclareMathOperator{\Int}{Int}
\DeclareMathOperator{\Span}{Span}
\DeclareMathOperator{\Var}{Var}
\DeclareMathOperator*{\argmin}{arg\,min}
\DeclareMathOperator*{\argmax}{arg\,max}

\title{Modern Statistical Methods --- Example Sheet 1} % subject
\author{Lucas Riedstra}
\begin{document}
\maketitle

\setcounter{question}{6}

\begin{question}
	Suppose we have a matrix of predictors $X \in \RR^{n \times p}$ where $p \gg n$. Explain how to obtain the fitted values of the following ridge regression using the kernel trick:
	\begin{gather*}
		\text{Minimise over $\beta \in \RR^p$, $\theta \in \RR^{p(p-1)/2}$, $\gamma \in \RR^p$,} \\
		\sum_{i=1}^n \qty(Y_i - \sum_{k=1}^p X_{ik}\beta_k - \sum_{k=1}^p \sum_{j=1}^{k-1} X_{ik} X_{ij} \theta_{jk} - \sum_{k=1}^p X_{ik}^2\gamma_k)^2 + \lambda_1 \norm{\beta}^2 + \lambda_2 \norm{\theta}^2 + \lambda_3\norm{\gamma}^2. 
	\end{gather*}
Note that we have indexed $\theta$ with two numbers for convenience. 
\end{question}

\begin{proof}
	Our linear model has predictors
	\[
	\qty{X_{ik} \mid 1 \leq k \leq p} \cup \qty{X_{ij} X_{ik} \mid 1 \leq j < k \leq p} \cup \qty{X_{ik}^2 \mid 1 \leq k \leq p}
	\]
	for $i = 1, \dotsc, k$. We want to assume that $\lambda_1 = \lambda_2 = \lambda_3$: to this end, define $\xi = \frac{\lambda_1}{\lambda_2}$ and $\eta = \frac{\lambda_3}{\lambda_1}$, and replace $\theta$ by $\xi\theta$ and $\gamma$ by $\eta\gamma$. This means we will have to replace our predictors by
	\[
	Z_i \ceq 
	(X_{ik} \mid 1 \leq k \leq p)\T \cup (\xi X_{ij} X_{ik} \mid 1 \leq j < k \leq p)\T \cup (\gamma X_{ik}^2 \mid 1 \leq k \leq p)\T
	\]
	We use the kernel trick: note that
	\begin{align*}
	\ang{Z_i, Z_j} &= \sum_k X_{ik}X_{jk} + \xi^2 \sum_{k < \ell} X_{ik} X_{i\ell} X_{jk} X_{j\ell} + \gamma^2 \sum_k X_{ik}^2 X_{jk}^2 \\
	&= \qty(\frac{1}{\sqrt 2\xi} + \frac{\xi}{\sqrt 2}\sum_k X_{ik}X_{jk})^2 + (\gamma^2 - \frac{\xi^2}{2})\sum_k X_{ik}^2 X_{jk}^2 - \frac{1}{2\xi^2}.
	\end{align*}
	We have therefore rewritten $\ang{Z_i, Z_j}$ into something that can be computed in $O(p)$, which is exactly the aim of the kernel trick, since this allows computation of the kernel matrix $K$ in $O(n^2 p)$. 
\end{proof}

\end{document}