\section{The Lasso}
\subsection{Model selection}
We go back to the linear model $Y = X\beta^0 + \eps$ with $\EE[\eps] = 0$ and $\Var(\eps) = \sigma^2 I$. Using the trace trick, one can easily compute that the MSPE of the OLS estimator is given by
\[
\frac1n \EE\norm{X\beta^0 - X\hat\beta^\Rm{OLS}}_2^2 = \frac{\sigma^2 p}{n}. 
\]
Defining $S = \qty{k \mid (\beta^0)_k \neq 0}$, there is often reason to assume that $S$ is small, i.e., $s \ceq \abs{S} \ll p$. If we could fit a model using only the variables in $S$, the MSPE would be much $\frac{\sigma^2 s}{n} \ll \frac{\sigma^2 p}{n}$. 

\paragraph{Best subset selection} A natural way to find $S$ is to consider all possible subsets of $\qty{1, \dotsc, p}$, and pick the best regression procedure using, for example, cross-validation. However, this can become computationally infeasible for moderately large $p$ (say $p \approx 10$). 

\paragraph{Forward selection} This is a greedy way of performing best subset regression. Given a target model size $m$, we first compute the intercept-only model $M_0$, and then one-by-one add the predictor variable that reduces the residual sum of squares the most, until we have a model with $m$ variables. 

\subsection{Lasso estimator}
The \emph{Least absolute shrinkage and selection operator} or \emph{Lasso} is given by
\[
(\hat\mu_\lambda^\Rm{L}, \hat\beta_\lambda^\Rm{L}) \ceq \argmin_{(\mu, \beta) \in \RR \times \RR^p} \frac1{2n}\norm{Y- \mu \vec 1 - X\beta}_2^2 + \lambda \norm{\beta}_1. 
\]
As with ridge regression, we usually centre and scale the matrix $X$, as well as centre the responses $Y$, in which case we find
\[
\hat\beta_\lambda^\Rm{L} = \argmin_{\beta \in \RR^p} \frac1{2n}\norm{Y - X\beta}_2^2 + \lambda \norm{\beta}_1. 
\]

The main difference between the lasso estimator and the ridge regression estimator is that it is likely that the lasso estimator has some zero components. This means that the lasso estimator also estimates which variables are relevant. 

\subsubsection{Slow prediction error rate}
We have the following bound on the prediction error:
\begin{theorem}[Slow rate] \label{thm:lasso_slow_rate}
	Assume $X$ has centred and scaled columns, and assume that $Y$ has been centred, so $Y = X\beta^0 + \eps - \bar\eps \vec 1$. Let $A > 0$ and suppose
	\[
	\lambda = A\sigma\sqrt{\frac{\log(p)}{n}}.
	\]
	Let $\hat\beta = \hat\beta_\lambda^\Rm{L}$, then with probability at least $1 - 2p^{-(A^2/2 - 1)}$ we have that the MSPE satisfies
	\[
	\frac1n \norm{X(\beta^0 - \hat\beta)}_2^2 \leq 4\lambda \norm{\beta^0}_1 = 4A\sigma\sqrt{\frac{\log(p)}{n}} \norm{\beta^0}_1. 
	\]
\end{theorem}

\begin{proof}
	By definition we have
	\[
	\frac1{2n}\norm{Y - X\hat\beta }_2^2 + \lambda \norm{\hat \beta}_1 \leq \frac1{2n}\norm{Y - X\beta^0}_2^2 + \lambda \norm{\beta^0}_1,
	\]
	and rearranging the terms gives
	\[
	\frac1{2n} \norm{X(\beta^0 - \hat\beta)}_2^2 \leq\frac1n \eps\T X (\hat\beta - \beta^0) + \lambda\norm{\beta^0}_1 - \lambda \norm{\hat\beta}_1. 
	\]
	By H\"older's inequality we have $\abs{\eps\T X(\beta^0 - \hat\beta)} \leq \norm{X\T\eps}_\infty \norm{\hat\beta - \beta^0}_1$. 
	Define the event \\ $\Omega = \qty{\norm{X\T\eps}_\infty/n \leq \lambda}$, then conditional on $\Omega$ we find
	\[
	\frac1{n} \norm{X(\beta^0 - \hat\beta)}_2^2 \leq 2\lambda \qty(\norm{\beta^0 - \hat\beta}_1 + \norm{\beta^0}_1 - \norm{\hat\beta}_1) \leq 4\lambda \norm{\beta^0}_1.
	\]
	In \cref{lem:probability_omega_bound}, we will show that $\PP(\Omega) \geq 1 - 2p^{-(A^2/2 - 1)}$, which completes the proof. 
\end{proof}

\subsubsection{Concentration inequalities}
Let $W$ be any random variable and $\phi \colon \RR\to[0,\infty)$ strictly increasing. Then by Markov's inequality we have
\[
\PP(W \geq t) = \PP(\phi(W) \geq \phi(t)) \leq \frac{\EE[\phi(W)]}{\phi(t)}. 
\]
Plugging in $\phi(x) = e^{\alpha x}$ (for some $\alpha > 0$), we get
\[
\PP(W \geq t) \leq e^{-\alpha t} \EE[e^{\alpha W}] = e^{-\alpha t} M_W(\alpha).
\]
Now we can take the infimum over all $\alpha$ on the right-hand side, and we get what is called the \emph{Chernoff bound}:
\[
\PP(W \geq t) \leq \inf_{\alpha > 0} e^{-\alpha t} M_W(\alpha). 
\]

\begin{definition}
	A random variable  $W$ with mean $\mu$ is called \emph{sub-Gaussian with parameter $\sigma > 0$} or $\sigma$-sub-Gaussian if 
	\[
	M_{W - \mu} \leq M_{N(0, \sigma^2)} \quad\text{or equivalently}\quad \EE[e^{\alpha(W - \mu)}] \leq e^{\alpha^2\sigma^2/2} \text{ for all $\alpha \in \RR$}. 
	\]
\end{definition}

We need the following lemma, which characterises an important class of sub-Gaussian random variables:
\begin{lemma}[Hoeffding]
	If $W$ is a mean-zero random variable which takes values in $[a, b]$, then $W$ is sub-Gaussian with parameter $(b-a)/2$. 
\end{lemma}

By Chernoff bounding, we obtain for a $\sigma$-sub-Gaussian random variable $W$ that \[\PP(W - \mu \geq t) \leq e^{-t^2/(2\sigma^2)}.\]

\begin{proposition}
	Let $W_1, \dotsc, W_n$ be independent, mean-zero random variables where $W_i$ is $\sigma_i$-sub-Gaussian. For any $\gamma \in \RR^n$, the random variable $\gamma\T W$ is sub-Gaussian with parameter $\qty(\sum_i \sigma_i^2\gamma_i^2)^{1/2}$.
\end{proposition}
\begin{proof}
	Since the $W_i$ are independent we have for all $\alpha \in \RR$ that
	\begin{align*}
		\EE[e^{\alpha \sum_i \gamma_i W_i}] = \prod_{i=1}^n \EE[e^{\alpha \gamma_i W_i}]  \leq \prod_{i=1}^n e^{\alpha^2\gamma_i^2\sigma_i^2/2} = e^{\alpha^2 \sum_i \gamma_i^2\sigma_i^2/2}.
	\end{align*}
\end{proof}

\begin{lemma} \label{lem:probability_omega_bound}
	Suppose $\eps_1, \dotsc, \eps_n$ are independent mean-zero $\sigma$-sub-Gaussian random variables and let $\lambda \ceq A\sigma \sqrt{\log(p)/n}$. Then
	\[
	\PP\qty(\frac{\norm{X\T\eps}_\infty}{n} \leq \lambda) \geq 1 - 2p\exp(\frac{-\lambda^2}{2(\sigma/\sqrt n)^2}) =  1 - 2p^{-(A^2/2 - 1)}. 
	\]
\end{lemma}

\begin{proof}
	We have
	\[
	\PP\qty(\frac{\norm{X\T\eps}_\infty}{n} > \lambda) = \PP\qty(\bigcup_i \frac{\abs{X_i\T\eps}}{n} > \lambda) \leq \sum_i \PP\qty(\frac{\abs{X_i\T\eps}}{n} > \lambda). 
	\]
	By the previous proposition, both $X_i\T\eps$ and $-X_i\T\eps$ are sub-Gaussian with parameter $\sigma/\sqrt n$, so we have
	\[
	\sum_i \PP\qty(\frac{\abs{X_i\T\eps}}{n} > \lambda) \leq 2p \exp(\frac{-\lambda^2}{2(\sigma/\sqrt n)^2}) = 2p\exp(-A^2\log(p)/2) = 2 p^{-(A^2/2 - 1)}.
	\]
\end{proof}

We now move to finding tail bounds for products of sub-Gaussian random variables. To this end, we consider Bernstein's in equality. 
\begin{definition}
	A random variable $W$ with mean $\mu$ is said to satisfy \emph{Bernstein's condition} with parameters $\sigma > 0, b > 0$ if 
	\[
	\EE\qty(\abs{W - \mu}^k) \leq \frac12 k! \sigma^2 b^{k-2} \qquad k = 2, 3, \dotsc
	\]
\end{definition}

\begin{proposition}[Bernstein's inequality]
	Let $W_1, W_2, \dotsc$ be independent random variables with mean $\mu$, where each $W_i$ satisfies Bernstein's condition with parameter $(\sigma, b)$. Then 
	\begin{enumerate}
		\item $\EE[\exp(\alpha(W_i - \mu))] \leq \exp\qty(\frac{\alpha^2\sigma^2}{2(1 - b \abs{\alpha})})$ for all $\abs{\alpha} < b^{-1}$;
		\item $\PP\qty(\overline{W} - \mu \geq t) \leq \exp\qty(\frac{-nt^2}{2(\sigma^2 + bt)})$ for all $t \geq 0$.  
	\end{enumerate}
\end{proposition}

\begin{proof}
	See notes from topics in statistical theory.
\end{proof}

\begin{lemma}
	Let $W, Z$ be mean-zero, sub-Gaussian random variables with parameters $\sigma_W, \sigma_Z$. Then $WZ$ satisfies the Bernstein condition with parameters $(8\sigma_W\sigma_Z, 4\sigma_W\sigma_Z)$. 
\end{lemma}

\begin{proof}
	Using the tail probability formula for expectations one can prove that any mean-zero sub-Gaussian random variable $X$ satisfies $\EE[X^{2k}] \leq 2^{k+1} \sigma_X^{2k} k!$. 
	
	Furthermore for any random variable $Y$ we have
	\begin{align*}
		\EE\abs{Y - \EE Y}^k &\leq \EE\qty(\abs{Y} + \abs{\EE Y})^k \\
		&= \EE \sum_{t=0}^k \binom kt \abs{Y}^t \abs{\EE Y}^{k-t} \\
		&\overset*\leq \sum_{t=0}^k \binom kt \EE(\abs{Y}^t) \EE(\abs{Y}^{k-t}) \\
		&\overset*\leq \sum_{t=0}^k \binom kt \EE(\abs{Y}^k)^{t/k} \EE(\abs{Y}^k)^{k - t/k} \\
		&= \EE(\abs{Y}^k) \sum_{t=0}^k \binom kt = 2^k \EE\abs{Y}^k,
	\end{align*}
where both equations $*$ are applications of Jensen's inequality. 

From the above bound, we obtian 
\begin{align*}
\EE\abs{WZ - \EE WZ}^k &\leq 2^k \EE(\abs{W}^k \abs{Z}^k) \overset{\Rm{CS}}\leq 2^k (\EE W^{2k})^{1/2} (\EE Z^{2k})^{1/2} \leq 2^{2k+1} \sigma_W^k \sigma_Z^k k! \\
&= \frac{k!}{2} (8\sigma_W\sigma_Z)^2 (4\sigma_W\sigma_Z)^{k-2},
\end{align*}
which proves the claim. 
\end{proof}

\subsubsection{Optimisation theory and convex analysis}
\subsubsection*{Lagrangian optimisation}
\begin{definition}
	For the optimisation problem
	\begin{equation} \label{eq:opt_primal}
	\inf \qty{f(x) \mid g(x) = 0, h(x) \leq 0},
	\end{equation}
 	its \emph{Lagrangian} is given by
 	\[
 	L(x, \lambda,\nu) \ceq f(x) + \lambda h(x) + \nu g(x),
 	\]
 	and its \emph{Lagrange dual function} is
 	\begin{equation} \label{eq:opt_dual}
 	\tilde f(\lambda, \nu) = \inf_x L(x, \lambda, \nu). 
 	\end{equation}
\end{definition}

Note that if $p^*$ is the minimum value of \eqref{eq:opt_primal}, then for any feasible $x$ (i.e., $x$ with $g(x) = 0$, $h(x) \leq 0$)
we have for $\lambda \geq 0$ that 
\[
f(x) \geq L(x, \lambda, \nu) \geq \tilde f(\lambda, \nu).
\]
By taking the infimum over $x$ in the left-hand side and the supremum over $\lambda, \nu$ in the right-hand side we find that
\[
p^*  \geq \sup_{\lambda \geq 0, \nu} \tilde f(\lambda, \nu),
\]
Sometimes we actually have equality here, and in this case we say that the \emph{duality gap} is zero. Note that $\tilde f$ is a concave function, and is therefore easy to maximise. 

A sufficient condition for a zero duality gap is that $L(\cdot, \lambda, \nu)$ is convex and that the problem is feasible. 

\subsubsection*{Convex analysis}

Let $\rext = \RR \cup \qty{\infty}$, and define for any $f \colon \RR^d \to \rext$ its (\emph{effective}) \emph{domain} $\dom(f) \ceq \qty{x \mid f(x) < \infty}$.
\begin{proposition}
	Let $f \colon \RR^d \to \RR$ be twice continuously differentiable. Then $f$ is convex (resp.\ strictly convex) if and only if its Hessian is positive semi-definite (resp.\ positive definite) for all $x \in \RR^d$.
\end{proposition}

\begin{definition}
	Let $f \colon \RR^d \to \rext$  be convex and $x \in \RR^d$. Then $v \in \RR^d$ is called a \emph{subgradient} of $f$ at $x$ if
	\[
	f(y) \geq f(x) + v\T(y-x) \qquad\forall y \in \RR^d. 
	\]
	The set of all subgradients of $f$ at $x$ is called the \emph{subdifferential} of $f$ at $x$, denoted by $\pt f(x)$. 
\end{definition}

\begin{proposition}
	If $f$ is convex and differentiable at $x \in \Int(\dom f)$, then $\pt f(x) = \qty{\nabla f(x)}$. 
\end{proposition}

\begin{proposition}
	Let $f, g \colon \RR^d \to \rext$ be convex with $\Int(\dom f) \cap \Int(\dom g) \neq\emptyset$, $x \in \RR^d$ and $\alpha > 0$. Then $\pt(\alpha f)(x) = \alpha \pt f(x)$ and $\pt(f+g)(x) = \pt f(x) + \pt g(x)$. 
\end{proposition}

\begin{proposition}[KKT conditions]
	Let $f \colon \RR^d \to \rext$ be convex. Then $x^*$ is a global minimiser of $f$ if and only if $0 \in \pt f(x^*)$. 
\end{proposition}

\begin{proof}
	Clearly $x^*$ is a global minimiser of $f$ if and only if $f(y) \geq f(x^*) + 0\T (y - x)$ for all $y \in \RR^d$. 
\end{proof}

We define for $x \in \RR$
\[
\sign(x) \ceq\begin{cases}
	-1, &x<0,\\ 0, &x=0, \\ 1, &x>0,
\end{cases}
\]
and for $x \in \RR^d$ we define $\sign(x) \ceq (\sign(x_1), \dotsc, \sign(x_n))$. 

\begin{proposition}
	Let $x \in \RR^d$ and $A \ceq \qty{j \mid x_j \neq 0}$. Then
	\[
	\pt \norm{x}_1 = \qty{v \in \RR^d : \norm{v}_\infty \leq 1 \text{ and } v_A = \sign(x_A)}. 
	\]
\end{proposition}

\begin{proof}
	Define $g_j(x) = \abs{x_j}$ so that $\norm{\cdot}_1 = \sum_j g_j$ and $\pt \norm{x}_1 = \sum_j \pt g_j(x)$. Now, if $x_j \neq 0$, then $g_j$ is differentiable at $x$ so $\pt g_j(x) = \qty{\sign(x_j) e_j}$. If $x_j = 0$, then it can be shown that
	\[
	\pt g_j(x) = \qty{c e_j \mid c \in [-1, 1]}.
	\]
	Combining the above facts proves the claim. 
\end{proof}

\subsubsection{Lasso solutions}

With the previous proposition, we can write characterise solutions to the Lasso: namely we have
\[
\hat\beta_\lambda^L = \argmin_\beta \frac1{2n} \norm{Y- X\beta}_2^2 + \lambda \norm{\beta}_1 = \argmin_\beta Q(\lambda)
\]
if and only if $0 \in \pt Q(\hat \beta_\lambda^L)$. A simple computation gives, with $A = \qty{k \mid \beta_k \neq 0}$, 
\begin{align*}
\pt Q(\beta) &= \pt(\frac1{2n} \norm{Y- X\beta}_2^2) + \pt(\lambda \norm{\beta}_1) \\
&= \qty{-\frac1n X\T(Y - X\beta)} + \qty{\lambda \nu : \norm{\nu}_\infty \leq 1, \nu_A = \sign(\beta_A)}. 
\end{align*}

We conclude that $\hat\beta$ is a Lasso solution if and only if there exists a vector $\hat\nu$ with $\norm{\hat\nu}_{\infty} \leq 1$ and $\hat\nu_A = \sign(\hat\beta_A)$ such that
\[
\frac1n X\T(Y - X\hat\beta) = \lambda \hat\nu. 
\]

Note that $\hat\beta^L_\lambda$ need not be unique, however, the fitted values are unique:
\begin{proposition}
	$X\hat\beta_\lambda^L$ is unique. 
\end{proposition}

\begin{proof}
	Write $\frac1{2n}\norm{Y- X\beta}_2^2 = Q_1(X \beta)$ and $\norm{\beta}_1 = Q_2(\beta)$. Note that $Q_1$ is strictly convex. Now, for any $t\in (0, 1)$ and any two lasso solutions $\hat\beta, \hat\gamma$ with $Q(\hat\beta) = Q(\hat\gamma) = c^*$ we have
	
	\begin{align*}
		c^* &\leq Q(t\hat\beta + (1-t)\hat\gamma) = Q_1(tX\hat\beta+(1-t)X\hat\gamma) + Q_2(t\hat\beta + (1-t)\hat\gamma) \\
		&\overset\star\leq t Q_1(X\hat\beta) + (1-t)Q_2(X\hat\gamma) + Q_2(t\hat\beta + (1-t)\hat\gamma) \\
		&\leq t Q(\hat\beta) + (1-t)Q(\hat\gamma) = c^*,
	\end{align*}
so all inequalities must be equalities. In particular, by $\star$ we find $Q_1(tX\hat\beta + (1-t)X\hat\gamma) = t Q_1(X\hat\beta) + (1-t)Q_1(X\hat\gamma)$, and since $Q_1$ is strictly convex this implies $X\hat\beta = X\hat\gamma$. 
\end{proof}

\subsubsection{Variable selection}
We consider the noiseless model $Y = X\beta^0$, where no randomness is involved. Let $S = \qty{k \mid \beta_k \neq 0}$, $N = \qty{1, \dotsc, p} \setminus S$, and assume that $S = \qty{1, \dotsc, s}$ (so $\beta_1, \dotsc, \beta_s$ are nonzero and $\beta_{s+1}, \dotsc, \beta_p$ are zero). 

Furthermore, write $X = \mqty[ X_S & X_N]$ and assume that $\rank(X_S) = s$, so that $X_S\T X_S$ is invertible (NB in particular this implies $n \geq s$). 
\begin{theorem}
Let $\lambda > 0$ and define $\Delta = X_N\T X_S(X_S\T X_S)^{-1} \sign(\beta^0_S)$. 
\begin{enumerate}
	\item If $\norm{\Delta}_{\infty} \leq 1$ and for $k = 1, \dotsc, s$ we have
	\[
	\abs{\beta^0_k} > \lambda \abs{\sign(\beta^0_S)\T \qty[\frac1n X_S\T X_S]^{-1}_k},
	\]
	then there exists a Lasso solution $\hat\beta_\lambda^L$ with $\sign(\hat\beta_\lambda^L) = \sign(\beta^0)$. 
	
	\item If there exists a Lasso solution $\hat\beta_\lambda^L$ with $\sign(\hat\beta_\lambda^L) = \sign(\beta^0)$, then $\norm{\Delta}_{\infty} = 1$. 
\end{enumerate}
\end{theorem}

\begin{remark}
	The condition $\norm{\Delta}_{\infty} \leq 1$ is called the \emph{irrepresentable condition}, and it roughly states that none of the insignificant predictors align ``too well'' with the response. \TODO Elaborate on this (lecture 12). 
\end{remark}
\begin{proof}
	Let $\hat\beta$ be any lasso solution and write $\hat S = \qty{k \mid \beta_k \neq 0}$. Using the fact that $\beta^0_N = 0$, the KKT conditions for the Lasso can be expanded as
	\begin{equation} \label{eq:expanded_lasso_conditions}
	\frac1n \mqty(X_S\T X_S & X_S\T X_N \\ X_N\T X_S & X_N\T X_N) \mqty(\beta^0_S - \hat\beta_S\\ -\hat\beta_N) = \lambda \mqty(\hat\nu_S\\ \hat\nu_N)
	\end{equation}
	for some $\hat\nu$ with $\norm{\hat\nu}_\infty \leq 1$ and $\hat\nu_{\hat S} = \sign(\hat\beta_{\hat S})$. 
	\begin{enumerate}
		\item It is easily checked that taking
		\begin{align*}
		(\hat\beta_S, \hat\beta_N) &= \qty(\beta^0_S - \lambda \qty(\frac1n X_S\T X_S)^{-1} \sign(\beta^0_S), 0) \\
		(\hat\nu_S, \hat\nu_N) &= (\sign(\beta^0_S), \Delta)
		\end{align*}
		is a Lasso solution with $\sign(\hat\beta^0_S) = \sign(\hat\beta_S)$. 
		\item If $\sign(\hat\beta) = \sign(\beta^0)$, then $\hat S = S$, $\hat \nu_S= \sign(\beta^0_S)$ and $\hat\beta_N = 0$. The top block of \cref{eq:expanded_lasso_conditions} can be written as
		\[
		\frac1n X_S\T X_S (\beta^0_S - \hat\beta_S) = \lambda \sign(\beta^0_S) \implies \beta^0_S - \hat\beta_S = \lambda \qty[\frac1n X_S\T X_S]^{-1} \sign(\beta^0_S). 
		\]
		Plugging that into the bottom block of the same equation, we get
		\[
		\lambda \frac1n X_N\T X_S \qty[\frac1nX_S\T X_S]^{-1} \sign(\beta^0_S) = \lambda \hat\nu_N \implies \norm{\Delta}_\infty \leq 1. 
		\]
	\end{enumerate}
\end{proof}

\subsubsection{Prediction and estimation}
\begin{definition}
	Let $X \in \RR^{n \times p}$ be a design matrix and $S \subseteq \qty{1, \dotsc, p}$ where $s \ceq \abs{S} > 0$, then the \emph{compatibility constant} $\phi^2 = \phi^2(X, S)$ is defined as
	\[
	\phi^2 \ceq \inf\qty{\frac{\frac1n \norm{X\beta}_2^2}{\frac1s \norm{\beta_S}_1^2} : \beta \in \RR^p, \beta_S \neq 0, \norm{\beta_N}_1 \leq 3 \norm{\beta_S}_1}. 
	\]
	The \emph{compatibility condition} is that $\phi^2 > 0$. 
\end{definition}

Note that if $X_S$ is rank-deficient, then there exists  $\beta_S \neq 0$ with $X_S\beta_S = 0$, and by setting $\beta_N = 0$ we find that $\phi^2 = 0$. Therefore, $X_S$ must have full column rank for the compatibility condition to be satisfied, and in particular we must have $n \geq s$. 

\begin{proposition}
	If $X\T X/n$ has minimal eigenvalue $c_\Rm{min}$ then $\phi^2 \geq c_\Rm{min}$. In particular, if $X$ has full column rank (i.e., $c_\Rm{min} > 0$), the compatibility condition is always satisfied. 
\end{proposition}

\begin{proof}
	We have
	\[
	\norm{\beta_S}_1 = \sign(\beta_S)\T \beta_S\leq \norm{\sign(\beta_S)}_2 \norm{\beta_S}_2 = \sqrt{s} \norm{\beta}_2.
	\]
	Therefore, we find
	\[
	\phi^2 \geq \inf_{\beta \neq 0}  \frac{\frac1n \norm{X\beta}_2^2}{\frac1s \norm{\beta}_1^2} \geq \frac1n \inf_{\beta \neq 0} \frac{\norm{X\beta}_2^2}{\norm{\beta}_2^2} = \inf_{\norm{\beta} =1 } \beta\T \qty(\frac{X\T X}{n}) \beta = c_\Rm{min}. 
	\]
\end{proof}

Note that the converse of this proposition does not necessarily hold: $X$ does not need to have full column rank to satisfy the compatibility condition. In particular, it does not need to be the case that $n < p$. 

Now we present the ``fast'' convergence rate of the lasso (compare with \cref{thm:lasso_slow_rate}):
\begin{theorem}[Fast rate] \label{thm:lasso_fast_rate}
	Let the conditions of \cref{thm:lasso_slow_rate} be satisfied and assume furthermore that the compatibility condition holds. Then with probability at least $1 - 2p^{-(A^2/8 - 1)}$ we have
	\[
	\frac1n \norm{X(\beta^0 - \hat\beta)}_2^2 + \lambda \norm{\hat\beta - \beta^0}_1 \leq \frac{16\lambda^2s}{\phi^2} = \frac{16A^2 \log(p) \sigma^2 s}{\phi^2n }. 
	\]
\end{theorem}

\begin{proof}
	As in the proof of \cref{thm:lasso_slow_rate}, we have
	\[
	\frac1{2n} \norm{X(\beta^0 - \hat\beta)}_2^2 + \lambda \norm{\hat\beta}_1 \leq \frac1n \eps\T X(\hat\beta - \beta^0) + \lambda \norm{\beta^0}_1. 
	\]
	Define the event $\Omega = \qty{2 \norm{X\T \eps}_\infty/n \leq \lambda}$, then by H\"older's inequality we get, conditional on $\Omega$,
	\[
	\frac1n \norm{X(\hat\beta - \beta^0)}_2^2 + 2\lambda \norm{\hat\beta}_1  \leq \lambda \norm{\hat\beta - \beta^0}_1 + 2\lambda \norm{\beta^0}_1. 
	\]
	By \cref{lem:probability_omega_bound}, we find that $\PP(\Omega) \geq 1 - 2p^{-(A^2/8 - 1)}$. 
	
	Dividing by $\lambda$, we have
	\begin{align*}
		\frac1{\lambda n} \norm{X(\hat\beta - \beta^0)}_2^2 + 2 \norm{\hat\beta}_1 &\leq \norm{\hat\beta - \beta^0}_1 + 2 \norm{\beta^0}_1 \\
		\frac1{\lambda n} \norm{X(\hat\beta - \beta^0)}_2^2 + 2 (\norm{\hat\beta_S}_1 + \norm{\hat\beta_N}_1) &\leq \norm{\hat\beta_S - \beta^0_S}_1 + \norm{\hat\beta_N}_1 + 2 \norm{\beta^0_S}_1\\
		\frac1{\lambda n} \norm{X(\hat\beta - \beta^0)}_2^2 +  \norm{\hat\beta_N}_1 &\leq \norm{\hat\beta_S - \beta^0_S}_1 - 2 \norm{\hat\beta_S}_1 + 2 \norm{\beta^0_S}_1 \\
		&\leq 3 \norm{\hat\beta_S - \beta^0_S}_1. 
	\end{align*}
	Now we note that $\norm{\hat\beta_N}_1 = \norm{\hat\beta_N + \beta^0_N}_1$, and we add $\norm{\hat\beta_S - \beta^0_S}_1$ to both sides of the inequality to get
	\begin{equation} \label{eq:error_inequality}
	\frac1{\lambda n} \norm{X(\hat\beta - \beta^0)}_2^2 + \norm{\hat\beta - \beta^0}_1  \leq 4 \norm{\beta_S - \beta^0_S}_1. 
	\end{equation}
	Assume that $\norm{\hat\beta - \beta^0} \neq 0$ (otherwise the theorem is definitely true), and note that $\norm{(\hat\beta - \beta^0)_N}_1 \leq 3 \norm{\hat\beta}_1 \leq 3 \norm{(\hat \beta - \beta^0)_S}$ by our previous inequality, we find that the compatibility constant satisfies
	\[
	\phi^2 \leq \frac{\frac1n \norm{X(\hat\beta - \beta^0)}_2^2}{\frac1s \norm{\hat\beta - \beta^0}_1^2} \implies \frac1{\phi} \geq \sqrt{\frac ns} \frac{\norm{\hat\beta - \beta^0}_1}{\norm{X(\hat\beta - \beta^0)}_2},
	\]
	and plugging this into \cref{eq:error_inequality} we get
	\begin{equation} \label{eq:error_inequality_2}
	\frac1{\lambda n} \norm{X(\hat\beta - \beta^0)}_2^2 + \norm{\hat\beta - \beta^0}_1  \leq 4 \norm{\beta_S - \beta^0_S}_1 \leq \frac4\phi \sqrt{\frac sn} \norm{X(\hat\beta - \beta^0)}_2,
	\end{equation}
	or equivalently
	\[
	\frac1{\lambda n} \norm{X(\hat\beta - \beta^0)}_2^2 - \frac4\phi \sqrt{\frac sn} \norm{X(\hat\beta - \beta^0)}_2 + \norm{\hat\beta - \beta^0}_1 \leq 0. 
	\]
	This is a quadratic inequality in $\norm{X(\hat\beta - \beta^0)}_2$, and solving it yields
	\[
	\frac1{\sqrt n} \norm{X(\hat\beta - \beta^0)}_2 \leq \frac{4\lambda \sqrt s}{\phi}. 
	\] 
	Plugging that into the right hand side of \cref{eq:error_inequality_2} yields the required result. 
\end{proof}

\subsubsection*{The compatibility condition and random design}
We will show that if $X$ is a random matrix, then the compatibility condition holds with high probability. 

Define, for $\Sigma \in \RR^{p \times p}$ and $S \subseteq \qty{1, \dotsc, p}$, the quantity
\[
\phi_\Sigma^2(S) = \inf_{\beta_S \neq 0, \norm{\beta_N}_1 \leq 3 \norm{\beta_S}_1} \frac{\beta\T \Sigma\beta}{\frac1{\abs{S}}\norm{\beta_S}_1^2},
\]
so that the compatiblity constant satisfies $\phi^2 = \phi_{\hat\Sigma}^2(S)$ with $\hat\Sigma \ceq X\T X /n $ and $S$ the support of $\beta^0$. 

To prove our main result, we first need a lemma:
\begin{lemma}
	Suppose $\phi_{\check\Sigma}^2(S) > 0$ and $\max_{jk} \abs{\hat\Sigma_{jk} - \check\Sigma_{jk}} \leq \phi_{\check\Sigma}^2(S) / (32\abs{S})$. Then $\phi_{\hat\Sigma}^2(S) \geq \phi_{\check\Sigma}^2(S)$. 
\end{lemma}

\begin{proof}
	Let $s = \abs{S}$ and $t \ceq \phi_{\check\Sigma}^2/(32s)$. Then by applying H\"older's inequality we obtain
	\[
	\abs{\beta\T (\check\Sigma - \hat\Sigma)\beta} \leq \norm{\beta}_1 \norm{(\check\Sigma - \hat\Sigma)\beta}_\infty \leq \norm{\beta}_1^2 \max_{j,k} \abs{\hat\Sigma_{jk} - \check\Sigma_{jk}}  \leq \norm{\beta}_1^2 t. 
	\]
	Furthermore, if $\norm{\beta_N}_1 \leq 3 \norm{\beta_S}_1$, then
	\[
	\norm{\beta}_1 = \norm{\beta_N}_1 + \norm{\beta_S}_1 \leq 4 \norm{\beta_S}_1 \leq \frac{4\sqrt{\beta\T \check\Sigma\beta}}{\phi_{\check\Sigma} / s} \implies \norm{\beta_1}^2 \leq \frac{16 \beta\T \check\Sigma\beta}{\phi_{\check\Sigma}^2/s^2},
	\]
	and therefore
	\[
	\frac12 \beta\T \check\Sigma\beta = \beta\T \check\Sigma\beta - \frac{\phi_{\check\Sigma}^2}{32s} \cdot \frac{16 \beta\T \check\Sigma\beta}{\phi_{\check\Sigma}^2/s^2} \leq \beta\T \check\Sigma \beta -  t \norm{\beta}_1^2 \leq \beta\T\check\Sigma\beta - \beta\T(\check\Sigma - \hat\Sigma)\beta = \beta\T \hat\Sigma\beta. 
	\]
	Taking the infimum over all $\beta$ with $\norm{\beta_N}_1 \leq 3 \norm{\beta_S}_1$ and $\beta_S \neq 0$, we obtain the required result. 
\end{proof}

Now we can state the main theorem. We consider a sequence of models with sample size $n$ (where $n\to\infty$), and with $p$ and $s$ increasing to $\infty$ as functions of $n$. 

\begin{theorem} \label{thm:lasso_compatibility_condition_asymptotics}
	Suppose the rows of $X$ are i.i.d.\ and each entry is a mean-zero sub-Gaussian random variable with parameter $v$. Let $\Sigma^0 \ceq\EE[X\T X/n]$, and suppose that $s \sqrt{\log(p)/n} \to 0$ as $n\to\infty$. Define
	\[
	\phi^2_{\hat\Sigma, s} \ceq \min_{\abs{S} = s} \phi_{\hat\Sigma}^2(S), \quad \phi_{\Sigma^0, s}^2 \ceq \min_{\abs{S} = s} \phi_{\Sigma^0}^2(S). 
	\]
	If $\phi_{\Sigma^0, s}^2 > c > 0$ (for all $n$), then $\PP(\phi_{\hat\Sigma, s}^2 \geq \phi_{\Sigma^0, s}^2/2) \to 1$ as $n\to\infty$. 
\end{theorem}

\begin{proof}
	By the previous lemma, it suffices to show that 
	\[
	(*) = \PP(\max_{jk}\abs{\hat\Sigma_{jk} - \Sigma^0_{jk}} \geq \phi_{\Sigma^0, s}^2 / (32s)) \to 0 \quad\text{ as } n \to\infty. 
	\]
	Letting $t \ceq \phi_{\Sigma^0, s}^2/(32s)$, then 
	by a union bound and Bernstein's theorem, we have that 
	\begin{align*}
	(*) &\leq p^2 \max_{j, k} \PP\qty(\abs{\sum_{i=1}^n X_{ij} X_{ik}/n - \Sigma^0_{jk}} \geq t) \\
	&\overset\star\leq p^2 2 \exp\qty(-\frac{nt^2}{2(64 v^4 + 4v^2 t)}) \\
	&\leq c_1 \exp(-c_2 \frac{n}{s^2} + c_3 \log(p)) \to 0 \quad\text{if $s \sqrt{\log(p)/n} \to 0$ as $n\to\infty$}. 
	\end{align*}
where $\star$ follows from the fact that $X_{ij}X_{ik}$ satisfies Bernstein's condition with parameters $(8v^2, 4v^2)$. 
\end{proof}

What does the previous theorem tell us? We have a weak growth condition: for example, the theorem holds if $p = n^d$ and $s = (\sqrt n)^{1 -\alpha}$ for any $d, \alpha > 0$. 

\begin{example}
	Suppose the rows of $X$ are taken form a $N_p(0, \Sigma^0)$ distribution, where $\Sigma^0$ has bounded diagonal entries and the smallest eigenvalue of $\Sigma^0$ is $c_\Rm{min} > c > 0$ (for all $n$). Then $\PP(\phi_{\hat\Sigma, s}^2 \geq \frac{c_\Rm{min}}{2}) \to 1$ as $n\to\infty$ if the growth condition is satisfied. This holds for example if $\Sigma^0 = I_p$. 
\end{example}

\begin{example}
	Suppose $X$ has i.i.d.\ entries $\PP(X_{ij} = 1) = \frac12 = \PP(X_{ij} = -1)$. Then by Hoeffding's lemma, $X_{ij}$ is sub-Gaussian, and therefore the theorem holds under appropriate conditions. 
\end{example}

\begin{example}
	Let $W$ be a matrix whose rows are vectors in a basis, and let $X$ have rows which are random subsets of $W$. This does not satisfy the conditions of \cref{thm:lasso_compatibility_condition_asymptotics}, yet the lasso still works very well in this case. 
\end{example}

Generally, the Lasso demonstrates a very low estimation if $n/p$ is much larger than $s/p$, but a very high estimation error if $s/p$ is much larger than $n/p$. Usually, there is a sharp ``phase transition'' here.

\subsubsection{Computation}
The \emph{coordinate descent} algorithm is an algorithm for minimising functions of the form $f \colon \RR^d \to \RR$, and works particularly well for functions of the form
\[
f(x) = g(x) + \sum_{j=1}^d h_j(x), 
\]
where $g$ is convex and differentiable and the $h_j \colon \RR \to \RR$ are convex and continuous. Note that the Lasso objective function is clearly of this form. 
We start with an initial guess $x^{(0)}$ for a minimiser, and then for $m = 1, 2, \dotsc$ we choose, for $i = 1, \dotsc, d$, 
\[
x^{(m)}_i = \argmin_{x_i \in \RR} f\qty(x_1^{(m)}, \dotsc, x_{i-1}^{(m)}, x_i, x_{i+1}^{(m-1)}, \dotsc, x_d^{(m-1)}). 
\]
It can be shown that for the Lasso objective function, each step of the coordinate descent algorithm admits a closed-form solution. 

We have the following:
\begin{theorem}[Tseng] \label{thm:tseng} If $A_0 \ceq \qty{x \mid f(x) \leq f(x^0)}$ is compact, then every converging subsequence of $(x^{(m)})$ converges to a minimiser of $f$. 
\end{theorem}

Using this theorem, we can see the following:
\begin{corollary}
	Suppose $A^0$ is compact. Then 
	\begin{enumerate}
		\item There exists a mimiser $x^*$ of $f$, and $f(x^{(m)}) \to f(x^*)$;
		\item If $x^*$ is the unique minimiser of $f$, then $x^{(m)} \to x^*$. 
	\end{enumerate}
\end{corollary}

\begin{proof}
	Since $f$ is continuous, it attains a minimum $f(x^*)$ on $A_0$. Suppose $f(x^{(m)}) \not\to f(x^*)$. Note that $f(x^{(m)})$ is a monotone decreasing sequence, so it must converge. Furthermore, $(x^{(m)}) \subseteq A_0$, so there exists a subsequence of $(f(x^{(m)}))$ converging to $f(x^*)$ by \cref{thm:tseng}. This proves that $f(x^{(m)}) \to f(x^*)$. 
	
	Suppose that $x^*$ is the unique minimiser of $f$, then by \cref{thm:tseng} we know that every convergent subsequence of $x^{(m)}$ converges to $x^*$. Since $x^{(m)}$ is bounded, this proves that $x^{(m)} \to x^*$ as well.  
\end{proof}

Often, we want to solve the Lasso on a grid of values $\lambda_0 > \dotsb > \lambda_L$. To speed up the coordinate descent procedure, we can use the minimiser for $\lambda_{\ell -1}$ as a starting point for coordinate descent on $\lambda_\ell$. An ``active set'' strategy can speed up computation even further: 
\begin{enumerate}
	\item Initialise $A_\ell \ceq \qty{k \mid \beta^L_{\ell - 1, k} \neq 0}$;
	\item Perform coordinate descent with $\lambda_\ell$ only on the coordinates in $A_\ell$, call this result $\hat\beta$. 
	\item Let $V$ be the set of coordinates which violate the KKT conditions;
	\item If $V = \emptyset$, set $\hat\beta_{\lambda_\ell}^L = \hat\beta$. Else, update $A_\ell = A_\ell \cup V$ and return to step 2. 
\end{enumerate}

\begin{remark}
	There are multiple possible algorithms for computing Lasso solutions, coordinate descent is simply one option. 
\end{remark}

\subsection{Extensions of the Lasso}
One of the ways to extend the Lasso is to change the loss function (for example,
the logistic loss function can be used if $Y_i \in \qty{-1, 1}$). 
There are other extensions. 

\subsubsection{Structural penalties}
\begin{example}[Group lasso]
	Let $G_1, \dotsc, G_q$ be a partition of $\qty{1, \dotsc, p}$, then instead of the Lasso penalty we can use
	\[
	\lambda \sum_{j=1}^q m_j \norm{\beta_{G_j}}_2,
	\]
	where $m_j$ is typically chosen as $\sqrt{\abs{G_j}}$. This estimator will generally ensure that either the entire group $G_j$ is shrunk to 0 or not at all. 
\end{example}

\begin{example}[Fused lasso]
	Suppose the $(\beta_j)$ are ordered, for example if $j$ fixes time. If we expect $(\beta_j)$ to be piecewise constant (or at least that $\beta^0_j$ is close to $\beta^0_{j+1}$), we can use the penalty 
	\[
	\lambda_1 \sum_{j=1}^{p-1} \abs{\beta_j - \beta_{j+1}} + \lambda_2 \norm{\beta}_1, 
	\]
	where the term $\lambda_2 \norm{\beta}_1$ can be omitted if shrinkage is not desired. 
\end{example}

\subsubsection{Debiasing the Lasso}
A useful property of the Lasso is that non-significant coefficients can be shrunk towards 0, but this same effect also causes many significant coefficients to be shrunk  towards 0. A possible solution is to only use the Lasso to estimate which coefficients are 0 and then use a different estimator (i.e., OLS) for the other coefficients. 

Another option is to use a non-convex penalty function which does not shrink large values of $\hat\beta$ as much. 
