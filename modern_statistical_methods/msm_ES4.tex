\documentclass{article}

\usepackage[utf8]{inputenc}
\usepackage[english]{babel}
\usepackage[margin=3cm]{geometry}
\usepackage[normalem]{ulem}
\usepackage{hyperref}
\usepackage[shortlabels]{enumitem}
\usepackage{mathtools, amsmath, amssymb, amsthm, mdframed, bbm, graphicx, float, physics, xcolor, cleveref}

\hypersetup{
    colorlinks   = true, %Colours links instead of ugly boxes
    urlcolor     = blue, %Colour for external hyperlinks
    linkcolor    = blue, %Colour of internal links
    citecolor   = red %Colour of citations
}

% Definition of numbered environments.
% Usage: \begin{theorem} ... \end{theorem}
% Remark has no numbering.
\theoremstyle{plain}
\newtheorem{question}{Question}
\newtheorem{theorem}{Theorem}
\newtheorem{lemma}{Lemma}
\theoremstyle{remark}
\newtheorem*{remark}{Remark}

% Question with box around it
\newenvironment{qbox}{\begin{mdframed}\begin{question}}{\end{question}\end{mdframed}}

% A proof with "solution" instead of "proof" and no QED symbol
\newenvironment{solution}{\begin{proof}[Solution]\renewcommand\qedsymbol{}}{\end{proof}}

% Some renewed commands
\renewcommand{\vec}{\mathbf}
\renewcommand{\emptyset}{\varnothing}
\renewcommand{\epsilon}{\varepsilon}
\renewcommand{\theta}{\vartheta}
\renewcommand{\phi}{\varphi}

% Frequently used math alphabets
\newcommand{\Bb}{\mathbb}
\newcommand{\Cal}{\mathcal}
\newcommand{\Bf}{\mathbf}
\newcommand{\Rm}{\mathrm}

% Frequently used letters in the blackboard alphabet
\newcommand{\CC}{\Bb C}
\newcommand{\NN}{\Bb N}
\newcommand{\PP}{\Bb P}
\newcommand{\QQ}{\Bb Q}
\newcommand{\RR}{\Bb R}
\newcommand{\EE}{\Bb E}

% Usage: \ang{...} is equivalent to \langle ... \rangle, while \ang*{...} is equivalent to \left\langle ... \right\rangle
% For other delimiters: use \qty from the physics package (i.e., \qty(...))
\DeclarePairedDelimiter{\ang}{\langle}{\rangle}
\DeclarePairedDelimiter{\floor}{\lfloor}{\rfloor}
\DeclarePairedDelimiter{\ceil}{\lceil}{\rceil}

% Frequently used commands
\newcommand{\T}{^\top} % Matrix transpose A\T
\newcommand{\C}{^\complement} % Set complement A\C
\renewcommand{\P}{^\perp} % Orth. complement A\P
\newcommand\ceq\coloneqq % Definitions :=
\newcommand\pow{\Cal P} % Power sets
\newcommand\eps\epsilon
\newcommand\ind{\mathbbm 1} % Blackboard 1 for indicator functions
\newcommand\restr{\mathord\restriction}
\newcommand\TODO{{\color{red} TODO: }}

% Functions that appear frequently
\DeclareMathOperator{\sign}{sign}
\DeclareMathOperator{\Int}{Int}
\DeclareMathOperator{\Span}{Span}
\DeclareMathOperator{\Var}{Var}
\DeclareMathOperator*{\argmin}{arg\,min}
\DeclareMathOperator*{\argmax}{arg\,max}

\title{Modern Statistical Methods --- Example Sheet 4} % subject
\author{Lucas Riedstra}
%\date{...} % date

\begin{document}
\maketitle
\begin{question}
	content...
\end{question}

\begin{proof}
	content...
\end{proof}

\begin{question}
	Consider the matrix $Q = \mqty[1 & 0 \\ 0 & 1.01]$ and its perturbation $\hat Q = Q + \mqty[0 & 0.01 \\ 0.01 & 0]$. Show that the eigenvalues are stable to perturbation, but the top eigenvector is not.
\end{question}

\begin{proof}
	Clearly $Q$ has eigenvalues $\lambda_1 = 1.01, \lambda_2 = 1$, with eigenvectors $\vec x_1 = \vec e_2, \vec x_2 = \vec e_1$. 
	
	Now, we have
	\[
	\hat Q = \mqty[ 1 & 0.01 \\ 0.01 & 1.01] \implies \det(\lambda I - \hat Q) = (\lambda - 1)(\lambda - 1.01) - 0.01^2 = \lambda^2  - 2.01\lambda + 1.0099,
	\]
	which gives eigenvectors
	\[
	\lambda = \frac{2.01 \pm \sqrt{4.0401 - 4.0396}}{2} = 1.005 \pm \frac{\sqrt 5}{2} \cdot 0.01 \approx \qty{0.9938, 1.0162}
	\]
	which is indeed very close to the eigenvalues 1 and 1.01. 
	
	Letting $\tilde \lambda_1 = 1.005 + 0.01\frac{\sqrt5}{2}$, we have by the top row that
	\[
	\hat Q x = \lambda x \iff x + 0.01 y = \lambda x \implies 
	x = \frac{0.01}{\lambda - 1}y = \frac12(\sqrt 5 - 1) y,
		\]
		so the top eigenvector has been perturbed from $(0, 1)\T$ to $(\frac12(\sqrt 5 - 1), 1)\T$, which upon normalisation is a change of more than $\pi/6$ radians. 
\end{proof}
\end{document}