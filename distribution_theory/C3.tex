
\section{Tempered distributions and Fourier analysis}
\subsection{Functions of rapid decay}
\begin{definition}
	For any $f \colon \RR^n \to \CC$ and multi-indices $\alpha, \beta$ we define $\norm{f}_{\alpha, \beta} \ceq \sup_{x \in \RR^n} \abs{x^\alpha \pt^\beta\phi}$. 
	
	We define the \emph{Schwartz space}
	\[
	\Cal S(\RR^n) \ceq \qty{f \in C^\infty(\RR^n) : \norm{f}_{\alpha, \beta} < \infty \text{ for all $\alpha, \beta$}}.
	\]
	
	We say that a sequence $(\phi_n) \subseteq \SS(\RR^n)$ converges to 0 if $\norm{\phi_m}_{\alpha, \beta} \to 0$ for every $\alpha, \beta$.  
\end{definition}

\begin{example}
	The function $x \mapsto \exp(-\norm{x}^2)$ lies in $\SS(\RR^n)$. 
\end{example}

\begin{proposition}
	For all $n$ we have that $\SS(\RR^n) \subseteq L^1(\RR^n)$. 
\end{proposition}

\begin{proof}
	Let $\phi \in \SS(\RR^n)$, then for all $N \in \NN$ we have
	\[
	\int_{\RR^n} \abs{\phi(x)} \dd{x} = \int_{\RR^n} (1 + \norm{x})^{-N} (1 + \norm{x})^N \abs{\phi(x)} \dd{x} \overset?\leq C\sum_{\abs{\alpha} \leq N} \norm{\phi}_{\alpha, 0} \int_{\RR^n} (1 + \norm{x})^{-N} \dd{x}.
	\]
	
	Since $\int_{\RR^n} (1 + \norm{x})^{-N} \dd{x}$ is finite for $N$ large enough (??), this proves the claim. 
\end{proof}


\begin{definition}
	A linear map $u \colon \SS(\RR^n) \to \CC$ is called a \emph{tempered distribtion} if there exists constants $C, N$ such that
	\[
	\abs{\ang{u, \phi}} \leq C \sum_{\abs{\alpha}, \abs{\beta} \leq N} \norm{\phi}_{\alpha, \beta} \quad\text{for all $\phi \in \SS(\RR^n)$}. 
	\]
	The space of tempered distributions is denoted by $\SS'(\RR^n)$. 
\end{definition}

This definition is equivalent to sequential continuity. 

\subsection{The Fourier transform on Schwartz functions}

\begin{convention}
	We write $D \ceq -i\pt$ and $D^\alpha = (-i)^{\abs{\alpha}} \pt^\alpha$. 
\end{convention}

\begin{definition}
	For $f \in L^1(\RR^n)$, define the \emph{Fourier transform} of $f$ by  
	\[
	[\FF(f)](\lambda) = \hat f(\lambda) \ceq \int_{\RR^n} e^{-i\lambda \vdot x} f(x) \dd{x} \quad\text{where $\lambda \in \RR^n$}. 
	\]
\end{definition}

\begin{lemma}
	If $f \in L^1(\RR^n)$, then $\hat f$ is continuous. 
\end{lemma}

\begin{proof}
	If $\lambda_m \to \lambda \in \RR^n$, then by the dominated convergence theorem we have
	\[
	\hat f(\lambda_m) = \int_{\RR^n} e^{-i\lambda_m \vdot x} f(x) \dd{x} \overset{\text{DCT}}= \int_{\RR^n} e^{-i\lambda \vdot x} f(x) \dd{x} = \hat f(\lambda),
	\]
	where we were able to use the dominated convergence theorem since the integrand is absolutely bounded by $\abs{f}$ and $f \in L^1$. 
\end{proof}

It turns out that this idea generalises: the faster the function $f$ decays, the smoother the Fourier transform $\hat f$ is.  

\begin{lemma} \label{lem:differentiate_fourier}
	For $\phi \in \SS(\RR^n)$, we have $\FF[D_x^\alpha \phi](\lambda) = \lambda^\alpha \hat\phi(\lambda)$ and $\FF[x^\beta\phi](\lambda) = (-1)^{\abs{\beta}} D_\lambda^\beta \hat\phi(\lambda)$. 
\end{lemma}

\begin{proof}
	Since $\abs{x^\alpha D^\beta\phi} \to 0$ as $\norm{x} \to \infty$, we have using integration by parts
	\begin{align*}
		\FF[D_\lambda^\alpha\phi](\lambda) &= \int_{\RR^n} e^{-i\lambda \vdot x} D_x^\alpha \phi(x) \dd{x} \\
		&= (-1)^{\abs{\alpha}} \int_{\RR^n} D_x^\alpha(e^{-i\lambda \vdot x}) \phi(x) \dd{x} \\
		&= \lambda^\alpha\hat\phi(\lambda). 
	\end{align*}
	
	For the second part of the lemma, by differentiation under the integral sign we get
	\begin{align*}
		 \FF[x^\beta \phi](\lambda) &=  \int_{\RR^n} e^{-i\lambda \vdot x} x^\beta \phi(x) \dd{x} \\
		 &= \int_{\RR^n} ((-D_\lambda)^\beta e^{-i\lambda \vdot x}) \phi(x) \dd{x} \\
		 &= (-1)^{\abs{\beta}} D_\lambda^\beta \hat\phi(\lambda).
	\end{align*}
\end{proof}


We define the \emph{inverse Fourier transform} by 
\[
\FF^{-1}[\hat f](x)  \ceq \frac{1}{(2\pi)^n} \int_{\RR^n} e^{i\lambda \vdot x} \hat f(\lambda) \dd{\lambda}. 
\]
We will now show that on $\SS(\RR^n)$, the inverse Fourier transform is indeed an inverse:
\begin{theorem} \label{thm:fourier_transform_isomorphism}
	The Fourier transform $\FF$ defines a continuous isomorphism from $\SS(\RR^n)$ to itself. 
\end{theorem}

\begin{proof}
	Let $\phi \in \SS(\RR^n)$. First, we show that $\hat\phi \in \SS(\RR^n)$: by the previous lemma we have for multi-indices $\alpha, \beta$ that
	\begin{align}
		\abs{\lambda^\alpha (-D_\lambda)^\beta \hat\phi(\lambda)} &= \abs{\lambda^\alpha \FF[x^\beta\phi](\lambda)} = \abs{\FF[D^\alpha_x (x^\beta\phi)](\lambda)} = \abs{\int_{\RR^n} e^{-i\lambda \vdot x} D^\alpha(x^\beta\phi) \dd{x}} \nonumber \\
		&\leq \int_{\RR^n} \abs{D^\alpha (x^\beta\phi)} \dd{x} \label{eq:fourier_Schwartz_estimate},
	\end{align}
which is finite since $D^\alpha (x^\beta\phi)$ is also a Schwartz function and therefore integrable. 

From the previous lemma we also infer that $\hat\phi$ is smooth, so indeed we have $\hat\phi \in \SS(\RR^n)$. 
From \cref{eq:fourier_Schwartz_estimate} it is also easily seen that if $\phi_m \to 0$ in $\SS(\RR^n)$, then $\hat\phi_m \to 0$ in $\SS(\RR^n)$ also, which shows that $\FF$ is continuous. 

To prove surjectivity and injectivity, we will show that $\FF^{-1}[\FF[f]](x) = f(x)$ (???). Indeed we have
\begin{align*}
	\FF^{-1} [\FF[\phi]](x) &= \frac{1}{(2\pi)^n} \int_{\RR^n} \int_{\RR^n} e^{i\lambda \vdot (x-y)} \phi(y) \dd{y}\dd{\lambda} \\
	&= \frac{1}{(2\pi)^n} \lim_{\eps \downarrow 0} \int_{\RR^n} \int_{\RR^n} e^{i \lambda \vdot ( x- y) - \eps\norm{\lambda}^2} \phi(y) \dd{y}\dd{\lambda} \\
	&\overset\star= \frac{1}{(2\pi)^n}\lim_{\eps \downarrow 0} \int_{\RR^n} \phi(y)\int_{\RR^n} e^{i\lambda \vdot (x- y) - \eps\norm{\lambda}^2} \dd{\lambda} \dd{y},
	%&= \frac{1}{(2\pi)^n} \int_{\RR^n}\int_{\RR^n} e^{i\lambda(x- y)} \dd{\lambda} \dd{y}
\end{align*}
where $\star$ follows from Fubini's theorem. 

Now we note that 
\[
\int_{\RR^n} e^{i\lambda\vdot (x-y) - \eps\norm{\lambda}^2} \dd{\lambda} = \prod_{i=1}^n \int_{\RR} e^{i\lambda_j(x_j - y_j) - \eps\lambda_j^2} \dd{\lambda} \overset{\star\star}= \prod_{i=1}^n \qty(\frac\pi e)^{1/2} e^{- \frac{(x_i - y_i)^2}{4\eps}} = \qty(\frac\pi\eps)^{n/2} e^{- \frac{\norm{x - y}^2}{4\eps}}.
\] 
To explain $\star\star$, \TODO. 


and plugging that into the above yields 
\begin{align*}
\FF^{-1}[\FF[\phi]](x) &= \lim_{\eps\downarrow 0} 2^{-n}(\pi \eps)^{-n/2} \int_{\RR^n} \phi(y) e^{-\norm{x-y}^2/(4\eps)} \dd{y} \\
&\overset{\star\star\star}= \pi^{-n/2} \lim_{\eps\downarrow0}\int_{\RR^n} \phi(x - 2\sqrt\eps y') e^{-\norm{y'}^2} \dd{y} \\
&\overset{\text{DCT}}=\pi^{-n/2} \phi(x) \int_{\RR^n} e^{-\norm{y'}^2} \dd{y} = \phi(x),
\end{align*}
where $\star\star\star$ follows from the substitution $x - y = 2\sqrt\eps y'$.

Finally, continuity of $\FF^{-1}$ is easily shown with an argument analogous to that for continuity of $\FF$ (????). 
\end{proof}

\subsection{The Fourier transform on tempered distributions}
\begin{lemma}
	For $\phi, \psi \in \SS(\RR^n)$, we have $\int_{\RR^n} \phi(x) \hat\psi(x) \dd{x} = \int_{\RR^n} \hat\phi(x) \psi(x) \dd{x}$. 
\end{lemma}

\begin{proof}
	This follows from Fubini's theorem:
	\[
	\int_{\RR^n}\phi(x) \hat \psi(x) \dd{x} = \int_{\RR^n}\int_{\RR^n} \phi(x) \psi(\lambda) e^{-i\lambda \vdot x} \dd{\lambda} \dd{x} = \int_{\RR^n}\int_{\RR^n} \psi(\lambda)  \phi(x) e^{-i\lambda \vd x} \dd{x} \dd{\lambda} = \psi(\lambda) \psi(\lambda) \hat\phi(\lambda) \dd{\lambda}. 
	\]
\end{proof}

The above result can be rewritten as $\ang{\hat\phi, \psi} = \ang{\phi, \hat\psi}$, which motivates the definition of the Fourier transform for tempered distributions:
\begin{definition}
	For $u \in \SS'(\RR^n)$, we define its Fourier transform by
	\[
	\ang{\hat u, \phi} \ceq \ang{u, \hat\phi} \quad\text{for all $\phi \in \SS(\RR^n)$}. 
	\]
\end{definition}
Using sequential continuity and \cref{thm:fourier_transform_isomorphism}, it is easily seen that $\hat u$ is indeed an element of $\SS'(\RR^n)$. 

\begin{example}
	It is easily checked that $\delta_0 \in \SS'(\RR^n)$, and we can compute
	\[
	\ang{\hat{\delta_0}, \phi} = \ang{\delta_0, \hat\phi} = \hat\phi(0) = \int_{\RR^n} \phi(x) \dd{x} = \ang{1, \phi},
	\]
	so we can write $\hat{\delta_0} = 1$. Analogously, by the Fourier inversion theorem we have
	\[
	\ang{\hat 1, \phi} = \ang{1, \hat\phi} = \int_{\RR^n} \hat\phi(\lambda) \dd{\lambda} = (2\pi)^n \phi(0) = \ang{(2\pi)^n \delta_0, \phi}, 
	\]
	so we write $\hat 1 = (2\pi)^n \delta_0$. 
\end{example}

We can easily generalise \cref{lem:differentiate_fourier} to the Fourier transform on $\SS'(\RR^n)$, so
\[
\FF[D^\alpha u] = \lambda^\alpha \hat u, \quad \FF[x^\beta u] = (-D^\beta) \hat u. 
\]

\begin{theorem}
	The Fourier transform $\FF$ extends to a continuous isomorphism on $\SS'(\RR^n)$. 
\end{theorem}

\begin{proof}
	We claim that $\refl u = (2\pi)^{-n} \FF[\FF[u]]$. To check this, note that by the Fourier inversion theorem we have for $\phi \in \SS(\RR^n)$ that
	\[
	\refl\phi(x) = \phi(-x) = (2\pi)^{-n} \int_{\RR^n} e^{-i\lambda\vd x} \hat\phi(\lambda) \dd{\lambda} = (2\pi)^{-n} \FF[\FF[\phi]](x),
	\]
	and therefore 
	\[
	\ang{\refl u, \phi} = \ang{u, \refl\phi} = \ang{u, (2\pi)^{-n} \FF[\FF[\phi]]} = \ang{(2\pi)^{-n} \FF[\FF[u]], \phi}. 
	\]
	
	This shows that $\FF$ is bijective (since $\FF\circ \FF$ is bijective). For continuity of $\FF$ and its inverse: using \cref{thm:fourier_transform_isomorphism}, we find
	\begin{align*}
	&\phantom{\iff} u_m \to 0 \text{ in $\SS'(\RR^n)$} \\
	&\iff \ang{u_m, \phi} \to 0 \ \forall \phi \in \SS(\RR^n) \\
	&\iff \ang{u_m, \hat\phi} \to 0 \ \forall \phi \in \SS(\RR^n) \\
	&\iff \ang{\hat u_m, \phi} \to 0 \ \forall \phi \in \SS(\RR^n) \\
	&\iff \hat u_m \to 0 \text{ in $\SS'(\RR^n)$}.
	\end{align*}
\end{proof}

\subsection{Sobolev spaces}
\begin{convention}
	We write $\ang{\lambda} \ceq (1 + \norm{\lambda}^2)^{1/2}$ for $\lambda \in \RR^n$. Note that $\ang{\lambda} \sim 1$ as $\norm{\lambda} \to 0$ and $\ang{\lambda} \to \norm{\lambda}$ as $\norm{\lambda} \to \infty$. 
\end{convention}

\begin{definition}
	For $s \in \RR$, define the \emph{Sobolev space} $H^s(\RR^n)$ to be the set of tempered distributions $u \in \SS'(\RR^n)$ for which $\hat u$ can be identified with a measurable function $\hat u \colon \RR^n \to \CC$ such that $\ang{\lambda}^s \hat u \in L^2(\RR^n)$. 
	
	For $X \subseteq \RR^n$ open, we define the \emph{localised Sobolev space} $H^s\loc(X)$ by setting
	\[
	u \in H^s\loc(X) \iff \phi u \in H^s(\RR^n) \text{ for all $\phi \in \DD(X)$}. 
	\]
\end{definition}

% Suppose we have an open subset $X \subseteq \RR^n$, then we can define the \emph{localised} Sobolev space $H^s_\Rm{loc}(X)$ by sayin

\begin{lemma}[Sobolev lemma]
	If $u \in H^s(\RR^n)$ with $s > \frac n2$, then  $u$ is continuous. 
\end{lemma}

\begin{proof}
	We will show that $\hat u$ is integrable. By Cauchy-Schwarz, we have
	\begin{align*}
		\int_{\RR^n} \abs{\hat u(\lambda)} \dd{\lambda} &= \int_{\RR^n}  \ang{\lambda}^{-s} \ang{\lambda}^s \abs{\hat u(\lambda)} \dd{\lambda} \\
		&\leq \qty(\int_{\RR^n}  \ang{\lambda}^{-2s}\dd{\lambda})^{1/2} \norm{u}_{H^s} \\
		&= C \norm{u}_{H^s} \qty(\int_0^\infty r^{n-1} (1 + r^2)^{-s} \dd{r})^{1/2}, 
	\end{align*}
where the last line follows from using polar coordinates and $C$ is the area of the $(n-1)$-sphere. 

Writing $s = \frac n2 + \eps$, we find that the integrand $r^{n-1}(1 + r^2)^{-s}$ is of order $O(r^{-1 - 2\eps})$ as $r \to\infty$, and therefore the integral is finite, so indeed we have $\hat u \in L^1(\RR^n)$. 

By applying \cref{thm:fourier_transform_isomorphism} to a test function, we can show that $u = (2\pi)^{-n} \int_{\RR^n} e^{i\lambda \vd x} \hat u(\lambda) \dd{\lambda}$, which is continuous by the dominated convergence theorem. 
\end{proof}

\begin{corollary}
	If $u \in H^s(\RR^n)$ for every $s > n/2$, then $u \in C^\infty(\RR^n)$. 
\end{corollary}

%\begin{proof}
%	We will show for any multi-index $\alpha$ of order $N$ that $\lambda^\alpha \hat u$ is integrable: this shows (??) that $D^\alpha u$ exists. We have for every $s > n/2$ that
%	\begin{align*}
%		\int_{\RR^n} \abs{\lambda^\alpha \hat u(\lambda)} \dd{\lambda} \les \int_{\RR^n} \ang{\lambda}^N  \abs{\hat u(\lambda)}\dd{\lambda} = \int_{\RR^n} \ang{\lambda}^{N - s} \ang{\lambda}^s \abs{\hat u(\lambda)}\dd\lambda \leq \qty(\int_{\RR^n} \ang{\lambda}^{2(N -s)})^{1/2} \norm{u}_{H^s}.
%	\end{align*}
%Analogously as in the proof of the previous theorem, we get an integrand $r^{n-1}(1 + r^2)^{N-s} = O(n-1-2s + 2N)$
%\end{proof}
