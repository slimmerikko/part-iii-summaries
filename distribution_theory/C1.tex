\section{Distributions}
%\begin{recap}
%    A \emph{topological field} $\KK$ is a field endowed with a topology such that addition, multiplication, and inversion are continuous.
%    
%    A \emph{topological vector space} $V$ over a topological field $\KK$ is a vector space endowed with a topology such that vector addition and scalar multiplication are continuous. 
%    
%    The \emph{topological dual space} $V^*$ of $V$ is the set of all continuous linear maps from $V$ to $\KK$. 
%\end{recap}
\subsection{Test functions and distributions}
\begin{definition}
    Let $X \subseteq \RR^n$ be open, then we define the set of \emph{test functions} on $X$ as
    \[
    \Cal D(X) \ceq C^\infty_0(X) = \qty{f \colon X \to \CC \mid f \text{ is smooth with compact support}}. 
    \]
\end{definition}

\begin{definition} \label{def:test_convergence}
    Let $(\phi_m) \subseteq \Cal D(X)$. We say that $(\phi_m) \to 0$ in $\Cal D(X)$ if
    \begin{enumerate}
        \item  there exists a compact $K \subseteq X$ such that $\supp\phi_m \subseteq K$ for all $m$;
        \item $\pt^\alpha \phi_m \to 0$ uniformly for each multi-index $\alpha$.
    \end{enumerate}
\end{definition}

Note that, for any $\phi, \psi \in \Cal D(X)$ and any multi-index $\alpha$ we have
\[
\int_X \phi\cdot  \pt^\alpha \psi \dd{x} = (-1)^{\abs{\alpha}} \int_X \psi \cdot \pt^\alpha \phi \dd{x}, 
\]
which follows from partial integration and the fact that all boundary terms vanish since $\phi$ and $\psi$ have compact support. 

Also, by Taylor's theorem, for any $\phi \in \Cal D(X)$, $x, h \in X$ and $N \in \NN$ we have
\[
\phi(x + h) = \sum_{\abs{\alpha} \leq N} \frac{h^\alpha}{\alpha!} \pt^\alpha \phi(x) + R_N(x, h) \quad\text{where } R_N(x, h) = o(\abs{h}^N) \text{ uniformly in $x$}. 
\]

\begin{definition}
    A \emph{distribution} on $X$ is a linear map $u \colon \Cal D(X) \to \CC$ if for every compact set $K \subseteq X$ there exist constants $C, N$ such that for all $\phi \in \Cal D(X)$ with $\supp\phi\subseteq K$ we have
    \begin{equation} \label{eq:seminorm_cond}
        \abs{u(\phi)} \leq C \sum_{\abs{\alpha} \leq N} \sup \abs{\pt^\alpha \phi}.
    \end{equation}

Condition \ref{eq:seminorm_cond} is called the \emph{seminorm condition}. If, in the seminorm condition, the same $N$ can be used for every compact set $K \subseteq X$, then the least such $N$ is called the \emph{order} of $u$, written $\ord(u)$. 

The set of all distributions in $X$ is denoted $\Cal D'(X)$. 
\end{definition}

If $u \in \DD'(X)$ and $\phi \in \DD(X)$, then instead of $u(\phi)$ we usually write $\ang{u, \phi}$.

\begin{recap}
    A function $f \colon X \to \CC$ is called \emph{locally integrable} if $\int_K \abs{f} \dd{x} < \infty$ for all compact $K \subseteq X$. 
    
    The set of locally integrable functions on $X$ is denoted $\locint(X)$. 
\end{recap}

\begin{example}
    Let $M \in \NN$ and let $f_\alpha \in \locint(X)$ for all $\abs{\alpha} \leq M$. Define the linear map $T \colon \DD(X) \to \CC$ by
    \[
    \ang{T, \phi} \ceq \sum_{\abs{\alpha} \leq M} \int_X f_\alpha \cdot \pt^\alpha \phi \dd{x}. 
    \]
    It is trivial that $T$ is linear, and we verify that $T$ is a distribution as follows: take $\phi \in \Cal D(X)$ with $\supp\phi \subseteq K$. Then we have
    \begin{align*}
        \abs{\ang{T, \phi}} &\leq \sum_{\abs{\alpha} \leq M} \int_K \abs{f_\alpha} \cdot \abs{\pt^\alpha \phi} \dd{x} \\
        &\leq \sum_{\abs{\alpha} \leq M} \sup \abs{\pt^\alpha \phi} \cdot \int_K \abs{f_\alpha} \dd{x} \\
        &\leq \qty(\max_\alpha \int_K \abs{f_\alpha} \dd{x})   \sum_{\abs{\alpha} \leq M} \sup \abs{\pt^\alpha \phi}. 
    \end{align*}
Therefore, the seminorm condition is satisfied with $N = M$. From this, it also follows that $\ord(T) \leq M$. 
\end{example}

A special case of the previous example is the case $M = 0$: in this case the distribution simply becomes
\[
\ang{\tau_f, \phi} = \int_X f \phi \dd{x}. 
\]
Henceforth we will abuse notation: if $f \in \locint(X)$, then we will write $f$ instead of $\tau_f$, i.e., $\ang{f, \phi} = \int_X f\phi \dd{x}$. 

\begin{lemma}[Sequential continuity] \label{lem:seq_continuity}
    Let $u \colon \DD(X) \to \CC$ be a linear map. Then $u$ is a distribution if and only if, for every sequence $(\phi_m) \subseteq \DD(X)$ with $\phi_m \to 0$ as in \cref{def:test_convergence}, we have $\ang{u, \phi_m} \to 0$. 
\end{lemma}

\begin{proof}
    `$\implies$' If $u$ is a distribution and $(\phi_m) \to 0$, then $\supp\phi_m \subseteq K$ for some compact $K$, and by \cref{eq:seminorm_cond} there exist $C, N$ such that
    \[
    \abs{\ang{u, \phi_m}} \leq C \sum_{\abs{\alpha} \leq N} \sup \abs{\pt^\alpha \phi_m} \to 0. 
    \]
    
    `$\impliedby$' Suppose there is a compact set $K$ such that \cref{eq:seminorm_cond} is not valid for any $C, N$. Let $m \in \NN$ and $C = N = m$, then there is some $\phi_m$ with $\supp(\phi_m) \subseteq K$, and 
    \[
    \abs{\ang{u, \phi_m}} > m \sum_{\abs{\alpha} \leq m} \sup \abs{\pt^\alpha \phi_m}. 
    \]
    By dividing $\phi_m$ by $\ang{u, \phi_m} \neq 0$, we may assume w.l.o.g.\ that $\ang{u, \phi_m} = 1$. 
    We now have a sequence $(\phi_m)$ such that
    \[
    \frac1m > \sum_{\abs{\alpha} \leq m} \sup \abs{\pt^\alpha \phi_m} \implies  \abs{\pt^\alpha \phi_m} < \frac1m \quad\text{ for $\abs{\alpha} \leq m$} \implies \pt^\alpha \phi_m \to 0 \text{ uniformly for all $\alpha$}. 
    \]
    Since each $\phi_m$ also satisfies $\supp\phi_m\subseteq K$, by \cref{def:test_convergence} we have that $\phi_m \to 0$, but also $\ang{u, \phi_m} \to 1$, a contradiction. 
\end{proof}

\subsection{Limits in $\DD'(X)$}
\begin{definition}
    We say that a sequence $(u_m) \subseteq \DD'(X)$ converges to $u \in \DD'(X)$ and write $u_m \to u$ if 
    \[
    \ang{u_m, \phi} \to \ang{u, \phi} \quad\text{for all $\phi \in \DD(X)$}. 
    \]
\end{definition}

\begin{mdframed}
The following theorem is non-examinable but interesting: 
\begin{theorem}
    Let $(u_m)$ be a sequence in $\DD'(X)$ such that $\lim\limits_{m\to\infty} \ang{u_m, \phi}$ exists for all $\phi \in \DD(X)$. Then the map $\ang{u, \phi} \ceq \lim\limits_{m\to\infty} \ang{u_m, \phi}$ is a distribution in $X$.
\end{theorem}

\begin{proof}
    This is a direct application of the uniform boundedness principle. 
\end{proof}
\end{mdframed}

\begin{example}
    Let $X = \RR$ and consider the sequence of functions $u_m \in \locint(\RR)$ defined by $u_m(x) = \sin(mx)$. Then, for all $\phi \in \DD(\RR)$, we have
    \[
    \ang{u_m, \phi} = \int_\RR \sin(mx) \phi(x) \dd{x} = \frac1m \int_\RR \cos(mx) \phi'(x) \dd{x} \leq \frac1m \int \abs{\phi'(x)} \dd{x} \to 0.   \]
    Therefore, it holds that $u_m \to 0$ in $\DD'(\RR)$. With our abuse of notation we write this as $\lim\limits_{m\to\infty} \sin(mx) = 0$ in $\DD'(\RR)$. 
\end{example}

\subsection{Basic operations}

\subsubsection{Differentiation and multiplication by smooth functions}
For $u \in C^\infty(X)$ and $\phi \in \DD(X)$, we have noted that
\[
\ang{\pt^\alpha u, \phi} = \int_X \pt^\alpha u \cdot \phi \dd{x} = (-1)^\abs{\alpha}  \int_X u \cdot \pt^\alpha \phi \dd{x} = \ang{u, (-1)^\abs{\alpha} \pt^\alpha \phi}. 
\]

Since the RHS makes sense for any distribution $u$, we define 
\begin{definition}
    For $f \in C^\infty(X)$, $u \in \DD'(X)$, we define $\pt^\alpha(fu)$ by
    \[
    \ang{\pt^\alpha(fu), \phi}  \ceq \ang{u, (-1)^\abs{\alpha} f \cdot  \pt^\alpha \phi}
    \]
\end{definition}

\begin{remark}
    This definition outlines a more general pattern when working with distributions: first we take some well-defined operator on the collection of smooth maps, then we rewrite it to a form that is sensible for any distribution, and then we \emph{define} that new form as the operator on distributions. This process is called \emph{extending the definition by duality}.
\end{remark}

\begin{example}
    Let $u = \delta_x$, then we have
    \[
    \ang{\pt^\alpha \delta_x, \phi} = \ang{\delta_x, (-1)^\abs{\alpha} \pt^\alpha\phi} = (-1)^\abs{\alpha} \pt^\alpha \phi(x)
    \]
    
    Furthermore, consider the Heaviside function $H(x) = \ind_{x \geq 0}$. We have
    \[
    \ang{H', \phi} = - \ang{H, \phi'} = -\int_\RR H(x) \phi'(x) \dd{x} = - \int_0^\infty \phi'(x) \dd{x} = \phi(0) = \ang{\delta_0, \phi}, 
    \]
    so we write $H'= \delta_0$ \emph{in the distributional sense}. 
\end{example}

\begin{lemma}
    Suppose $u' \in \DD'(\RR)$ satisfies $u' = 0$. Then $u$ is constant (i.e., $\ang{u, \phi} = \ang{c, \phi} = c \int_\RR \phi \dd{x}$ for some $c$). 
\end{lemma}

\begin{proof}
    Fix  any $\theta \in \DD(\RR)$ with $\ang{1, \theta} = 1$. Let $\phi \in \DD(\RR)$ and define
    \[
    \phi_A \ceq \phi - \ang{1, \phi}\theta, \quad \phi_B \ceq \ang{1, \phi}\theta \quad\text{such that }\phi = \phi_A + \phi_B.
    \]
    
    Note that $\ang{1, \phi_A} = \ang{1, \phi} - \ang{1, \phi}\ang{1, \theta} = 0$.
    
    We claim that the function $\Phi_A(x) \ceq \int_{-\infty}^x \phi_A(y) \dd{y}$ has compact support: since $\supp\phi_A \subseteq [a, b]$ for some $a, b \in \RR$, clearly $\Phi_A(x) = 0$ for $x < a$, while for $x > b$ we have $\Phi_A(x) = \ang{1, \phi_A} = 0$. 
    Obviously, it holds that $\Phi_A' = \phi_a$. Now we compute
    \[
    \ang{u, \phi} = \ang{u, \phi_A} + \ang{u, \phi_B} = \ang{u, \Phi_A'} + \ang{1, \phi}\ang{u, \theta} = - \ang{u', \Phi_A} + c \ang{1, \phi} = \ang{c, \phi}. 
    \]
    Since $\phi$ was chosen arbitrarily this shows that $u$ is constant. 
\end{proof}

\subsubsection{Reflection and translation} 
For $\phi \in \DD(\RR^n), h \in \RR^n$, define the \emph{translation of $\phi$ by $h$} by 
\[
(\tau_h\phi)(x) \ceq \phi(x - h),
\]
and the \emph{reflection of $\phi$} by $\refl\phi(x) \ceq \phi(-x)$. 

Extending the definitions of translation and reflection by duality yields the following:
\begin{definition}
    For $u \in \DD'(\RR^n)$, $h \in \RR^n$, $\phi \in \DD(\RR^n)$, define
    \[
    \ang{\tau_hu, \phi} \ceq \ang{u, \tau_{-h}\phi} \quad\text{and}\quad \ang{\refl u, \phi} \ceq \ang{u, \refl{\phi}}. 
    \]
\end{definition}

\begin{lemma}
    For $u \in \DD'(\RR^n)$, define $V_h \in \DD'(\RR^n)$ for $0 \neq h \in \RR^n$ by
    \[
    V_h \ceq \frac{\tau_{-h}u - u}{\norm{h}}
    \]
    If $(h_j) \subseteq \RR^n$ is a sequence for which $\lim_{j \to \infty} \frac{h_j}{\norm{h_j}} = m \in S^{n-1}$, then $V_{h_j} \to \sum_i m_i \pdv{u}{x_i}$ in $\DD'(\RR^n)$. 
\end{lemma}

\begin{proof}
    By definition, we can write $\ang{V_h, \phi} = \frac{1}{\norm{h}} \ang{u, \tau_h\phi - \phi}$. Now Taylor's theorem tells us that
    \[
    (\tau_h\phi - \phi)(x) = \phi(x - h) - \phi(x) = - \sum_i h_i \pdv{\phi}{x_i}(x) + R(x, h),
    \]
    where $R(x, h) = o(\norm{h})$ in $D(\RR^n)$ (exercise sheet 1, question 2). 
    
    By sequential continuity, we have
    \[
    \lim_{j \to \infty} \ang{V_{h_j}, \phi} = \ang{u, - \sum_i m_i \pdv{\phi}{x_i}} = \ang{\sum_i m_i \pdv{u}{x_i}, \phi}, 
    \]
    which shows that $V_{h_j} \to \sum_i m_i \pdv{u}{x_i}$ in $\DD'(\RR^n)$. 
\end{proof}

\subsubsection{Convolution}
For $\phi \in \DD(\RR^n)$, note that $(\tau_x \refl\phi)(y) = \refl\phi(y-x) = \phi(x-y)$. 
\begin{definition}
    For $u \in C^\infty(\RR^n), \phi \in \DD(\RR^n)$, we define the \emph{convolution} $u * \phi \colon \RR^n \to \CC$ as
    \[
    (u * \phi)(x) \ceq \int_{\RR^n} u(x-y) \phi(y) \dd{y} = \int_{\RR^n} u(y) \phi(x - y) \dd{y} = \ang{u, \tau_x \refl\phi}. 
    \]
    Since the RHS makes sense for any $u \in \DD'(\RR^n)$, we extend the definition this way: for $u \in \DD'(\RR^n), \phi \in \DD(\RR^n)$, we define the convolution $u * \phi$ as
    \[
    (u * \phi)(x) \ceq \ang{u, \tau_x\refl\phi}. 
    \]
\end{definition}


\begin{lemma}
    Let $\phi \in C^\infty(\RR^n \times \RR^m)$ and define $\Phi_x(y) \ceq \phi(x, y)$. Suppose for any $x \in \RR^n$ there exists a neighbourhood $N(x)$ and a compact $K \subseteq \RR^m$ such that $\phi(x, y)$ for all $x \in N(x), y \notin K$. 
    
    Then $x \mapsto \ang{u, \Phi_x}$ is differentiable with
    \[
    \pt_x^\alpha \ang{u, \Phi_x} = \ang{u, \pt_x^\alpha \Phi_x}
    \]
    for any $u \in \DD'(\RR^m)$. 
\end{lemma}

\begin{proof}
    Fix $x \in \RR^n$, then by Taylor's formula we have
    \[
    \Phi_{x + h}(y) = \Phi_x(y) + \sum_{i=1}^n h_i \pdv{\phi}{x_i} (x, y)+ R(x, y, h), 
    \]
    where $\pt_y^\alpha R(x, y, h) = o(\norm{h})$, uniformly in $y$, for any multi-index $\alpha$. Furthermore, by assumption there exists a compact $K$ such that for $h$ small enough, $\supp R(x, \cdot, h) \subseteq K$. Therefore, $R(x, \cdot, h)$ is a test function for $h$ small enough.
    
    Combining the previous two facts shows that $R(x, \cdot, h) = o(\norm{h})$ in $\DD(\RR^m)$ as $h \to 0$. 
    
    Let $u \in \DD'(\RR^m)$, then we find by sequential continuity that $\ang{u, R(x, \cdot, h)}$ is also $o(\norm{h})$, and therefore
    \[
    \ang{u, \Phi_{x+h}} = \ang{u, \Phi_x} + \sum_{i=1}^n h_i \ang{u, \pdv{\Phi_x}{x_i}} + o(\norm{h}). 
    \]
    This shows that $x \mapsto \ang{u, \Phi_x}$ is differentiable with 
    \[
    \pdv{x_i} \ang{u, \Phi_x}  = \ang{u, \pdv{\Phi_x}{x_i}}. 
    \]
    From this the result follows. 
\end{proof}

\begin{corollary}
    If $u \in \DD'(\RR^n)$, $\phi \in \DD(\RR^n)$, then $u * \phi$ is differentiable with $\pt^\alpha (u * \phi) = u * \pt^\alpha\phi$. 
\end{corollary}

\begin{proof}
    Apply the previous lemma with $\Phi_x(y) \ceq  \phi(x-y)$. 
\end{proof}

Due to the previous corollary, we often call $u * \phi$ a \emph{regularisation} of $u$. 

\begin{convention}
	If $u \in \DD'(X)$ and $\phi \in \DD(X)$, then instead of $\ang{u, \phi}$ we also write $\ang{u(t), \phi(t)}$ (or with any other dummy variable) when the variable used for $\phi$ is not directly clear. 
\end{convention}

\subsection{Density of $\DD(\RR^n)$ in $\DD'(\RR^n)$}
\begin{lemma}
	If $u \in \DD'(\RR^n)$ and $\phi, \psi \in \DD(\RR^n)$, then
	\[
	(u * \phi) * \psi = u * (\phi * \psi). 
	\]
\end{lemma}

\begin{proof}
	Fix $x \in \RR^n$. Now we write
	\begin{align*}
		((u * \phi) * \psi)(x) &= \int_{\RR^n} \ang{u(z), (\tau_y \refl\phi)(z)} \cdot \psi(x-y) \dd{y} \\
		&= \int_{\RR^n} \ang{u(z), (\tau_{x-y}\refl\phi)(z)} \cdot \psi(y) \dd{y} \\
		&= \int_{\RR^n} \ang{u(z), \psi(y) (\tau_{x-y} \refl\phi)(z)} \dd{y}.
	\end{align*}
We would like to interchange integral and application of $u$, and we will have to justify this using Riemann sums: 
\begin{align*}
\int_{\RR^n} \ang{u(z), \psi(y) (\tau_{x-y} \refl\phi)(z)} \dd{y} &= \lim_{\eps \downarrow 0} \sum_{m \in \ZZ^n} \ang{u(z), \psi(\eps m) \phi(x - z - \eps m)} \eps^n \\
&\overset*= \lim_{\eps\downarrow0} \ang{u(z), \sum_{m \in \ZZ^n} \psi(\eps m) \phi(x - z - \eps m)\eps^n } \\
&\overset{**}= \ang*{u(z), \int_{\RR^n} \psi(y) \phi(x - z - y) \dd{y}} \\
&= \ang{u(z), (\phi * \psi)(x - z)} = \ang{u(z), (\tau_x\widecheck{\phi * \psi})(z)} = (u * (\phi * \psi))(x). 
\end{align*}
Here, $*$ is by the fact that the sum is finite since $\psi$ has compact support, while $**$ is by sequential continuity of $u$ and the fact that the Riemann sum converges to the convolution integral \emph{in the space of test functions} (non-examinable fact). 
\end{proof}

We will use the following trick many times:
\begin{proposition}
	For any $u \in \DD'(\RR^n), \phi \in \DD(\RR^n)$ we have $\ang{u, \phi} = (u * \refl\phi)(0)$. 
\end{proposition}

\begin{proof}
	We have $(u * \refl\phi)(0) = \ang{u, \tau_0 \phi} = \ang{u, \phi}$. 
\end{proof}
For example, from this trick it follows that if $u * \phi = 0$ for all $\phi$, then $u = 0$. 

\begin{theorem}
	If $u \in \DD'(\RR^n)$, there exists a sequence $(\phi_k) \subseteq \DD(\RR^n)$ such that $\phi_k \to u$ in $\DD'(\RR^n)$. 
\end{theorem}

\begin{proof}
	Fix $\psi \in \DD(\RR^n)$ with $\int_{\RR^n} \psi \dd{x} = 1$, and set $\psi_k(x) \ceq k^n \psi(kx)$. Note that $\int_{\RR^n} \psi_k \dd{x} = 1$.
	
	Now, fix any $\chi \in \DD(\RR^n)$ with $\chi \equiv 1$ on $\qty{\norm{x} < 1}$ and $\chi \equiv 0$ on $\qty{\norm{x} < 2}$. Define $\chi_k(x) \ceq \chi(x/k)$, so that $\lim_{k\to\infty} \chi_k(x) = 1$ for all $x$. We will set
	\[
	\phi_k(x) \ceq \chi_k(x) (u * \psi_k)(x). 
	\]
	Clearly we have $\phi_k \in \DD(\RR^n)$ since each $\chi_k$ has compact support. 
	
	Now, take any $\theta \in \DD(\RR^n)$, then we have
	
	\begin{align*}
		\ang{\phi_k, \theta} &= \ang{\chi_k (u * \psi_k), \theta} = \ang{u * \psi_k, \chi_k \theta} = \qty[ (u * \psi_k) * \refl{\chi_k\theta}](0) \\
		&= \qty[u * \qty(\psi_k * \refl{\chi_k \theta})](0).
	\end{align*}

	Now we compute $\psi_k * \refl{\chi_k\theta}$: note that 
	\begin{align*}
		(\psi_k * \refl{\chi_k\theta})(x) &= \int_{\RR^n} k^n\psi(k(x-y)) \chi\qty(-\frac yk) \theta(-y) \dd{y} \\
		&= \int_{\RR^n} \psi(y) \chi\qty(\frac{y}{k^2} - \frac xk) \theta(\frac yk - x) \dd{y} \\
		&= \theta(-x) + R_k(-x) = (\refl{\theta + R_k})(x)
	\end{align*}
where $R_k(x) = \int_{\RR^n} \psi(y) \qty[ \chi\qty(\frac y{k^2} + \frac xk) \theta\qty(\frac yk + x )- \theta(x)] \dd{y}$.

So \[
\ang{\phi_k, \theta} = (u * (\refl{\theta + R_k}))(0) = (u *\refl\theta)(0) + (u * \refl{R_k})(0) = \ang{u, \theta} + \ang{u, R_k}. 
\]

We must now only prove that $R_k \to 0$ in $\DD(\RR^n)$, and then by sequential continuity it follows that $\phi_k \to u$  in $\DD'(\RR^n)$. 
\end{proof}

