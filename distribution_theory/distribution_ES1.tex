\documentclass{article}

\usepackage[utf8]{inputenc}
\usepackage[english]{babel}
\usepackage[margin=3cm]{geometry}
\usepackage[normalem]{ulem}
\usepackage{hyperref}
\usepackage{mathtools, amsmath, amssymb, amsthm, mathabx, enumerate, mdframed, bbm, graphicx, float, physics, xcolor, cleveref}

\hypersetup{
    colorlinks   = true, %Colours links instead of ugly boxes
    urlcolor     = blue, %Colour for external hyperlinks
    linkcolor    = blue, %Colour of internal links
    citecolor   = red %Colour of citations
}

% Definition of numbered environments.
% Usage: \begin{theorem} ... \end{theorem}
% Remark has no numbering.
\theoremstyle{plain}
\newtheorem{question}{Question}
\newtheorem{theorem}{Theorem}
\newtheorem{lemma}{Lemma}
\theoremstyle{remark}
\newtheorem*{remark}{Remark}

% Question with box around it
\newenvironment{qbox}{\begin{mdframed}\begin{question}}{\end{question}\end{mdframed}}

% A proof with "solution" instead of "proof" and no QED symbol
\newenvironment{solution}{\begin{proof}[Solution]\renewcommand\qedsymbol{}}{\end{proof}}

% Some renewed commands
\renewcommand{\vec}{\mathbf}
\renewcommand{\emptyset}{\varnothing}
\renewcommand{\epsilon}{\varepsilon}
\renewcommand{\theta}{\vartheta}
\renewcommand{\phi}{\varphi}

% Frequently used math alphabets
\newcommand{\Bb}{\mathbb}
\newcommand{\Cal}{\mathcal}
\newcommand{\Bf}{\mathbf}
\newcommand{\Rm}{\mathrm}

% Frequently used letters in the blackboard alphabet
\newcommand{\CC}{\Bb C}
\newcommand{\NN}{\Bb N}
\newcommand{\PP}{\Bb P}
\newcommand{\QQ}{\Bb Q}
\newcommand{\RR}{\Bb R}
\newcommand{\EE}{\Cal E}
\newcommand{\DD}{\Cal D}
\newcommand\ZZ{\Bb Z}

% Usage: \ang{...} is equivalent to \langle ... \rangle, while \ang*{...} is equivalent to \left\langle ... \right\rangle
% For other delimiters: use \qty from the physics package (i.e., \qty(...))
\DeclarePairedDelimiter{\ang}{\langle}{\rangle}
\DeclarePairedDelimiter{\floor}{\lfloor}{\rfloor}
\DeclarePairedDelimiter{\ceil}{\lceil}{\rceil}

% Frequently used commands
\newcommand{\T}{^\top} % Matrix transpose A\T
\newcommand{\C}{^\complement} % Set complement A\C
\newcommand\ceq\coloneqq % Definitions :=
\newcommand\pow{\Cal P} % Power sets
\newcommand\eps\epsilon
\newcommand\ind{\mathbbm 1} % Blackboard 1 for indicator functions
\newcommand\restr{\mathord\restriction}
\newcommand\TODO{{\color{red} TODO: }}
\newcommand\pt\partial
\newcommand\locint{L^1_{\Rm{loc}}}
\newcommand\refl\widecheck

% Functions that appear frequently
\DeclareMathOperator{\sign}{sign}
\DeclareMathOperator{\Int}{Int}
\DeclareMathOperator{\Span}{Span}
\DeclareMathOperator{\Var}{Var}
\DeclareMathOperator*{\argmin}{arg\,min}
\DeclareMathOperator*{\argmax}{arg\,max}
\DeclareMathOperator{\supp}{supp}
\DeclareMathOperator{\ord}{ord}

\newcommand\clos\overline

\title{Distribution Theory and Applications --- Example Sheet 1} % subject
\author{Lucas Riedstra}
\date{...} % date

\begin{document}
\maketitle

\begin{question}
	Construct a non-zero element of $\DD(\RR)$ that vanishes outside $(0, 1)$. Construct a non-zero of $\DD(\RR^n)$ that vanishes outside the ball $B_\eps = \qty{x \in \RR^n : \norm{x} < \eps}$. 
\end{question}

\begin{proof}
	It is well-known that the function 
	\[
	\phi \colon \RR \to \RR \colon x \mapsto \begin{cases}
		0 &\text{if $x \leq 0$}; \\
		e^{-1/x} &\text{if $x > 0$},
	\end{cases}
	\]
	is smooth and vanishes outside $(0, \infty)$. The function $\psi(x) \ceq \phi(x)\phi(1-x)$ is therefore also smooth and vanishes outside $(0, 1)$.  
	
	Since $\psi$ vanishes outside $(0, 1)$, the function $\psi(x/\eps)$ vanishes outside $(0, \eps)$, and therefore the function $\vec x \mapsto \psi(\norm{\vec x}/\eps)$ vanishes outside $B_\eps$. 
\end{proof}

\begin{question}
	Given $\phi \in \DD(X)$, Taylor's theorem gives
	\[
	\phi(x+h) = \sum_{\abs{\alpha} \leq N} \frac{h^\alpha}{\alpha!} \pt^\alpha \phi(x) + R_N(x, h).
	\]
	Prove that $\supp(R_N)$ is contained in some fixed compact $K \subseteq X$ for $\abs{h}$ sufficiently small. Show also that $\pt^\alpha R_N = o(\abs{h}^N)$ uniformly in $x$ for each multi-index $\alpha$, i.e.\ prove 
	\[
	\lim_{\abs{h}\to 0} \frac{\sup_x \abs{\pt^\alpha R_N(x, h)}}{\abs{h}^N} = 0
	\]
	for each multi-index $\alpha$.  
	
	\emph{Hint: you may find it convenient to use the following form of the remainder}
	\[
	R_N(x, h) = \sum_{\abs{\alpha} = N+1} \frac{h^\alpha}{\alpha!}(N+1)\int_0^1 (1-t)^N (\pt^\alpha\phi)(x + th)\dd{t},
	\]
	\emph{and note that} $(N+1)! \sum_{\abs{\alpha} = N+1} \frac{h^\alpha}{\alpha!} = (h_1 + \dotsb + h_n)^{N+1}$. 
\end{question}

\begin{proof}
	Since $\phi \in \DD(X)$, we know that $\supp\phi \subseteq \clos{B_N}$ for some $N \in \NN$. Now, suppose $\norm{h} < 1$, then
	\[
	\phi(x+h) \neq 0 \implies \norm{x + h} \leq N \implies \norm{x} \leq \norm{x + h} + \norm{h} \leq N + 1, 
	\]
	so if we define $\psi_h(x) = \phi(x+h)$ then we know that $\supp\phi_h \subseteq \clos{B_{N+1}}$. 
	
	By Taylor's theorem, we have
	\[
	\phi(x+h) = \sum_{\abs{\alpha} \leq N} \frac{h^\alpha}{\alpha!} \pt^\alpha \phi(x) + R_N(x, h),  
	\]
	and since $\sum_{\abs{\alpha} \leq N} \frac{h^\alpha}{\alpha!} \pt^\alpha \phi(x)$ vanishes for $x \notin \clos{B_N}$, it is clear that $\supp(R_N(\cdot, h))$ must also be contained in $\clos{B_{N+1}}$ (again, for $\norm{h} \leq 1$). This shows that $\supp(R_N)$ is contained in $\clos{B_{N+1}}$ for $\abs{h}$ sufficiently small. 
	
	Now let $\beta$ be a multi-index and define $C \ceq \frac{1}{N!} \max_{\abs{\alpha} = N+1, x \in \RR^n} \abs{(\pt^{\alpha + \beta})\phi(x)}$ (note that $C$ exists and is finite since all partial derivatives of $\phi$ have compact support), then we have
	\begin{align*}
	\abs{\pt^\beta R_N(x, h)} &= \abs{\pt^\beta \sum_{\abs{\alpha} = N+1} \frac{h^\alpha}{\alpha!}(N+1)\int_0^1 (1-t)^N (\pt^\alpha\phi)(x + th)\dd{t}} \\
	%&= \abs{\sum_{\abs{\alpha} = N+1} \frac{h^\alpha}{\alpha!}(N+1) \cdot \pt^\beta\int_0^1 (1-t)^N (\pt^\alpha\phi)(x + th)\dd{t}} \\
	&\overset\star= \abs{\sum_{\abs{\alpha} = N+1} \frac{h^\alpha}{\alpha!}(N+1) \int_0^1 (1-t)^N (\pt^{\alpha + \beta}\phi)(x + th)\dd{t}} \\
	&\leq \sum_{\abs{\alpha} = N+1} \frac{\abs{h^\alpha}}{\alpha!}(N+1) \int_0^1 (1-t)^N \abs{\qty(\pt^{\alpha + \beta}\phi)(x+th)} \dd{t} \\
	&\leq \qty[\max_{\abs{\alpha} = N+1, x \in \RR^n} \abs{(\pt^{\alpha + \beta})\phi(x)}] \sum_{\abs{\alpha} = N+1} \frac{\abs{h^\alpha}}{\alpha!} (N+1) \\
	&\leq C (N+1)! \sum_{\abs{\alpha}= N+1}\frac{\abs{h^\alpha}}{\alpha!} = C(\abs{h_1} + \dotsb + \abs{h_n})^{N+1}.
\end{align*}
Since this upper bound does not depend on $x$, we also have
\[
\sup_x \abs{\pt^\beta R_N(x, h)} \leq C(\abs{h_1} + \dotsb + \abs{h_n})^{N+1}, 
\]
and we conclude that
\[ 
\frac{\sup_x \abs{\pt^\beta R_N(x, h)}}{\norm{h}^N} \leq \frac{C(\abs{h_1} + \dotsb + \abs{h_n})^{N+1}}{\norm{h}^N} \leq \frac{C N^{N+1} \norm{h}^{N+1}}{\norm{h}^N} = CN^{N+1} \norm{h} \to 0, 
\]
and therefore that $\pt^\beta R_N(x, h) = o(\norm{h}^n)$ for all multi-indices $\beta$. 
\end{proof}

\begin{question}
	Which elements of $\DD(X)$ can be represented as a power series on $X$? 
\end{question}

\begin{solution}
	It is known that if two power series agree on an open set, they agree on the entire space. Since every $\phi \in \DD(X)$ is identically zero on some open set (outside its support), the only element of $\DD(X)$ with a power series representation is the zero function. 
\end{solution}

\begin{question}
	Prove the $C^\infty$ Urysohn lemma: if $K$ is a compact subset of $X \subseteq \RR^n$, show that one can find a $\phi \in \DD(X)$ such that $0 \leq \phi \leq 1$ and $\phi = 1$ on a neighborhood of $K$. 
\end{question}

\begin{solution}
	Let $K \subseteq U_1$. Define $U_2 \ceq U_1 + B(0, 1)$ and let $\chi = \ind_{U_2}$. Now let $\psi \in \DD(\RR^n)$ satisfy $\int_{\RR^n} \psi \dd{x} = 1$ and $\supp\psi \subseteq B(0, 1)$. The we compute
	\begin{align*}
		(\chi * \psi)(x) &= \int_{\RR^n} \chi(y) \psi(x - y) \dd{y} = \int_{U_2} \psi(x - y) \dd{y}.
	\end{align*}
	Clearly, $\chi * \psi \in \DD(X)$, and furthermore, we have for $x \in U_1$ that
	\[
	\int_{U_2} \psi(x - y) \dd{y} = \int_{U_2 - x} \psi(z) \dd{z} \overset\star= 1,
	\]
	since $B(0, 1) \subseteq U_2 - x$. This proves the claim.
\end{solution}

\begin{question}
	Given $T \in \DD'(X)$, the derivative $\pt^\alpha T$ is defined by
	\[
	\ang{\pt^\alpha T, \phi} = (-1)^\abs{\alpha} \ang{T, \pt^\alpha\phi} \quad \forall \phi \in \DD(X). 
	\]
	Show that $\pt^\alpha T \in \DD'(X)$. If $\ord(T) = m$ what can you say about $\ord(\pt^\alpha T)$? 
\end{question}

\begin{proof}
	Let $K \subseteq X$ be compact and $\phi \in \DD(X)$. Since $T$ is a distribution, we know that there exists constants $C, N$ such that
	\[
	\abs{\ang{T, \phi}} \leq C \sum_{\abs{\beta} \leq N} \sup \abs{\pt^\beta \phi}. 
	\]
	
	Letting $M \ceq \abs{\alpha}$, we find
		\begin{align*}
		\abs{\ang{\pt^\alpha T, \phi}} &= \abs{\ang{T, \pt^\alpha\phi}} \leq C \sum_{\abs{\beta} \leq N} \sup \abs{\pt^{\alpha + \beta}\phi} \leq C \sum_{\abs{\beta} \leq M + N} \sup \abs{\pt^\beta \phi}.
	\end{align*}
We conclude that $\pt^\alpha T$ is a distribution, and that if $\ord(T) = m$, $\ord(\pt^\alpha T) \leq m + \abs{\alpha}$. 
\end{proof}

\begin{question}
	Given $T \in \DD'(X)$ and $f \in C^\infty(X)$, prove that for each multi-index $\alpha$
	\[
	\pt^\alpha(Tf) = \sum_{\beta \leq \alpha} \binom{\alpha}{\beta} \pt^\beta f \pt^{\alpha - \beta}T
	\]
	in $\DD'(X)$. 
\end{question}

\begin{proof}
	Let $\phi \in \DD(X)$, then by definition we have $\ang{\pt^\alpha(Tf), \phi} = \ang{T, (-1)^\abs{\alpha} f \pt^\alpha\phi}$. Approximate $T$ by a sequence $(\psi_n) \subseteq \DD'(X)$, then we find
	\begin{align*}
	\ang{T, (-1)^{\abs{\alpha}} f \pt^\alpha\phi} &= \lim_{n\to\infty} \ang{\psi_n,  (-1)^{\abs{\alpha}} f \pt^\alpha\phi} = \lim_{n\to\infty}  (-1)^{\abs{\alpha}} \int_X \psi_n(x) f(x) \pt^\alpha\phi(x) \dd{x} \\
	&= \lim_{n\to\infty} \int_X \pt^\alpha (\psi_n(x) f(x)) \phi(x) \dd{x} \\
	&= \lim_{n\to\infty} \int_X \qty(\sum_{\beta \leq \alpha} \binom\alpha\beta\pt^\beta f(x) \cdot \pt^{\alpha - \beta} \psi_n(x)) \phi(x) \dd{x} \\
	&= \lim_{n\to\infty} \sum_{\beta \leq \alpha} \binom\alpha\beta \ang{\pt^\beta f \pt^{\alpha - \beta} \psi_n, \phi} =  \lim_{n\to\infty} \sum_{\beta \leq \alpha} \binom\alpha\beta (-1)^{\abs{\alpha - \beta}} \ang{\psi_n, \pt^\beta f \pt^{\alpha - \beta} \phi} \\
	&= \sum_{\beta \leq \alpha} \binom\alpha\beta (-1)^{\abs{\alpha - \beta}} \ang{T, \pt^\beta f \pt^{\alpha - \beta} \phi} = \sum_{\beta \leq \alpha} \binom\alpha\beta \ang{\pt^\beta f \pt^{\alpha - \beta} T, \phi} \\
	&= \ang*{\sum_{\beta \leq \alpha} \binom{\alpha}{\beta} \pt^\beta f \pt^{\alpha - \beta}T, \phi}. 
	%&= \lim_{n\to\infty} \ang*{\sum_{\abs{\beta} \leq \abs{\alpha}} \binom\alpha\beta \pt^\beta f\cdot \pt^{\alpha - \beta} \psi_n, \phi} = \lim_{n\to\infty} \sum_{\abs{\beta} \leq \abs{\alpha}
	\end{align*}
\end{proof}

\begin{question}
	Let $(x_k)$ be a sequence in $X$ with no limit point in $X$. Consider the family of linear maps
	$u_\alpha \colon \DD(X) \to \CC$ defined by 
	\[ \ang{u_\alpha, \phi} = \sum_{k=1}^\infty \pt^\alpha \phi(x_k)
	 \]
	 for each multi-index $\alpha$. For what $\alpha$ is $u_\alpha \in \DD'(X)$? What is $\ord(u_\alpha)$? 
\end{question}

\begin{solution}
	Let $K \subseteq X$ be compact. Since $(x_k)$ does not have a limit point, only finitely many of the $x_k$ lie in $K$ (otherwise $(x_k)$ would have a subsequence contained in $K$ which would have a convergent subsequence). Without loss of generality, assume that $x_1, \dotsc, x_n \in K$, and $x_{n+1}, x_{n+2}, \dotsc, \notin K$. Now, for any $\phi \in \DD(X)$ with $\supp(\phi) \subseteq K$ we find
	\begin{align*}
		\abs{\ang{u, \phi}} &= \abs{\sum_{k=1}^\infty \pt^\alpha\phi(x_k)} = \abs{\sum_{k=1}^n \pt^\alpha\phi(x_k)} \leq \sum_{k=1}^n \abs{\pt^\alpha \phi(x_k)} \leq n \cdot \sup \abs{\pt^\alpha\phi} \leq n \cdot \sum_{\abs{\beta} \leq \abs{\alpha}} \sup \abs{\pt^\beta \phi}. 
	\end{align*}
	This shows that $u_\alpha \in \DD'(X)$ for any $\alpha$, with $\ord(u_\alpha) \leq \abs{\alpha}$. We claim that this is an equality, i.e., $\ord(u_\alpha) = \abs{\alpha}$. \TODO How to show??
\end{solution}

\begin{question}
	Find the most general solution to the equations
	\begin{enumerate}[(a)]
		\item $u'  = 1$,
		\item $xu' = \delta_0$,
		\item $(e^{2\pi i x} - 1)u' = 0$
	\end{enumerate}
	in $\DD'(\RR)$. 
\end{question}

\begin{solution}
	Let $\phi \in \DD(X)$. 
	\begin{enumerate}[(a)]
		\item If $u' = 1$ then we find 
		\[
		\int_\RR \phi(x) \dd{x} = \ang{1, \phi} = \ang{u', \phi} = -\ang{u, \phi'},
		\]
		so for any $c \in \RR$ we find by partial integration $\ang{u, \phi'} = - \int_\RR\phi(x) \dd{x} = \int_\RR (x + c) \phi'(x) \dd{x}$. From this we deduce that $u = x + c$ for some $c$. 
		
		\item If $xu' = \delta_0$ then we have
		\[
		\phi(0) = \ang{\delta_0, \phi} = \ang{xu', \phi} = \ang{u', x\phi} = \ang{u, -(x\phi)'} = \ang{u, -x\phi' - \phi} %=  \ang{u, -x\phi'} + \ang{u, -\phi}. 
		\]
		Note that the above is satisfied for $u = - \delta_0 + c$ for any constant $c$. \TODO is this the most general solution?
		
		\item Since $e^{2\pi i x} = 1 \iff x \in \ZZ$, intuitively it must be the case that $u'$ is 0, except ``on $\ZZ$'', whatever that may mean. Therefore, we guess that, for any sequence $(\alpha_n)_{n \in \ZZ} \subseteq \CC$ and constant $c \in \CC$, the map \[u = c + \sum_{n \in \ZZ} \alpha_n \ind_{x \geq n}.\] We compute the derivative of $u$. It is easily seen that $u' = \sum_{n \in \ZZ} \alpha_n \delta_n$ (the infiniteness of the sum does not pose a problem since the test functions are compactly supported, so $\ang{u, \phi}$ will always be a finite sum). 
%		We have for arbirary $\phi$ that
%		\begin{align*}
%		\ang{u', \phi} &= - \ang{u, \phi'} = -\int_\RR \sum_{n \in \ZZ} \qty(\alpha_n \ind_{x \geq n}) \phi'(x) \dd{x}	\overset\star= - \sum_{n \in \ZZ} \alpha_n \int_n^\infty \phi'(x) \dd{x} = \sum_{n \in \ZZ} \alpha_n \phi(n),	
%		\end{align*}
%	where $\star$ follows from the fact that the sum is finite since $\phi'$ is compactly supported. 
%	Therefore, we can conclude $u' = \sum_{n \in \ZZ} \alpha_n \delta_n$, and indeed we find
From this, we see that
	\[
	\ang{(e^{2\pi i x} - 1) u', \phi} = \ang{u', (e^{2\pi i x} -1 )\phi} = \sum_{n \in \ZZ} \alpha_n (e^{2\pi i n} - 1) \phi(n) = 0,
	\]
	so $u$ satisfies the equation. \TODO Why is this the most general solution? Intuitively clear, but how to make this rigorous?
	\end{enumerate}
\end{solution}

\begin{question}
	Define the distribution $u \in \DD'(\RR^2)$ by the locally integrable function $u(x, y) = \ind_{x \geq y}$. Show that $\pt_x^2   u - \pt_y^2 u = 0$ in $\DD'(\RR^2)$. Can you give a physical interpretation of this result?
\end{question}

\begin{proof}
	Let $f \in \DD(\RR^2)$, then we have
	\begin{align*}
		\ang{\pt_x^2 u - \pt_y^2 u, f} &= \ang{\pt_x^2 u, f}- \ang{\pt_y^2 u, f} = \ang{u, \pt_x^2 f} - \ang{u, \pt_y^2 f} = \ang{u, \pt_x^2 f - \pt_y^2 f} \\
		&\overset\star= \int_{-\infty}^\infty \int_{y}^\infty \pt_x^2 f(x, y) \dd{x}\dd{y} - \int_{-\infty}^\infty \int_{-\infty}^x \pt_y^2 f(x, y) \dd{y} \dd{x} \\
		&= - \int_{-\infty}^\infty \pt_x f(y, y) \dd{y} - \int_{-\infty}^\infty \pt_y f(x, x) \dd{x}
		\\
		&= - \int_{-\infty}^\infty (\pt_xf + \pt_yf)(x, x) \dd{x}.
	\end{align*}
Here, $\star$ follows from Fubini's theorem. 
Define $g(x) = f(x, x)$, then it is easily seen that $g'(x) = \pt_x f(x, x) + \pt_y f(x, x)$, so we find that
\[
\ang{\pt_x^2 u - \pt_y^2 u, f} = 
- \int_{-\infty}^\infty g'(x) \dd{x} = \lim_{x\to -\infty} g(x) - \lim_{x\to\infty} g(x) = 0 - 0 = 0. 
\]
This shows that $\pt_x u - \pt_y u = 0$, or equivalently, that $u$ satisfies the wave equation. 
\end{proof}

\begin{question}
	Compute $\Delta(\norm{x}^{2-n})$ in $\DD'(\RR^n)$ for $n \geq 3$, i.e.\ compute
	\[
	\ang{\Delta (\norm{x}^{2-n}), \phi} = \ang{\norm{x}^{2-n}, \Delta\phi} = \int \frac{\Delta\phi}{\norm{x}^{n-2}} \dd{x}
	\]
	for arbitrary $\phi \in \DD(\RR^n)$. Note that $\norm{x} = (x_1^2 +\dotsb +x_n^2)^{1/2}$ and $\Delta = \sum_i (\pdv{x_i})^2$. 
	
	\emph{Hint:} use $\int\dd{x} = \int_{\norm{x}\leq\eps}\dd{x} + \int_{\norm{x}>\eps}\dd{x}$ and treat each integral separately. 
\end{question}

\begin{solution}
	We follow the hint: let $\eps > 0$, then we first compute
	\[
	\int_{\norm{x} > \eps} \frac{\Delta\phi}{\norm{x}^{n-2}} \dd{x} = \sum_{i=1}^n \int_{\norm{x} > \eps} \frac{\pdv[2]{x_i} \phi}{\norm{x}^{n-2}}\dd{x} = \sum_{i=1}^n \int_{\norm{x} > \eps} \phi \cdot \pdv[2]{x_i} \norm{x}^{2-n} \dd{x}. 
	\]
	Now, it is easily computed that $\pdv{x_i} \norm{x} = \frac{x_i}{\norm{x}}$, and therefore 
	\begin{align*}
	\pdv[2]{x_i} \norm{x}^{2-n} &= \pdv{x_i} (2-n) x_i \norm{x}^{-n} = (2-n) \qty(\norm{x}^{-n} - n x_i^2 \norm{x}^{-n-2}) \\
	&= (n-2)\norm{x}^{-n} \qty(n \qty(\frac{x_i}{\norm{x}})^2 - 1). 
	\end{align*}
\end{solution}

\end{document}