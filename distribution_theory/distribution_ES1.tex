\documentclass{article}

\usepackage[utf8]{inputenc}
\usepackage[english]{babel}
\usepackage[margin=3cm]{geometry}
\usepackage[normalem]{ulem}
\usepackage{hyperref}
\usepackage{mathtools, amsmath, amssymb, amsthm, mathabx, enumerate, mdframed, bbm, graphicx, float, physics, xcolor, cleveref}

\hypersetup{
    colorlinks   = true, %Colours links instead of ugly boxes
    urlcolor     = blue, %Colour for external hyperlinks
    linkcolor    = blue, %Colour of internal links
    citecolor   = red %Colour of citations
}

% Definition of numbered environments.
% Usage: \begin{theorem} ... \end{theorem}
% Remark has no numbering.
\theoremstyle{plain}
\newtheorem{question}{Question}
\newtheorem{theorem}{Theorem}
\newtheorem{lemma}{Lemma}
\theoremstyle{remark}
\newtheorem*{remark}{Remark}

% Question with box around it
\newenvironment{qbox}{\begin{mdframed}\begin{question}}{\end{question}\end{mdframed}}

% A proof with "solution" instead of "proof" and no QED symbol
\newenvironment{solution}{\begin{proof}[Solution]\renewcommand\qedsymbol{}}{\end{proof}}

% Some renewed commands
\renewcommand{\vec}{\mathbf}
\renewcommand{\emptyset}{\varnothing}
\renewcommand{\epsilon}{\varepsilon}
\renewcommand{\theta}{\vartheta}
\renewcommand{\phi}{\varphi}

% Frequently used math alphabets
\newcommand{\Bb}{\mathbb}
\newcommand{\Cal}{\mathcal}
\newcommand{\Bf}{\mathbf}
\newcommand{\Rm}{\mathrm}

% Frequently used letters in the blackboard alphabet
\newcommand{\CC}{\Bb C}
\newcommand{\NN}{\Bb N}
\newcommand{\PP}{\Bb P}
\newcommand{\QQ}{\Bb Q}
\newcommand{\RR}{\Bb R}
\newcommand{\EE}{\Bb E}
\newcommand{\DD}{\Cal D}

% Usage: \ang{...} is equivalent to \langle ... \rangle, while \ang*{...} is equivalent to \left\langle ... \right\rangle
% For other delimiters: use \qty from the physics package (i.e., \qty(...))
\DeclarePairedDelimiter{\ang}{\langle}{\rangle}
\DeclarePairedDelimiter{\floor}{\lfloor}{\rfloor}
\DeclarePairedDelimiter{\ceil}{\lceil}{\rceil}

% Frequently used commands
\newcommand{\T}{^\top} % Matrix transpose A\T
\newcommand{\C}{^\complement} % Set complement A\C
\newcommand\ceq\coloneqq % Definitions :=
\newcommand\pow{\Cal P} % Power sets
\newcommand\eps\epsilon
\newcommand\ind{\mathbbm 1} % Blackboard 1 for indicator functions
\newcommand\restr{\mathord\restriction}
\newcommand\TODO{{\color{red} TODO: }}
\newcommand\pt\partial
\newcommand\locint{L^1_{\Rm{loc}}}
\newcommand\refl\widecheck

% Functions that appear frequently
\DeclareMathOperator{\sign}{sign}
\DeclareMathOperator{\Int}{Int}
\DeclareMathOperator{\Span}{Span}
\DeclareMathOperator{\Var}{Var}
\DeclareMathOperator*{\argmin}{arg\,min}
\DeclareMathOperator*{\argmax}{arg\,max}
\DeclareMathOperator{\supp}{supp}
\DeclareMathOperator{\ord}{ord}

\newcommand\clos\overline

\title{Distribution Theory and Applications --- Example Sheet 1} % subject
\author{Lucas Riedstra}
\date{...} % date

\begin{document}
\maketitle

\begin{question}
	Construct a non-zero element of $\DD(\RR)$ that vanishes outside $(0, 1)$. Construct a non-zero of $\DD(\RR^n)$ that vanishes outside the ball $B_\eps = \qty{x \in \RR^n : \norm{x} < \eps}$. 
\end{question}

\begin{proof}
	It is well-known that the function 
	\[
	\phi \colon \RR \to \RR \colon x \mapsto \begin{cases}
		0 &\text{if $x \leq 0$}; \\
		e^{-1/x} &\text{if $x > 0$},
	\end{cases}
	\]
	is smooth and vanishes outside $(0, \infty)$. The function $\psi(x) \ceq \phi(x)\phi(1-x)$ is therefore also smooth and vanishes outside $(0, 1)$.  
	
	Since $\psi$ vanishes outside $(0, 1)$, the function $\psi(x/\eps)$ vanishes outside $(0, \eps)$, and therefore the function $\vec x \mapsto \psi(\norm{\vec x}/\eps)$ vanishes outside $B_\eps$. 
\end{proof}

\begin{question}
	Given $\phi \in \DD(X)$, Taylor's theorem gives
	\[
	\phi(x+h) = \sum_{\abs{\alpha} \leq N} \frac{h^\alpha}{\alpha!} \pt^\alpha \phi(x) + R_N(x, h).
	\]
	Prove that $\supp(R_N)$ is contained in some fixed compact $K \subseteq X$ for $\abs{h}$ sufficiently small. Show also that $\pt^\alpha R_N = o(\abs{h}^N)$ uniformly in $x$ for each multi-index $\alpha$, i.e.\ prove 
	\[
	\lim_{\abs{h}\to 0} \frac{\sup_x \abs{\pt^\alpha R_N(x, h)}}{\abs{h}^N} = 0
	\]
	for each multi-index $\alpha$.  
	
	\emph{Hint: you may find it convenient to use the following form of the remainder}
	\[
	R_N(x, h) = \sum_{\abs{\alpha} = N+1} \frac{h^\alpha}{\alpha!}(N+1)\int_0^1 (1-t)^N (\pt^\alpha\phi)(x + th)\dd{t},
	\]
	\emph{and note that} $(N+1)! \sum_{\abs{\alpha} = N+1} \frac{h^\alpha}{\alpha!} = (h_1 + \dotsb + h_n)^{N+1}$. 
\end{question}

\begin{solution}
	Since $\phi \in \DD(X)$, we know that $\supp\phi \subseteq \clos{B_N}$ for some $N \in \NN$. Now, suppose $\norm{h} < 1$, then
	\[
	\phi(x+h) \neq 0 \implies \norm{x + h} \leq N \implies \norm{x} \leq \norm{x + h} + \norm{h} \leq N + 1, 
	\]
	so if we define $\psi_h(x) = \phi(x+h)$ then we know that $\supp\phi_h \subseteq \clos{B_{N+1}}$. 
	
	By Taylor's theorem, we have
	\[
	\phi(x+h) = \sum_{\abs{\alpha} \leq N} \frac{h^\alpha}{\alpha!} \pt^\alpha \phi(x) + R_N(x, h),  
	\]
	and since $\sum_{\abs{\alpha} \leq N} \frac{h^\alpha}{\alpha!} \pt^\alpha \phi(x)$ vanishes for $x \notin \clos{B_N}$, it is clear that $\supp(R_N(\cdot, h))$ must also be contained in $\clos{B_{N+1}}$ (again, for $\norm{h} \leq 1$). This shows that $\supp(R_N)$ is contained in $\clos{B_{N+1}}$ for $\abs{h}$ sufficiently small. 
	
	Now let $\beta$ be a multi-index and define $C \ceq \frac{1}{N!} \max_{\abs{\alpha} = N+1, x \in \RR^n} \abs{(\pt^{\alpha + \beta})\phi(x)}$ (note that $C$ exists and is finite since all partial derivatives of $\phi$ have compact support), then we have
	\begin{align*}
	\abs{\pt^\beta R_N(x, h)} &= \abs{\pt^\beta \sum_{\abs{\alpha} = N+1} \frac{h^\alpha}{\alpha!}(N+1)\int_0^1 (1-t)^N (\pt^\alpha\phi)(x + th)\dd{t}} \\
	%&= \abs{\sum_{\abs{\alpha} = N+1} \frac{h^\alpha}{\alpha!}(N+1) \cdot \pt^\beta\int_0^1 (1-t)^N (\pt^\alpha\phi)(x + th)\dd{t}} \\
	&\overset\star= \abs{\sum_{\abs{\alpha} = N+1} \frac{h^\alpha}{\alpha!}(N+1) \int_0^1 (1-t)^N (\pt^{\alpha + \beta}\phi)(x + th)\dd{t}} \\
	&\leq \sum_{\abs{\alpha} = N+1} \frac{\abs{h^\alpha}}{\alpha!}(N+1) \int_0^1 (1-t)^N \abs{\qty(\pt^{\alpha + \beta}\phi)(x+th)} \dd{t} \\
	&\leq \qty[\max_{\abs{\alpha} = N+1, x \in \RR^n} \abs{(\pt^{\alpha + \beta})\phi(x)}] \sum_{\abs{\alpha} = N+1} \frac{\abs{h^\alpha}}{\alpha!} (N+1) \\
	&\leq C (N+1)! \sum_{\abs{\alpha}= N+1}\frac{\abs{h^\alpha}}{\alpha!} = C(\abs{h_1} + \dotsb + \abs{h_n})^{N+1}.
\end{align*}
Since this upper bound does not depend on $x$, we also have
\[
\sup_x \abs{\pt^\beta R_N(x, h)} \leq C(\abs{h_1} + \dotsb + \abs{h_n})^{N+1}, 
\]
and we conclude that
\[ 
\frac{\sup_x \abs{\pt^\beta R_N(x, h)}}{\norm{h}^N} \leq \frac{C(\abs{h_1} + \dotsb + \abs{h_n})^{N+1}}{\norm{h}^N} \leq \frac{C N^{N+1} \norm{h}^{N+1}}{\norm{h}^N} = CN^{N+1} \norm{h} \to 0, 
\]
and therefore that $\pt^\beta R_N(x, h) = o(\norm{h}^n)$ for all multi-indices $\beta$. 
\end{solution}
\end{document}