\section{Applications of Fourier transform}
\subsection{Elliptic regularity}
Recall that $D = -i\pt$. If $p$ is an $N$-th order polynomial, then $p(D)$ is called an $N$-th order differential operator. 

\begin{definition}
	For an $N$-th order differential operator $p(D) = \sum_{\abs{\alpha} \leq N} c_\alpha D^\alpha$, define its \emph{principal symbol} $\sigma_p(\lambda)$ by \[
	\sigma_p(\lambda) \ceq \sum_{\abs{\alpha} = N} c_\alpha \lambda^\alpha \qquad(\lambda \in \RR^n). 
	\]
	The operator $p(D)$ is called \emph{elliptic} if $\sigma_p(\lambda) \neq 0$ for $\lambda \neq 0$. 
\end{definition}

\begin{lemma}
	If $p(D)$ is an $N$-th order elliptic partial differential operator, then there exist $R > 0$ such that, $C > 0$ such that 
	\[
	\abs{p(\lambda)} \geq C \ang{\lambda}^N \qquad\text{if } \norm{\lambda} > R. 
	\]
\end{lemma}

\begin{proof}
Let $C_0 > 0$ be the minimum of $\abs{\sigma_p}$ on $S^{n-1}$, then for $\lambda \neq 0$ we have
\[
\abs{\sigma_p(\lambda)} = \abs{\sum_{\abs{\alpha} = N} c_\alpha \lambda^\alpha} = \norm{\lambda}^N \abs{\sigma_p(\lambda/\norm{\lambda})}  \geq \norm{\lambda}^N C_0. 
\]
By the triangle inequality we find
\[
\abs{p(\lambda)} \geq \abs{\sigma_p(\lambda)} - \abs{\sigma_p(\lambda) - p(\lambda)} \geq \qty[ C_0 - \abs{\frac{p(\lambda) - \sigma_p(\lambda)}{\norm{\lambda}^N}}] \norm{\lambda}^N
\]
Since $p - \sigma_p$ is a polynomial of order $N - 1$, we can choose $R$ sufficiently large s.t.\ $\abs{\frac{p(\lambda) - \sigma_p(\lambda)}{\norm{\lambda}^N}} < C_0/2$. Since $\ang{\lambda} \sim \norm{\lambda}$ for $\lambda$ large enough, we find that there exists $C$ such that
\[
\abs{p(\lambda)} \geq \frac{C_0}{2} \norm{\lambda}^N \geq C \ang{\lambda}^N
\]
for $\norm{\lambda} > R$. 
\end{proof}

We will try to prove the \emph{elliptic regularity theorem}:
\begin{theorem}[Elliptic regularity] \label{thm:elliptic_regularity}
	Suppose $p(D)$ is an $N$-th order elliptic partial differential operator and $u \in \DD'(X)$ satisfies $p(D) u \in H^s\loc(X)$, then $u \in H^{s+N}\loc(X)$. 
\end{theorem}

\begin{corollary}
	If $p(D)$ is $N$-th order elliptic and $p(D)u \in C^\infty(X)$, then $u \in C^\infty(X)$. 
\end{corollary}

We will first prove an ``easy version'' of \cref{thm:elliptic_regularity} using a \emph{parametrix}: 
\begin{definition}
	If $p(D)$ is an $N$-th order differential operator, then $E \in D'(\RR^n)$ is called a \emph{parametrix} for $p(D)$ if
	\[
	p(D)E = \delta_0 + \omega \qquad\text{for some $\omega \in \EE(\RR^n)$}. 
	\]
\end{definition}

\begin{lemma}
	Every elliptic partial differential operator $p(D)$ has a parametrix which is smooth on $\RR^n \setminus \qty{0}$. 
\end{lemma}

\begin{proof}
	Since $p(D)$ is elliptic, we can choose $R > 0$, $C > 0$ such that $\abs{p(\lambda)} \geq C \ang{\lambda}^N$  for $\norm{\lambda} > R$. 
	
	Fix some $\chi_R \in \DD(\RR^n)$ such that $\chi_R = 1$ on $\norm{\lambda} \leq R$ and $\chi_R = 0$ on $\norm{\lambda} > R + 1$, and define
	\[
	\hat E(\lambda) \ceq \frac{1 - \chi_R(\lambda)}{p(\lambda)}. 
	\]
	Then $\tilde E$ is smooth and for $\lambda$ sufficiently large we have $\abs*{\hat E(\lambda)} \les \ang{\lambda}^{-N}$ since $\chi_R$ vanishes for large $\lambda$, which implies $\hat E \in \SS'(\RR^n)$. Therefore, $p(\lambda) \hat E = 1 - \chi_R(\lambda)$ is also a tempered distribution and we can take its inverse Fourier transform $p(D) E = \delta_0 + \omega$ for some $\omega \in \SS(\RR^n)$, which shows that $E$ is a parametrix. 
	
	To prove that $E$ is smooth on $\RR^n \setminus \qty{0}$, consider for $\norm{\lambda} > R + 1$
	\[
	\abs{\FF[D^\beta (x^\alpha E)]} = \abs{\lambda^\beta D^\alpha \hat E} = \abs{\lambda^\beta D^\alpha\qty(\frac1{p(\lambda)})} \overset\star\les \norm{\lambda}^{\abs{\beta} - \abs{\alpha} - N},
	\]
	where $\star$ can be shown with an induction argument. For each $\beta$, we can simply choose $\abs{\alpha}$ large enough such that $\FF[D^\beta (x^\alpha E)] \in L^1(\RR^n)$, and therefore $D^\beta(x^\alpha E)$ is continuous for $\abs{\alpha}$ large enough. Since $\beta$ was randomly chosen, $E$ will be smooth outside the origin. 
\end{proof}

We will now consider the proof of \cref{thm:elliptic_regularity} in the special case that $u$ and $f \ceq p(D)u$ have compact support. 
\begin{proof}
	Let $E$ be a parametrix for $P$, then we have
	\[
	u= \delta_0 * u = [p(D) E - \omega] * u = p(D)E * u - \omega * u = E * f - \omega * u. 
	\]
	Since $u$ has compact support, $\omega * u$ will be a Schwartz function, and it can be shown that
	\[
	\abs{\FF[E * f](\lambda)} = \abs{\hat E(\lambda) \hat f(\lambda)} \les \ang{\lambda}^{-N} \abs{\hat f(\lambda)}, 
	\]
	which shows that $f \in H^s(\RR^n) \implies u \in H^{s + N}(\RR^n)$. 
\end{proof}

To prove \cref{thm:elliptic_regularity} in general, we will need some facts which are proved on the second example sheet:
\begin{enumerate}
	\item If $s > t$ then $H^s(\RR^n) \subseteq H^t(\RR^n)$;
	\item If $\phi \in \SS(\RR^n)$, $u \in H^s(\RR^n)$, then $\phi u \in H^s(\RR^n)$;
	\item If $u \in \EE'(\RR^n)$, then $u \in H^t(\RR^n)$ for some $t \in \RR$;
	\item If $u \in H^s(\RR^n)$, then $D^\alpha u \in H^{s - \abs{\alpha}}(\RR^n)$. 
\end{enumerate}
Now we prove the theorem:
\begin{proof}
	Fix $\phi \in \DD(X)$, we wish to prove that $\phi u \in H^{s + N}(\RR^n)$ given that $p(D) u \in H^s\loc(X)$.  Choosing $M \in \NN$, we introduce a collection $\qty{\psi_0, \dotsc, \psi_M} \subseteq \DD(X)$ such that
	\[
	\supp(\phi) \subseteq \supp(\psi_M) \subseteq \dotsb \subseteq \supp(\psi_0), \quad \psi_{i-1} = 1 \text{ on $\supp \psi_i$}, \quad \psi_M = 1 \text{ on $\supp\phi$}. 
	\]
	
	Consider $\psi_0 u \in \EE'(\RR^n)$. Then there exists $t \in \RR$ for which $\phi_0 u \in H^t(\RR^n)$. We compute
	\begin{align*}
		p(D)(\psi_1 u) = \psi_1 p(D) u + [p(D), \psi_1](u) = \psi_1 f + [p(D), \psi_1](\psi_0 u),
	\end{align*}
where the last equality follows from the fact that $\psi_0 u \equiv u$ on $\supp \psi_1$. Now note that $[p(D), \psi_1]$ is an order $N-1$ differential operator. So we have $\psi_1 f \in H^s(\RR^n)$ and $[p(D), \psi_1](\psi_0 u) \in H^{t - N + 1}(\RR^n)$.
Setting $\tilde A_1 \ceq \min(s, t - N + 1)$ we find that $p(D)(\psi_1 u) \in H^{\tilde A_1}(\RR^n)$. 

Since $\abs{p(\lambda)} \ges \ang{\lambda}^N$, we find that
\begin{align*}
p(D)(\psi_1 u) \in H^{\tilde A_1}(\RR^n) &\implies \int_{\RR^n} \ang{\lambda}^{2 \tilde A_1} \abs{p(\lambda) \FF[\psi_1u](\lambda)}^2 \dd{\lambda} \\
&\implies \int_{\RR^n} \ang{\lambda}^{2\tilde A_1 + 2N} \abs{\FF[\psi_1 u](\lambda)}^2 \dd{\lambda} \\
&\implies \psi_1 u \in H^{\tilde A_1 + N}(\RR^n).
\end{align*}
Define $A_1 \ceq \tilde A_1 + N = \min\qty{s + N, t + 1}$, then we have shown that $\psi_1 u \in H^{A_1}(\RR^n)$. By carrying on inductively, we can show that $\psi_M u \in H^{A_M}(\RR^n)$ where $A_M = \min\qty{s + N, t + M}$. By choosing $M$ large enough we conclude $\psi_M u \in H^{s + N}(\RR^n)$, and since $\psi_M = 1$ on $\supp\phi$, this also shows that $\phi u \in H^{s + N}(\RR^n)$. Since $\phi$ was randomly chosen, it follows that $u \in H^{s+N}\loc(X)$.  
\end{proof}

\subsection{Fundamental solutions}
\begin{definition}
	Let $p(D)$ be a partial differential operator, then $E \in \DD'(\RR^n)$ is called a \emph{fundamental solution} for $p(D)$ if $p(D)E = \delta_0$. 
\end{definition}

\begin{example}
	Let $z = x_1 + i x_2 \in \CC$ and define the Cauchy-Riemann operator as $\pdv{\bar z} \ceq \frac12 \qty(\pdv{x_1} + i \pdv{x_2})$. 
	It can be shown that $E \ceq \frac1{\pi z}$ is a fundamental solution of this equation. 
\end{example}

\begin{example}
	Let $p(D) = \pdv{t} - \Delta x$ be the heat operator (where $\Delta = \pdv[2]x_1 + \dotsb + \pdv[2]{x_n}$) with coordinates $(t, x) \in \RR \times \RR^n$. Then it can be shown that
	\[
	E \ceq \begin{cases}
		(4\pi t)^{-n/2} \exp(-\frac{\norm{x}}{4t}), &t > 0, \\ 0, & t \leq 0,
	\end{cases}
	\]
	is a fundamental solution. 
	
	Furthermore, if $f$ has compact support, then $u = E * f$ solves $p(D) u = f$, since in this case
	\[
	p(D)(E * f) = (p(D) E * f) = \delta_0 * f = f. 
	\]
\end{example}

As a guess to construct fundamental solutions, we can use the Fourier transform: we have
\begin{align*}
	p(D) E = \delta_0 &\implies p(\lambda) \hat E = 1 \implies \hat E = \frac{1}{p(\lambda)}  \\
	&\implies \ang{E, \phi} = \ang{E,\frac{1}{(2\pi)^n} \FF[\refl{\FF[\phi]}]} = \frac{1}{(2\pi)^n}\ang{\hat E, \refl{\FF[\phi]}} =\frac{1}{(2\pi)^n} \int \frac{\hat\phi(-\lambda)}{p(\lambda)} \dd{\lambda}. 
\end{align*}

Indeed, one can check that this $E$ ``works'', but the problem is that we have no guarantee that $E \in \DD'(\RR^n)$, since $p(\lambda)$ may cause problems at its roots. To circumvent this, we have to use a construction called \emph{H\"ormander's staircase}.
For this, we will first need a lemma. For $x \in \RR^n$, we will write $x = (x', x_n)$ with $x' \in \RR^{n-1}$. 
\begin{lemma}
	For each $\phi \in \DD(\RR^n)$ and $\lambda' \in \RR^{n-1}$, the function $z \mapsto \hat\phi(\lambda', z)$ is analytic in $z\in\CC$. Furthermore, for each $m \in \NN_0$ there exists constants $c_m, \delta > 0$ (independent of $\lambda'$) such that
	\[
	\abs{\hat\phi(\lambda', z)} \leq c_m(1 + \abs{z})^{-m} e^{\delta \abs{\Im z}}. 
	\] 
\end{lemma}

\begin{proof}
	By definition of the Fourier transform and Fubini's theorem, we have
	\[
	\hat\phi(\lambda', z) = \int e^{-i\lambda' \cdot x'} \int e^{-i z x_n} \phi(x', x) \dd{x_n} \dd{x'}. 
	\]
	It is easily seen that this function is smooth in $z$ and satisfies the Cauchy-Riemann equations, which means it is analytic. 
	
	Integrating by parts we find
	\begin{align*}
		\abs{z^m \hat\phi(\lambda', z)} &= \abs{\int e^{-i\lambda' \vdot x'} \int \qty(i \pdv{x_n})^m e^{-i z x_n} \phi(x', x_n) \dd{x_n} \dd{x'} } \\
		&= \abs{\int e^{-i\lambda' \vdot x'} \int e^{-i z x_n} \qty(\pdv[m]{x_n} \phi(x', x_m)) \dd{x_m} \dd{x'}} \\
		&\leq \iint \abs{e^{-iz x_n}} \cdot \abs{\pdv[m]{x_n} \phi(x', x_n)} \dd{x_n} \dd{x'} \\
	&\leq c_m e^{\delta \abs{\Im z}}, 
	\end{align*}
	where $\delta$ is chosen such that $\phi(x', x_n) = 0$ if $\abs{x_n}> \delta$. 
\end{proof}

Now, we can prove the main theorem of this section, which \emph{almost} gives an explicit construction for a fundamental solution:
\begin{theorem}
	Every nonzero constant-coefficient partial differential operator has a fundamental solution. 
\end{theorem}

\begin{proof}
	By rotating our coordinate axes, we can assume $p$ takes the form
	\[
	p(\lambda', \lambda_n) = \lambda_n^M + \sum_{m=1}^{M-1} a_m(\lambda') \lambda_n^m, 
	\]
	(??) (i.e., we simply write $p$ as a polynomial in $\lambda_n$). Fix $\mu' \in \RR^{n-1}$, then we can write
	\[
	p(\mu', \lambda_n) = \prod_{i=1}^M (\lambda_n - \tau_i(\mu')), 
	\]
	where the $\tau_i$ are the roots of the polynomial $\lambda_n \mapsto p(\mu', \lambda_n)$. Now, by the pigeonhole principle, there exists a horizontal line $\Im\lambda_n = c(\mu')$ in the region $\abs{\Im\lambda_n} \leq M + 1$ such that
	\[
	\abs{\lambda_n - \tau_i(\mu')} > 1 \quad\text{on $\Im\lambda_n = c(\mu')$} \quad (i = 1, \dotsc, m)
	\]
	Therefore, on $\Im(\lambda_n) = c(\mu')$ we have $\abs{p(\lambda', \lambda_n)} \ges 1$. 
	
	Since roots of a polynomial vary continuously with its coefficients, we can use the same horizontal line $\Im\lambda_n = c(\mu')$ for all $\lambda'$ in a (small) neighbourhood $N(\mu')$ of $\mu'$. We can cover all of $\RR^{n-1}$ with such neighbourhoods, and by the Heine-Borel theorem, we can extract a locally finite subcover $N_1 = N(\mu_1'), N_2 = N(\mu_2'), \dotsc$. Furthermore, we can modify these neighbourhoods so that they are disjoint by defining 
	\[
	\Delta_i = N_i \setminus \qty(\bigcup_{j=1}^{i-1} \overline{N_j}). 
	\]
	The $\Delta_i$ are all open, disjoint, and satisfy $\RR^{n-1} = \cup_i \overline{\Delta_i}$.
	Now we define
	\[
	\ang{E, \phi} \ceq \frac1{(2\pi)^n} \sum_{i=1}^\infty \int_{\Delta_i} \int_{\Im\lambda_n = c_i} \frac{\hat\phi(-\lambda' - \lambda_n)}{p(\lambda', \lambda_n)} \dd{\lambda_n} \dd{\lambda'}. 
	\]
	In ES3, it is shown that $E \in \DD'(\RR^n)$. Furthermore, we have
	\begin{align*}
		\ang{p(D)E, \phi} &= \ang{E, p(-D)\phi} = \frac1{(2\pi)^n} \sum_{i=1}^\infty \int_{\Delta_i} \int_{\Im\lambda_n = c_i} \frac{p(\lambda', \lambda_n)}{p(\lambda', \lambda_n)} \hat\phi(-\lambda' - \lambda_n) \dd{\lambda_n} \dd{\lambda'} \\
		&= \frac1{(2\pi)^n} \sum_{i=1}^\infty \int_{\Delta_i} \int_{\Im\lambda_n = c_i}  \hat\phi(-\lambda' - \lambda_n) \dd{\lambda_n} \dd{\lambda'}  \\
		&\overset\star= \frac1{(2\pi)^n} \sum_{i=1}^\infty \int_{\Delta_i} \int_{\Im\lambda_n = 0}  \hat\phi(-\lambda' - \lambda_n) \dd{\lambda_n} \dd{\lambda'} = \frac1{(2\pi)^n} \int_{\RR^n} \hat\phi(-\lambda) \dd{\lambda} = \phi(0).
	\end{align*} 
by the Fourier inversion theorem. Here, $\star$ follows from the Cauchy's theorem and the previous lemma ($\hat\phi$ decays rapidly in the horizontal direction, so taking a contour integral over a rectangle and letting the vertical side go to infinity shows that the integral over $\Im \lambda_n = c_i$ equals the integral over $\Im\lambda_n = 0$). We conclude that $p(D)E = \delta_0$. 
\end{proof}

Note that the only nonconstructive part of the theorem is the extraction of a locally finite subcover of the neighbourhoods $N(\mu')$. 

\subsection{Structure theorem for distributions of compact support}
In this section, we will prove that every $u \in \EE'(X)$ can be written as a finite sum $u = \sum_\alpha \pt^\alpha f_\alpha$ where the $f_\alpha$ are continuous. The theorem generalises to $u \in \DD'(X)$ (the sum can then be infinite, but locally finite), but we will not prove this, since it requires the use of partitions of unity. 

We start with a lemma:
\begin{lemma}
	For $u \in \EE'(\RR^n)$, the Fourier transform $\hat u \in \SS'(\RR^n)$ can be identified with the smooth (real-analytic) function $\lambda \mapsto \ang{u(x), e^{-i\lambda\vdot x}}$, which we will denote $\hat u(\lambda)$. 
\end{lemma}

\begin{proof}
	We will first prove the density of $\DD(\RR^n)$ in $\SS(\RR^n)$. 
	Fix $\phi \in \SS(\RR^n)$ and $\chi \in \DD(\RR^n)$ with $\chi = 1$ on $\norm{x}\leq1$ and $\chi = 0$ on $\norm{x}>2$. Define $\phi_m \in \DD(\RR^n)$ by $\phi_m(x) \ceq \phi(x) \chi(x/m)$. We will show that $\phi_m \to \phi \in \SS(\RR^n)$. 
	
	For each pair of multi-indices $\alpha, \beta$, we have
	\begin{align*}
		\norm{\phi - \phi_m}_{\alpha, \beta} &= \norm{x^\alpha D^\beta( \phi - \phi_m)}_\infty = \norm{x^\alpha D^\beta(\phi \cdot \qty{1 - \chi(x/m)})} \\
		&= \norm{x^\alpha \sum_{\gamma \leq \beta} \binom\beta\gamma (D^\gamma\phi)(x) \cdot D^{\beta - \gamma} (1 - \chi(x/m))}. 
	\end{align*}
	For $\gamma \neq \beta$, the derivative $D^\gamma\phi$ is bounded uniformly while the derivative $D^{\beta - \gamma} (1 - \chi(x/m))$ will converge uniformly to 0 since it will have at least one factor $1/m$. For $\gamma = \beta$, we have
	\[
	\norm{x^\alpha (1 - \chi(x/m)) D^\beta\phi}_\infty \leq \sup_{\norm{x} > M} \norm{x^\alpha D^\beta\phi} \to 0,
	\]
	since $D^\beta\phi$ decays rapidly. We conclude that $\norm{\phi - \phi_m}_{\alpha, \beta} \to 0$, and therefore that $\phi_m \to \phi$ in $\SS(\RR^n)$. 
	
	Now, by a Riemann sum argument (like the one we have used in \cref{lem:convolution_associative}) we have
	\begin{align*}
		\ang{\hat u, \phi_m} &= \ang{u, \hat \phi_m} = \ang*{u(x), \int e^{-i\lambda\vdot x} \phi_m(\lambda) \dd{\lambda}} 
		\overset\star= \int \ang{u(x), e^{-i\lambda\cdot x}} \phi_m(\lambda) \dd{\lambda}, 
		\end{align*}
	where $\star$ is the Riemann sum argument (here, we need that $\phi_m$ has compact support).  Now, since $u \in \EE'(\RR^n)$, there exists a compact $K$ and constants $C', N > 0$ such that
	\[
	\abs{\ang{u(x), e^{-i\lambda\cdot x}}} \leq C' \sum_{\abs{\alpha} \leq N} \sup_K \abs{D_x e^{-i\lambda\cdot x}} \leq C \ang{\lambda}^N, 
	\]
	so $\lambda \mapsto \ang{u(x), e^{-i\lambda\cdot x}}$ is polynomially bounded, and therefore we can use the dominated convergence theorem to conclude
	\[
	\ang{\hat u, \phi} = \lim_{n\to\infty} \ang{\hat u, \phi_m} = \lim_{n\to\infty} \int \ang{u(x), e^{-i\lambda\cdot x}} \phi_m(\lambda) \dd{\lambda} = \int \ang{u(x), e^{-i\lambda\cdot x}} \phi(\lambda) \dd{\lambda}, 
	\]
	which proves that $\hat u$ can be identified with the function $\hat u(\lambda) = \ang{u(x), e^{-i\lambda\vdot x}}$. 
\end{proof}

Furthermore, it is clear that for $u \in \EE'(\RR^n)$, we have 
\begin{equation} \label{eq:fourier_estimate}
	\abs{\hat u(\lambda)} \leq C \sum_{\abs{\alpha} \leq N} \sup_K \abs{\pt^\alpha_x e^{-i\lambda x}} \les \ang{\lambda}^N.
\end{equation}

\begin{theorem}[Structure theorem]
	For $u \in \EE'(X)$, there exists a finite collection $\qty{f_\alpha} \subseteq C(X)$ with $\supp(f_\alpha) \subseteq X$ such that $u = \sum_\alpha \pt^\alpha f_\alpha$ in $\EE'(X)$. 
\end{theorem}

\begin{proof}
	Fix $\rho \in \DD(X)$ with $\rho = 1$ on a neighbourhood of $u$, then we can extend $u$ to $\EE'(\RR^n)$ by setting $\ang{u, \phi} \ceq \ang{u, \rho\phi}$ (note that $\rho\phi \in \DD(X)$ for all $\phi \in \EE(\RR^n)$). Since $\rho\phi \in \SS(\RR^n)$, we know there exist $\psi \in \SS(\RR^n)$ such that
	\[
	\rho\phi = \FF[\FF[\psi]] = (2\pi)^n \refl\psi, 
	\]
	and therefore
	\[
	\ang{u, \rho\phi} = \ang{u, \FF[\FF[\psi]]} = \ang{\hat u, \hat\psi}. 
	\]
	Using the Laplacian $\Delta = \sum_i \pt^i \pt^i$, we can write for any $m \in \NN$
	\[
	\hat\psi = \ang{\lambda}^{-2M} \FF\qty[(1 - \Delta)^M \psi],
	\]
	since $\FF\qty[(1 - \Delta)^m \psi] = (1 + \norm{\lambda}^2)^m \hat\psi = \ang{\lambda}^{2M}\hat\psi$. 
	
	Plugging this back into our previous equations, we have
	\begin{equation} \label{eq:fourier_laplace}
		\ang{\hat u, \hat\psi} = \ang{\hat u, \ang{\lambda}^{-2M} \FF[(1 - \Delta)^M \psi]} = \ang*{\FF[\hat u \ang{\lambda}^{-2M}], (1 - \Delta)^M \psi}.
	\end{equation}
Now, by \cref{eq:fourier_estimate}, we have $\abs{\hat u(\lambda)} \les \ang{\lambda}^N$, so we can choose $M$ large enough such that $\hat u(\lambda) \cdot \ang{\lambda}^{-2M} \in L^1(\RR^n)$, and by the dominated convergence theorem, the function
\[
f(x) = \frac1{(2\pi)^n} \int e^{-i\lambda \vdot x} \ang{\lambda}^{-2M} \hat u(\lambda) \dd{\lambda} 
\]
is continuous, and it is easily checked that $(2\pi)^n \refl f = \FF[\ang{\lambda}^{2M} \hat u(\lambda)]$. 

Using the fact that $(2\pi)^n \refl\psi = \rho\phi$, and going back to \cref{eq:fourier_laplace} we see
\[
\ang{u, \rho\phi} = \ang{(2\pi)^n \refl f, (1 - \Delta)^M \psi} = \ang{\refl f, (1 -\Delta)^M \refl{(\rho\phi)}} = \ang{f, (1 - \Delta)^M (\rho\phi)}, 
\]
where the last step follows from the fact that the Laplacian is reflection invariant. 

Expanding the derivatives using the Leibniz rule yields
\[
(1 -\Delta)^M (\rho\phi) = \sum_\alpha (-1)^{\abs{\alpha}} \rho_\alpha \pt^\alpha\phi
\]
for suitable $\rho_\alpha \in \DD(X)$, and therefore we have
\[
\ang{u, \phi} = \sum_\alpha \ang{f, (-1)^{\abs{\alpha}} \rho_\alpha \pt^\alpha\phi} = \ang*{\sum_\alpha \pt^\alpha(\rho_\alpha f), \phi},
\]
so $u = \sum_\alpha \pt^\alpha(\rho_\alpha f) = \sum_\alpha \pt^\alpha f_\alpha \in \EE'(\RR^n)$, where $f_\alpha$ is continuous and $\supp(f_\alpha) = \supp(\rho_\alpha f) \subseteq X$. 
\end{proof}

There also exist nonconstructive proofs for the previous theorem using Hahn-Banach. 

\subsection{Paley-Wiener-Schwartz theorem}
We have seen that if $u \in \EE'(\RR^n)$, then $\hat u$ is equivalent to the real-analytic function $\lambda \mapsto \ang{u(x), e^{-i\lambda \vdot x}}$ and $\abs{\hat u(\lambda)} \les \ang{\lambda}^N$ for some $N \geq 0$. 
We consider the complex extension of $\hat u$, and we call $\hat u(z)$ the \emph{Fourier-Laplace transform} of $u$. 
Note that
\[
\pdv{\hat u}{\bar z_i} = \ang{u, \pdv{\bar z_i} e^{-iz\vdot x}} = 0 \quad(i = 1, \dotsc, n), 
\]
so $\hat u(z)$ is complex-analytic in each component $z_i$. 

We can also estimate the size of $\hat u(z)$:
\begin{lemma}
	If $u \in \EE'(\RR^n)$ and $\supp(u) \subseteq \clos{B_\delta}$, then there exist nonnegative constant $C, N$ such that
	\[
	\norm{\hat u(z)} \leq C (1 + \norm{z})^N e^{\delta \abs{\Im z}} \qquad \forall z \in \CC^n. 
	\]
\end{lemma}

\begin{proof}
	Let $\psi \in C^\infty(\RR)$ be such that $\psi(\tau) = 1$ for $\tau \geq -\frac12$ and $\psi(\tau) = 0$ for $\tau \leq -1$.  Introduce for $\eps > 0$
	\[
	\phi_\eps(x) \ceq \psi(\eps(\delta - \norm{x})),
	\]
	then we have $\phi_\eps \in \DD(\RR^n)$ with $\phi_\eps(x) = 0$ for $\norm{x}\geq \delta + \eps^{-1}$, and $\phi_\eps(x) = 1$ for $\norm{x} \leq \delta + \frac12\eps^{-1}$. In particular, every $\phi_\eps$ is 1 on a neighbourhood of $\supp(u) \subseteq \clos{B_\delta}$. 
	
	Therefore, we have
	\[
	\hat u(z) = \ang{u(x), e^{-iz\vdot x}} = \ang{u(x), \phi_\eps(x) e^{-iz\vdot x}},
	\]
	and since $u \in \EE'(\RR^n)$, by the seminorm condition we have nonnegative $C, N$ such that
	\[
	\hat u(z) \leq C \sum_{\abs{\alpha} \leq N} \sup \abs{\pt^\alpha_x \qty(\phi_\eps(x) e^{-iz \vdot x})}.
	\]
	By definition we have $\abs{\pt_x^\beta \phi_\eps} \les \eps^{\abs{\beta}}$ while $\abs{\pt_x^\gamma e^{-iz \vdot x}} \les \norm{z}^{\abs{\gamma}} e^{(\delta + \eps^{-1}) \abs{\Im z}}$ on $\supp\phi_\eps$. 
	
	Applying Leibniz, we obtain
	\[
	\abs{\hat u(z)} \leq C \sum_{\abs{\beta} + \abs{\gamma} \leq N} \norm{z}^{\abs{\beta}} e^{\abs{\gamma}} e^{(\delta + \eps^{-1}) \Im z}, 
	\]
	and this holds for all $\eps > 0$, so we can plug in $\eps = \norm{z}$ and obtain the result (since $\Im z / \norm{z}$ is bounded). 
\end{proof}

So if $u \in \EE'(\RR^n)$ with $\supp(u) \subseteq \clos{B_\delta}$, then $\hat u(z)$ is complex analytic and obeys $\abs{\hat u(z)} \leq (1 + \norm{z})^N e^{\delta \abs{\Im z}}$. The PWS theorem addresses the converse: if a complex analytic function obeys the estimate we just saw, is it the fourier transform of some distribution? 

\begin{theorem}
	\begin{enumerate}[(a)]
		\item If $u \in \EE'(\RR^n)$ with $\ord(u) = N$ and $\supp u \subseteq \clos{B_\delta}$, then $\hat u(z)$ is entire and
		\begin{equation} \label{eq:fourier_transform_estimate}
			\abs{\hat u(z)} \les (1 + \norm{z})^N e^{\delta \abs{\Im z}}. 
		\end{equation}
	
	Conversely, if $U(z)$ is entire and satisfies \cref{eq:fourier_transform_estimate} for some $N$, then $U = \hat u$ for some $u \in \EE'(\RR^n)$ with $\supp(u) \subseteq \clos{B_\delta}$. 
	
	\item If $u \in \DD(\RR^n)$ and $\supp(u) \subseteq \clos{B_\delta}$, then for $M = 0, 1, 2, \dotsc$ we have 
	\begin{equation}\label{eq:fourier_transform_estimate_2}
		\abs{\hat u(z)} \les_M (1 + \norm{z})^{-M} e^{\delta \abs{\Im z}}.
	\end{equation}
	Conversely, if $U(z)$ is entire and satisfies \cref{eq:fourier_transform_estimate_2}, then $U = \hat u$ for some $u \in \DD(\RR^n)$ with $\supp(u) \subseteq \clos{B_\delta}$. 
	\end{enumerate}
\end{theorem}

\begin{proof}
	\TODO write this out (lecture 12)
\end{proof}

\subsection{Oscillatory integrals}
We will study integrals of the form $\int_{\RR^n} e^{i\Phi(x, \theta)} a(x, \theta) \dd{\theta}$, where $\Phi$ is called the \emph{phase function} and $a$ the \emph{amplitude} of the signal. We wil use oscillations from the $e^{i\Phi}$-term to control the growth, while $a(\cdot, \theta)$ is allowed to grow modestly with $\theta$. 

\begin{lemma}[Riemann-Lebesgue]
	If $f \in L^1(\RR)$, then $\abs{\hat f(\lambda)} \to 0$ when $\abs{\lambda} \to\infty$. 
\end{lemma}
\begin{proof}
	Assume $f \in L^1$ is continuous, then setting $x' = x + \pi/\lambda$, we have
	\begin{align*}
		\hat f(\lambda) &= \int e^{-i\lambda \cdot x} f(x) \dd{x} = \frac12\int \qty( e^{-i\lambda \vd x}f(x) \dd{x} + e^{-i\lambda \vd (x + \pi/\lambda)} f(x + \pi/\lambda))\dd{x} \\
		&= \frac12 \int e^{-i\lambda \vd x} \qty[ f(x) - f(x + \pi/\lambda)] \dd{x}
	\end{align*}
	Now let $\eps > 0$ and choose $R = R(\eps)$ such that $\int_{\norm{x}> R} \abs{f(x) - f(x + \pi/\lambda)} \dd{x} < \frac\eps2$.
	By the dominated convergence theorem, we can also choose $\lambda = \lambda(\eps, R)$ large enough such that
	\[
	\int_{\norm{x}< R} \abs{e^{-i\lambda x} \qty[f(x) - f(x + \pi/\lambda)]} \dd{x} < \frac\eps2. 
	\]  
	So for $f$ we have $\abs{\hat f(\lambda)} < \frac\eps2$ for $\abs{\lambda}$ large enough, which proves the result for continuous functions.
	
	Now we use that continuous functions are dense in $L^1$, so for any $f \in L^1$, pick $g \in C(\RR) \cap L^1$ such that $\norm{f - g}_{L^1} < \frac\eps2$, then it is easily checked that $\abs{\hat f(\lambda)} \leq \norm{f - g}_{L^1} + \abs{\hat g(\lambda)} < \eps $ for $\lambda$ sufficiently large, which proves the claim. 
	
\end{proof}

Now suppose $\phi \in \DD(\RR)$ and $\Phi \in C^\infty(\RR)$ with $\Phi'$ nowhere 0. Consider
\[
I(\lambda) \ceq \int e^{i\lambda \Phi(\theta)} \phi(\theta) \dd{\theta}. 
\]
Note that $\abs{\Phi'(\theta)} \ges 1$ for $\theta \in \supp\phi$ (since $\supp\phi$ is compact), and we can write
\[
I(\lambda) = \int \frac1{i\lambda} \frac{\phi(\theta)}{\Phi'(\theta)} \dv{\theta}\qty(e^{i\lambda \Phi(\theta)}) \dd{\theta},
\]
and with repeated integration by parts we find that
\[
\abs{I(\lambda)} \leq \ang{\lambda}^{-N} \quad\text{for any $N \geq 0$}. 
\]
A natural question is what happens if $\Phi'(\theta) = 0$ somewhere. 

\begin{lemma}[Stationary phase]
	Suppose $\Phi \in C^\infty(\RR)$ with $\Phi' \neq 0$ on $\RR \setminus \qty{0}$ and $\Phi(0) = \Phi'(0) = 0$, $\Phi''(0) \neq 0$. Then, for $\chi \in \DD(\RR)$, we have
	\[
	\int e^{i\lambda \Phi(\theta)} \chi(\theta) \dd{\theta} \les \ang{\lambda}^{-1/2}. 
	\]
\end{lemma}

\begin{proof}
	\TODO write this (lecture 13, a terrible proof with lots of differentiation and garbage). 
\end{proof}