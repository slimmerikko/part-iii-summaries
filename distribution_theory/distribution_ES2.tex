\documentclass{article}

\usepackage[utf8]{inputenc}
\usepackage[english]{babel}
\usepackage[margin=3cm]{geometry}
\usepackage[normalem]{ulem}
\usepackage{hyperref}
\usepackage[shortlabels]{enumitem}
\usepackage{mathtools, amsmath, amssymb, amsthm, mathabx, mdframed, bbm, graphicx, float, physics, xcolor, cleveref}

\hypersetup{
    colorlinks   = true, %Colours links instead of ugly boxes
    urlcolor     = blue, %Colour for external hyperlinks
    linkcolor    = blue, %Colour of internal links
    citecolor   = red %Colour of citations
}

% Definition of numbered environments.
% Usage: \begin{theorem} ... \end{theorem}
% Remark has no numbering.
\theoremstyle{plain}
\newtheorem{question}{Question}
\newtheorem{theorem}{Theorem}
\newtheorem{lemma}{Lemma}
\theoremstyle{remark}
\newtheorem*{remark}{Remark}

% Question with box around it
\newenvironment{qbox}{\begin{mdframed}\begin{question}}{\end{question}\end{mdframed}}

% A proof with "solution" instead of "proof" and no QED symbol
\newenvironment{solution}{\begin{proof}[Solution]\renewcommand\qedsymbol{}}{\end{proof}}

% Some renewed commands
\renewcommand{\vec}{\mathbf}
\renewcommand{\emptyset}{\varnothing}
\renewcommand{\epsilon}{\varepsilon}
\renewcommand{\theta}{\vartheta}
\renewcommand{\phi}{\varphi}

% Frequently used math alphabets
\newcommand{\Bb}{\mathbb}
\newcommand{\Cal}{\mathcal}
\newcommand{\Bf}{\mathbf}
\newcommand{\Rm}{\mathrm}

% Frequently used letters in the blackboard alphabet
\newcommand{\CC}{\Bb C}
\newcommand{\NN}{\Bb N}
\newcommand\ZZ{\Bb Z}
\newcommand{\PP}{\Bb P}
\newcommand{\QQ}{\Bb Q}
\newcommand{\RR}{\Bb R}
\newcommand{\KK}{\Bb K}
\newcommand{\DD}{\Cal D}
\newcommand\EE{\Cal E}
\renewcommand\SS{\Cal S}
\newcommand\FF{\Cal F}
\newcommand\vd{\vdot}
%\newcommand\ZZ{\Bb Z}


% Usage: \ang{...} is equivalent to \langle ... \rangle, while \ang*{...} is equivalent to \left\langle ... \right\rangle
% For other delimiters: use \qty from the physics package (i.e., \qty(...))
\DeclarePairedDelimiter{\ang}{\langle}{\rangle}
\DeclarePairedDelimiter{\floor}{\lfloor}{\rfloor}
\DeclarePairedDelimiter{\ceil}{\lceil}{\rceil}

% Frequently used commands
\newcommand{\T}{^\top} % Matrix transpose A\T
\newcommand{\C}{^\complement} % Set complement A\C
\newcommand\ceq\coloneqq % Definitions :=
\newcommand\pow{\Cal P} % Power sets
\newcommand\eps\epsilon
\newcommand\ind{\mathbbm 1} % Blackboard 1 for indicator functions
\newcommand\restr{\mathord\restriction}
\newcommand\TODO{{\color{red} TODO: }}
\newcommand\pt\partial
\newcommand\locint{L^1_{\Rm{loc}}}
\newcommand\refl\widecheck
\newcommand\les\lesssim
\newcommand\ges\gtrsim
\newcommand\loc{_\Rm{loc}}

% Functions that appear frequently
\DeclareMathOperator{\sign}{sign}
\DeclareMathOperator{\Int}{Int}
\DeclareMathOperator{\Span}{Span}
\DeclareMathOperator{\Var}{Var}
\DeclareMathOperator*{\argmin}{arg\,min}
\DeclareMathOperator*{\argmax}{arg\,max}
\DeclareMathOperator{\supp}{supp}
\DeclareMathOperator{\ord}{ord}

\title{Distribution Theory --- Example Sheet 2} % subject
\author{Lucas Riedstra}
\date{18 November 2020} % date

\begin{document}
\maketitle
\begin{question}
Let $u, v \in \DD'(\RR^n)$, one of which has compact support. Show that the convolution $u * v$, defined as in your notes, is uniquely defined and gives rise to an element of $\DD'(\RR^n)$. 
\end{question}

\begin{proof}
	The convolution between $u, v \in \DD'(\RR^n)$ is defined by the formula
	\begin{equation} \label{def:convolution}
	(u * v) * \phi = u * (v * \phi) \quad\text{for all $\phi \in \DD(\RR^n)$}. 
	\end{equation}
	To show that this is uniquely defined, recall that for all $u \in \DD'(\RR^n) \phi\in \DD(\RR^n)$, we have $\ang{u, \phi} = (u * \refl\phi)(0)$. Therefore, we have
	\[
	\ang{u * v, \phi} = ((u * v) * \refl\phi)(0) = (u * (v * \refl\phi))(0),
	\]
	which shows that the formula \cref{def:convolution} uniquely defines $\ang{u * v, \phi}$ for any $\phi \in \DD(\RR^n)$, and therefore that $u * v$ is well-defined. 
	
	Now we prove that $u * v \in \DD'(\RR^n)$: by the previous equation we have
	\[
	\ang{u * v, \phi} = (u * (v * \refl\phi))(0) = \ang{u, \refl{v * \refl\phi}}.
	\]
	
	Suppose $u$ is compactly supported. Since $\refl{v * \refl\phi} \in \EE(\RR^n)$, there exists a compact $K \subseteq X$ and nonnegative $C, N$ such that
	\begin{align*}
		\abs{\ang{u, \refl{v * \refl\phi}}} &\leq C \sum_{\alpha \leq N} \sup_{x \in K} \abs{\pt^\alpha (\refl{v * \refl\phi})} = C \sum_{\alpha \leq N} \sup_{x \in -K} \abs{\pt^\alpha (v * \refl\phi)} = C \sum_{\alpha \leq N} \sup_{x \in -K} \abs{v * \pt^\alpha\refl\phi}\\
		&= C \sum_{\abs{\alpha} \leq N} \sup_{x \in -K} \abs{\ang{v, \tau_x \refl{\pt^\alpha \refl\phi}}}.
	\end{align*}
	Note that if $\supp\phi \subseteq K'$, then $\supp\refl\phi \subseteq -K'$, and for $x \in -K$ we find $\supp \tau_x \refl{\pt^\alpha \refl\phi} \subseteq -K' -K$. Then by the previous equation we find that there exists $C', M$ with
	\[
	\abs{\ang{u * v, \phi}} \leq C' \sum_{\abs{\alpha} \leq N} \sum_{\abs{\beta} \leq M} \sup_{x \in -K' - K} \pt^\beta (\tau_x \refl{\pt^\alpha \refl\phi}) = C' \sum_{\abs{\alpha} \leq N} \sum_{\abs{\beta} \leq M} \sup_x \abs{\pt^{\alpha + \beta} \phi} \leq C'' \sum_{\abs{\alpha} \leq M + N} \sup_x \abs{\pt^\alpha \phi},
	\]
	which shows that $u* v \in \DD'(\RR^n)$. An analogous argument holds if $v$ is compactly supported. 
\end{proof}

\begin{question}
	Show that if $u, v, w \in \DD'(\RR^n)$ and at least two of them have compact support, then the convolution is associative (i.e., $(u * v) * w) = u* (v * w)$). 
\end{question}

\begin{proof}
	Note that the convolution between two compactly supported distributions is again compactly supported, which ensures that both expressions `make sense'. Now, let $\phi \in \DD(\RR^n)$, then we have
	\begin{align*}
		((u * v) * w) * \phi &= (u* v) * (w * \phi) = u * (v * ( w* \phi)) = u * ((v * w) * \phi) = (u * (v * w)) * \phi,
	\end{align*}
	which proves the theorem. 
\end{proof}

\begin{question}
	Let $\phi \in \DD(\RR)$ and choose $\eps > 0$ sufficiently small so that $\supp(\phi) \subset I_\eps = (-1/\eps, 1/\eps)$. Given that $\phi$ has a uniformly convergent Fourier series on $I_\eps$ in the form
	\[
	\phi(x) = \sum_{n \in \ZZ} c_n e^{i \eps \pi n x}, \quad c_n = \frac\eps2 \int_\RR \phi(x) e^{-i\eps \pi n x} \dd{x},
	\]
	prove the Fourier inversion theorem on $\DD(\RR)$ by taking a suitable limit. 
\end{question}

\begin{proof}
	Let $\psi \in \DD(\RR)$, then we want to prove that \[
	\psi(x) = \frac1{(2\pi)^n} \iint e^{i \lambda (x - y)} \psi(y) \dd{y} \dd{\lambda}. 
	\]
	??
\end{proof}

\begin{question}
	For $\phi \in \SS(\RR^n)$ prove that $\sum_m \phi(m) = \sum_n \hat\phi(2\pi n)$. This is the famous Poisson summation formula. 
\end{question}

\begin{proof}
	We have
	\begin{align*}
		\sum_m \phi(m) = \frac1{(2\pi)^n} \sum_m \int e^{i\lambda m} \hat \phi(\lambda) \dd{\lambda} = \sum_m \int e^{2\pi i \lambda m} \hat \phi(2\pi \lambda) \dd{\lambda}
	\end{align*}
\end{proof}
\end{document}