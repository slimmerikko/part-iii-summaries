\documentclass{article}

\usepackage[utf8]{inputenc}
\usepackage[english]{babel}
\usepackage[margin=3cm]{geometry}
\usepackage[normalem]{ulem}
\usepackage{hyperref}
\usepackage[shortlabels]{enumitem}
\usepackage{mathtools, amsmath, amssymb, amsthm, mathabx, mdframed, bbm, graphicx, float, physics, xcolor, cleveref}

\hypersetup{
    colorlinks   = true, %Colours links instead of ugly boxes
    urlcolor     = blue, %Colour for external hyperlinks
    linkcolor    = blue, %Colour of internal links
    citecolor   = red %Colour of citations
}

% Definition of numbered environments.
% Usage: \begin{theorem} ... \end{theorem}
% Remark has no numbering.
\theoremstyle{plain}
\newtheorem{question}{Question}
\newtheorem{theorem}{Theorem}
\newtheorem{lemma}{Lemma}
\theoremstyle{remark}
\newtheorem*{remark}{Remark}

% Question with box around it
\newenvironment{qbox}{\begin{mdframed}\begin{question}}{\end{question}\end{mdframed}}

% A proof with "solution" instead of "proof" and no QED symbol
\newenvironment{solution}{\begin{proof}[Solution]\renewcommand\qedsymbol{}}{\end{proof}}

% Some renewed commands
\renewcommand{\vec}{\mathbf}
\renewcommand{\emptyset}{\varnothing}
\renewcommand{\epsilon}{\varepsilon}
\renewcommand{\theta}{\vartheta}
\renewcommand{\phi}{\varphi}

% Frequently used math alphabets
\newcommand{\Bb}{\mathbb}
\newcommand{\Cal}{\mathcal}
\newcommand{\Bf}{\mathbf}
\newcommand{\Rm}{\mathrm}

% Frequently used letters in the blackboard alphabet
\newcommand{\CC}{\Bb C}
\newcommand{\NN}{\Bb N}
\newcommand\ZZ{\Bb Z}
\newcommand{\PP}{\Bb P}
\newcommand{\QQ}{\Bb Q}
\newcommand{\RR}{\Bb R}
\newcommand{\KK}{\Bb K}
\newcommand{\DD}{\Cal D}
\newcommand\EE{\Cal E}
\renewcommand\SS{\Cal S}
\newcommand\FF{\Cal F}
\newcommand\vd{\vdot}
\newcommand\Rr{\Cal R}
%\newcommand\ZZ{\Bb Z}


% Usage: \ang{...} is equivalent to \langle ... \rangle, while \ang*{...} is equivalent to \left\langle ... \right\rangle
% For other delimiters: use \qty from the physics package (i.e., \qty(...))
\DeclarePairedDelimiter{\ang}{\langle}{\rangle}
\DeclarePairedDelimiter{\floor}{\lfloor}{\rfloor}
\DeclarePairedDelimiter{\ceil}{\lceil}{\rceil}

% Frequently used commands
\newcommand{\T}{^\top} % Matrix transpose A\T
\newcommand{\C}{^\complement} % Set complement A\C
\newcommand\ceq\coloneqq % Definitions :=
\newcommand\pow{\Cal P} % Power sets
\newcommand\eps\epsilon
\newcommand\ind{\mathbbm 1} % Blackboard 1 for indicator functions
\newcommand\restr{\mathord\restriction}
\newcommand\TODO{{\color{red} TODO: }}
\newcommand\pt\partial
\newcommand\locint{L^1_{\Rm{loc}}}
\newcommand\refl\widecheck
\newcommand\les\lesssim
\newcommand\ges\gtrsim
\newcommand\loc{_\Rm{loc}}

% Functions that appear frequently
\DeclareMathOperator{\sign}{sign}
\DeclareMathOperator{\Int}{Int}
\DeclareMathOperator{\Span}{Span}
\DeclareMathOperator{\Var}{Var}
\DeclareMathOperator*{\argmin}{arg\,min}
\DeclareMathOperator*{\argmax}{arg\,max}
\DeclareMathOperator{\supp}{supp}
\DeclareMathOperator{\ord}{ord}
%\DeclareMathOperator{\PV}{PV}

\title{Distribution Theory --- Example Sheet 2} % subject
\author{Lucas Riedstra}
\date{18 November 2020} % date

\begin{document}
\maketitle

\begin{mdframed}
	We will write $\Rr$ and $\FF$ for the reflection and Fourier transform operators. 
\end{mdframed}
\begin{question}
Let $u, v \in \DD'(\RR^n)$, one of which has compact support. Show that the convolution $u * v$, defined as in your notes, is uniquely defined and gives rise to an element of $\DD'(\RR^n)$. 
\end{question}

\begin{proof}
	The convolution between $u, v \in \DD'(\RR^n)$ is defined by the formula
	\begin{equation} \label{def:convolution}
	(u * v) * \phi = u * (v * \phi) \quad\text{for all $\phi \in \DD(\RR^n)$}. 
	\end{equation}
	To show that this is uniquely defined, recall that for all $u \in \DD'(\RR^n) \phi\in \DD(\RR^n)$, we have $\ang{u, \phi} = (u * \refl\phi)(0)$. Therefore, we have
	\[
	\ang{u * v, \phi} = ((u * v) * \refl\phi)(0) = (u * (v * \refl\phi))(0),
	\]
	which shows that the formula \cref{def:convolution} uniquely defines $\ang{u * v, \phi}$ for any $\phi \in \DD(\RR^n)$, and therefore that $u * v$ is well-defined. For uniqueness, 
	
	Now we prove that $u * v \in \DD'(\RR^n)$: by the previous equation we have
	\[
	\ang{u * v, \phi} = (u * (v * \refl\phi))(0) = \ang{u, \refl{v * \refl\phi}}.
	\]
	
	Suppose $u$ is compactly supported. Since $\refl{v * \refl\phi} \in \EE(\RR^n)$, there exists a compact $K \subseteq X$ and nonnegative $C, N$ such that
	\begin{align*}
		\abs{\ang{u, \refl{v * \refl\phi}}} &\leq C \sum_{\alpha \leq N} \sup_{x \in K} \abs{\pt^\alpha (\refl{v * \refl\phi})} = C \sum_{\alpha \leq N} \sup_{x \in -K} \abs{\pt^\alpha (v * \refl\phi)} = C \sum_{\alpha \leq N} \sup_{x \in -K} \abs{v * \pt^\alpha\refl\phi}\\
		&= C \sum_{\abs{\alpha} \leq N} \sup_{x \in -K} \abs{\ang{v, \tau_x \refl{\pt^\alpha \refl\phi}}}.
	\end{align*}
	Note that if $\supp\phi \subseteq K'$, then $\supp\refl\phi \subseteq -K'$, and for $x \in -K$ we find $\supp \tau_x \refl{\pt^\alpha \refl\phi} \subseteq -K' -K$. Then by the previous equation we find that there exists $C', M$ with
	\[
	\abs{\ang{u * v, \phi}} \leq C' \sum_{\abs{\alpha} \leq N} \sum_{\abs{\beta} \leq M} \sup_{x \in -K' - K} \pt^\beta (\tau_x \refl{\pt^\alpha \refl\phi}) \leq C' \sum_{\abs{\alpha} \leq N} \sum_{\abs{\beta} \leq M} \sup_x \abs{\pt^{\alpha + \beta} \phi} \leq C'' \sum_{\abs{\alpha} \leq M + N} \sup_x \abs{\pt^\alpha \phi},
	\]
	which shows that $u* v \in \DD'(\RR^n)$. An analogous argument holds if $v$ is compactly supported. 
\end{proof}

\begin{question}
	Show that if $u, v, w \in \DD'(\RR^n)$ and at least two of them have compact support, then the convolution is associative (i.e., $(u * v) * w) = u* (v * w)$). 
\end{question}

\begin{proof}
	Note that the convolution between two compactly supported distributions is again compactly supported, which ensures that both expressions `make sense'. Now, let $\phi \in \DD(\RR^n)$, then we have
	\begin{align*}
		((u * v) * w) * \phi &= (u* v) * (w * \phi) = u * (v * ( w* \phi)) = u * ((v * w) * \phi) = (u * (v * w)) * \phi,
	\end{align*}
	which proves the theorem. 
\end{proof}

\begin{question}
	Let $\phi \in \DD(\RR)$ and choose $\eps > 0$ sufficiently small so that $\supp(\phi) \subset I_\eps = (-1/\eps, 1/\eps)$. Given that $\phi$ has a uniformly convergent Fourier series on $I_\eps$ in the form
	\[
	\phi(x) = \sum_{n \in \ZZ} c_n e^{i \eps \pi n x}, \quad c_n = \frac\eps2 \int_\RR \phi(x) e^{-i\eps \pi n x} \dd{x},
	\]
	prove the Fourier inversion theorem on $\DD(\RR)$ by taking a suitable limit. 
\end{question}

\begin{proof}
Since $\DD(\RR) \subseteq \SS(\RR)$, we know that the Fourier inversion formula holds. We only need to show that the Fourier tranform of $\phi$ is again an element of $\DD(\RR)$. (??)
\end{proof}

\begin{question}
	For $\phi \in \SS(\RR^n)$ prove that $\sum_m \phi(m) = \sum_n \hat\phi(2\pi n)$. This is the famous Poisson summation formula. 
\end{question}

\begin{proof}
	We have
	\begin{align*}
		\sum_m \phi(m) = \frac1{(2\pi)^n} \sum_m \int e^{i\lambda m} \hat \phi(\lambda) \dd{\lambda} = \sum_m \int e^{2\pi i \lambda m} \hat \phi(2\pi \lambda) \dd{\lambda}
	\end{align*}
(??)
\end{proof}

\begin{question}
	If $u \in H^s(\RR^n)$ show that $D^\alpha u\in H^{s - \abs{\alpha}} (\RR^n)$. If $s > t$ show that $H^s(\RR^n) \subseteq H^t(\RR^n)$. 
\end{question}

\begin{proof}
	Assuming $u \in H^s(\RR^n)$, we have
	\begin{align*}
		\int_{\RR^n} \ang{\lambda}^{2(s - \abs{\alpha})} \abs{\widehat{D^\alpha u}(\lambda)}^2 \dd{\lambda}  &= \int_{\RR^n} \ang{\lambda}^{2(s-\abs{\alpha})} \norm{\lambda}^{2\abs{\alpha}} \abs{\hat u(\lambda)}^2 \dd{\lambda} \\
		&\les \int_{\RR^n} \ang{\lambda}^{2s} \abs{\hat u(\lambda)}^2 \dd{\lambda} < \infty, 
	\end{align*}
which proves $D^\alpha u \in H^{s- \abs{\alpha}}(\RR^n)$.

The second claim follows immediately from the fact that $\ang{\lambda}^{t} \leq \ang{\lambda}^s$ for $s \geq t$ and $\lambda$ sufficiently large. 
\end{proof}


\begin{question}
	Show that $S(\RR^n)$ is dense in $L^2(\RR^n) = H^0(\RR^n)$ and deduce that $\SS(\RR^n)$ is dense in $H^s(\RR^n)$.
	
	\emph{Hint:} Use Parseval's theorem. 
\end{question}

\begin{proof}
	We will show that $\DD(\RR^n)$ is dense in $L^2(\RR^n)$: since $\DD(\RR^n) \subseteq \SS(\RR^n)$, this shows that $\SS(\RR^n)$ is dense in $L^2(\RR^n)$ as well. 
	
	\TODO Give proof.
	
	Now, for $u \in H^s(\RR^n)$, let $(\phi_m) \to u$ in $L^2$. Then we have
	\begin{align*}
		\norm{\phi_m - u}_{H^s}^2 = \int \ang{\lambda}^{2s} \abs{(\phi_m - u)(\lambda)}^2 \dd{\lambda}
	\end{align*}
\end{proof}
%-----------------------------------------------------------------------------
\setcounter{question}{18}

\begin{question}
	Compute the Fourier transforms of the functions
	\begin{enumerate}[(a)]
		\item $\sign(x)$;
		\item $\arctan(x)$;
		\item $x\log \abs{x} - x$;
		\item $\exp(i\omega x^2)$
	\end{enumerate}
	in $\SS'(\RR)$, where $\omega \in \RR$. 
\end{question}

\begin{proof}
	\begin{enumerate}[(a)]
		\item We have for $\phi \in \SS(\RR)$ that 
		\begin{align*}
			\ang{\widehat{\sign}, \phi} &= \ang{\sign, \hat\phi} = \int_\RR \sign(\lambda) \hat\phi(\lambda)\dd{\lambda} = \int_\RR \sign(\lambda) \int_\RR e^{-i\lambda x} \phi(x) \dd{x} \dd{\lambda}
			\\
			&\overset\star= \int_\RR \phi(x) \int_\RR \sign(\lambda) e^{-i\lambda x} \dd{\lambda} \dd{x} 
			=  \int_\RR \phi(x) \lim_{R\to\infty} \qty(\int_0^R e^{-i\lambda x} \dd{\lambda}  -  \int_{-R}^0 e^{-i\lambda x}  \dd{\lambda})\dd{x} \\
			&= \lim_{\eps\to0} \lim_{R\to\infty} \int_{\abs{x}>\eps}  \phi(x) \qty(\frac{e^{ix R} + e^{-ix R}}{ix} - \frac2{ix}) \dd{x}  \\
			&= \lim_{\eps \to 0} \lim_{R\to\infty} \int_\RR \frac{\phi(x) \ind_{\abs{x}> \eps}}{ix} \cdot e^{ixR} \dd{x} + \lim_{\eps \to 0} \lim_{R\to\infty} \int_\RR \frac{\phi(x) \ind_{\abs{x}> \eps}}{ix} \cdot e^{-ixR} \dd{x} + 2i \PV\qty(\frac1x).
%			&= \lim_{R\to\infty}\qty( \int_\RR \phi(x) \int_0^R e^{-i\lambda x}\dd{x} - \int_\RR \phi(x) \int_{-R}^0 e^{-i\lambda x} \dd{x})
		\end{align*}
	We claim the first two terms go to 0: this is because the term in the integral is the Fourier transform of $\frac{\phi(x)\ind_{\abs{x} > \eps}}{ix}$ evaluated at $\pm R$, and since the function is in $L^1$, its Fourier transform decays to 0 as $\abs{R} \to\infty$. 
	
	We conclude $\widehat\sign = 2i \PV(\frac1x)$. 
	
	\item We know that $\arctan'(x) = \frac{1}{1 + x^2} \eqqcolon f(x)$, then we have $\widehat\arctan(\lambda) = \frac1{i\lambda} \hat f(\lambda)$ (in the distributional sense). 
	
	We have, using Fubini and the fact that $\ang{\hat f, \phi}$ is finite, that 
	\begin{align*}
		\ang{\hat f, \phi} &= \int_\RR \frac{\hat\phi(\lambda)}{1 + \lambda^2} \dd{\lambda} = \int_\RR \phi(x) \lim_{R\to\infty} \int_{-R}^R \frac{e^{-i\lambda x}}{1 + \lambda^2} \dd{\lambda} \dd{x} \overset\star= \int_\RR \phi(x) \qty(\pi e^{-\abs{x}}) \dd{x}, 
	\end{align*}
	from which it follows that the Fourier transform of $\frac{1}{1 + x^2}$ is given by $\pi e^{-\abs{\lambda}}$, and therefore the Fourier transform of $\arctan$ is given by $\frac{\pi}{i\lambda} e^{-\abs{\lambda}}$. 
	
	\item The derivative of this function, outside of 0, is $\log\abs{x}$. 
	
	\item  Clearly, if $\omega = 0$, the function is 1 and its Fourier transform is $2\pi\delta_0$, so assume $\omega \neq 0$. 
	We have analogously to (b), with $f(x) = \exp(i\omega x^2)$, that 
	\[
	\ang{\widehat{f}, \phi} = \int_\RR \hat\phi(\lambda) e^{i\omega \lambda^2} \dd{\lambda} = \int_\RR \phi(x) \lim_{R\to\infty} \int_{-R}^R e^{i\omega\lambda^2 - i\lambda x} \dd{\lambda} \dd{x}.
	\]
	Now, by completing the square we have
	\[
	i(\omega\lambda^2 - x\lambda) = i\qty(\sqrt\omega\lambda - \frac{x}{2\sqrt\omega})^2 - \frac{ix^2}{4\omega},
	\]
	and therefore
	\begin{align*}
	\lim_{R\to\infty} \int_{-R}^R e^{i\omega\lambda^2 - i\lambda x} \dd{\lambda} &= e^{-ix^2/(4\omega)} \lim_{R\to\infty} \int_{-R}^R e^{i(\sqrt\omega\lambda - \frac{x}{2\sqrt\omega})^2}\dd{\lambda} = \frac{e^{-ix^2/(4\omega)}}{\sqrt\omega} \lim_{R\to\infty} \int_{-R}^R e^{i\lambda^2} \dd{\lambda} \\
	&= \frac{e^{-ix^2/(4\omega)}}{\sqrt\omega} (1+i)\sqrt{\frac\pi2}, 
	\end{align*}
	where we use that the \emph{Fresnel integral} $\int_{-\infty}^\infty e^{ix^2} \dd{x}$ is known. 
	
	Plugging this back into our original equation yields
	\[
	\ang{\hat f, \phi} = \int_\RR \phi(x) \frac{e^{-ix^2/(4\omega)}}{\sqrt\omega} (1+i)\sqrt{\frac\pi2} \dd{x}, 
	\]
	which shows that
	\[
	\hat f(\lambda) =  (1+i) e^{-i\lambda^2/(4\omega)} \sqrt{\frac{\pi}{2\omega}}. 
	\]
	in the distributional sense. 
	\end{enumerate}
\end{proof}
\end{document}