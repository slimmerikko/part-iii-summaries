\documentclass{article}

\usepackage[utf8]{inputenc}
\usepackage[english]{babel}
\usepackage[margin=3cm]{geometry}
\usepackage[normalem]{ulem}
\usepackage{hyperref}
\usepackage[shortlabels]{enumitem}
\usepackage{mathtools, amsmath, amssymb, amsthm, mathabx, mdframed, bbm, graphicx, float, physics, xcolor, cleveref}

\hypersetup{
    colorlinks   = true, %Colours links instead of ugly boxes
    urlcolor     = blue, %Colour for external hyperlinks
    linkcolor    = blue, %Colour of internal links
    citecolor   = red %Colour of citations
}

% Definition of numbered environments.
% Usage: \begin{theorem} ... \end{theorem}
% Remark has no numbering.
\theoremstyle{plain}
\newtheorem{question}{Question}
\newtheorem{theorem}{Theorem}
\newtheorem{lemma}{Lemma}
\theoremstyle{remark}
\newtheorem*{remark}{Remark}

% Question with box around it
\newenvironment{qbox}{\begin{mdframed}\begin{question}}{\end{question}\end{mdframed}}

% A proof with "solution" instead of "proof" and no QED symbol
\newenvironment{solution}{\begin{proof}[Solution]\renewcommand\qedsymbol{}}{\end{proof}}

% Some renewed commands
\renewcommand{\vec}{\mathbf}
\renewcommand{\emptyset}{\varnothing}
\renewcommand{\epsilon}{\varepsilon}
\renewcommand{\theta}{\vartheta}
\renewcommand{\phi}{\varphi}

% Frequently used math alphabets
\newcommand{\Bb}{\mathbb}
\newcommand{\Cal}{\mathcal}
\newcommand{\Bf}{\mathbf}
\newcommand{\Rm}{\mathrm}

% Frequently used letters in the blackboard alphabet
\newcommand{\CC}{\Bb C}
\newcommand{\NN}{\Bb N}
\newcommand{\PP}{\Bb P}
\newcommand{\QQ}{\Bb Q}
\newcommand{\RR}{\Bb R}
\newcommand{\EE}{\Cal E}
\newcommand{\DD}{\Cal D}
\newcommand\ZZ{\Bb Z}

% Usage: \ang{...} is equivalent to \langle ... \rangle, while \ang*{...} is equivalent to \left\langle ... \right\rangle
% For other delimiters: use \qty from the physics package (i.e., \qty(...))
\DeclarePairedDelimiter{\ang}{\langle}{\rangle}
\DeclarePairedDelimiter{\floor}{\lfloor}{\rfloor}
\DeclarePairedDelimiter{\ceil}{\lceil}{\rceil}

% Frequently used commands
\newcommand{\T}{^\top} % Matrix transpose A\T
\newcommand{\C}{^\complement} % Set complement A\C
\newcommand\ceq\coloneqq % Definitions :=
\newcommand\pow{\Cal P} % Power sets
\newcommand\eps\epsilon
\newcommand\ind{\mathbbm 1} % Blackboard 1 for indicator functions
\newcommand\restr{\mathord\restriction}
\newcommand\TODO{{\color{red} TODO: }}
\newcommand\pt\partial
\newcommand\locint{L^1_{\Rm{loc}}}
\newcommand\refl\widecheck

% Functions that appear frequently
\DeclareMathOperator{\sign}{sign}
\DeclareMathOperator{\Int}{Int}
\DeclareMathOperator{\Span}{Span}
\DeclareMathOperator{\Var}{Var}
\DeclareMathOperator*{\argmin}{arg\,min}
\DeclareMathOperator*{\argmax}{arg\,max}
\DeclareMathOperator{\supp}{supp}
\DeclareMathOperator{\ord}{ord}
\DeclareMathOperator{\prv}{p.\!v.}

\newcommand\clos\overline

\title{Distribution Theory and Applications --- Example Sheet 1} % subject
\author{Lucas Riedstra}
\date{4 November 2020} % date

\begin{document}
\maketitle


\begin{mdframed}
For both question 2, we need the following lemma:
\begin{lemma} \label{lem:distance_compact_closed}
	Let $K, V \subseteq \RR^n$ where $K$ is compact, $V$ is closed, and $K \cap V = \emptyset$. Then there is a nonzero distance between $K$ and $V$, i.e., 
	\[
	\inf_{x \in K, v \in V} \norm{x - v} > 0. 
	\]
\end{lemma}

\begin{proof}
	We know that $K \subseteq V\C$ and that $V\C$ is open, so for every $x \in K$ there exists an open ball $B(x, \eps_x)$ around $x$ such that $B(x, 2\eps_x) \subseteq V\C$. Since $\qty{B(x, \eps_x)}$ is an open covering of $K$, there exist finitely many balls $B(x_1, \eps_1), \dotsc, B(x_n, \eps_n)$ that cover $K$. Let $\eps \ceq \min\qty{\eps_1, \dotsc, \eps_n}$ and $x \in K$, then there is an $x_i$ such that $\norm{x - x_i} < \eps$, and since $B(x_i, 2\eps) \subseteq B(x_i, 2\eps_i) \subseteq V\C$ it is clear that $B(x, \eps) \subseteq V\C$ as well. 
	
	We conclude that $B(x, \eps) \subseteq V\C$ for any $x \in K$, and therefore that $\inf\limits_{x \in K, v \in V} \norm{x - v} \geq \eps > 0$. 
\end{proof}
\end{mdframed}

\setcounter{question}{1}
\begin{question}
	Given $\phi \in \DD(X)$, Taylor's theorem gives
	\[
	\phi(x+h) = \sum_{\abs{\alpha} \leq N} \frac{h^\alpha}{\alpha!} \pt^\alpha \phi(x) + R_N(x, h).
	\]
	Prove that $\supp(R_N)$ is contained in some fixed compact $K \subseteq X$ for $\abs{h}$ sufficiently small. Show also that $\pt^\alpha R_N = o(\abs{h}^N)$ uniformly in $x$ for each multi-index $\alpha$, i.e.\ prove 
	\[
	\lim_{\abs{h}\to 0} \frac{\sup_x \abs{\pt^\alpha R_N(x, h)}}{\abs{h}^N} = 0
	\]
	for each multi-index $\alpha$.  
	
	\emph{Hint:} you may find it convenient to use the following form of the remainder
	\[
	R_N(x, h) = \sum_{\abs{\alpha} = N+1} \frac{h^\alpha}{\alpha!}(N+1)\int_0^1 (1-t)^N (\pt^\alpha\phi)(x + th)\dd{t},
	\]
	and note that $(N+1)! \sum_{\abs{\alpha} = N+1} \frac{h^\alpha}{\alpha!} = (h_1 + \dotsb + h_n)^{N+1}$. 
\end{question}

\begin{proof}
	Let $\phi \in \DD(X)$ with $K = \supp\phi$, then by \cref{lem:distance_compact_closed} we know there exists a nonzero distance $d > 0$ between $K$ and $\RR^n \setminus X$. 
	We claim that if $\norm{h} \leq \frac d2$, then 
	\[
	\supp(R_N) \subseteq \qty{x \in X \mid d(x, K) \leq \frac d2} \eqqcolon \hat K,
	\]
	which is clearly a compact set contained in $X$. Indeed, if $\norm{h} \leq \frac d2$ we have
	\[
	\phi(x+h) \neq 0 \implies x + h \in K \implies d(x, K) \leq \norm{h} \leq \frac d2. 
	\]
	
	
%	Since $\phi \in \DD(X)$, we know that $\supp\phi \subseteq \clos{B_N}$ for some $N \in \NN$. Now, suppose $\norm{h} < 1$, then
%	\[
%	\phi(x+h) \neq 0 \implies \norm{x + h} \leq N \implies \norm{x} \leq \norm{x + h} + \norm{h} \leq N + 1, 
%	\]
%	so if we define $\psi_h(x) = \phi(x+h)$ then we know that $\supp\phi_h \subseteq \clos{B_{N+1}}$. 
%	
	By Taylor's theorem, we have
	\[
	\phi(x+h) = \sum_{\abs{\alpha} \leq N} \frac{h^\alpha}{\alpha!} \pt^\alpha \phi(x) + R_N(x, h),  
	\]
	and since $\sum_{\abs{\alpha} \leq N} \frac{h^\alpha}{\alpha!} \pt^\alpha \phi(x)$ vanishes for $x \notin K$, it is clear that $\supp(R_N(\cdot, h))$ must be contained in $\hat K$ (for $\norm{h} \leq \frac d2$). 
	
	Now let $\beta$ be a multi-index and define $C \ceq \frac{1}{N!} \max_{\abs{\alpha} = N+1, x \in \RR^n} \abs{(\pt^{\alpha + \beta})\phi(x)}$ (note that $C$ exists and is finite since all partial derivatives of $\phi$ have compact support), then we have
	\begin{align*}
	\abs{\pt^\beta R_N(x, h)} &= \abs{\pt^\beta \sum_{\abs{\alpha} = N+1} \frac{h^\alpha}{\alpha!}(N+1)\int_0^1 (1-t)^N (\pt^\alpha\phi)(x + th)\dd{t}} \\
	%&= \abs{\sum_{\abs{\alpha} = N+1} \frac{h^\alpha}{\alpha!}(N+1) \cdot \pt^\beta\int_0^1 (1-t)^N (\pt^\alpha\phi)(x + th)\dd{t}} \\
	&\overset\star= \abs{\sum_{\abs{\alpha} = N+1} \frac{h^\alpha}{\alpha!}(N+1) \int_0^1 (1-t)^N (\pt^{\alpha + \beta}\phi)(x + th)\dd{t}} \\
	&\leq \sum_{\abs{\alpha} = N+1} \frac{\abs{h^\alpha}}{\alpha!}(N+1) \int_0^1 (1-t)^N \abs{\qty(\pt^{\alpha + \beta}\phi)(x+th)} \dd{t} \\
	&\leq \qty[\max_{\abs{\alpha} = N+1, x \in \RR^n} \abs{(\pt^{\alpha + \beta})\phi(x)}] \sum_{\abs{\alpha} = N+1} \frac{\abs{h^\alpha}}{\alpha!} (N+1) \\
	&= C (N+1)! \sum_{\abs{\alpha}= N+1}\frac{\abs{h^\alpha}}{\alpha!} = C(\abs{h_1} + \dotsb + \abs{h_n})^{N+1}.
\end{align*}
Here, $\star$ follows from differentiation under the integral sign since the integrand is bounded. 

Since this upper bound does not depend on $x$, we also have
\[
\sup_x \abs{\pt^\beta R_N(x, h)} \leq C(\abs{h_1} + \dotsb + \abs{h_n})^{N+1}, 
\]
and we conclude that
\[ 
\frac{\sup_x \abs{\pt^\beta R_N(x, h)}}{\norm{h}^N} \leq \frac{C(\abs{h_1} + \dotsb + \abs{h_n})^{N+1}}{\norm{h}^N} \leq \frac{C N^{N+1} \norm{h}^{N+1}}{\norm{h}^N} = CN^{N+1} \norm{h} \to 0, 
\]
and therefore that $\pt^\beta R_N(x, h) = o(\norm{h}^n)$ for all multi-indices $\beta$. 
\end{proof}


\setcounter{question}{7}
\begin{question}
	Find the most general solution to the equations
	\begin{enumerate}[(a)]
		\item $u'  = 1$,
		\item $xu' = \delta_0$,
		\item $(e^{2\pi i x} - 1)u' = 0$
	\end{enumerate}
	in $\DD'(\RR)$. 
\end{question}

\begin{solution}
	Let $\phi \in \DD(X)$. 
	\begin{enumerate}[(a)]
		\item It is clear that $x' = 1$ in the distributional sense, since for any test function $\phi$ we have
		\[
		\ang{x', \phi} = -\ang{x, \phi'} = - \int_\RR x \phi'(x) \dd{x} = \int_\RR \phi(x)\dd{x} = \ang{1, \phi}. 
		\]
		Therefore, the equation $u' = 1$ is equivalent to the equation $(u - x)' = 0$. We know that this implies that $u - x = c$ for some constant $c \in \CC$, so the most general solution is $u = x + c$. 
		
		\item If $xu' = \delta_0$ then we have
		\[
		\phi(0) = \ang{\delta_0, \phi} = \ang{xu', \phi} = \ang{u', x\phi} = \ang{u, -(x\phi)'} = \ang{u, -x\phi' - \phi} %=  \ang{u, -x\phi'} + \ang{u, -\phi}. 
		\]
		Note that the above is satisfied for $u = - \delta_0 + c$ for any constant $c\in \CC$.
		
		I do not know if this is the most general solution and/or how one would show this. 
		
		\item Since $e^{2\pi i x} = 1 \iff x \in \ZZ$, it is clear that $\supp(u') \subseteq \ZZ$. We will show that this is also sufficient, i.e., that any distribution $u$ with $\supp(u') \subseteq \ZZ$ yields a solution. 
		
		It is easily seen that 
		\begin{align*}
		\supp(u') \subseteq  \ZZ &\iff u' = \sum_{n \in \ZZ} \alpha_n \delta_n \quad\text{for some $(\alpha_n)_{n \in \ZZ} \subseteq \CC$} \\
		&\iff u = c + \sum_{n\in \ZZ} \alpha_n \ind_{x \geq n} \quad\text{for some $c \in \CC$, $(\alpha_n)_{n \in \ZZ} \subseteq \CC$}. 
		\end{align*}
		
		Indeed, if $u = c + \sum_{n \in \ZZ} \alpha_n \ind_{x \geq n}$ then 
		\[
		\ang{u', \phi} = -\ang{u, \phi'} \overset\star= - \alpha_n \sum_{n \in \ZZ \cap \supp\phi} \int_n^\infty \phi'(x) \dd{x} = \sum_{n \in \ZZ \cap \supp\phi} \alpha_n \phi(n) = \ang{\sum_{n \in \ZZ} \alpha_n\delta_n, \phi}, 
		\]
		where $\star$ follows from the fact that there are only finitely many $n$ in $\ZZ \cap \supp\phi$ (since $\phi$ has compact support). 
 		
%		Therefore, we guess that, for any sequence $(\alpha_n)_{n \in \ZZ} \subseteq \CC$ and constant $c \in \CC$, the map \[u = c + \sum_{n \in \ZZ} \alpha_n \ind_{x \geq n}.\] We compute the derivative of $u$. It is easily seen that $u' = \sum_{n \in \ZZ} \alpha_n \delta_n$ (the infiniteness of the sum does not pose a problem since the test functions are compactly supported, so $\ang{u, \phi}$ will always be a finite sum). 
%		We have for arbirary $\phi$ that
%		\begin{align*}
%		\ang{u', \phi} &= - \ang{u, \phi'} = -\int_\RR \sum_{n \in \ZZ} \qty(\alpha_n \ind_{x \geq n}) \phi'(x) \dd{x}	\overset\star= - \sum_{n \in \ZZ} \alpha_n \int_n^\infty \phi'(x) \dd{x} = \sum_{n \in \ZZ} \alpha_n \phi(n),	
%		\end{align*}
%	where $\star$ follows from the fact that the sum is finite since $\phi'$ is compactly supported. 
%	Therefore, we can conclude $u' = \sum_{n \in \ZZ} \alpha_n \delta_n$, and indeed we find
Finally, we compute that 
	\[
	\ang{(e^{2\pi i x} - 1) u', \phi} = \ang{u', (e^{2\pi i x} -1 )\phi} = \sum_{n \in \ZZ} \alpha_n (e^{2\pi i n} - 1) \phi(n) = 0,
	\]
	so $u$ satisfies the equation. 
	\end{enumerate}
\end{solution}


\end{document}