\section{Distributions with compact support}
\begin{definition}
	Let $Y \subseteq X$ be open and $u \in \DD'(X)$. We say that $u$ \emph{vanishes} on $Y$ if $\ang{u, \phi} = 0$ for all $\phi \in \DD(Y)$. 
\end{definition}

\begin{definition}
	For $u \in \DD'(X)$, we define the \emph{support} of $u$ as
	\[
	\supp u \ceq X \setminus \bigcup\qty{Y \subseteq X \mid Y \text{ open}, u \text{ vanishes on $Y$}}. 
	\]
\end{definition}
For example, the support of $\delta_x$ is simply $\qty{x}$. 

\subsection{Test functions and distributions}
We will now consider a bigger space of test functions, which yields a smaller space of distributions. 
\begin{definition}
	We define $\EE(X)$ as the space of smooth functions $\phi \colon X \to \CC$. We say that a sequence $(\phi_m) \subseteq \EE(X)$ converges to 0 if $\pt^\alpha\phi \to 0$ uniformly on compact subsets of $X$ for every multi-index $\alpha$. 
\end{definition}

\begin{definition}
	We define 
	%the space of \emph{distributions with compact support}, denoted 
	$\EE'(X)$ as the space of linear maps $u \colon \EE(X) \to \CC$ for which there exists a compact $K \subseteq X$ and nonnegative constants $C, N$ such that
	\begin{equation} \label{eq:dist_compact_support}
	\abs{\ang{u, \phi}} \leq C \sum_{\alpha \leq N} \sup_K \abs{\pt^\alpha\phi}
	\end{equation}
	for all $\phi \in \EE(X)$. 
\end{definition}

\begin{lemma}[Sequential continuity] \label{lem:seq_continuity_compact_support}
	Let $u \colon \EE(X) \to \CC$ be a linear map. Then $u \in \EE'(X)$ if and only if, for every sequence $(\phi_m) \subseteq \EE(X)$ with $\phi_m \to 0$, we have $\ang{u, \phi_m} \to 0$. 
\end{lemma}

\begin{proof}
	\TODO
\end{proof}

\begin{lemma}
	If $u \in \EE'(X)$, then $u\restr_{\DD(X)}$ defines an element of $\DD'(X)$ with compact support and finite order. 
	
	Conversely, for each $u \in \DD'(X)$ with compact support there exists a unique extension $\tilde u \in \EE'(X)$ with $\supp(\tilde u) = \supp(u)$ and $\tilde u\restr_{\DD(X)} = u$. 
\end{lemma}

\begin{proof}
	Let $u \in \EE'(X)$, so that there exists a compact $K \subseteq X$ with $\abs{\ang{u, \phi}} \leq C \sum_{\abs{\alpha} \leq N} \sup_K \abs{\pt^\alpha \phi}$. 
	Now, for any compact $K' \subseteq X$ and any $\phi$ with $\supp\phi \subseteq K'$, \cref{eq:seminorm_cond} is clearly satisfied, and we can use the same $N$ for all compact $K'$, so clearly $u\restr_{\DD(X)}$ is an element of $\DD'(X)$ with finite order.
	Finally, suppose $\phi$ is supported in $X \setminus K$, then it is clear that $\ang{u, \phi} = 0$, which proves that $\supp u \subseteq K$ and therefore that $u$ has compact support.
	
	Now suppose $u \in \DD'(X)$ has compact support, let $\rho \in \DD(X)$ be 1 in a neighbourhood of $\supp u$, and define $\tilde u \in \EE'(X)$ by
	\[
	\ang{\tilde u, \phi} \ceq \ang{u, \rho\phi} \quad\forall \phi \in \EE(X). 
	\]
	Clearly $\tilde u$ is an element of $\EE'(X)$ since $\supp(\rho\phi) \subseteq \supp\rho$ and 
	\[
	\abs{\ang{\tilde u, \phi}} = \abs{\ang{u, \rho\phi}} \leq C \sum_{\abs{\alpha} \leq N} \sup_{\supp(\rho)} \abs{\pt^\alpha(\rho \phi)} \overset\star\leq C' \sum_{\abs{\alpha} \leq N} \sup_{\supp\rho} \abs{\pt^\alpha\phi},
	\]
	where $\star$ follows from the Leibniz rule. It is also clear that $\supp \tilde u = \supp u$. 
	
	Finally we will show uniqueness: suppose $\tilde v$ is an extension of $u$ with $\supp \tilde v = \supp u$, and write any $\phi \in \EE(X)$ as $\phi = \rho\phi + (1-\rho)\phi = \phi_0 + \phi_1$. Then since $\phi_0 \in \DD(X)$ and $\phi_1$ vanishes on a neighbourhood of $\supp u$, we find
	\[
	\ang{\tilde v, \phi} = \ang{\tilde v, \phi_0 + \phi_1} = \ang{\tilde v, \phi_0} = \ang{u, \phi_0},
	\]
	which is independent of choice of extension. 
\end{proof}

\subsection{Convolution between $\EE'(\RR^n)$ and $\DD'(\RR^n)$}
\begin{definition}
	Define for $u \in \EE'(\RR^n), \phi \in \EE(\RR^n)$ the \emph{convolution}
	\[
	(u * \phi)(x) = \ang{u, \tau_x \refl\phi}. 
	\]
\end{definition}
This convolution satisfies the same properties as the convolution between $\DD'(\RR^n)$ and $\DD(\RR^n)$. 
Also, if $\phi \in \DD(\RR^n)$, then $u * \phi \in \DD(\RR^n)$. 

\begin{definition}
	Let $u, v \in \DD'(\RR^n)$, at least one of which has compact support, define $u * v \colon \DD(\RR^n) \to \EE(\RR^n)$ by
	\[
	(u * v) * \phi = u * (v * \phi) \quad \forall \phi \in \DD(\RR^n). 
	\]
\end{definition}
We will show (example sheet 2, question 1) that $u * v$ is uniquely defined and gives rise to an element of $\DD'(\RR^n)$ via $\ang{u * v, \phi} \ceq [(u * v) * \refl\phi](0)$. 

\begin{lemma}
	Given $u, v \in \DD'(\RR^n)$, at least one of which has compact support, we have $u * v = v * u$. 
\end{lemma}

\begin{proof}
	First we note that $(u * \phi) * \psi = u * (\phi * \psi)$ holds if $u$ has compact support and at least one of $\phi, \psi$ has compact support. 
	
	Fix $\phi, \psi \in \DD(\RR^n)$, we see from our earlier shown properties that 
	\begin{align*}
	(u * v) * (\phi * \psi) &= u * \qty(v * (\phi * \psi)) = u * ((v * \phi) * \psi) = u * (\psi * (v * \phi)) = (u * \psi) * (v * \phi).
	\end{align*}
	If we interchange $u$ and $v$ in the above, that is equivalent to interchanging $\phi$ and $\psi$, which we know must yield the same result. This shows $u * v$ and $v * u$ agree on $\phi * \psi$ for all $\phi, \psi \in \DD(\RR^n)$. Defining $E = u * v - v * u$, we find that $0 = E * (\phi * \psi) = (E * \phi) * \psi$ for all $\phi, \psi \in \DD(\RR^n)$, so $E * \phi = 0$ for all $\phi \in \DD(\RR^n)$, so $E = 0$. 
\end{proof}

\section{Tempered distributions and Fourier analysis}
\subsection{Functions of rapid decay}
\begin{definition}
	For any $f \colon \RR^n \to \CC$ and multi-indices $\alpha, \beta$ we define $\norm{f}_{\alpha, \beta} \ceq \sup_{x \in \RR^n} \abs{x^\alpha \pt^\beta\phi}$. 
	
	We define the \emph{Schwarz space}
	\[
	\Cal S(\RR^n) \ceq \qty{f \in C^\infty(\RR^n) : \norm{f}_{\alpha, \beta} < \infty \text{ for all $\alpha, \beta$}}.
	\]
	
	We say that a sequence $(\phi_n) \subseteq \SS(\RR^n)$ converges to 0 if $\norm{\phi_m}_{\alpha, \beta} \to 0$ for every $\alpha, \beta$.  
\end{definition}

\begin{example}
	The function $x \mapsto \exp(-\norm{x}^2)$ lies in $\SS(\RR^n)$. 
\end{example}

\begin{proposition}
	For all $n$ we have that $\SS(\RR^n) \subseteq L^1(\RR^n)$. 
\end{proposition}

\begin{proof}
	Let $\phi \in \SS(\RR^n)$, then for all $N \in \NN$ we have
	\[
	\int_{\RR^n} \abs{\phi(x)} \dd{x} = \int_{\RR^n} (1 + \norm{x})^{-N} (1 + \norm{x})^N \abs{\phi(x)} \dd{x} \overset?\leq C\sum_{\abs{\alpha} \leq N} \norm{\phi}_{\alpha, 0} \int_{\RR^n} (1 + \norm{x})^{-N} \dd{x}.
	\]
	
	Since $\int_{\RR^n} (1 + \norm{x})^{-N} \dd{x}$ is finite for $N$ large enough (??), this proves the claim. 
\end{proof}


\begin{definition}
	A linear map $u \colon \SS(\RR^n) \to \CC$ is called a \emph{tempered distribtion} if there exists constants $C, N$ such that
	\[
	\abs{\ang{u, \phi}} \leq C \sum_{\abs{\alpha}, \abs{\beta} \leq N} \norm{\phi}_{\alpha, \beta} \quad\text{for all $\phi \in \SS(\RR^n)$}. 
	\]
\end{definition}
This definition is equivalent to sequential continuity. 

\subsection{The Fourier transform on $\SS(\RR^n)$}

\begin{convention}
	We write $D \ceq -i\pt$ and $D^\alpha = (-i)^{\abs{\alpha}} \pt^\alpha$. 
\end{convention}

\begin{definition}
	For $f \in L^1(\RR^n)$, define the \emph{Fourier transform} of $f$ by  
	\[
	[\FF(f)](\lambda) = \hat f(\lambda) \ceq \int_{\RR^n} e^{-i\lambda x} f(x) \dd{x} \quad\text{where $\lambda \in \RR^n$}. 
	\]
\end{definition}

\begin{lemma}
	If $f \in L^1(\RR^n)$, then $\hat f$ is continuous. 
\end{lemma}

\begin{proof}
	If $\lambda_m \to \lambda \in \RR^n$, then by the dominated convergence theorem we have
	\[
	\hat f(\lambda_m) = \int_{\RR^n} e^{-i\lambda_m x} f(x) \dd{x} \overset{\text{DCT}}= \int_{\RR^n} e^{-i\lambda x} f(x) \dd{x} = \hat f(\lambda),
	\]
	where we were able to use the dominated convergence theorem since the integrand is absolutely bounded by $\abs{f}$ and $f \in L^1$. 
\end{proof}

It turns out that this idea generalises: the faster the function $f$ decays, the smoother the Fourier transform $\hat f$ is.  

\begin{lemma}
	For $\phi \in \SS(\RR^n)$, we have $\FF[D_x^\alpha \phi](\lambda) = \lambda^\alpha \hat\phi(\lambda)$ and $\FF[x^\beta\phi](\lambda) = (-1)^{\abs{\beta}} D_\lambda^\beta \hat\phi(\lambda)$. 
\end{lemma}

\begin{proof}
	Since $\abs{x^\alpha D^\beta\phi} \to 0$ as $\norm{x} \to \infty$, we have using integration by parts
	\begin{align*}
		\FF[D_\lambda^\alpha\phi](\lambda) &= \int_{\RR^n} e^{-i\lambda x} D_x^\alpha \phi(x) \dd{x} \\
		&= (-1)^{\abs{\alpha}} \int_{\RR^n} D_x^\alpha(e^{-i\lambda x}) \phi(x) \dd{x} \\
		&= \lambda^\alpha\hat\phi(\lambda). 
	\end{align*}
	
	For the second part of the lemma, by differentiation under the integral sign we get
	\begin{align*}
		 \FF[x^\beta \phi](\lambda) &=  \int_{\RR^n} e^{-i\lambda x} x^\beta \phi(x) \dd{x} \\
		 &= \int_{\RR^n} ((-D_\lambda)^\beta e^{-i\lambda x}) \phi(x) \dd{x} \\
		 &= (-1)^{\abs{\beta}} D_\lambda^\beta \hat\phi(\lambda).
	\end{align*}
\end{proof}


We define the \emph{inverse Fourier transform} by 
\[
\FF^{-1}[\hat f](x)  \ceq \frac{1}{(2\pi)^n} \int_{\RR^n} e^{i\lambda x} \hat f(\lambda) \dd{\lambda}. 
\]
We will now show that on $\SS(\RR^n)$, the inverse Fourier transform is indeed an inverse:
\begin{theorem}
	The Fourier transform $\FF$ defines a continuous isomorphism from $\SS(\RR^n)$ to itself. 
\end{theorem}

\begin{proof}
	Let $\phi \in \SS(\RR^n)$. First, we show that $\hat\phi \in \SS(\RR^n)$: by the previous lemma we have for multi-indices $\alpha, \beta$ that
	\begin{align}
		\abs{\lambda^\alpha (-D_\lambda)^\beta \hat\phi(\lambda)} &= \abs{\lambda^\alpha \FF[x^\beta\phi](\lambda)} = \abs{\FF[D^\alpha_x (x^\beta\phi)](\lambda)} = \abs{\int_{\RR^n} e^{-i\lambda x} D^\alpha(x^\beta\phi) \dd{x}} \nonumber \\
		&\leq \int_{\RR^n} \abs{D^\alpha (x^\beta\phi)} \dd{x} \label{eq:fourier_schwarz_estimate},
	\end{align}
which is finite since $D^\alpha (x^\beta\phi)$ is also a Schwarz function and therefore integrable. 

From the previous lemma we also infer that $\hat\phi$ is smooth, so indeed we have $\hat\phi \in \SS(\RR^n)$. 
From \cref{eq:fourier_schwarz_estimate} it is also easily seen that if $\phi_m \to 0$ in $\SS(\RR^n)$, then $\hat\phi_m \to 0$ in $\SS(\RR^n)$ also, which shows that $\FF$ is continuous. 

To prove surjectivity and injectivity, we will show that $\FF^{-1}[\FF[f]](x) = f(x)$ (???). Indeed we have
\begin{align*}
	\FF^{-1} [\FF[\phi]](x) &= \frac{1}{(2\pi)^n} \int_{\RR^n} \int_{\RR^n} e^{i\lambda (x-y)} \phi(y) \dd{y}\dd{\lambda} \\
	&= \frac{1}{(2\pi)^n} \lim_{\eps \downarrow 0} \int_{\RR^n} \int_{\RR^n} e^{i \lambda ( x- y) - \eps\norm{\lambda}^2} \phi(y) \dd{y}\dd{\lambda} \\
	&\overset\star= \frac{1}{(2\pi)^n}\lim_{\eps \downarrow 0} \int_{\RR^n} \phi(y)\int_{\RR^n} e^{i\lambda(x- y) - \eps\norm{\lambda}^2} \dd{\lambda} \dd{y},
	%&= \frac{1}{(2\pi)^n} \int_{\RR^n}\int_{\RR^n} e^{i\lambda(x- y)} \dd{\lambda} \dd{y}
\end{align*}
where $\star$ follows from Fubini's theorem. 

Now we note that 
\[
\int_{\RR^n} e^{i\lambda(x-y) - \eps\norm{\lambda}^2} \dd{\lambda} = \prod_{i=1}^n \int_{\RR} e^{i\lambda_j(x_j - y_j) - \eps\lambda_j^2} \dd{\lambda} \overset{\star\star}= \prod_{i=1}^n \qty(\frac\pi e)^{1/2} e^{- \frac{(x_i - y_i)^2}{4\eps}} = \qty(\frac\pi\eps)^{n/2} e^{- \frac{\norm{x - y}^2}{4\eps}}.
\] 
To explain $\star\star$, \TODO. 


and plugging that into the above yields 
\begin{align*}
\FF^{-1}[\FF[\phi]](x) &= \lim_{\eps\downarrow 0} 2^{-n}(\pi \eps)^{-n/2} \int_{\RR^n} \phi(y) e^{-\norm{x-y}^2/(4\eps)} \dd{y} \\
&\overset{\star\star\star}= \pi^{-n/2} \lim_{\eps\downarrow0}\int_{\RR^n} \phi(x - 2\sqrt\eps y') e^{-\norm{y'}^2} \dd{y} \\
&\overset{\text{DCT}}=\pi^{-n/2} \phi(x) \int_{\RR^n} e^{-\norm{y'}^2} \dd{y} = \phi(x),
\end{align*}
where $\star\star\star$ follows from the substitution $x - y = 2\sqrt\eps y'$.

Finally, continuity of $\FF^{-1}$ is easily shown with an argument analogous to that for continuity of $\FF$ (????). 
\end{proof}


